\documentclass[10pt]{article}
\usepackage[T1]{fontenc}
\usepackage[utf8]{inputenc}%ATTENTION codage en utf8 !!! 
\usepackage[upright]{fourier}
\usepackage[scaled=0.875]{helvet}
\renewcommand{\ttdefault}{lmtt}
\usepackage{amsmath,amssymb,makeidx}
\usepackage{fancybox}
\usepackage{tabularx}
\usepackage[normalem]{ulem}
\usepackage{pifont}
\usepackage{textcomp} 
\usepackage{multicol}
\newcommand{\euro}{\eurologo{}}
%Tapuscrit : Denis Vergès
%Merci à Florence Francillette et à Vincent Tolleron pour le sujet
\usepackage{pst-plot,pst-text}
\newcommand{\R}{\mathbb{R}}
\newcommand{\N}{\mathbb{N}}
\newcommand{\D}{\mathbb{D}}
\newcommand{\Z}{\mathbb{Z}}
\newcommand{\Q}{\mathbb{Q}}
\newcommand{\C}{\mathbb{C}}
\newcommand{\abs}[1]{\left\vert#1\right\vert}
\usepackage[left=3cm, right=3cm, top=3cm, bottom=3cm]{geometry}
\newcommand{\vect}[1]{\overrightarrow{\,\mathstrut#1\,}}
\newcommand{\barre}[1]{\overline{\,\mathstrut#1\,}}
\renewcommand{\theenumi}{\textbf{\arabic{enumi}}}
\renewcommand{\labelenumi}{\textbf{\theenumi.}}
\renewcommand{\theenumii}{\textbf{\alph{enumii}}}
\renewcommand{\labelenumii}{\textbf{\theenumii.}}
\def\Oij{$\left(\text{O},~\vect{\imath},~\vect{\jmath}\right)$}
\def\Oijk{$\left(\text{O},~\vect{\imath},~\vect{\jmath},~\vect{k}\right)$}
\def\Ouv{$\left(\text{O},~\vect{u},~\vect{v}\right)$}
\def\e{\text{e}}
\def\i{\text{i}}
\setlength{\voffset}{-1,5cm}
\usepackage{fancyhdr}
\usepackage{hyperref}
\hypersetup{%
pdfauthor = {APMEP},
pdfsubject = {TS Antilles-Guyane},
pdftitle = {19 juin 2014},
allbordercolors = white,
pdfstartview=FitH}  
\usepackage[frenchb]{babel}
\usepackage[np]{numprint}
\usepackage{pstricks-add}
\usepackage{pst-bezier}
\usepackage{graphicx}
\begin{document}
\setlength\parindent{0mm}
\rhead{\textbf{A. P{}. M. E. P{}.}}
\lhead{\small Baccalauréat S}
\lfoot{\small{Antilles-Guyane}}
\rfoot{\small{19 juin 2014}}
\pagestyle{fancy}
\thispagestyle{empty}
\begin{center}\textbf{Durée : 4 heures}

\vspace{0,5cm}

{\Large \textbf{Baccalauréat S Antilles-Guyane 19 juin 2014}}
\end{center}

\vspace{0,5cm}

\textbf{\textsc{Exercice 1} \hfill 5 points}
 
\textbf{Commun à tous les candidats}

\medskip

\emph{Les parties A et B sont indépendantes}

\medskip
 
\emph{Les résultats seront arrondis à $10^{-4}$ près}

\medskip
 
\textbf{Partie A}

\medskip
 
Un ostréiculteur élève deux espèces d'huîtres : \og la plate \fg{} et \og la japonaise \fg. Chaque année, les huîtres plates représentent 15\,\% de sa production.\index{probabilités}
 
Les huîtres sont dites de calibre \no 3 lorsque leur masse est comprise entre 66~g et 85~g.
 
Seulement 10\,\% des huîtres plates sont de calibre \no 3, alors que 80\,\% des huîtres japonaises le sont.

\medskip
 
\begin{enumerate}
\item Le service sanitaire prélève une huître au hasard dans la production de l'ostréiculteur. On suppose que toutes les huitres ont la même chance d'être choisies.
 
On considère les évènements suivants : 

\setlength\parindent{8mm}
\begin{itemize}
\item[$\bullet~~$] $J$ : \og l'huître prélevée est une huître japonaise \fg, 
\item[$\bullet~~$] $C$ : \og l'huître prélevée est de calibre \no 3 \fg.
\end{itemize}
\setlength\parindent{0mm}
 
	\begin{enumerate}
		\item Construire un arbre pondéré complet traduisant la situation.\index{arbre} 
		\item Calculer la probabilité que l'huître prélevée soit une huître plate de calibre \no 3. 
		\item Justifier que la probabilité d'obtenir une huître de calibre \no 3 est $0,695$. 
		\item Le service sanitaire a prélevé une huître de calibre \no 3. 
Quelle est la probabilité que ce soit une huître plate ?
	\end{enumerate} 
\item La masse d'une huître peut être modélisée par une variable aléatoire $X$ suivant la loi normale de moyenne $\mu = 90$ et d'écart-type $\sigma = 2$. \index{loi normale}
	\begin{enumerate}
		\item Donner la probabilité que l'huître prélevée dans la production de l'ostréiculteur ait une masse comprise entre 87~g et 89~g. 
		\item Donner $P(X \geqslant 91)$.
	\end{enumerate}
\end{enumerate}

\bigskip
	 
\textbf{Partie B}

\medskip
 
Cet ostréiculteur affirme que 60\,\% de ses huîtres ont une masse supérieure à 91~g.
 
Un restaurateur souhaiterait lui acheter une grande quantité d'huîtres mais il voudrait, auparavant, vérifier l'affirmation de l'ostréiculteur.

\medskip
 
Le restaurateur achète auprès de cet ostréiculteur 10 douzaines d'huîtres qu'on considèrera comme un échantillon de $120$~huîtres tirées au hasard. Sa production est suffisamment importante pour qu'on l'assimile à un tirage avec remise.
 
Il constate que $65$ de ces huîtres ont une masse supérieure à 91~g.

\medskip
 
\begin{enumerate}
\item Soit $F$ la variable aléatoire qui à tout échantillon de $120$~huîtres associe la fréquence de celles qui ont une masse supérieure à 91~g. 

Après en avoir vérifié les conditions d'application, donner un intervalle de fluctuation asymptotique au seuil de 95\,\% de la variable aléatoire $F$.\index{intervalle de fluctuation} 
\item Que peut penser le restaurateur de l'affirmation de l'ostréiculteur ? 
\end{enumerate} 

\vspace{0,5cm}

\textbf{\textsc{Exercice 2} \hfill 6 points}
 
\textbf{Commun à tous les candidats}

\medskip 

On considère la fonction $f$ définie et dérivable sur l'ensemble $\R$ des nombres réels par 

\[f(x) = x + 1 + \dfrac{x}{\text{e}^x}.\]

On note $\mathcal{C}$ sa courbe représentative dans un repère orthonormé \Oij. 

\medskip

\textbf{Partie A}

\medskip

\begin{enumerate}
\item Soit $g$ la fonction définie et dérivable sur l'ensemble $\R$ par 
\index{fonction exponentielle}
\[g(x) = 1 - x + \text{e}^x.\]
 
Dresser, en le justifiant, le tableau donnant les variations de la fonction $g$ sur $\R$ (les limites de $g$ aux bornes de son ensemble de définition ne sont pas attendues).
 
En déduire le signe de $g(x)$. 
\item Déterminer la limite de $f$ en $- \infty$ puis la limite de $f$ en $+ \infty$. 
\item On appelle $f'$ la dérivée de la fonction $f$ sur $\R$. 

Démontrer que, pour tout réel $x$, 

\[f'(x) = \text{e}^{- x}g(x).\]
 
\item En déduire le tableau de variation de la fonction $f$ sur $\R$. 
\item Démontrer que l'équation $f(x) = 0$ admet une unique solution réelle $\alpha$ sur $\R$.
 
Démontrer que $- 1 < \alpha < 0$.  
\item 
	\begin{enumerate}
		\item Démontrer que la droite $T$ d'équation $y = 2x + 1$ est tangente à la courbe $\mathcal{C}$  au point d'abscisse $0$.
		\item Étudier la position relative de la courbe $\mathcal{C}$ et de la droite $T$.
	\end{enumerate} 
\end{enumerate}

\bigskip

\textbf{Partie B}

\medskip
 
\begin{enumerate}
\item Soit $H$ la fonction définie et dérivable sur $\R$ par 

\[H(x) = (- x - 1)\text{e}^{- x}.\]
 
Démontrer que $H$ est une primitive sur $\R$ de la fonction $h$ définie par $h(x) = x\text{e}^{- x}$.\index{primitive} 
\item On note $\mathcal{D}$ le domaine délimité par la courbe $\mathcal{C}$, la droite $T$ et les droites d'équation $x = 1$ et $x = 3$.
 
Calculer, en unité d'aire, l'aire du domaine $\mathcal{D}$. 
\end{enumerate}


\vspace{0,5cm}

\textbf{\textsc{Exercice 3} \hfill 4 points}
 
\textbf{Commun à tous les candidats}

\medskip 

\emph{Pour chacune des quatre propositions suivantes, indiquer si elle est vraie ou fausse en justifiant la réponse.\\ 
Il est attribué un point par réponse exacte correctement justifiée. Une réponse non justifiée n'est pas prise en compte. Une absence de réponse n'est pas pénalisée.}

\medskip
 
L'espace est muni d'un repère orthonormé \Oijk.\index{géométrie dans l'espace}
 
On considère les points A(1~;~2~;~5), B$(-1~;~6~;~4)$, C$(7~;~- 10~;~8)$ et D$(-1~;~3~;~4)$.

\medskip
 
\begin{enumerate}
\item \textbf{Proposition 1 :} Les points A, B et C définissent un plan. 
\item On admet que les points A, B et D définissent un plan. 

\textbf{Proposition 2 :} Une équation cartésienne du plan (ABD) est $x - 2z + 9 = 0$. 
\item \textbf{Proposition 3 :} Une représentation paramétrique de la droite (AC) est 
\index{equation parametrique de droite@équation paramétrique de droite}
\[\left\{\begin{array}{l c l} 
x &=& \phantom{-}\dfrac{3}{2}t - 5\\ 
y &=& - 3t + 14\\
z &=&- \dfrac{3}{2}t + 2
\end{array}\right. \quad  t \in \R\]
 
\item Soit $\mathcal{P}$ le plan d'équation cartésienne $2x - y + 5z + 7 = 0$ et $\mathcal{P}'$ le plan d'équation cartésienne $- 3x - y + z + 5 = 0$.
 
\textbf{Proposition 4 :} Les plans $\mathcal{P}$ et $\mathcal{P}'$ sont parallèles. 
\end{enumerate}



\vspace{0,5cm}

\textbf{\textsc{Exercice 4} \hfill 5 points}
 
\textbf{Candidats n'ayant pas choisi l'enseignement de spécialité}

\medskip 

Soit la suite numérique $\left(u_{n}\right)$ définie sur l'ensemble des entiers naturels $\N$ par \index{suite}

\[\left\{\begin{array}{r c l}
u_{0}& =& 2\\ 
\text{et pour tout entier naturel }\:n,\: u_{n+1} &=& \dfrac{1}{5} u_{n} + 3 \times 0,5^n.
\end{array}\right.\] 

\begin{enumerate}
\item 
	\begin{enumerate}
		\item Recopier et, à l'aide de la calculatrice, compléter le tableau des valeurs de la suite $\left(u_{n}\right)$ approchées à $10^{-2}$ près: 

\begin{center}
\begin{tabularx}{\linewidth}{|*{10}{>{\centering \arraybackslash}X|}}\hline
$n$		&0	&1	&2	&3	&4	&5	&6	&7	&8\\ \hline 
$u_{n}$	&2	&	&	&	&	&	&	&	&\\ \hline
\end{tabularx}
\end{center} 
 
		\item D'après ce tableau, énoncer une conjecture sur le sens de variation de la suite $\left(u_{n}\right)$.
	\end{enumerate} 
\item
	\begin{enumerate}
		\item Démontrer, par récurrence, que pour tout entier naturel $n$ non nul on a 
		
		\[u_{n} \geqslant  \dfrac{15}{4} \times 0,5^n.\] 
 
		\item En déduire que, pour tout entier naturel $n$ non nul, $u_{n+1} - u_{n} \leqslant  0$. 
		\item Démontrer que la suite $\left(u_{n}\right)$ est convergente.
	\end{enumerate} 
\item \emph{On se propose, dans cette question de déterminer la limite de la suite } \:$\left(u_{n}\right)$.
 
Soit $\left(v_{n}\right)$ la suite définie sur $\N$ par $v_{n} = u_{n} - 10 \times  0,5^n$. 
	\begin{enumerate}
		\item Démontrer que la suite $\left(v_{n}\right)$ est une suite géométrique de raison $\dfrac{1}{5}$. On précisera le  
premier terme de la suite $\left(v_{n}\right)$. \index{suite géométrique}
		\item En déduire, que pour tout entier naturel $n$,
 
\[u_{n} = - 8 \times  \left(\dfrac{1}{5}\right)^n + 10 \times  0,5^n.\]
 
		\item Déterminer la limite de la suite $\left(u_{n}\right)$
	\end{enumerate} 
\item Recopier et compléter les lignes (1), (2) et (3) de l'algorithme suivant, afin qu'il affiche la plus petite valeur de $n$ telle que $u_{n} \leqslant  0,01$.
\index{algorithme}
\begin{center}
\begin{tabularx}{0.6\linewidth}{|l X|} \hline
\textbf{Entrée :}& $n$ et $u$ sont des nombres\\ 
\textbf{Initialisation :}& $n$ prend la valeur 0\\
& $u$ prend la valeur 2\\  
\textbf{Traitement :}&Tant que ...\hfill (1)\\  
	&\hspace{0,5cm} $n$ prend la valeur ... 	\hfill(2)\\ 
	&\hspace{0,5cm} $u$ prend la valeur ... 	\hfill(3)\\ 
&Fin Tant que\\ 
\textbf{Sortie :}&Afficher $n$\\ \hline 
\end{tabularx}
\end{center}
\end{enumerate}

\vspace{0,5cm}

\textbf{\textsc{Exercice 4} \hfill 5 points}
 
\textbf{Candidats ayant  choisi l'enseignement de spécialité}

\medskip 

En montagne, un randonneur a effectué des réservations dans deux types d'hébergements:

L'hébergement A et l'hébergement B.

Une nuit en hébergement A coûte 24 \euro{} et une nuit en  hébergement B coûte 45 ~\euro{}.

Il se rappelle que le coût  total de sa réservation est de 438\euro{}.
 
\emph{On souhaite retrouver les nombres $x$ et $y$ de nuitées passées respectivement en hébergement A et en  hébergement B}

\medskip
 
\begin{enumerate}
\item 
	\begin{enumerate}
		\item Montrer que les nombres $x$ et $y$ sont respectivement inférieurs ou égaux à 18 et 9.		
		\item Recopier et compléter les lignes (1), (2) et (3) de l'algorithme suivant afin qu'il affiche les couples ($x$ ; $y$) possibles.

\begin{center}
\fbox{
\begin{minipage}{8.5cm}
\begin{tabular}{ll}
\textbf{Entrée :} & $x$ et $y$ sont des nombres\\
\textbf{Traitement :} & Pour $x$ variant de $0$ \ldots\hfill~~(1)\\
&\phantom{XXX} Pour $y$ variant de 0 \ldots\hfill~~(2)\\
&\phantom{XXXXXX} Si \ldots\hfill~~(3)\\
&\phantom{XXXXXXXXX}Afficher $x$ et $y$\\
&\phantom{XXXXXX}Fin Si\\
&\phantom{XXX}Fin Pour\\
&Fin Pour\\
\textbf{Fin traitement}
\end{tabular}
\end{minipage}
}
\end{center}
%
%\begin{center}
%\begin{tabular}{|l l|}\hline
%\textbf{EntrŽe :}&$x$ et $y$ sont deux nombres \\
%\textbf{Traitement :}& Pour $x$ variant de 0 ˆ \ldots \hfill (1)\\ 
%& \hspace{0.2cm}\begin{tabular}{|l}
%Pour $y$ allant de 0 ˆ \ldots \hfill (2)\\
%	\hspace{0.2cm}\begin{tabular}{|l}
%	Si \ldots \hfill (3)\\ 
%	Afficher $x$ et $y$\\
%		\end{tabular}\\ 
%	Fin si\\
%Fin Pour\\
%\end{tabular}\\ 
%&Fin Pour\\ \hline
%\end{tabular}
%\end{center}
		
	\end{enumerate}
\item Justifier que le coût total de la réservation est un multiple de 3.
\item 
	\begin{enumerate}
	\item Justifier que l'équation $8x + 15y = 1$ admet pour solution au moins un couple d'entiers relatifs.
	\item Déterminer une telle solution.
	\item Résoudre l'équation (E) : $8x + 15y = 146$ où $x$ et $y$ sont des nombres entiers relatifs.
	\end{enumerate}
\item Le randonneur se souvient avoir passé au maximum 13 nuits en hébergement A.

Montrer alors qu'il peut retrouver le nombre exact de nuits passées en hébergement A et celui des nuits passées en hébergement B.

Calculer ces nombres.
\end{enumerate}
\end{document}