\documentclass[10pt]{article}
\usepackage[T1]{fontenc}
\usepackage[utf8]{inputenc}
\usepackage[upright]{fourier}
\usepackage[scaled=0.875]{helvet} 
\renewcommand{\ttdefault}{lmtt}
\usepackage{amsmath,amssymb,makeidx}
\usepackage[normalem]{ulem}
\usepackage{fancybox}
\usepackage{tabularx}
\usepackage{multirow}
\usepackage{mathrsfs}%police script \mathscr{}
\usepackage{enumitem}
\usepackage{lscape}
\usepackage{tikz}
\usepackage{dcolumn}
\usepackage{textcomp}
\newcommand{\euro}{\eurologo{}}
\usepackage{pstricks,pst-plot,pst-tree,pst-node,pstricks-add,pst-func}
\usepackage[left=3.5cm, right=3.5cm, top=3cm, bottom=3cm]{geometry}
% Tapuscrit Denis Vergès
% Corrigé Jérôme Bonifas relu par François Hache
\newcommand{\R}{\mathbb{R}}
\newcommand{\N}{\mathbb{N}}
\newcommand{\D}{\mathbb{D}}
\newcommand{\Z}{\mathbb{Z}}
\newcommand{\Q}{\mathbb{Q}}
\newcommand{\C}{\mathbb{C}}
\newcommand{\intff}[2]{\left[#1 \, ; \, #2\right]} %intervalle fermé fermé
\newcommand{\intfo}[2]{\left[#1 \, ; \, #2\right[} %intervalle fermé ouvert
\newcommand{\intof}[2]{\left]#1 \, ; \, #2\right]} %intervalle ouvert fermé
\newcommand{\intoo}[2]{\left]#1 \, ; \, #2\right[} %intervalle ouvert ouvert
\newcommand{\vect}[1]{\overrightarrow{\,\mathstrut#1\,}}
\renewcommand{\theenumi}{\textbf{\arabic{enumi}}}
\renewcommand{\labelenumi}{\textbf{\theenumi.}}
\renewcommand{\theenumii}{\textbf{\alph{enumii}}}
\renewcommand{\labelenumii}{\textbf{\theenumii.}}
\def\Oij{$\left(\text{O}~;~\vect{\imath},~\vect{\jmath}\right)$}
\def\Oijk{$\left(\text{O}~;~\vect{\imath},~\vect{\jmath},~\vect{k}\right)$}
\def\Ouv{$\left(\text{O}~;~\vect{u},~\vect{v}\right)$}
\usepackage{fancyhdr}
\usepackage[frenchb]{babel}
\DecimalMathComma
\usepackage{cancel}
\usepackage{hyperref}
\hypersetup{%
pdfauthor = {APMEP},
pdfsubject = {S Métropole - La Réunion - Corrigé},
pdftitle = {22 juin 2018},
allbordercolors = white,
pdfstartview=FitH}    
\usepackage[np]{numprint}
\newcommand{\cg}{\texttt{]}}%    crochet gauche
\newcommand{\cd}{\texttt{[}}%    crochet droit
\renewcommand{\d}{\mathrm{\,d}}%     le d de différentiation
\newcommand{\e}{\mathrm{\,e\,}}%      le e de l'exponentielle
\renewcommand{\i}{\mathrm{\,i\,}}%    le i des complexes
\newcommand{\ds}{\displaystyle}

\begin{document}
\setlength\parindent{0mm}
\rhead{\textbf{A. P{}. M. E. P{}.}}
\lhead{\small Corrigé du baccalauréat S}
\lfoot{\small{Métropole - La Réunion}}
\rfoot{\small 22 juin 2018}
\pagestyle{fancy}
\label{Metropolejuin}
\thispagestyle{empty}

\begin{center}
\textbf{\begin{Large}
Corrigé du baccalauréat TS   Métropole--La Réunion  22 juin 2018
\end{Large}}
\end{center}

\bigskip

\textbf{\textsc{Exercice 1}} \smallskip  \hrule \medskip 

\begin{enumerate}
\item La largeur de l'arc de chaînette est égal à $2x$ et sa hauteur est égale à $\dfrac{1}{2}(\e^x + \e^{-x}-2)$.\smallskip 
	
Le problème étudié revient à résoudre l'équation $\dfrac{1}{2}(\e^x + \e^{-x}-2)=2x$\smallskip 
	
$\dfrac{1}{2}(\e^x + \e^{-x}-2)=2x \iff \e^x + \e^{-x}-2=4x \iff \e^x + \e^{-x}-2-4x=0$\item 
	\begin{enumerate}
		\item Pour $x > 0$, $x\left(\dfrac{\e^x}{x}-4 \right) = \cancel{x} \times \dfrac{e^{x}}{ \cancel{x} } - 4x=  \e^x -4$ donc $f(x)$ peut bien s'écrire sous la forme proposée.
		\item $\displaystyle\lim_{x \to + \infty} \dfrac{\e^x}{x}=+\infty$ par croissance comparée, donc par somme puis produit, 
		
$\displaystyle\lim_{x \to + \infty} x\left( \dfrac{\e^x}{x}-4 \right) = +\infty$ ; 
		 $\displaystyle\lim_{x \to + \infty}\e^{-x}=0$\smallskip 
	
Par somme, on obtient $\boxed{\displaystyle\lim_{x \to + \infty} f(x)=+\infty}$
	\end{enumerate}
\item 
	\begin{enumerate}
		\item $\boxed{f'(x) = \e^x - \e^{-x} - 4}$
		\item $f'(x) = 0 \iff \e^x - \dfrac{1}{\e^x}-4=0 \iff \dfrac{(\e^x)^2-4\e^x-1}{\e^x}=0 \iff \left(\e^x\right)^2-4\e^x-1 = 0$.
		\item Si on pose $X =\e^x$ alors $(\e^x)^2-4\e^x-1=0 \iff X^2-4X-1 = 0$ \smallskip 
		
$\Delta = 16-4\times 1 \times (-1) = 16+4=20 > 0$ donc l'équation admet deux solutions :\smallskip 
		
$X_1 = \dfrac{4 - \sqrt{20}}{2}=\dfrac{4-2\sqrt{5}}{2}=2-\sqrt{5} \approx -0,24 <0$ et $X_2=2+\sqrt{5} \approx 4,24 > 0$ \smallskip 
		
$\e^x=2-\sqrt{5}$ n'a pas de solution car $\e^x >0$ et $\e^x = 2 + \sqrt{5} \iff x=\ln(2+\sqrt{5})$.\smallskip 
		
Donc $f'(x)$ s'annule pour une seule valeur égale à $\ln(2 + \sqrt{5})$
	\end{enumerate}
	\item 
	\begin{enumerate}
		\item On obtient le tableau de variations suivant : \smallskip 
		
\begin{minipage}{0.6\linewidth}
\begin{center}
%		\begin{tikzpicture}
%		\tkzTabInit[lgt=1.5,espcl=3]{$x$/0.6, $f(x)$/2}	{$0$,$\ln(2+\sqrt{5})$,$+\infty$}
%		\tkzTabVar{+/$0$ ,-/$f(\ln(2+\sqrt{5}))$ ,+/$+\infty$}
%		\end{tikzpicture}
\psset{unit=1cm}
\begin{pspicture}(5.5,2.75)
\psframe(5.5,2.75)\psline(0,2)(5.5,2)\psline(1.5,0)(1.5,2.75)
\uput[u](0.75,1.9){$x$} \uput[u](1.6,1.9){0} 
\uput[u](3.5,1.9){\small $\ln \left(2 + \sqrt{5}\right)$} \uput[u](5.1,1.9){$+ \infty$} 
\rput(0.75,1){$f(x)$} \uput[d](1.6,2){0} \uput[u](3.5,0){\small $f\left(\ln \left(2 + \sqrt{5}\right)\right)$}\uput[d](5.1,2){$+ \infty$}
\psline{->}(1.75,1.5)(3,0.75)\psline{->}(3.5,0.75)(5.1,1.5)
\end{pspicture}
\end{center}
\end{minipage}
\begin{minipage}{0.4\linewidth}
		avec $f(0) = 1+1-0-2 =0 $ \smallskip 
		
		et $f\left( \ln \left(2+\sqrt{5}\right) \right) \approx -3,3$
\end{minipage}
		\item 
		\begin{itemize}
			\item Sur $\intof{ 0 }{ \ln(2+\sqrt{5}) }$, $f(x) < 0$ donc l'équation $f(x)=0$ n'a pas de solution.\smallskip 
			
			\item Sur $\intfo{ \ln(2+\sqrt{5} }{ +\infty }$, $f$ est continue et strictement croissante.\smallskip 
			
$0\in \intfo{ f(\ln(2+\sqrt{5})) }{ \displaystyle\lim_{x \to + \infty} f(x) } $ car $f(\ln(2+\sqrt{5}))\approx -3,3 <0$ et $\displaystyle\lim_{x \to + \infty} f(x)=+\infty$\smallskip 
			
D'après le corollaire du théorème des valeurs intermédiaires, l'équation admet une unique solution $\alpha$.
		\end{itemize}
	\end{enumerate}
	\item
	\begin{enumerate}
		\item 
$\begin{array}[t]{|*{5}{c|}}\hline
m  & a & b & b-a & f(m) \\ \hline
& 2 & 3 & 1 & \\ \hline
2,5 & 2 & 2,5 & 0,5>0,1 & \approx 0,26 >0 \\ \hline
2,25 & 2,25 & 2,5 & 0,25>0,1 & \approx -1,4 <0 \\ \hline 
2,375 & 2,375 & 2,5 & 0,125>0,1 & \approx -0,66 <0 \\ \hline 
2,4375 & 2,4375 & 2,5 & 0,0625<0,1 & \approx -0,22 <0 \\ \hline 
\end{array}$
		\item Grâce à cet algorithme, on obtient un encadrement de $\alpha$ : $\boxed{ 2,4375 < \alpha < 2,5 }$
	\end{enumerate}
\item $\e^{ \frac{t}{39} } + \e^{- \frac{t}{39} } - 4 \dfrac{t}{39}-2 = 0 \iff \e^x + \e^{-x} -4x-2=0$ avec $x=\dfrac{t}{39}$ \smallskip  
	
	Cette équation a une unique solution $\alpha$ et $\alpha = \dfrac{t}{39} \iff t = 39 \alpha$ donc la hauteur de l'arche est $2t = 78 \alpha$\smallskip 
	
$2,4375 < \alpha < 2,5 \iff 190,125 < 78\alpha < 195$\smallskip 
	
donc la hauteur de l'arche est comprise entre 190 et 195 mètres.
\end{enumerate} \bigskip 

\textbf{\textsc{Exercice 2}} \smallskip  \hrule 

\bigskip 

\textbf{Partie A}

\medskip

\begin{enumerate}
	\item 
	\begin{enumerate}
		\item $P(G)=0,2$ car 20\% de la population a contracté la grippe.
		\item On obtient :
		\begin{center}
\begin{tikzpicture}[scale=2,grow'=right]  
\tikzstyle{level 1}=[sibling distance=16mm,level distance=20mm]  
\tikzstyle{level 2}=[sibling distance=7mm,level distance=13mm] 
\node{}  
	child{ node{$V$}
		child{ node{$G$} edge from parent node[above]{$0,08$} }
		child{ node{$\overline{G}$} edge from parent node[below]{$0,92$} }
		edge from parent node[above] {$0,4$}} 
	child{node{$\overline{V}$}
		child{ node{$G$} edge from parent node[above ] {$ $} }
		child{ node{$\overline{G}$} edge from parent node[below ] {$ $} }
		edge from parent node[below ] {$0,6 \quad$} };  
\end{tikzpicture}
\end{center}
	\end{enumerate}
	\item On calcule $P(G \cap V) = 0,4 \times 0,08 = 0,032$ soit $3,2\%$ de chances que la personne ait contractée la grippe et soit vaccinée.
	\item On calcule $P_{\overline{V}} (G) =\dfrac{P\left(\overline{V} \cap G\right) }{P\left(\overline{V}\right)}$\smallskip 
	
D'après la formule des probabilités totales, $P(V \cap G) + P\left(\overline{V} \cap G\right) = P\left(G\right)$.\smallskip 
	
Donc $P\left(\overline{V} \cap G\right) = P(G) - P(V \cap G) = 0,2 - 0,032 = 0,168$ puis $P_{\overline{V}} (G) = \dfrac{0,168}{0,6} =0,28$.

La probabilité qu'une personne non vaccinée ait contracté la grippe est égale à 0,28.
\end{enumerate}

\bigskip 

\textbf{Partie B}

\medskip

\begin{enumerate}
	\item Il s'agit de $n$ expériences aléatoires identiques et indépendantes à 2 issues (la personne est vaccinée ou non) avec une probabilité de succès de $0,4$.\smallskip 
	
La variable aléatoire $X$ compte le nombre de succès donc $X$ suit la loi binomiale $\mathscr{B}(n~;~0,4)$.
	\item Avec la loi $\mathscr{B}(40~;~0,4)$
	\begin{enumerate}
		\item $P(X=15) \approx 0,123$
		\item $P(X \geqslant 20) = 1- P(X < 20) =  1- P(X \leqslant 19) \approx 0,130$
	\end{enumerate}
	\item On calcule $P(1450 < X < 1550) = P\left( \dfrac{1450-1500}{30} < Z < \dfrac{1550-1500}{30} \right) =  P\left( \dfrac{-5}{3} < Z < \dfrac{5}{3} \right) \approx 0,904$
	\end{enumerate}\bigskip 
	
\textbf{\textsc{Exercice 3}} \smallskip  \hrule \medskip 

\textbf{Partie A}
\begin{enumerate}
	\item 
	\begin{enumerate}
		\item $(EA) \perp (ABC)$ donc $(EA)$ est la hauteur issue de E dans le tétraèdre ABCE.\smallskip 
		
		$(CB) \perp (ABE)$ donc $(CB)$ est la hauteur issue de C dans le tétraèdre ABCE.
		
		\item Les droites $(EA)$ et $(BC)$ sont non coplanaires donc non sécantes.\smallskip 
		
		Avec deux hauteurs non sécantes, il est impossible d'avoir 4 hauteurs concourantes !
	\end{enumerate}
	\item 
	\begin{enumerate}
		\item $x-y+z=0$ est bien l'équation cartésienne d'un plan donc je vérifie que les points A, C et H appartiennent bien à ce plan :
		

$A(0~;~0~;~0)$ donc $x_A-y_A+z_A=0$

$C(1~;~1~;~0)$ donc $x_C-y_C+z_C=1-1-0=0$

$H(0~;~1~;~1)$ donc $x_H-y_H+z_H=0-1+1=0$
		\item $F(1~;~0~;~1)$ et $D(0~;~1~;~0)$ donc $\vect{DF}(1~;~-1~;~1)$ qui est bien un vecteur normal au plan d'après les coefficients de l'équation cartésienne donc $(FD) \perp (ACH)$ puis $(FD)$ est bien la hauteur issue de F du tétraèdre ACHF.
		\item Par analogie, on en déduit que $(AG)$ est la hauteur issue de A, $(CE)$ est la hauteur issue de H et $(HB)$ est la hauteur issue de H.
		
D'après l'énoncé, les 4 hauteurs correspondent aux \og grandes diagonales \fg{} du cube et sont donc concourantes.
	\end{enumerate}
\end{enumerate}

\medskip 

\textbf{Partie B}
\begin{enumerate}
	\item 
	\begin{enumerate}
		\item $(MK)$ est orthogonale au plan $(NPQ)$ donc d'après le théorème de la porte, $(MK)$ est orthogonale à toute droite de ce plan;  en particulier, $(MK) \perp (PQ)$.
		\item On a montré que $(PQ)$ est orthogonale à $(NK)$ et $(MK)$ qui sont deux droites sécantes du plan $(MNK)$ donc par définition, $(PQ)$ est orthogonale au plan $(MNK)$.
	\end{enumerate}
	\item $(PQ)$ est orthogonale au plan $(MNK)$ donc d'après le théorème de la porte, $(PQ)$ est orthogonale à toute droite de ce plan;  en particulier, $(PQ) \perp (MN)$.
\end{enumerate}\medskip 

\textbf{Partie C}\smallskip 

$\vect{RS}(4~;~-1~;~-4)$ \quad $\vect{ST}(3~;~-5~;~7)$ \quad $\vect{TU} (0~;~8~;~-2)$ \quad $\vect{RU}(7~;~2~;~1)$ \quad $\vect{RT}(7~;~-6~;~3)$ \quad $\vect{SU} (3~;~3~;~5)$\smallskip 

$\vect{ST}\cdot\vect{RU}=3\times 7 + (-5) \times 2 + 7 \times 1 =21-10+7\neq 0$ donc $(ST)$ n'est pas orthogonale à $(TU)$.\smallskip 

Avec deux arêtes opposées non orthogonales, ce tétraèdre n'est pas orthocentrique.\bigskip 

\textbf{\textsc{Exercice 4 obligatoire}} \smallskip  \hrule \medskip 

\begin{enumerate}
	\item
	\begin{enumerate}
		\item $\dfrac{\sqrt{3}}{2} \e^{-\text{i} \frac{\pi}{6}} = \dfrac{\sqrt{3}}{2} \left( \cos\left(-\dfrac{\pi}{6}\right) + \text{i}\sin\left(-\dfrac{\pi}{6}\right) \right) = \dfrac{\sqrt{3}}{2} \left( \dfrac{\sqrt{3}}{2} - \text{i}\dfrac{1}{2} \right) = \dfrac{3}{4} - \text{i} \dfrac{\sqrt{3}}{4} = \dfrac{3-\text{i}\sqrt{3}}{4}$
		
	\item $z_1=\dfrac{\sqrt{3}}{2} \e^{-\text{i} \frac{\pi}{6}}z_0=\dfrac{\sqrt{3}}{2} \e^{-\text{i} \frac{\pi}{6}} \times 8$ donc $\boxed{ z_1=4\sqrt{3} \e^{-\text{i} \frac{\pi}{6}} }$\smallskip 
	
	$z_2=\dfrac{\sqrt{3}}{2} \e^{-\text{i} \frac{\pi}{6}}z_1 = \dfrac{\sqrt{3}}{2} \e^{-\text{i} \frac{\pi}{6}} \times 4\sqrt{3} \e^{-\text{i} \frac{\pi}{6}} = 6 \e^{-\text{i} \frac{2\pi}{6}}$ donc $\boxed{ z_2=6 \e^{-\text{i} \frac{\pi}{3}} }$\smallskip 
	
	$z_3=\dfrac{\sqrt{3}}{2} \e^{-\text{i} \frac{\pi}{6}}z_2 = \dfrac{\sqrt{3}}{2} \e^{-\text{i} \frac{\pi}{6}} \times 6 \e^{-\text{i} \frac{\pi}{3}} = 3\sqrt{3} \e^{-\text{i} \frac{3\pi}{6}}$ donc $\boxed{ z_3=3\sqrt{3} \e^{-\text{i} \frac{\pi}{2}} }$\medskip 
	
	$\arg(z_3) = \dfrac{-\pi}{2}$ donc $z_3$ est un imaginaire pur dont la partie imaginaire est négative et 
	
$\boxed{ \text{Im}\left(z_3\right) = - 3\sqrt{3} }$
	\item \textsc{Figure représentation des points $A_0$, \:$A_1$, \:$A_2$, \:$A_3$}

La relation $z_{n+1} = \dfrac{\sqrt{3}}{2}\text{e}^{- \text{i}\frac{\pi}{6}}z_n$ montre en prenant les arguments que 

arg$\left(z_{n+1} \right) = \text{arg}\left(\dfrac{\sqrt{3}}{2}\text{e}^{- \text{i}\frac{\pi}{6}}\right) + \text{arg}\left(z_n\right)$.
Or $\text{arg}\left(\dfrac{\sqrt{3}}{2}\text{e}^{- \text{i}\frac{\pi}{6}} \right) = - \dfrac{\pi}{6}$.

On a donc pour tout $n \in \N$, $\left(\vect{\text{O}A_n},~\vect{\text{O}A_{n+1}} \right) = - \dfrac{\pi}{6}$.

On a donc $\left(\vect{\text{O}A_0},~\vect{\text{O}A_{1}} \right) = - \dfrac{\pi}{6}$, puis 
$\left(\vect{\text{O}A_0},~\vect{\text{O}A_{2}} \right) = - \dfrac{\pi}{3}$ et $\left(\vect{\text{O}A_0},~\vect{\text{O}A_{3}} \right) = - \dfrac{\pi}{2}$.

\smallskip

$\bullet~~$$A_0$ a pour affixe 8 ;

$\bullet~~$ On sait que $\sin - \frac{\pi}{6} = - \frac{1}{2}$. On trace donc l'horizontale partant du point de coordonnées $(0~;~- 4)$ qui coupe le cercle de centre O de rayon 8 en un point B d'abscisse positive. La droite verticale d'équation $x = 6$ coupe OB en $A_1$.

$\bullet~~$ On sait que $\cos - \frac{\pi}{3} =  \frac{1}{2}$. On trace donc la verticale ale partant du point de coordonnées $(4~;~0)$ qui coupe le cercle de centre O de rayon 8 en un point C d'ordonnée négative. La droite verticale  d'équation $x = 3$ coupe OC en $A_2$.

$\bullet~~$ Enfin $A_3$ est le projeté orthogonal de $A_2$ sur l'axe des ordonnées puisque $\text{O}A_3 = \dfrac{\sqrt{3}}{2}\text{O}A_2$ ou encore $\text{O}A_3 = \cos \frac{\pi}{6}\text{O}A_2$.
\begin{center}
\psset{unit=1cm}
\begin{pspicture}(-0.5,-8)(8.1,0.5)
\psaxes[linewidth=1.25pt]{->}(0,0)(-0.5,-8)(8.1,0.5)
\psaxes[linewidth=1.25pt](0,0)(-0.5,-8)(8.1,0.5)
\psarc(0,0){8}{-90}{5}
\uput[dr](8,0){$A_0$}
\psline[linestyle=dotted](0,-4)(7.1,-4)
\psline[linestyle=dotted](8;-30)
\psline[linestyle=dotted](6,0)(6,-3.464)\uput[r](6,-3.464){$A_1$}
\psline[linestyle=dotted](4,0)(4,-7.1)
\psline[linestyle=dotted](8;-60)
\psline[linestyle=dotted](3,0)(3,-5.196)\uput[ur](3,-5.196){$A_2$}
\psline(8;0)(6.928;-30)(6;-60)(0,-5.196)\uput[dl](0,-5.196){$A_3$}
\psdots(8,0)(6,-3.464)(3,-5.196)(0,-5.196)(8;-30)(8;-60)
\psarc(0,0){0.4}{-30}{0}\psarc(0,0){0.5}{-60}{-30}\psarc(0,0){0.6}{-90}{-60}
\rput(0.7,-0.2){$\frac{\pi}{6}$}
\rput(0.6,-0.6){$\frac{\pi}{6}$}
\rput(0.3,-1){$\frac{\pi}{6}$}
\uput[dr](8;-30){B}\uput[dr](8;-60){C}
\psarc[linecolor=red](4,0){4}{-180}{0}
\rput{-30}(0,0){\psarc[linecolor=red](3.4641,0){3.4641}{-180}{0}}
\rput{-60}(0,0){\psarc[linecolor=red](3,0){3}{-180}{0}}
\end{pspicture}
\end{center}

\smallskip

\emph{Remarque} :

Puisque que pour tout naturel $n$, \: O$A_{n+1} = \cos \frac{\pi}{6} \text{O}A_n$, le point $A_{n+}$ est la projeté orthogonal de $A_n$ sur la droite O$A_{n+1}$.

$A_1$ est donc le point d'intersection de la droite (OB) avec le demi-cercle de diamètre $\left[\text{O}A_0\right]$ contenant les points d'ordonnée négative.

$A_2$ est le point d'intersection de la droite (OC) avec le demi-cercle de diamètre $\left[\text{O}A_1\right]$. (voir les demi-cercles tracés en rouge)

$A_3$ est le point d'intersection de l'axe des ordonnées  avec le demi-cercle de diamètre $\left[\text{O}A_2\right]$.
	\end{enumerate}
	\item 
	\begin{enumerate}
		\item \emph{Initialisation} $z_0 = 8 \times 1 \times 1 = 8$ donc la propriété est vraie pour $n=0$.\smallskip 
		
		\emph{Hérédité} : On suppose que pour $n \geqslant 0$,  $z_n=8 \times \left( \dfrac{\sqrt{3}}{2} \right)^n \e^{-\text{i} \frac{n\pi}{6}}$ et on va montrer que 
		
$z_{n+1} = 8 \times \left( \dfrac{\sqrt{3}}{2} \right)^{n+1} \e^{-\text{i} \frac{(n+1)\pi}{6}}$\medskip 
		
On a $z_{n+1}= \dfrac{\sqrt{3}}{2} z_n= \dfrac{\sqrt{3}}{2}\e^{-\text{i} \frac{\pi}{6}} \times 8\times \left( \dfrac{\sqrt{3}}{2} \right)^n \e^{-\text{i} \frac{n\pi}{6}}$ \textit{(par hypothèse de récurrence)}.\smallskip 
		
Donc $z_{n+1}=8 \times \left( \dfrac{\sqrt{3}}{2} \right)^{n+1} \e^{-\text{i} \frac{(n+1)\pi}{6}}$ \textit{ (en utilisant la propriété $a^n \times a = a^{n+1}$ pour tout nombre réel $a$) }.\smallskip 
		
Donc la propriété est héréditaire.\smallskip 

La propriété est vraie au rang $0$, et si elle est vraie au rang $n \geqslant 0$, elle l'est aussi au rang $n + 1$
		
Conclusion : d'après le principe de récurrence la propriété est vraie pour tout entier naturel $n$.
	\item On a donc $u_n=\left|z_n\right| =8 \times \left( \dfrac{\sqrt{3}}{2} \right)^n$\smallskip 
	
Il s'agit d'une suite géométrique de premier terme $u_0=8$ et de raison $\dfrac{\sqrt{3}}{2}$. \medskip 
	
$0 < \dfrac{\sqrt{3}}{2} < 1$ donc $\displaystyle\lim_{n \to + \infty} \left( \dfrac{\sqrt{3}}{2} \right)^n = 0$ puis $\boxed{ \displaystyle\lim_{n \to + \infty} u_n = 8 \times 0 = 0 }$		
	\end{enumerate}
	\item 
	\begin{enumerate}
		\item $\dfrac{z_{k+1} - z_k}{z_{k+1} } = \dfrac{ \dfrac{3-\text{i}\sqrt{3}}{4}z_k - z_k}{ \dfrac{3-\text{i}\sqrt{3}}{4}z_k } = \dfrac{ \cancel{z_k} \left( \dfrac{3-\text{i}\sqrt{3}}{4} - 1 \right)}{ \dfrac{3-\text{i}\sqrt{3}}{4}\cancel{z_k} } 
		=\dfrac{ \dfrac{3-\text{i}\sqrt{3}}{4} - 1 }{\dfrac{3-\text{i}\sqrt{3}}{4}}
		= \dfrac{ -1-\text{i}\sqrt{3} }{4} \times \dfrac{4}{3-\text{i}\sqrt{3}} 
		= \dfrac{ -1-\text{i}\sqrt{3} }{3-\text{i}\sqrt{3}}$\medskip 
		
On multiplie par le conjugué du dénominateur :\smallskip 
		
$\dfrac{z_{k+1} - z_k}{z_{k+1}}
		=\dfrac{ (-1-\text{i}\sqrt{3})(3+\text{i}\sqrt{3}) }{ (3-\text{i}\sqrt{3})(3+\text{i}\sqrt{3})} 
		= \dfrac{-3-\text{i}\sqrt{3}-3\text{i}\sqrt{3}+3 }{9+3} = \dfrac{-4\text{i}\sqrt{3}\times \sqrt{3} }{12 \times \sqrt{3} } = \dfrac{-12\text{i}}{12\sqrt{3}} = - \dfrac{1}{\sqrt{3}} \text{i}$\medskip 
		
On a donc $\left|\dfrac{z_{k+1} - z_k}{z_{k+1} }\right| = \left|- \dfrac{1}{\sqrt{3}} \text{i}\right|
		\iff \dfrac{\left|z_{k+1} - z_k \right|}{\left|z_{k+1} \right|} = \dfrac{1}{\sqrt{3}} 
		\iff \dfrac{A_kA_{k+1}}{OA_{k+1}} = \dfrac{1}{\sqrt{3}}
		\iff $
		
$A_kA_{k+1} = \dfrac{1}{\sqrt{3}} OA_{k+1}$. \medskip 
		
		\item D'après la question précédente, pour tout entier naturel $k$,\smallskip 
		
$A_kA_{k+1} = \dfrac{1}{\sqrt{3}} OA_{k+1} = \dfrac{1}{\sqrt{3}} \left|z_{k+1} \right| = \dfrac{1}{\sqrt{3}} \times 8 \times \left( \dfrac{\sqrt{3}}{2} \right)^{k+1} = \dfrac{8}{\sqrt{3}} \left( \dfrac{\sqrt{3}}{2} \right)^{k+1}$ \medskip 
		
Donc $\ell_n=\dfrac{8}{\sqrt{3}} \left( \dfrac{\sqrt{3}}{2} \right)^1 + \dfrac{8}{\sqrt{3}} \left( \dfrac{\sqrt{3}}{2} \right)^2 + \cdots + \dfrac{8}{\sqrt{3}} \left( \dfrac{\sqrt{3}}{2} \right)^n
		= \dfrac{8}{\sqrt{3}} \times \dfrac{\sqrt{3}}{2} \left( 1 +  \left( \dfrac{\sqrt{3}}{2} \right)^1 + \cdots + \left( \dfrac{\sqrt{3}}{2} \right)^{n-1} \right)$\medskip 
		
Puis $\ell_n=4 \times \dfrac{1 - \left( \dfrac{\sqrt{3}}{2} \right)^{n} }{1-\left( \dfrac{\sqrt{3}}{2} \right)}
		=4 \times \dfrac{1 - \left( \dfrac{\sqrt{3}}{2} \right)^{n} }{\dfrac{2-\sqrt{3}}{2}}
		=\dfrac{8}{2-\sqrt{3}} \times \left( 1 - \left( \dfrac{\sqrt{3}}{2} \right)^{n} \right)$\medskip 
		
Pour finir, $\displaystyle\lim_{n \to + \infty} \ell_n=\dfrac{8}{2-\sqrt{3}}(1-0) =\dfrac{8}{2-\sqrt{3}} \approx 29,86$
	\end{enumerate}
\end{enumerate}


\bigskip

\textbf{\textsc{Exercice 4  spécialité}} \smallskip  \hrule \medskip 

\bigskip

\textbf{Partie A}

%\medskip

%On considère l'équation suivante dont les inconnues $x$ et $y$ sont des entiers naturels :

\[x^2 - 8y^2 = 1. \quad(E)\]

\medskip

\begin{enumerate}
\item %Déterminer un couple solution $(x~;~y)$ où $x$ et $y$ sont deux entiers naturels.
Le couple (1~;~0) est solution ; avec $y = 1$, on trouve aussitôt $x = 3$.

Le couple (3~;~1) est aussi solution.
\item %On considère la matrice $A = \begin{pmatrix}3&8\\1&3\end{pmatrix}$.

%On définit les suites d'entiers naturels $\left(x_n\right)$ et $\left(y_n\right)$ par :

\[x_0 = 1,\: y_0 = 0,\: \text{et pour tout entier naturel }\:n,\: \begin{pmatrix}x_{n+1}\\y_{n+1}\end{pmatrix} = A\begin{pmatrix}x_{n}\\y_{n}\end{pmatrix}.\]
	\begin{enumerate}
		\item %Démontrer par récurrence que pour tout entier naturel $n$, le couple 
%		$\left(x_n~;~y_n\right)$ est solution de l'équation $(E)$.

\emph{Initialisation}

On a vu  que le couple $\left(x_0 = 1~;~y_0 = 0 \right)$ est un couple solution. Donc la proposition est vraie au rang $0$.

\emph{Hérédité}

Soit $n \in \N$ et supposons que le couple $\left(x_n~;~y_n\right)$ est solution de l'équation $(E)$, c'est-à-dire que $x_n^2 - 8y_n^2 = 1$.

Alors $\begin{pmatrix}x_{n+1}\\y_{n+1}\end{pmatrix} = A\begin{pmatrix}x_{n}\\y_{n}\end{pmatrix} =  \begin{pmatrix}3&8\\1&3\end{pmatrix}\begin{pmatrix}x_{n}\\y_{n}\end{pmatrix} = \begin{pmatrix}3x_{n} + 8y_n\\x_n + 3y_{n}\end{pmatrix}$

Donc $x_{n+1} = 3x_n + 8y_n$ et $y_{n+1} = x_n + 3y_n$.

Calculons la différence :

$x_{n+1}^2 - 8y_{n+1}^2 = \left(3x_n + 8y_n\right)^2 -  8\left(x_n + 3y_n \right)^2 = 9x_n^2 64y_n^2 + 48x_ny_n - 8\left(x_n^2 + 9y_n^2 + 6x_ny_n \right) = 9x_n^2 +  64y_n^2 + 48x_ny_n - 8x_n^2 - 72y_n^2 - 48x_ny_n = x_n - 8y_n^2 = 1$, d'après l'hypothèse de récurrence.

Le couple $\left(x_{n+1}~;~y_{n+1} \right)$ est aussi un couple solution.

On a montré que la proposition est vraie au rang $0$ et que si elle est vraie à un rang $n \in \N$ elle l'est aussi au rang $n + 1$ : d'après le principe de récurrence on a montré que pour tout naturel $n$, \: le couple $\left(x_n~;~y_n \right)$ est une solution de $(E)$.
		\item %En admettant que la suite $\left(x_n\right)$ est à valeurs strictement positives, démontrer que pour tout entier naturel $n$, on a : $x_{n+1} > x_n$.
		
On calcule la différence :

$x_{n+1} - x_n = 3x_n + 8y_n - x_n = 2x_n + 8y_n$ ; cette somme est positive car on suppose que $x_n > 0$ et  $y_n \in \N,\: y_n \geqslant 0$.

On a donc $x_{n+1} - x_n > 0 \iff x_{n+1} > x_n$ : la suite $\left(x_n\right)$ est donc strictement croissante.


 	\end{enumerate}
\item  %En déduire que l'équation $(E)$ admet une infinité de couples solutions.
On a vu qu'il existe au moins un couple $\left(x_0~;~y_0\right)$ solution de $(E)$ et on a démontré que pour chaque couple solution $\left(x_n~;~y_n\right)$ le couple $\left(x_{n+1}~;~y_{n+1}\right)$ est aussi solution ; comme on a montré que $x_{n+1} > x_n$ le couple  $\left(x_{n+1}~;~y_{n+1}\right)$ est une solution différente.

Conclusion : l'équation $(E)$ a une infinité de solutions, les premiers termes étant de plus en plus grands. Les premiers couples sont (1~;0), \: (3~;~1), \: (17~;~6), \ldots
\end{enumerate}

\bigskip

\textbf{Partie B}

\medskip

Un entier naturel $n$ est appelé un nombre puissant lorsque, pour tout diviseur premier $p$ de $n$,\: $p^2$ divise $n$.

\medskip

\begin{enumerate}
\item ~%Vérifier qu'il existe deux nombres entiers consécutifs inférieurs à $10$ qui sont puissants.
$\bullet~~$ On a $8 = 2^3$ ; 8 est divisible par 2 qui est premier et aussi par $2^2$: il est puissant ;

$\bullet~~$ On a $9 = 3^2$ ; 9 est divisible par 3 qui est premier et aussi par $3^2$: il est puissant ;

8 et 9 sont deux naturels consécutifs inférieurs à 10  puissants.
\end{enumerate}
\medskip

%L'objectif de cette partie est de démontrer, à l'aide des résultats de la partie A, qu'il existe une infinité de couples de nombres entiers naturels consécutifs puissants et d'en trouver quelques exemples.

\medskip

\begin{enumerate}[resume]
\item  %Soient $a$ et $b$ deux entiers naturels.

%Montrer que l'entier naturel $n = a^2 b^3$ est un nombre puissant.
On suppose que $ab\neq 0$, que $a \neq b$ et $a \geqslant 1$.

Tout diviseur premier de $n$ est un diviseur de $a$ ou de $b$.

$\bullet~~$Si $p$ est un diviseur premier de $a$, alors il existe $a' \in \N$ tel que $a = p \times a'$, donc $n$ s'écrit 

$n = p^2a'^2b^3$, donc $p^2$ divise $n$ ;

$\bullet~~$Si $p$ est un diviseur premier de $b$, alors il existe $b' \in \N$ tel que $b = p \times b'$, donc $n$ s'écrit 

$n = a^2p^3\left(b'\right)^3 = p^2pa^2\left(b'\right)^3$, donc $p^2$ divise $n$.

Conclusion $n$ est puissant.
\item  %Montrer que si $(x~;~y)$ est un couple solution de l'équation $(E)$ définie dans la partie A, alors $x^2 - 1$ et $x^2$ sont des entiers consécutifs puissants.
$\bullet~~$Il est évident que $x^2 - 1$ précède $x^2$ ; les deux nombres sont consécutifs ;

$\bullet~~$Puisque $(x~;~y)$ est un couple solution de l'équation $(E)$, on a donc $x^2 - 8y^2 = 1 \iff$

$ x^2 - 1 = 8y^2$ qui est un nombre puissant puisque divisible par 2  premier et son carré 4.

D'autre part $x$ supérieur à 1 a au moins un diviseur premier $p$ ; il existe $q \in \N$ tel que $x = pq$ et par conséquent $x^2 = p^2q^2$ qui est puissant puisque divisible par le premier $p$ et le carré de ce premier.

Conclusion : si $(x~;~y)$ est un couple solution de l'équation $(E)$, $x^2 - 1$ et $x^2$ sont deux naturels consécutifs puissants.

On a vu que 8 et 9 sont consécutifs et puissants.
\item  %Conclure quant à l'objectif fixé pour cette partie, en démontrant qu'il existe une infinité de couples de nombres entiers consécutifs puissants.

%Déterminer deux nombres entiers consécutifs puissants supérieurs à $2018$.

On a vu l'équation $(E)$ a une infinité de couples solutions.

On a démontré que pour chaque couple $(x~;~y)$ solution de $(E)$, les nombres $x^2 - 1$ et $x^2$ sont consécutifs et puissants et que la suite des premiers termes est strictement croissante.

Il existe donc une infinité de naturels consécutifs et puissants.

La calculatrice donne $\left(x_3~;~y_3\right) = (99~;~35)$.

D'après la question précédente $99^2$ et $99^2 - 1$ sont deux nombres consécutifs puissants

On a  $99^2 = (100 - 1)^2 = \np{10000} - 200 + 1 = \np{9801}$ et $99^2 - 1 = \np{9800}$.

$\np{9801} = (9 \times 11)^2 = \left(3^2 \times 11\right)^2 = 3^4 \times 11^2$.

$\np{9801}$ est effectivement  divisible par 3 et par $3^2$, par 11 et $11^2$ ;

$\np{9801} - 1 = \np{9800} = 98 \times 100 = 2 \times 49 \times (2 \times 5)^2 = 2^3 \times 5^2 \times 7^2$ est divisible par 2 et par $2^2$, par 5 et $5^2$ par 7 et $7^2$.

\np{9800} et \np{9801} sont des naturels consécutifs et puissants supérieurs à 2018.
\end{enumerate}
\end{document}
%%%%%%%%%%%%%%%
La question 1. permettait de faire la question 2 en trois lignes. 
x2 et x2−1 sont des entiers consécutifs. 
x2 est puissant d'après la question 1 en prenant a=x et b=1. 
x2−1=8y2=y2.23 est puissant d'après la question 1 en prenant a=y et b=2.
%%%%%%%%%%%%%%%