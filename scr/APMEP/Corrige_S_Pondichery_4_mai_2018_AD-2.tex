\documentclass[11pt,a4paper,answers,fancyhdr]{exam} 
% remplacer answers par noanswers pour ne pas afficher les solutions
%\documentclass[a4paper ,11pt]{article} 
\usepackage{color}
\renewcommand{\solutiontitle}{\noindent\textbf{\blue Solution : }\noindent}
\shadedsolutions
\definecolor{SolutionColor}{rgb}{0.95,0.9,1}
%\framedsolutions
\usepackage[T1]{fontenc}
\usepackage[utf8]{inputenc}
\usepackage{fourier}
\usepackage[scaled=0.875]{helvet}
\renewcommand{\ttdefault}{lmtt} 
\usepackage{amsmath,amssymb,makeidx} 
\usepackage{fancybox}
\usepackage{tabularx}
\usepackage{graphicx} 
\usepackage{colortbl} 
\usepackage[normalem]{ulem} 
\usepackage{pifont}
\usepackage{array} 
\usepackage{ulem} 
\usepackage{textcomp}
\usepackage{lscape} 
\newcommand{\euro}{\eurologo{}} 
\usepackage[left=3.5cm, right=3.5cm, top=3cm, bottom=3cm]{geometry}
\newcommand{\vect}[1]{\overrightarrow{\,\mathstrut#1\,}}
\renewcommand{\theenumi}{\textbf{\arabic{enumi}}} 
\renewcommand{\labelenumi}{\textbf{\theenumi.}} 
\renewcommand{\theenumii}{\textbf{\alph{enumii}}} 
\renewcommand{\labelenumii}{\textbf{\theenumii.}} 
\usepackage[frenchb]{babel} 
\DecimalMathComma
\usepackage{pstricks,pst-plot,pst-text,pst-tree,pst-eps,pst-fill,pst-node,pst-math,pst-eucl,pstricks-add}
\usepackage{lastpage}
\usepackage{titlesec}
\usepackage{enumitem}
\usepackage[np]{numprint}
\usepackage{tabularx}
\usepackage{fp}
\usepackage{multido}
\newcommand{\entetecor}[4]{%
                  
\rhead{#3}
\chead{\shadowbox{Corrigé du baccalauréat S  #1 du #2}}
\lfoot{Baccalauréat 2018}
\rfoot{A. Detant}
\setcounter{page}{1}
\cfoot{page \thepage ~sur~ \pageref{LastPage}}
           % \rfoot{#3}
\newgeometry{left=1.5cm, right=1.5cm , top=2cm , bottom=2.5cm}             
                  }

 %** surlignage en jaune, rouge, vert ***
 \newcommand{\surj}[1]{\fcolorbox{yellow}{yellow}{#1}}
 \newcommand{\surr}[1]{\fcolorbox{red}{red}{#1}}
 \newcommand{\surv}[1]{\fcolorbox{green}{green}{#1}}
 
 % % % % % % % % % % % % % %
 % COMMANDES MATHÉMATIQUES %
 % % % % % % % % % % % % % %
 
 %\input{tabvar} %** permet d'utiliser Pst+ pour les tableaux de variation et arbres (ATTENTION tabvar.tex doit être dans le répertoire de compilation du document)
 
 %** écrire un repère (hors environnement math)
 
 \def\Oij{$\left(\text{O},~\vect{\imath},~\vect{\jmath}\right)$} 
 \def\Oijk{$\left(\text{O},~\vect{\imath},~\vect{\jmath},~\vect{k}\right)$} 
 \def\Ouv{$\left(\text{O},~\vect{u},~\vect{v}\right)$} 
  
 %** écrire une équivalence, une implication (dans un environnement math)
 
 \newcommand{\equi}{\Longleftrightarrow}
 \newcommand{\imp}{\Longrightarrow} 
 %** écrire une relation d'ordre (dans un environnement math)
 \newcommand{\pp}{\leqslant}
 \newcommand{\pg}{\geqslant}
 %*** Ensembles de nombres ***
 \newcommand{\R}{\mathbb{R}} 
 \newcommand{\N}{\mathbb{N}} 
 \newcommand{\D}{\mathbb{D}} 
 \newcommand{\Z}{\mathbb{Z}} 
 \newcommand{\Q}{\mathbb{Q}} 
 \newcommand{\C}{\mathbb{C}} 
 %**** écrire un vecteur (dans un environnement math)*****
 %\newcommand{\vect}[1]{\overrightarrow{#1\,\,}}
 %**** écrire une variable aléatoire (dans un environnement math)*****
 \newcommand{\binomiale}[2]{\hookrightarrow \mathcal{B}\left(#1~;~#2 \right) }
 \newcommand{\expo}[1]{\hookrightarrow \mathcal{E}\left(#1 \right) }
 \newcommand{\normale}[2]{\hookrightarrow \mathcal{N}\left(#1~;~#2^2 \right) }
 %**** écrire un angle géométrique (en dehors de l'environnement math)*****
 \newcommand{\anglg}[1]{$\widehat{#1\,\,}$}
 %**** écrire un angle orienté de vecteurs *****
 \newcommand{\anglor}[2]{$\left( ~\vect{#1}~;~\vect{#2}~\right) $}
 %***** Écrire un repère (en dehors de l'environnement math) *******
 \def\Oij{$\left(\text{O};~\vect{\imath},~\vect{\jmath}\right)$} 
 \def\Oijk{$\left(\text{O};~\vect{\imath},~ \vect{\jmath},~ \vect{k}\right)$} 
 \def\Ouv{$\left(\text{O};~\vect{u},~\vect{v}\right)$} 
 %*** écrire i (dans un environnement mathématique)****
 \newcommand{\im}{\text{i}}
 %*** écrire e (dans un environnement mathématique)****
 \newcommand{\ex}{\text{e}}
 %*** écrire une exponentielle (dans un environnement mathématique)****
 \renewcommand{\exp}[1]{\text{e}^{#1}}
 %*** écrire le symbole multiplié (dans un environnement mathématique)
 \newcommand{\x}{\times}
 %**** écrire un module (en dehors de l'environnement math)*****
 \renewcommand{\mod}[1]{$\left| #1\right| ~$}
 %**** écrire un conjugué (en dehors de l'environnement math)*****
 \newcommand{\conj}[1]{$\overline{~#1~}~$}
 %**** écrire une intégrale (dans un environnement math)*****
 \newcommand{\integ}[4]{\displaystyle \int_{#1}^{#2}~ #3~\text{d}#4}
 %**** écrire une somme (dans un environnement math)*****
 \newcommand{\somme}[4]{\displaystyle \sum_{#1=#2}^{#1=#3} \left(#4\right) ~}
 %**** écrire des limites (en dehors de l'environnement math)*****
 \newcommand{\Lim}[3]{$\displaystyle \lim_{#1 \to #2} #3~$}
 \newcommand{\Limg}[3]{$\displaystyle \lim_{\substack{#1 \to #2\\#1~<~#2}} #3~$}
 \newcommand{\Limd}[3]{$\displaystyle \lim_{\substack{#1 \to #2\\#1~>~#2}} #3~$}
 \usepackage{fancyhdr}
\begin{document}
%\SetWatermarkText{Lycée Henri IV}
%\SetWatermarkScale{0.6}
%\entete{9}{vendredi 13 mai 2016}{TS 2}{2 heures}{B}
%\entetecor{Pondichéry}{4 mai 2018}{TS}{}

\titlespacing*{\subsubsection}
{0pt}%␣retrait␣à␣gauche
{0ex plus 0ex minus 0ex}%␣espace␣avant
{0ex plus 0ex}%␣espace␣après
[0pt]%␣retrait␣à␣droite
\rhead{\textbf{A. P{}. M. E. P{}.}}
\lhead{\small Corrigé du baccalauréat S --  A. Detant}
\lfoot{\small{Pondichéry}}
\rfoot{\small{4 mai 2018}}
\pagestyle{fancy}
\thispagestyle{empty}
\begin{center}{\Large\textbf{\decofourleft~Corrigé du baccalauréat S Pondichéry  4 mai 2018~\decofourright}}
\end{center}

\label{exercice 1}
\textbf{\textsc{Exercice 1} \hfill 6 points}
 
\textbf{Commun à tous les candidats}

\medskip

%\emph{Les parties } A \emph{et}   B \emph{peuvent être traitées de façon indépendante.}
%
%\bigskip
%
%
%\begin{minipage}{0.9\linewidth}
%Dans une usine, un four cuit des céramiques à la température de \np{1000}~\degres C. À la fin de la
%cuisson, il est éteint et il refroidit.
%
%\smallskip
%
%On s'intéresse à la phase de refroidissement du four, qui débute dès l'instant où il est éteint.
%
%\smallskip
%La température du four est exprimée en degré Celsius (\degres~C).
%
%\smallskip
%
%La porte du four peut être ouverte sans risque pour les céramiques dès que sa température est
%inférieure à $70$\degres~C. Sinon les céramiques peuvent se fissurer, voire se casser.
%\end{minipage}
%
%\bigskip

\textbf{Partie A}

\medskip

%\begin{minipage}{0.9\linewidth}
%Pour un nombre entier naturel $n$, on note $T_n$ la température en degré Celsius du four au bout
%de $n$ heures écoulées à partir de l'instant où il a été éteint. On a donc $T_0 = \np{1000}$.
%\end{minipage}

%La température $T_n$ est calculée par l'algorithme suivant :
%
%\begin{center}
%\begin{tabularx}{0.35\linewidth}{|X|}\hline
%$T \gets \np{1000}$\\
%Pour $i$ allant de $1$ à $n$\\
%\hspace{1cm}$T \gets 0,82 \times T + 3,6$\\
%Fin Pour\\\hline
%\end{tabularx}
%\end{center}
%
%\medskip

\begin{enumerate}
\item %Déterminer la température du four, arrondie à l'unité, au bout de $4$ heures de refroidissement.

\begin{solution}
On cherche $T_4$. 

On applique l'algorithme pour $n = 4$ à l'aide de la calculatrice on trouve 

$T_4\approx 463$~\degres C
\end{solution}

\item %Démontrer que, pour tout nombre entier naturel $n$, on a : $T_n = 980 \times 0,82^n + 20$.

\begin{solution}

\textbf{\emph{Initialisation}}: pour $n=0$
	
	$T_0=\np{1000}$ et $980 \times 0,82^0 + 20=\np{1000}$
	
\vspace{0.3cm}
	
\textbf{\emph{Hérédité}}: Soit $n$ un entier naturel tel que $T_n=980 \times 0,82^n + 20$
		alors 
		
$T_{n+1}=0,82T_n+3,6$
	
$\hphantom{T_{n+1}}=0,82\times \left( 980 \times 0,82^n + 20\right) +3,6$ d'après l'hypothèse de récurrence 
	
$\hphantom{T_{n+1}}=980 \times 0,82^{n+1} + 16,4+3,6 $
		
$\hphantom{T_{n+1}}=980 \times 0,82^{n+1} + 20 $
			
La propriété est donc héréditaire à partir du rang $n = 0$ or elle est vérifiée à ce rang 0 donc par le principe de récurrence on vient de montrer que 
	
$\forall n \in \N~,~T_n=980 \times 0,82^n + 20$
	
\end{solution}

\item  %Au bout de combien d'heures le four peut-il être ouvert sans risque pour les céramiques ?

\begin{solution}

On cherche le plus petit entier naturel $n$ tel que $T_n\pp70$

On peut utiliser la calculatrice pour trouver $T_{14}\approx 80,9 >70$ et $T_{15}\approx 69,9 <70$

donc il faut attendre au minimum 15 heures avant de pouvoir ouvrir le four sans dommage.

\vspace{0.2cm}
\textbf{On peut aussi résoudre l'inéquation:} 

$T_n\pp70\equi980 \times 0,82^n + 20 \pp 70 \equi0,82^n \pp \dfrac{5}{98} \equi n\ln(0,82) \pp \ln\left( \dfrac{5}{98}\right) \equi n \pg \dfrac{\ln\left( \dfrac{5}{98}\right)}{\ln(0,82)}$ ~car~ $\ln(0,82)<0$ et on a $\dfrac{\ln\left( \dfrac{5}{98}\right)}{\ln(0,82)}\approx 14,99$
\end{solution}

\end{enumerate}
 
\bigskip

\textbf{Partie B}

\medskip

%\begin{minipage}{0.9\linewidth}
%Dans cette partie, on note $t$ le temps (en heure) écoulé depuis l'instant où le four a été éteint.
%
%La température du four (en degré Celsius) à l'instant $t$ est donnée par la fonction $f$ définie,
%pour tout nombre réel $t$ positif, par : 
%
%\[f(t) = a\text{e}^{- \frac{t}{5}} + b,\]
%
%où $a$ et $b$ sont deux nombres réels.
%
%On admet que $f$ vérifie la relation suivante : $f'(t) + \dfrac{1}{5}f(t) = 4$.
%\end{minipage}
%\medskip

\begin{enumerate}
\item %Déterminer les valeurs de $a$ et $b$ sachant qu'initialement, la température du four est de $\np{1000}$~\degres C, c'est-à-dire que $f(0) = \np{1000}$.

\begin{solution}
$f(0) = \np{1000}\equi a+b=\np{1000}$

de plus $f'(0)+\dfrac{1}{5}f(0)=4~\equi~f'(0)=-196$

$f$ est dérivable sur $\R$ et $\forall t \in \R~,~f'(t)=- \dfrac{1}{5}a\text{e}^{- \frac{t}{5}}$ d'où $f'(0)=-196 \equi \dfrac{1}{5}a = 196$

On a donc $\begin{cases}
a+b=\np{1000}\\\dfrac{1}{5}a=196
\end{cases}\equi~
\begin{cases}
b=20\\a=980
\end{cases}$

Finalement on a $\forall t \in [0~;~+\infty[~,~f(t)=980\text{e}^{- \frac{t}{5}} + 20$
\end{solution}

\item % Pour la suite, on admet que, pour tout nombre réel positif $t$: 

\[f(t) = 980\text{e}^{- \frac{t}{5}} + 20.\]

\medskip

	\begin{enumerate}
		\item %Déterminer la limite de $f$ lorsque $t$ tend vers $+ \infty$.
		
		\begin{solution}
$\displaystyle\lim_{t \to +\infty}\left( - \dfrac{t}{5}\right) = - \infty$, donc en posant $T = -\dfrac{t}{5}, 
\: \Lim{T}{-\infty}{\exp{T}}=~0\: \\\text{et par opération sur les limites on obtient} \:\Lim{t}{+\infty}{f(t) }=~20$.
		\end{solution}
		
		\item %Étudier les variations de $f$ sur $[0~;~+ \infty[$. 
		
%En déduire son tableau de variations complet.

\begin{solution}

D'après la question 1, $\forall t \in [0~;~+\infty[~,~f'(t)=-196\text{e}^{- \frac{t}{5}}$, d'où $f'(t) < 0$.
		
On en déduit que $f$ est strictement décroissante sur $[0~;~+\infty[$
		
\begin{center}
\psset{unit=1cm}
\begin{pspicture}(9,3)
\psframe(9,3)\psline(0,2)(9,2)\psline(0,2.5)(9,2.5)\psline(1,0)(1,3)
\uput[u](0.5,2.4){$t$} \uput[u](1.4,2.4){$0$}  \uput[u](8.5,2.4){$+\infty$} 
\rput(0.5,2.25){$f'(t)$} \rput(5,2.25){$-$}
\uput[u](1.5,1.5){$\np{1000}$}
\uput[u](8.5,0){$20 $}
\rput(0.5,1){$f(t)$}
\psline{->}(2,1.7)(8,0.3)
\end{pspicture}
\end{center}
\end{solution}
		\item %Avec ce modèle, après combien de minutes le four peut-il être ouvert sans risque pour les céramiques ?

\begin{solution}
On cherche à résoudre l'équation $f(t)=70$
		
Sur $[0~;~+\infty[,~f$ est continue et strictement décroissante à valeurs dans 

$]20~;~\np{1000}]$, or $70 \in ]20~;~\np{1000}]$ donc d'après le théorème des valeurs intermédiaires, l'équation $f(t)=70$ admet une unique solution $\alpha$ sur $[0~;~+\infty[$
		
Par dichotomie on trouve $\alpha \approx 14,9$ et comme $f$ est strictement décroissante sur $[0~;~+\infty[$, on en déduit que $f(t)\pp 70 \equi t \pg \alpha$.
		
D'après ce modèle on peut donc ouvrir le four après environ 15 heures de refroidissement.

\emph{Remarque : on peut aussi résoudre l'inéquation :}

$f(t) \leqslant  70 \iff 980\text{e}^{- \frac{t}{5}} + 20 \leqslant 70 \iff 980\text{e}^{- \frac{t}{5}}  \leqslant 50 \iff \text{e}^{- \frac{t}{5}} \leqslant \dfrac{5}{98} \iff$

$ - \dfrac{t}{5} \leqslant \ln \left(\dfrac{5}{98}\right)$ (par croissance de la fonction logarithme népérien) $ \iff $

$\dfrac{t}{5} \geqslant - \ln \left(\dfrac{5}{98}\right) \iff t \geqslant - 5 \ln \left(\dfrac{5}{98}\right) \approx 14,877$ soit en minutes au moins 893~min.
		
\end{solution}
 	\end{enumerate}
\item  %La température moyenne (en degré Celsius) du four entre deux instants $t_l$ et $t_2$ est donnée
%par: $\dfrac{1}{t_2 - t_1}\displaystyle\int_{t_1}^{t_2} f(t)\:\text{d}t$.

	\begin{enumerate}
		\item ~
		%À l'aide de la représentation graphique de $f$ ci-dessous, donner une estimation de la
%température moyenne $\theta$ du four sur les $15$ premières heures de refroidissement.
		
%Expliquer votre démarche.
		
%\begin{center}
%\psset{xunit=0.6cm,yunit=0.01cm}
%\begin{pspicture}(-1,-50)(19,1100)
%\multido{\n=0+1}{20}{\psline[linestyle=dashed,linewidth=0.5pt](\n,0)(\n,1100)}
%\multido{\n=0+100}{11}{\psline[linestyle=dashed,linewidth=0.5pt](0,\n)(19,\n)}
%\psaxes[linewidth=1.25pt,Dy=200]{->}(0,0)(0,0)(19,1100)
%\psaxes[linewidth=1.25pt,Dy=200](0,0)(0,0)(19,1100)
%\uput[d](16.4,-40){temps écoulé (en heure)}
%\uput[r](0,1080){température (en \degres C)}
%\psplot[plotpoints=3000,linewidth=1.25pt,linecolor=blue]{0}{19}{980 2.71828 0.2 x mul exp div 20 add}
%\end{pspicture}
%\end{center}

\begin{center}
\psset{xunit=0.6cm,yunit=0.01cm}
\begin{pspicture}(-1,-50)(19,1100)
\pspolygon[linecolor=green,fillstyle=solid,fillcolor=green,opacity=0.4](0.,0.)(0.,1000.)(4.,400)(4.,0.)

\pspolygon[linecolor=red,fillstyle=solid,fillcolor=red,opacity=0.4](4.,400)(4.,0.)(8.,0.)(8.,200.)

\pspolygon[linecolor=yellow,fillstyle=solid,fillcolor=yellow,opacity=0.4](8.,200.)(8.,0.)(12.,0.)(12.,100.)

\pspolygon[linecolor=blue,fillstyle=solid,fillcolor=blue,opacity=0.4](12.,0.)(12.,100.)(15.,100)(15.,0)


\multido{\n=0+1}{20}{\psline[linestyle=dashed,linewidth=0.5pt](\n,0)(\n,1100)}
\multido{\n=0+100}{11}{\psline[linestyle=dashed,linewidth=0.5pt](0,\n)(19,\n)}
\psaxes[linewidth=1.25pt,Dy=200]{->}(0,0)(0,0)(19,1100)
\psaxes[linewidth=1.25pt,Dy=200](0,0)(0,0)(19,1100)
\uput[d](16,-40){temps écoulé (en heure)}
\uput[r](0,1080){température (en \degres C)}
\psplot[plotpoints=3000,linewidth=1.25pt,linecolor=blue]{0}{19}{980 2.71828 0.2 x mul exp div 20 add}
\end{pspicture}
\end{center}

\bigskip

\begin{solution}
Sur $[0~;~15]$ l'aire entre la courbe $\mathcal{C}_f$ et l'axe des abscisses peut-être approchée par les quatre trapèzes ci-dessus et on a alors 

$\integ{0}{15}{f(t)}{t}\approx \np{2800}+\np{1200}+\np{600}+\np{300} = \np{4900} $ et donc la température moyenne est $\theta \approx \dfrac{\np{4900}}{15}\approx 327$~(\degres~C).

\textit{Remarque: cette question laisse place à toute méthode d'approche de l'aire (par des rectangles, par des trapèzes...) et donc les résultats attendus peuvent être très divers.
Ici le choix a été fait de trouver un minorant de l'aire par des rectangles}
\end{solution}
%\begin{solution}
%Sur $[0~;~15]$ l'aire entre la courbe $\mathcal{C}_f$ et l'axe des abscisses contient entièrement l'équivalent d'environ 47 rectangles d'aire 100 donc à l'aide de cette approximation grossière on aurait 
%
%$\integ{0}{15}{f(t)}{t}\approx \np{4700}$ et donc la température moyenne est $\theta \approx \dfrac{\np{4700}}{15}\approx 313$\degres~C
%
%\vspace{0.5cm}
%
%\end{solution}

		\item  %Calculer la valeur exacte de cette température moyenne $\theta$ et en donner la valeur
%arrondie au degré Celsius.

\begin{solution}

$\integ{0}{15}{f(t)}{t}=\integ{0}{15}{\left(980\exp{- \frac{t}{5}} + 20 \right) }{t}=\left[-\np{4900}\exp{- \frac{t}{5}} + 20t \right]_{_0}^{^{15}}=$

$\left(-\np{4900}\exp{- 3}+300 \right) -\left(-\np{4900} \right) = \np{4900}\left(1-\exp{- 3} \right)+300$

d'où $\theta =\dfrac{1}{15}\integ{0}{15}{f(t)}{t}=\dfrac{980}{3}\left(1-\exp{- 3} \right)+20 \approx 330,4$~(\degres C).
\end{solution}

	\end{enumerate}
\item %Dans cette question, on s'intéresse à l'abaissement de température (en degré Celsius) du
%four au cours d'une heure, soit entre deux instants $t$ et $(t + 1)$. Cet abaissement est donné
%par la fonction $d$ définie, pour tout nombre réel $t$ positif, par : $d(t) = f(t) - f(t + 1)$.
	\begin{enumerate}
		\item %Vérifier que. pour tout nombre réel $t$ positif: $d(t) = 980\left(1 - \text{e}^{- \frac{1}{5}}\right)\text{e}^{- \frac{t}{5}}$.
		
\begin{solution}
		
$d(t) = f(t) - f(t + 1)=\left(980\exp{- \frac{t}{5}} + 20 \right) -\left(980\exp{- \frac{t+1}{5}} + 20 \right) =980\left(\exp{- \frac{t}{5}}- \exp{- \frac{t+1}{5}}\right) $
		
On a donc bien $\forall t \in [0~;~+\infty[,~d(t) = 980\left(1 - \exp{- \frac{1}{5}}\right)\exp{- \frac{t}{5}}$
\end{solution}
		
	\item %Déterminer la limite de $d(t)$ lorsque $t$ tend vers $+ \infty$.
		
%Quelle interprétation peut-on en donner ?

\begin{solution}
\Lim{t}{+\infty}{\left( -\dfrac{t}{5}\right) }=~$-\infty$
		or \Lim{T}{-\infty}{\exp{T}}=~0 donc en posant $T=-\dfrac{t}{5}$ et par opération sur les limites on obtient \Lim{t}{+\infty}{d(t) }=~0.
		
On peut donc en conclure que la température finira par se stabiliser et comme on a \Lim{t}{+\infty}{f(t)}=~20, on en déduit que la température se stabilisera avec le temps à 20 \degres~C
\end{solution}
 	\end{enumerate}
\end{enumerate}

%\newpage

\label{exercice 2}
\textbf{\textsc{Exercice 2} \hfill 4 points}
 
\textbf{Commun  à tous les candidats}

\medskip

%Le plan est muni d'un repère orthonormé \Ouv.

\smallskip

Les points A, B et C ont pour affixes respectives $a = - 4,\: b = 2$ et $c = 4$.

\medskip

\begin{enumerate}
\item %On considère les trois points A$'$, B$'$ et C$'$ d'affixes respectives $a'= \text{j}a$, $b'= \text{j}b$ et $c'= \text{j}c$ où. j est le nombre complexe $-\dfrac{1}{2} + \text{i}\dfrac{\sqrt{3}}{2}$.

	\begin{enumerate}
		\item %Donner la forme trigonométrique et la forme exponentielle de j.
		
%En déduire les formes algébriques et exponentielles de $a'$, $b'$ et $c'$.

\begin{solution}
$\text{j}=-\dfrac{1}{2} + \text{i}\dfrac{\sqrt{3}}{2}=\cos\left(\dfrac{2\pi}{3}\right) + \text{i}\sin\left(\dfrac{2\pi}{3}\right)=\exp{\frac{2\text{i}\pi}{3}}$

$a'=a\text{j}=-4\text{j}=2-2\text{i}\sqrt{3}=4\left( -\exp{\frac{2\text{i}\pi}{3}}\right)=4\left( \exp{\text{i}\pi}\exp{\frac{2\text{i}\pi}{3}}\right)=4\exp{\text{i}\left( \pi+\frac{2\pi}{3}\right)  }=4\exp{\frac{5\text{i}\pi}{3}}=4\exp{-\frac{\text{i}\pi}{3}}$

$b'= b\text{j}=2\text{j}=-1+\text{i}\sqrt{3}=2\exp{\frac{2\text{i}\pi}{3}}$

$c'= c\text{j}=4\text{j}=-2+2\text{i}\sqrt{3}=4\exp{\frac{2\text{i}\pi}{3}}$
\end{solution}
		\item %Les points A, B et C ainsi que les cercles de centre O et de rayon 2, 3 et 4 sont
%représentés sur le graphique fourni en Annexe.
		
%Placer les points A$'$, B$'$ et C$'$ sur ce graphique.

\begin{solution}

\begin{minipage}{5.25cm}
\psset{unit=0.6cm}
\begin{pspicture}(-4,-4)(4,4)
\psgrid[gridlabels=0pt,subgriddiv=1,gridwidth=0.1pt]
\psaxes[linewidth=1pt,Dx=10,Dy=10](0,0)(-4,-4)(4,4)
\psaxes[linewidth=1.5pt,Dx=10,Dy=10]{->}(0,0)(1,1)
\uput[d](0.5,0){$\vect{u}$}\uput[l](0,0.5){$\vect{v}$}\uput[dl](0,0){O}
\uput[dr](2,0){B} \uput[dr](4,0){C}\uput[dl](-4,0){A}
\pscircle(0,0){2}\pscircle(0,0){3}\pscircle(0,0){4}
\psdots(-4,0)(2,0)(4,0)

\psdots[linecolor=red](2,-3.464)(-1,1.732)(-2,3.464)
\uput[dr](2,-3.464){\red A$'$}
\uput[dr](-1,1.732){\red B$'$}
\uput[dr](-2,3.464){\red C$'$}

\psdots[linecolor=blue](3,-1.732)(1,1.732)(-3,1.732)
\uput[dr](3,-1.732){\blue M}
\uput[ur](1,1.732){\blue N}
\uput[ul](-3,1.732){\blue P}
\psline[linecolor=blue](3,-1.732)(1,1.732)(-3,1.732)(3,-1.732)
\end{pspicture}
\end{minipage}
\hfill\begin{minipage}{6.5cm}
$|a'|=4$ donc A$'$ est sur le cercle de centre O et de rayon 4 et on a $Re\left(a' \right) =2$ et $Im\left(a' \right)<0$, on peut donc placer A$'$

\vspace{0.8cm}
$|b'|=2$ donc B$'$ est sur le cercle de centre O et de rayon 2 et on a $Re\left(b' \right) =-1$ et $Im\left(b' \right)>0$, on peut donc placer B$'$

\vspace{0.8cm}
$|c'|=4$ donc C$'$ est sur le cercle de centre O et de rayon 4 et on a $Re\left(c' \right) =-2$ et $Im\left(c' \right)>0$, on peut donc placer C$'$
\end{minipage}
\end{solution}

	\end{enumerate}
\item  %Montrer que les points A$'$, B$'$ et C$'$ sont alignés.

\begin{solution}

$a'=-c'$ donc A$'$ et C$'$ sont symétriques par rapport à O alors O, A$'$ et C$'$ sont alignés

$arg\left( b'\right) =arg\left( c'\right) =\dfrac{2\pi}{3} (2\pi)$ donc $\vect{\text{OB}'}$ et $\vect{\text{OC}'}$ sont colinéaires d'où O, B$'$ et C$'$ sont alignés.

Finalement O, A$'$, B$'$ et C$'$ sont alignés.
\end{solution}

\item  %On note M le milieu du segment [A$'$C], N le milieu du segment [C$'$C] et P le milieu du segment [C$'$A]. 
	
%Démontrer que le triangle MNP est isocèle.

\begin{solution}

$z_{\text{M}}=\dfrac{a'+c}{2}=3-\text{i}\sqrt{3}$~~~,~~~
$z_{\text{N}}=\dfrac{c'+c}{2}=1+\text{i}\sqrt{3}$~~~,~~~
$z_{\text{P}}=\dfrac{c'+a}{2}=-3+\text{i}\sqrt{3}$

MNP semble isocèle en N d'après le dessin

MN=$\left|z_{\text{N}}-z_{\text{M}} \right| = \left|2-2\text{i}\sqrt{3} \right| =4$ et PN=$\left|z_{\text{N}}-z_{\text{P}} \right| =\left|4 \right| = 4$

On a MN=NP donc MNP est bien isocèle en N
\end{solution}

\end{enumerate}

\vspace{0,5cm}
\newpage
\label{exercice 3}
\textbf{\textsc{Exercice 3} \hfill 5 points}
 
\textbf{Commun  à tous les candidats}

\medskip

%\begin{minipage}{0.9\linewidth}
%Une entreprise conditionne du sucre blanc provenant de deux exploitations U et V en paquets
%de 1 kg et de différentes qualités.
%\end{minipage}
%
%\smallskip
%
%Le sucre extra fin est conditionné séparément dans des paquets portant le label \og  extra fin \fg.
%
%\smallskip
%
%\emph{Les parties \rm{A}, \rm{B} et \rm{C} peuvent être traitées de façon indépendante.}
%
%\smallskip
%
%Dans tout l'exercice, les résultats seront arrondis, si nécessaire, au millième.
%
%\bigskip

\textbf{Partie A}

\medskip

%\begin{minipage}{0.9\linewidth}
%Pour calibrer le sucre en fonction de la taille de ses cristaux, on le fait passer au travers d'une
%série de trois tamis positionnés les uns au-dessus des autres et posés sur un récipient à fond
%étanche.
%Les ouvertures des mailles sont les suivantes :
%\end{minipage}
%
%\begin{center}
%\psset{unit=1cm}
%\begin{pspicture}(12,4)
%%\psgrid
%\uput[r](0.5,3){Tamis 1 : 0,8 mm} \psline[linewidth=1.5pt]{->}(4.6,1)(6.8,1)
%\uput[r](0.5,2){Tamis 2 : 0,5 mm} \psline[linewidth=1.5pt]{->}(4.6,2)(6.8,2)
%\uput[r](0.5,1){Tamis 3 : 0,2 mm} \psline[linewidth=1.5pt]{->}(4.6,3)(6.8,3)
%\uput[r](0.5,0.5){Récipient à fond étanche}  \psline[linewidth=1.5pt]{->}(4.6,0.5)(7,0.5)
%\psline[linewidth=2pt](7,3.75)(7,0)(12,0)(12,3.75)
%\multido{\n=7.00+0.12}{42}{\psframe(\n,2.94)(\n,3.06)}
%\multido{\n=7.00+0.1}{50}{\psframe(\n,1.96)(\n,2.04)}
%\multido{\n=7.00+0.06}{84}{\psframe(\n,0.97)(\n,1.03)}
%\end{pspicture}
%\end{center}
%
%\begin{minipage}{0.9\linewidth}
%Les cristaux de sucre dont la taille est inférieure à $0,2$ mm se trouvent dans le récipient à fond
%étanche à la fin du calibrage. lis seront conditionnés dans des paquets portant le label \og  sucre
%extra fin \fg.
%\end{minipage}
%
%\medskip

\begin{enumerate}
\item %On prélève au hasard un cristal de sucre de l'exploitation U. La taille de ce cristal,
%exprimée en millimètre, est modélisée par la variable aléatoire $X_{\text{ U}}$ qui suit la loi normale
%de moyenne $\mu_{\text{ U}} = 0,58$~mm et d'écart type $\sigma_{\text{ U}} = 0,21$~mm.
	\begin{enumerate}
		\item %Calculer les probabilités des évènements suivants : $X_{\text{ U}} < 0,2$ et $0,5 \leqslant X_{\text{ U}} < 0,8$.
	
\begin{solution}
	
$X_{\text{ U}} \normale{0,58}{0,21}$ donc $P\left(X_{\text{ U}}<0,2 \right)\approx 0,035$ et $P\left(0,5 \pp X_{\text{ U}}<0,8 \right)\approx 0,501$  
\end{solution}
	
		\item %On fait passer \np{1800} grammes de sucre provenant de l'exploitation U au travers de la
%série de tamis.
	
%Déduire de la question précédente une estimation de la masse de sucre récupérée dans
%le récipient à fond étanche et une estimation de la masse de sucre récupérée dans le
%tamis 2.

\begin{solution}
$P\left(X_{\text{ U}}<0,2 \right)\approx 0,035$ signifie que la probabilité qu'un cristal se retrouve dans le récipient à fond étanche est de 0,035 environ ; 
donc pour \np{1800}g de sucre on peut estimer que \np{1800}$\x$ 0,035~=~63 g se retrouvent dans le récipient à fond étanche.

\vspace{0.3cm}
$P\left(0,5 \pp X_{\text{ U}}<0,8 \right)\approx 0,501$ signifie que la probabilité qu'un cristal se retrouve dans le tamis 2 est de 0,501 environ.

Donc pour \np{1800}~g de sucre on peut estimer que $\np{1800} \times 0,501~=~901,8$~g se retrouvent dans le tamis 2.
\end{solution}
 		\end{enumerate}

\item %On prélève au hasard un cristal de sucre de l'exploitation V. La taille de ce cristal,
%exprimée en millimètre, est modélisée par la variable aléatoire $X_{\text{V}}$ qui suit la loi normale
%de moyenne $\mu_{\text{V}} = 0,65$ mm et d'écart type $\sigma_{\text{V}}$ à déterminer.

%Lors du calibrage d'une grande quantité de cristaux de sucre provenant de l'exploitation V,
%on constate que 40\,\% de ces cristaux se retrouvent dans le tamis 2.

%Quelle est la valeur de l'écart type $\sigma_{\text{V}}$ de la variable aléatoire $X_{\text{V}}$ ?

\begin{solution}
On sait que si $X_{\text{V}} \normale{0,65}{\sigma_{\text{V}}}$ alors $Z=\dfrac{X_{\text{V}}-0,65}{\sigma_{\text{V}}}\normale{0}{1}$

On a ici $P\left(0,5 \pp X_{\text{ V}}<0,8 \right)= 0,4$

$0,5 \pp X_{\text{ V}}<0,8~ \equi~-\dfrac{0,15}{\sigma_{\text{V}}} \pp Z<\dfrac{0,15}{\sigma_{\text{V}}}$

on a donc $P\left(-\dfrac{0,15}{\sigma_{\text{V}}} \pp Z<\dfrac{0,15}{\sigma_{\text{V}}} \right) =0,4$ avec $Z \normale{0}{1}$.

À l'aide de la calculatrice on trouve $\dfrac{0,15}{\sigma_{\text{V}}}\approx 0,524$ d'où $\sigma_{\text{V}}\approx 0,286$.
\end{solution}
\end{enumerate}

\bigskip
%\newpage
\textbf{Partie B}

\medskip

%\begin{minipage}{0.9\linewidth}
%Dans cette partie, on admet que 3\,\% du sucre provenant de l'exploitation U est extra fin et que
%5\,\% du sucre provenant de l'exploitation V est extra fin.
%
%On prélève au hasard un paquet de sucre dans la production de l'entreprise et, dans un souci
%de traçabilité, on s'intéresse à la provenance de ce paquet.
%
%On considère les évènements suivants:
%\end{minipage}
%
%\setlength\parindent{9mm}
%\begin{itemize}
%\item[$\bullet~~$] $U$ : \og  Le paquet contient du sucre provenant de l'exploitation U \fg{} ;
%\item[$\bullet~~$] $V$ : \og Le paquet contient du sucre provenant de l'exploitation V \fg{} ;
%\item[$\bullet~~$] $E$ : \og Le paquet porte le label "extra fin" \fg{}.
% \end{itemize}
%\setlength\parindent{0mm} 
%
%\medskip
 
\begin{enumerate}
\item %Dans cette question, on admet que l'entreprise fabrique 30\,\% de ses paquets avec du sucre
%provenant de l'exploitation U et les autres avec du sucre provenant de l'exploitation V,
%sans mélanger les sucres des deux exploitations.
	\begin{enumerate}
		\item %Quelle est la probabilité que le paquet prélevé porte le label \og extra fin \fg{} ?
		
\begin{solution}
L'énoncé donne $P_U(E)=0,03~,~P_V(E)=0,05~,~P(U) = 0,3~\text{et}$

$P(V) = 0,7$$ 
		
On cherche $P(E)$.
		
$U$ et $V$ forment une partition de l'univers, 
donc d'après les probabilités totales on a
		
$P(E)=P\left(E \cap U \right) +P\left(E \cap V  \right)$
		
$\hphantom{P(E)}=P(U)\x P_U(E) +P(V)\x P_V(E)$
		
		$\hphantom{P(E)}=0,009 +0,035=0,044$
		 
		
		\end{solution}
		
		\item %Sachant qu'un paquet porte le label \og extra fin \fg, quelle est la probabilité que le sucre qu'il contient provienne de l'exploitation U ?

\begin{solution}
On cherche $P_E(U)$

$P_E(U)=\dfrac{P(E\cap U)}{P(E)}=\dfrac{0,009}{0,044}=\dfrac{9}{44}\approx 0,205$
\end{solution}
 	\end{enumerate}
 	
 	\newpage
\item %L'entreprise souhaite modifier son approvisionnement auprès des deux exploitations afin
%que parmi les paquets portant le label « extra fin », 30\,\% d'entre eux contiennent du sucre
%provenant de l'exploitation U.
	
%Comment doit-elle s'approvisionner auprès des exploitations U et V ?
	
%Toute trace de recherche sera valorisée dans cette question.

\begin{solution}on veut $P_E(U)=0,3$

$P_E(U)=\dfrac{P(E\cap U)}{P(E)}=0,3 \equi \dfrac{P(U)\x P_U(E)}{P(U)\x P_U(E) +P(V)\x P_V(E)}=0,3$

$\hphantom{P_E(U)=\dfrac{P(E\cap U)}{P(E)}=0,3} \equi \dfrac{0,03P(U)}{0,03P(U) +0,05P(V)}=0,3$

$\hphantom{P_E(U)=\dfrac{P(E\cap U)}{P(E)}=0,3} \equi 0,03P(U)=0,009P(U) +0,015P(V)$

$\hphantom{P_E(U)=\dfrac{P(E\cap U)}{P(E)}=0,3} \equi 0,021P(U)=0,015P(V)$

$\hphantom{P_E(U)=\dfrac{P(E\cap U)}{P(E)}=0,3} \equi P(U)=\dfrac{5}{7}P(V)$

or $P(V)=1-P(U)$ donc on a $P(U)=\dfrac{5}{7}\left( 1-P(U)\right) \equi \dfrac{12}{7}P(U)=\dfrac{5}{7} \equi$

$P(U)=\dfrac{5}{12}\approx 0,417 $

Si l'entreprise veut atteindre son objectif, elle doit donc acheter environ 41,7\% de son sucre à l'exploitation U et 58,3\% à V
\end{solution}
\end{enumerate}

\bigskip

%\newpage
\textbf{Partie C}

\medskip

\begin{enumerate}
\item %L'entreprise annonce que 30\,\% des paquets de sucre portant le label «extra fin» qu'elle conditionne contiennent du sucre provenant de l'exploitation U.

%Avant de valider une commande, un acheteur veut vérifier cette proportion annoncée. Il prélève $150$ paquets pris au hasard dans la production de paquets labellisés \og extra fin \fg{} de l'entreprise. Parmi ces paquets, $30$ contiennent du sucre provenant de l'exploitation U.

%A-t-il des raisons de remettre en question l'annonce de l'entreprise ?
\begin{solution}
Ici on répète $n=150$ fois de manière indépendante le prélèvement d'un paquet 
	
La proportion annoncée de paquets provenant de l'exploitation U est $p = 0,3$.
	
On a $n\geqslant 30~~,~~np=45\geqslant 5~~\text{et}~~n(1-p)=105\geqslant 5$.
	
On peut donc bâtir l'intervalle de fluctuation asymptotique.
	
On peut affirmer au seuil de 95\% que la fréquence observée de paquets de sucre provenant de l'exploitation U devrait appartenir à l'intervalle 

$I_n=\left[ p-1,96 \dfrac{\sqrt{p(1-p)}}{\sqrt{n}}   ; p+1,96 \dfrac{\sqrt{p(1-p)}}{\sqrt{n}}  \right]$.
	
$p - 1,96 \dfrac{\sqrt{p(1-p)}}{\sqrt{n}} \approx 0,227$ et $p+1,96 \dfrac{\sqrt{p(1-p)}}{\sqrt{n}} \approx 0,373$
	
La fréquence observée est $f=\dfrac{30}{150}=0,2 \notin I_n$.
	
On peut donc rejeter l'affirmation au seuil de 95\%. 
\end{solution}
\item  %L'année suivante, l'entreprise déclare avoir modifié sa production. L'acheteur souhaiteestimer la nouvelle proportion de paquets de sucre provenant de l'exploitation U parmi lespaquets portant le label \og extra fin \fg. 

%Il prélève 150 paquets pris au hasard dans la production de paquets labellisés \og extra fin \fg{} de l'entreprise. Parmi ces paquets 42\,\% contiennent du sucre provenant de l'exploitation U.

%Donner un intervalle de confiance, au niveau de confiance 95\,\%, de la nouvelle proportion de paquets labellisés \og extra fin \fg{} contenant du sucre provenant de l'exploitation U.

\begin{solution}l'échantillon est de taille $n=150$, la fréquence observée de paquets provenant de l'exploitation U est $f=0,42$

On a $n\geqslant 30~~,~~nf=63\geqslant 5~~\text{et}~~n(1-f)=87\geqslant 5$.
	
On peut donc bâtir l'intervalle de confiance.
	
On peut affirmer avec une confiance à 95\,\% que la proportion $p$ de paquets de sucre provenant de l'exploitation U devrait appartenir à l'intervalle 

$I=\left[ f-\dfrac{1}{\sqrt{n}}  ; f+\dfrac{1}{\sqrt{n}}  \right]$.
	$f-\dfrac{1}{\sqrt{n}} \approx 0,3384$ et $f+\dfrac{1}{\sqrt{n}} \approx 0,5016$
	
Un intervalle de confiance, au niveau de confiance 95\,\% est donc environ 

$I=[0,338~;~0,502]$. 
\end{solution}
\end{enumerate}

\newpage

\label{exercice 4-obli}
\textbf{\textsc{Exercice 4} \hfill 5 points}
 
\textbf{Candidats n'ayant pas suivi l'enseignement de spécialité}

\medskip

%Dans l'espace muni du repère orthonormé \Oijk d'unité 1~cm, on considère les points
%A, B, C et D de coordonnées respectives (2~;~1~;~4), $(4~;~-1~;~0)$, $(0~;~3~;~2)$ et $(4~;~3~;~-2)$.
%
%\medskip

\begin{enumerate}
\item Déterminer une représentation paramétrique de la droite (CD).

\begin{solution}
$\vect{\text{CD}}\begin{pmatrix}
4\\0\\-4
\end{pmatrix}$ donc $\vect{u}\begin{pmatrix}
1\\0\\-1
\end{pmatrix}$ est un vecteur directeur 
de (CD), et (CD) passe par C$(0~;~3~;~2)$ 

on obtient une représentation paramétrique de (CD): $\begin{cases}
x=t\\y=3\\z=2-t
\end{cases}~~(t\in \R)$
\end{solution}

\item Soit M un point de la droite (CD).
	\begin{enumerate}
		\item Déterminer les coordonnées du point M tel que la distance BM soit minimale.
		
		\begin{solution}B$(4~;~-1~;~0)$. 		
		Soit $t$ le paramètre associé à M alors M$(t~;~3~;~2-t)$. 
		
Alors BM$^2=\left(t-4 \right)^2+\left(4 \right)^2+\left(2-t \right)^2=2t^2-12t+36=2(t^2-6t+18)$. 
		
BM est minimale quand BM$^2$ l'est c'est-à-dire quand $t^2 - 6t + 18$ est minimal.
		
On sait que tout polynôme de la forme $at^2+bt+c$ avec $a > 0$ admet un minimum en $t = -\dfrac{b}{2a}$
		
ici BM sera donc minimale pour $t = 3$ soit pour M$(3~;~3~;~-1)$.
	
\end{solution}
		
		\item %On note H le point de la droite (CD) ayant pour coordonnées $(3~;~3~;~- 1)$.
%Vérifier que les droites (BH) et (CD) sont perpendiculaires.

\begin{solution}
$\vect{\text{BH}}\begin{pmatrix}
-1\\4\\-1
\end{pmatrix}$ et $\vect{\text{CD}}\begin{pmatrix}
4\\0\\-4
\end{pmatrix}$ donc $\vect{\text{HB}}\cdot \vect{\text{CD}} = 0$.

On en déduit que (BH) et (CD) sont perpendiculaires.

\end{solution}

		\item %Montrer que l'aire du triangle BCD est égale à 12 cm$^2$.
		
		\begin{solution}
D'après ce qui précède, on a (BH) est la hauteur issue de B dans BCD.
		
On a alors $\mathcal{A}_{\text{BCD}}=\dfrac{1}{2}\x \text{CD}\x \text{BH}=\dfrac{1}{2}\x \sqrt{32}\x \sqrt{18}=\sqrt{144}=12$~(en u. a.).
		
L'aire de BCD est donc bien de 12~cm$^2$.
		\end{solution}
		
	\end{enumerate}
\item 
	\begin{enumerate}
		\item %Démontrer que le vecteur $\vect{n}\begin{pmatrix}2\\1\\2\end{pmatrix}$  est un vecteur normal au plan (BCD).
		
\begin{solution}
		
$\vect{\text{BC}}\begin{pmatrix}
-4\\4\\2
\end{pmatrix}$ et $\vect{\text{CD}}\begin{pmatrix}
4\\0\\-4
\end{pmatrix}$ ne sont évidement pas colinéaires donc B, C et D définissent bien un plan.

$\vect{n}\cdot\vect{\text{BC}}=-8+4+4=0$ et $\vect{n}\cdot\vect{\text{CD}}=8+0-8=0$. 

$\vect{n}$ est donc bien normal au plan (BCD) car orthogonal à deux vecteurs non colinéaires de ce plan.
		\end{solution}
		
		\item %Déterminer une équation cartésienne du plan (BCD).
		
\begin{solution}
		
$\vect{n}\begin{pmatrix}2\\1\\2\end{pmatrix}$ est normal au plan (BCD) donc (BCD):~$2x+y+2z+d=0$.
		
C$(0~;~3~;~2)$ appartient à (BCD) donc $2x_{\text{C}}+y_{\text{C}}+2z_{\text{C}}+d=0$ ce qui donne $d = -7$.
		
Finalement (BCD)~:~$2x+y+2z-7= 0$.
		\end{solution}
		
		\item %Déterminer une représentation paramétrique de la droite $\Delta$ passant par A et orthogonale au plan (BCD).

\begin{solution}
$\Delta$ est orthogonale au plan (BCD) donc elle admet $\vect{n}$ pour vecteur directeur, on a alors

$\Delta~:~\begin{cases}
x=2+2t\\y=1+t\\z=4+2t
\end{cases}~~(t\in \R)$

\end{solution}

		\item %Démontrer que le point I, intersection de la droite $\Delta$ et du plan (BCD) a pour coordonnées $\left(\dfrac{2}{3}~;~\dfrac{1}{3}~;~\dfrac{8}{3}\right)$.

\begin{solution}
I est un point de (BCD) donc $2x_{\text
I}+y_{\text
I}+2z_{\text
I}-7=0$ 

De plus I $\in \Delta$ donc il existe un réel $t$ tel que 

$2(2+2t)+(1+t)+2(4+2t)-7=0~\equi~9t=-6 \equi t=-\dfrac{2}{3}$.

On en déduit I$\left(\dfrac{2}{3}~;~\dfrac{1}{3}~;~\dfrac{8}{3}\right)$.

\emph{Remarque}: on pouvait aussi simplement vérifier que les coordonnées proposées correspondaient à un point de $\Delta$ et à un point de (BCD).
\end{solution}

	\end{enumerate}
\item  %Calculer le volume du tétraèdre ABCD.

\begin{solution}
$\Delta$ est perpendiculaire au plan (BCD) en I et passe par A, on en déduit que AI est la hauteur du tétraèdre ABCD de base BCD. 

AI=$\sqrt{\left(-\dfrac{4}{3} \right)^2+\left(-\dfrac{2}{3} \right)^2+\left(-\dfrac{4}{3} \right)^2}=2$.

$\mathcal{V}_{\text{ABCD}}=\dfrac{1}{3}\x AI \x \mathcal{A}_{\text{BCD}}=8$ (en cm$^3$).

Le volume du tétraèdre est 8~cm$^3$.
\end{solution}

\end{enumerate}

\vspace{0,5cm}

\newpage

\label{exercice 4-spé}

\textbf{\textsc{Exercice 4} \hfill 5points}
 
\textbf{Candidats ayant suivi l'enseignement de spécialité}

\medskip

%À toute lettre de l'alphabet on associe un nombre entier $x$ compris entre 0 et 25 comme
%indiqué dans le tableau ci-dessous:
%
%\begin{center}
%\begin{tabularx}{\linewidth}{|c|*{13}{>{\centering \arraybackslash}X|}}\hline
%Lettre 	&A &B &C &D &E &F &G &H &I &J &K 	&L 	&M\\ \hline
%$x$ 	&0 &1 &2 &3 &4 &5 &6 &7 &8 &9 &10 	&11 &12\\ \hline\hline
%Lettre 	&N &O &P &Q &R &S &T &U &V &W &X 	&Y 	&Z\\ \hline
%$x$ 	&13&14&15&16&17&18&19&20&21&22&23 	&24 &25\\ \hline
%\end{tabularx}
%\end{center}
%
%\medskip
%
%Le \og chiffre de RABIN \fg{} est un dispositif de cryptage asymétrique inventé en 1979 par
%l'informaticien Michael RABIN.
%
%Alice veut communiquer de manière sécurisée en utilisant ce cryptosystème. Elle choisit deux
%nombres premiers distincts $p$ et $q$. Ce couple de nombres est sa clé privée qu'elle garde
%secrète.
%
%Elle calcule ensuite $n = p \times q$ et elle choisit un nombre entier naturel $B$ tel que $0 \leqslant B \leqslant n -1$.
%
%Si Bob veut envoyer un message secret à Alice, il le code lettre par lettre.
%
%Le codage d'une lettre représentée par le nombre entier $x$ est le nombre $y$ tel que :
%
%\[y \equiv  x(x + B)\quad  [n] \:\text{ avec }\: 0 \leqslant y \leqslant n.\]
%
%Dans tout l'exercice on prend $p = 3,\: q = 11$ donc $n = p \times q = 33$ et $B = 13$.
%
%\bigskip

\textbf{Partie A : Cryptage}

\medskip

%Bob veut envoyer le mot \og  NO \fg{} à Alice.

\begin{enumerate}
\item %Montrer que Bob code la lettre \og N \fg{} avec le nombre 8.

\begin{solution}

Dans l'alphabet la lettre \og N \fg{} est associée à $x=13$ alors $x(x+B) = 13\x26 = 338 = 33 \x  10 + 8 \equiv 8\: [33]$

Bob code la lettre \og N \fg{} avec le nombre $y = 8$.
\end{solution}

\item %Déterminer le nombre qui code la lettre \og O \fg.
\begin{solution}

Dans l'alphabet la lettre \og O \fg{} est associée à $x = 14$ alors $x(x + B) = 14\x27=378 = 33\x11 + 15 \equiv 15 [33]$

Bob code la lettre \og O \fg{} avec le nombre $y=15$.
\end{solution}
\end{enumerate}

\bigskip

\textbf{Partie B : Décryptage}

\medskip

%Alice a reçu un message crypté qui commence par le nombre 3.

%Pour décoder ce premier nombre, elle doit déterminer le nombre entier $x$ tel que :

%\[x(x + 13) \equiv  3 \quad [33]\:  \text{ avec }\: 0 \leqslant  x \leqslant 26.\]

%\medskip

\begin{enumerate}
\item %Montrer que $x(x + 13) \equiv 3\quad [33]$ équivaut à $(x + 23)^2 \equiv 4\quad [33]$.

\begin{solution}
Soit $x$ un entier tel que $x(x + 13) \equiv 3\quad [33] \equi x^2+13x \equiv 3\:\: [33]$.

$\hphantom{x(x + 13) \equiv 3\:\:[33]} \equi x^2+13x + 33x  \equiv 3\:\: [33]$~~~~ car~~~~ $33x \equiv 0\:\: [33]$

$\hphantom{x(x + 13) \equiv 3\:\: [33]} \equi x^2+46x+529 \equiv 3+1\:\: [33]$~~~~ car~~~~ $529=16\x33+1 \equiv 1\:\: [33]$

$\hphantom{x(x + 13) \equiv 3\:\: [33]} \equi \left(x+23 \right)^2  \equiv 4\:\:[33]$.
\end{solution}

\item
	\begin{enumerate}
		\item %Montrer que si $(x + 23)^2 \equiv 4\quad [33]$ alors le système d'équations $\left\{\begin{array}{l c l}(x + 23)^2 &\equiv &4 \quad [3]\\ (x + 23)^2 &\equiv &4 \quad [11] \end{array}\right.$ est vérifié.

\begin{solution}

Soit $x$ un entier tel que $(x + 23)^2 \equiv 4\:\: [33]$  alors il existe un entier $k$ tel que $(x+23)^2=33k+4$.

Alors $(x+23)^2=33k+4 =3\x11k + 4 \equiv 4 \:\: [3]$

$(x+23)^2=33k+4=11\x3k+4 \equiv 4 \:\: [11]$

finalement si $(x + 23)^2 \equiv 4\:\: [33]$ alors le système d'équations 

$\left\{\begin{array}{l c l}
(x + 23)^2 &\equiv &4 \:\: [3]\\ 
(x + 23)^2 &\equiv &4 \:\: [11]
\end{array}\right.$ est vérifié.
\end{solution}

		\item %Réciproquement, montrer que si  $\left\{\begin{array}{l c l}(x + 23)^2 &\equiv &4\quad [3]\\ (x + 23)^2 &\equiv &4 \quad [11]\end{array}\right.$ alors $(x + 23)^2 \equiv 4\quad [33]$.

\begin{solution}
si $(x + 23)^2 \equiv 4\:\: [11]$ et $(x + 23)^2\equiv 4 \:\: [33]$, alors il existe un entier $k'$ tel que $(x+23)^2=11k'+4$

Or $(x + 23)^2 \equiv 4\:\: [3]$. 
On a alors $11k'+ 4 \equiv 4\:\: [3]  $ ou encore $11k' \equiv 0\:\: [3]$

donc 3 divise $11k'$ or 3 et  11 sont premiers entre eux
donc d'après le théorème de Gauss, 
on en déduit que 3 divise $k'$.

Il existe donc un entier $r$ tel que $k'=3r$

alors $(x + 23)^2 = 11k' + 4 = 33r + 4\equiv 4\:\: [33]$, d'où $(x + 23)^2 \equiv 4\:\: [33]$

Finalement si  $\left\{\begin{array}{l c l}
(x + 23)^2 &\equiv &4\:\: [3]\\ 
(x + 23)^2 &\equiv &4 \:\: [11]
\end{array}\right.$ alors $(x + 23)^2 \equiv 4\:\: [33]$.
\end{solution}

\emph{Remarque} :

Plus rapide : 3 et 11 divisent $(x + 23)^2 - 4$. Or 3 et 11 sont premiers entre eux donc leur produit $ 3 \times 11 = 33$ divise $(x + 23)^2 - 4$, d'où $(x + 23)^2 \equiv 4\:\: [33]$.

		\item %En déduire que $x(x + 13) \equiv 3\quad [33] \iff  \left\{\begin{array}{l c l}(x + 23)^2 &\equiv&1 \quad [3]\\(x + 23)^2 &\equiv& 4 \quad [11]\end{array}\right.$

\begin{solution}
$(x + 23)^2 \equiv 4\:\: [3] \equi (x + 23)^2 \equiv 1\:\: [3] $ car $4 \equiv 1\:\:[3]$.

Dans la question \textbf{a.} on a montré que $x(x + 13) \equiv 3\:\: [33] \imp  $

$\left\{\begin{array}{l c l}
(x + 23)^2 &\equiv&1 \:\: [3]\\
(x + 23)^2 &\equiv& 4 \:\: [11]
\end{array}\right.$

Dans la question \textbf{b.} on a montré ainsi que 

$ \left\{\begin{array}{l c l}
(x + 23)^2 &\equiv&1 \:\: [3]\\
(x + 23)^2 &\equiv& 4 \:\: [11]
\end{array}\right.\imp$

$ x(x + 13) \equiv 3\:\: [33]  $

On a donc bien $x(x + 13) \equiv 3\:\: [33] \iff  \left\{\begin{array}{l c l}
(x + 23)^2 &\equiv&1 \:\: [3]\\
(x + 23)^2 &\equiv& 4 \:\: [11]
\end{array}\right.$
\end{solution}

	\end{enumerate}
\item
	\begin{enumerate}
		\item %Déterminer les nombres entiers naturels $a$ tels que $0 \leqslant a < 3$ et $a^2 \equiv 1 \quad  [3]$.
		
\begin{solution}
		
		\begin{center}
		\begin{tabularx}{7cm}{|c|*{3}{>{\centering \arraybackslash}X|}}\hline
		$a$ 							&0 &1 &2  \\ \hline
		$a^2$ 							&0 &1 &4  \\ \hline
		modulo 3, $a^2$ est congru à 	&0 &1 &1  \\ \hline
		\end{tabularx}
		\end{center}
		
Les entiers naturels $a$ tels que $0 \leqslant a < 3$ et $a^2 \equiv 1 \:\:[3]$ sont 1 et 2.
		\end{solution}
		
		\item %Déterminer les nombres entiers naturels $b$ tels que $0 \leqslant b < 11$ et $b^2 \equiv 4\quad [11]$.
		
\begin{solution}
\begin{center}
\begin{tabularx}{\linewidth}{|c|*{11}{>{\centering \arraybackslash}X|}}\hline
$b$ 	&0 &1 &2 &3 &4 &5 &6 &7 &8 &9 &10 \\ \hline
$b^2$ 	&0 &1 &4 &9 &16 &25 &36 &49 &64 &81 &100 \\ \hline
modulo 11, $b^2$ est congru à 	&0 &1 &4 &9 &5 &3 &3 &5 &9 &4 &1  \\ \hline
\end{tabularx}
\end{center}

Les entiers naturels $b$ tels que $0 \leqslant b < 11$ et $b^2 \equiv 4\:\: [11]$ sont 2 et 9.
\end{solution}
 	\end{enumerate}
\item
	\begin{enumerate}
		\item %En déduire que $x(x + 13) \equiv 3 \quad[33]$ équivaut aux quatre systèmes suivants :
		
%\[\left\{\begin{array}{l c l}
%x &\equiv&2\quad [3]\\
%x&\equiv &8\quad[11]
%\end{array}\right. \: \text{ ou } \left\{\begin{array}{l c l}
% x &\equiv& 0\quad[3]\\
% x &\equiv& 1 \quad[11]
% \end{array}\right.\: \text{ ou } \left\{\begin{array}{l c l}
%x  &\equiv& 2\quad[3]\\
%x &\equiv&1 \quad[11]
%\end{array}\right.\: \text{ ou } \left\{\begin{array}{l c l}
%x &\equiv& 0\quad [3]\\
%x &\equiv& 8 \quad [11]
%\end{array}\right.\]

\begin{solution}

On sait que $x(x + 13) \equiv 3\quad [33] \iff  \left\{\begin{array}{l c l}
(x + 23)^2 &\equiv&1 \:\: [3]\\
(x + 23)^2 &\equiv& 4 \:\:[11]
\end{array}\right.$

D'après ce qui précède on a 

$(x+23)^2\equiv 1\:\:[3] \equi \begin{cases}
x+23 \equiv 1\:\:[3]\\\text{ou}\\x+23 \equiv 2\:\:[3]
\end{cases}\equi \begin{cases}
x \equiv -22\:\:[3]\\\text{ou}\\x \equiv -21\:\:[3]
\end{cases} \equi $

$\begin{cases}
x \equiv 2\:\:[3]\\\text{ou}\\x \equiv 0\:\:[3]
\end{cases}$

$(x+23)^2\equiv 4\quad[11] \equi \begin{cases}
x+23 \equiv 2\:\:[11]\\\text{ou}\\x+23 \equiv 9\:\:[11]
\end{cases}\equi \begin{cases}
x \equiv -21\:\:[11]\\\text{ou}\\x \equiv -14\:\:[11]
\end{cases} \equi $

$\begin{cases}
x \equiv 1\quad[11]\\\text{ou}\\x \equiv 8\:\:[11]
\end{cases}$

$x(x + 13) \equiv 3\quad [33] \iff  \left\{\begin{array}{l c l}
(x + 23)^2 &\equiv&1 \:\: [3]\\
(x + 23)^2 &\equiv& 4 \:\: [11]
\end{array}\right.$ est donc équivalent aux quatre systèmes donnés 
\end{solution}

		\item %On admet que chacun de ces systèmes admet une unique solution entière $x$ telle que

%$0 \leqslant x < 33$.

%Déterminer, sans justification, chacune de ces solutions.

\begin{solution}
Avec $x$ entier et $0 \leqslant x < 33$\: $\left\{\begin{array}{l c l}
x &\equiv&2\:\: [3]\\
x&\equiv &8\:\:[11]
\end{array}\right. \equi x=8$.

$\left\{\begin{array}{l c l}
 x &\equiv& 0\:\:[3]\\
 x &\equiv& 1 \:\:[11]
 \end{array}\right.\equi $
$x=12$
 
$\left\{\begin{array}{l c l}
x  &\equiv& 2\:\:[3]\\
x &\equiv&1 \:\:[11]
\end{array}\right.\equi x=23$\qquad,\qquad
$\left\{\begin{array}{l c l}
x &\equiv& 0\:\: [3]\\
x &\equiv& 8 \:\: [11]
\end{array}\right.\equi x=30$.
\end{solution}

	\end{enumerate}
\item %Compléter l'algorithme en \textbf{Annexe} pour qu'il affiche les quatre solutions trouvées dans laquestion précédente.

\begin{solution}

\begin{center}
\begin{tabularx}{0.7\linewidth}{|X|}\hline
Pour \surj{$x$} allant de \surj{$0$} à \surj{$32$}\\
\quad Si le reste de la division de \surj{$x(x+13$)} par \surj{$33$} est égal à \surj{$3$} alors\\
\qquad Afficher \surj{$x$}\\
\quad Fin Si\\
Fin Pour\\ \hline
\end{tabularx}
\end{center}
\end{solution}

\item %Alice peut-elle connaître la première lettre du message envoyé par Bob ? 
	
%Le \og chiffre de RABIN \fg{} est-il utilisable pour décoder un message lettre par lettre ?

\begin{solution}
La première lettre du message de Bob a été codée par 3, d'après ce qui précède cela signifie que cette première lettre peut être I, M, X.

Il est donc impossible pour Alice d'utiliser le \og chiffre de RABIN \fg{} pour décoder un message lettre par lettre.
\end{solution}
\end{enumerate}
\end{document}