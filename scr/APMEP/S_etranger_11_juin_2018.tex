%!TEX encoding = UTF-8 Unicode
\documentclass[10pt]{article}
\usepackage[T1]{fontenc}
\usepackage[utf8]{inputenc}
\usepackage{fourier}
\usepackage[scaled=0.875]{helvet}
\renewcommand{\ttdefault}{lmtt}
\usepackage{amsmath,amssymb}
\usepackage{fancybox}
\usepackage[normalem]{ulem}
\usepackage{pifont}
\usepackage{lscape}
\usepackage{diagbox}
\usepackage{eucal}
\usepackage{tabularx}
\usepackage{multirow}
\usepackage{mathrsfs}
\usepackage{textcomp} 
\usepackage{enumerate}
\usepackage{enumitem} 
\usepackage{colortbl} 
\usepackage{variations}
\newcommand{\euro}{\eurologo{}}
%Tapuscrit : Denis Vergès
\usepackage{pst-plot,pst-tree,pstricks,pst-node,pst-text}
\usepackage{pst-eucl}
\usepackage{pstricks-add}
\newcommand{\R}{\mathbb{R}}
\newcommand{\N}{\mathbb{N}}
\newcommand{\D}{\mathbb{D}}
\newcommand{\Z}{\mathbb{Z}}
\newcommand{\Q}{\mathbb{Q}}
\newcommand{\C}{\mathbb{C}}
\usepackage[left=3.5cm, right=3.5cm, top=3cm, bottom=3cm]{geometry}
\newcommand{\vect}[1]{\overrightarrow{\,\mathstrut#1\,}}
\newcommand{\barre}[1]{\overline{\,\mathstrut#1\,}}
\renewcommand{\theenumi}{\textbf{\arabic{enumi}}}
\renewcommand{\labelenumi}{\textbf{\theenumi.}}
\renewcommand{\theenumii}{\textbf{\alph{enumii}}}
\renewcommand{\labelenumii}{\textbf{\theenumii.}}
\renewcommand{\thesubsection}{Exercice \Roman{subsection}} 
\def\Oij{$\left(\text{O}~;~\vect{\imath},~\vect{\jmath}\right)$}
\def\Oijk{$\left(\text{O}~;~\vect{\imath},~\vect{\jmath},~\vect{k}\right)$}
\def\Ouv{$\left(\text{O}~;~\vect{u},~\vect{v}\right)$}
\everymath{\displaystyle}
\usepackage{fancyhdr}
\usepackage[dvips]{hyperref}
\hypersetup{%
pdfauthor = {APMEP},
pdfsubject = {Baccalauréat S},
pdftitle = {Centres étrangers - 11 juin 2018},
allbordercolors = white,
pdfstartview=FitH} 
\usepackage[frenchb]{babel}
\usepackage[np]{numprint}
\begin{document}
\setlength\parindent{0mm}
\marginpar{\rotatebox{90}{\textbf{A. P{}. M. E. P{}.}}}
\rhead{\textbf{A. P{}. M. E. P{}.}}
\lhead{\small Baccalauréat S }
\lfoot{\small{Centres étrangers}}
\rfoot{\small{11 juin 2018}}
\pagestyle{fancy}
\thispagestyle{empty} 

\begin{center} {\Large{\textbf{\decofourleft~Baccalauréat S  Centres étrangers 11 juin 2018~\decofourright}}}
\end{center}

\subsection{\hfill 4 points}
 
\textbf{Pour tous les candidats}

\medskip

Dans une usine, on se propose de tester un prototype de hotte aspirante pour un local industriel.

Avant de lancer la fabrication en série, on réalise l'expérience suivante : dans un local clos équipé
du prototype de hotte aspirante, on diffuse du dioxyde de carbone (CO$_2$) à débit constant.

Dans ce qui suit, $t$ est le temps exprimé en minute.

À l'instant $t = 0$, la hotte est mise en marche et on la laisse fonctionner pendant $20$ minutes. Les
mesures réalisées permettent de modéliser le taux (en pourcentage) de CO$_2$ contenu dans le local au
bout de $t$ minutes de fonctionnement de la hotte par l'expression $f(t)$, où $f$ est la fonction définie
pour tout réel $t$ de l'intervalle [0~;~20] par :

\[f(t) = (0,8t + 0,2)\text{e}^{-0,5t} + 0,03.\]

\smallskip

\parbox{0.57\linewidth}{On donne ci-contre le tableau de variation de la fonction $f$ sur l'intervalle [0~;~20].

Ainsi, la valeur $f(0) = 0,23$ traduit le fait que le taux 
de CO$_2$ à l'instant $0$ est égal à 23\,\%.}\hfill
\parbox{0.41\linewidth}{\psset{unit=1.2cm}
\begin{pspicture}(5,2.5)
\psframe(5,2.5)\psline(0,1.5)(5,1.5)\psline(0,2)(5,2)\psline(1,0)(1,2.5)
\uput[u](0.5,1.9){$t$}\uput[u](1.1,1.9){$0$}\uput[u](3,1.9){$1,75$}\uput[u](4.8,1.9){$20$}
\rput(0.5,1.75){$f'(t)$}\rput(2,1.75){$+$}\rput(3,1.75){$0$}\rput(4,1.75){$-$}
\rput(0.5,0.75){$f$}\uput[u](1.3,0){\small $0,23$}
\psline{->}(1.5,0.5)(2.5,1.25)\psline{->}(3.5,1.25)(4.5,0.5)
\end{pspicture}}

\bigskip

\begin{enumerate}
\item Dans cette question, on arrondira les deux résultats au millième.
	\begin{enumerate}
		\item Calculer $f (20)$.
		\item Déterminer le taux maximal de CO$_2$ présent dans le local pendant l'expérience.
 	\end{enumerate}
\item  On souhaite que le taux de CO$_2$ dans le local retrouve une valeur $V$ inférieure ou égale à $3,5$\,\%.
	\begin{enumerate}
		\item Justifier qu'il existe un unique instant $T$ satisfaisant cette condition.
		\item  On considère l'algorithme suivant :
		
\begin{center}
\begin{tabularx}{0.5\linewidth}{|X|}\hline
$t \gets 1,75$\\
$p \gets 0,1$\\
$V \gets 0,7$\\
Tant que $V > 0,035$\\
\hspace{0.75cm}$t \gets t + p$\\
\hspace{0.75cm}$V \gets (0,8t + 0,2)\text{e}^{-0,5t} + 0,03$\\
Fin Tant que\\ \hline
\end{tabularx}
\end{center}		
		
Quelle est la valeur de la variable $t$ à la fin de l'algorithme ?
		
Que représente cette valeur dans le contexte de l'exercice ?
 	\end{enumerate}
\item  On désigne par $V_m$ le taux moyen ( en pourcentage) de CO$_2$ présent dans le local pendant les $11$
premières minutes de fonctionnement de la hotte aspirante.
	\begin{enumerate}
		\item Soit $F$ la fonction définie sur l'intervalle [0~;~11] par : 
		
		\[F(t) = (-1,6t -3,6)\text{e}^{-0,5t} + 0,03t.\]
		
Montrer que la fonction $F$ est une primitive de la fonction $f$ sur l'intervalle [0~;~11].\index{primitive}
		\item En déduire le taux moyen $V_m$, valeur moyenne de la fonction $f$ sur l'intervalle [0~;~11].
Arrondir le résultat au millième, soit à $0,1$\,\%.
	\end{enumerate}
\end{enumerate}

\subsection{\hfill 4 points}
 
\textbf{Pour tous les candidats}

\medskip

Pour chacune des quatre affirmations suivantes, indiquer si elle est vraie ou fausse, en justifiant la
réponse. Il est attribué un point par réponse exacte correctement justifiée. Une réponse inexacte ou
non justifiée ne rapporte ni n'enlève aucun point.

\bigskip

\begin{enumerate}
\item Un type d'oscilloscope a une durée de vie, exprimée en année, qui peut être modélisée par une
variable aléatoire $D$ qui suit une loi exponentielle de paramètre $\lambda$.

On sait que la durée de vie moyenne de ce type d'oscilloscope est de $8$ ans.

\smallskip

\textbf{Affirmation 1 :} pour un oscilloscope de ce type choisi au hasard et ayant déjà fonctionné $3$ ans,
la probabilité que la durée de vie soit supérieure ou égale à $10$ ans, arrondie au centième, est
égale à $0,42$.

\emph{On rappelle que si $X$ est une variable aléatoire qui suit une loi exponentielle de paramètre $\lambda$, on a pour tout réel $t$ positif :} $P(X \leqslant t) = 1 - \text{e}^{-\lambda t}$.
\item  En 2016, en France, les forces de l'ordre ont réalisé $9,8$ millions de dépistages d'alcoolémie
auprès des automobilistes, et 3,1\,\% de ces dépistages étaient positifs.

Source : \emph{OFDT (Observatoire Français des Drogues et des Toxicomanies)}

Dans une région donnée, le 15 juin 2016, une brigade de gendarmerie a effectué un dépistage
sur $200$ automobilistes.

\smallskip

\textbf{Affirmation 2 :} en arrondissant au centième, la probabilité que, sur les $200$ dépistages, il y ait
eu strictement plus de $5$ dépistages positifs, est égale à $0,59$.
\item  On considère dans $\R$ l'équation :

\[\ln (6 x - 2) + \ln (2x - 1) = \ln (x).\]

\smallskip

\textbf{Affirmation 3 :} l'équation admet deux solutions dans l'intervalle $\left]\dfrac{1}{2}~;~+ \infty\right[$.
\item  On considère dans $\C$ l'équation : 

\[\left(4z^2 - 20z + 37\right)(2z -7 + 2\text{i}) = 0.\]


\smallskip

\textbf{Affirmation 4 :} les solutions de l'équation sont les affixes de points appartenant à un même
cercle de centre le point P d'affixe $2$.
\end{enumerate}

\subsection{\hfill 7 points}
 
\textbf{Pour tous les candidats}

\medskip

\emph{Les parties} A \emph{et} B \emph{sont indépendantes}

\medskip

Un détaillant en fruits et légumes étudie l'évolution de ses ventes de melons afin de pouvoir
anticiper ses commandes.

\bigskip

\textbf{Partie A}

\medskip

Le détaillant constate que ses melons se vendent bien lorsque leur masse est comprise entre $900$ g et
\np{1200}~g. Dans la suite, de tels melons sont qualifiés \og conformes \fg.

Le détaillant achète ses melons auprès de trois maraîchers, notés respectivement A, B et C.

Pour les melons du maraîcher A, on modélise la masse en gramme par une variable aléatoire $M_{\text{A}}$
qui suit une loi uniforme sur l'intervalle $[850~;~x]$, où $x$ est un nombre réel supérieur à \np{1200}.

La masse en gramme des melons du maraîcher B est modélisée par une variable aléatoire $M_{\text{B}}$ qui
suit une loi normale de moyenne \np{1050} et d'écart-type inconnu $\sigma$.

Le maraîcher C affirme, quant à lui, que 80\,\% des melons de sa production sont conformes.

\medskip

\begin{enumerate}
\item Le détaillant constate que 75\,\% des melons du maraîcher A sont conformes. Déterminer $x$.
\item Il constate que 85\,\% des melons fournis par le maraîcher B sont conformes.

Déterminer l'écart-type $\sigma$ de la variable aléatoire $M_{\text{B}}$. En donner la valeur arrondie à l'unité.
\item  Le détaillant doute de l'affirmation du maraîcher C. Il constate que sur $400$ melons livrés par ce
maraîcher au cours d'une semaine, seulement $294$ sont conformes.

Le détaillant a-t-il raison de douter de l'affirmation du maraîcher C ?
\end{enumerate}

\bigskip

\textbf{Partie B}

\medskip

Le détaillant réalise une étude sur ses clients. Il constate que:

\begin{itemize}
\item parmi les clients qui achètent un melon une semaine donnée, 90\,\% d'entre eux achètent un
melon la semaine suivante;
\item parmi les clients qui n'achètent pas de melon une semaine donnée, 60\,\% d'entre eux n'achètent
pas de melon la semaine suivante.
\end{itemize}

\smallskip

On choisit au hasard un client ayant acheté un melon au cours de la semaine 1 et, pour $n \geqslant 1$, on
note $A_n$ l'évènement : \og le client achète un melon au cours de la semaine $n$ \fg.

On a ainsi $p\left(A_1\right) = 1$.

\medskip

\parbox{0.6\linewidth}{\begin{enumerate}
\item 
	\begin{enumerate}
		\item Reproduire et compléter l'arbre de probabilités
ci-contre, relatif aux trois premières semaines.
		\item Démontrer que $p\left(A_3\right) = 0,85$.
		\item Sachant que le client achète un melon au cours
de la semaine 3, quelle est la probabilité qu'il en ait acheté un au cours de la semaine 2 ?
		
Arrondir au centième.
	\end{enumerate}
\end{enumerate}
}\hfill 
\parbox{0.31\linewidth}{\pstree[treemode=R,nodesepA=0pt,nodesepB=3pt]{\TR{$A_1$~}}
{
   \pstree{\TR{$A_2$~}}
      {
      \TR{$A_3$} 
      \TR{$\overline{A_3}$}
      }

   \pstree{\TR{$\overline{A_2}$~} }
     {
     \TR{$A_3$} 
     \TR{$\overline{A_3}$}
     }
}
}
\medskip
	
Dans la suite, on pose pour tout entier $n \geqslant 1$ : \:$p_n = P\left(A_n\right)$. On a ainsi $p_1 = 1$.

\medskip

\begin{enumerate}[resume,start=2]
\item Démontrer que, pour tout entier $n \geqslant 1$ : $p_{n+1} = 0,5p_n + 0,4$.
\item 
	\begin{enumerate}
		\item Démontrer par récurrence que, pour tout entier $n \geqslant 1$ : $p_n > 0,8$.
		\item Démontrer que la suite $\left(p_n\right)$ est décroissante.
		\item La suite $\left(p_n\right)$ est-elle convergente ?
 	\end{enumerate}
\item On pose pour tout entier $n \geqslant 1$ : $v_n = p_n - 0,8$.
	\begin{enumerate}
		\item Démontrer que $\left(v_n\right)$ est une suite géométrique dont on donnera le premier terme $v_1$ et la raison.
		\item  Exprimer $v_n$ en fonction de $n$.
		
En déduire que, pour tout $n \geqslant 1$,\: $p_n = 0,8 + 0,2 \times  0,5^{n-1}$.
		\item  Déterminer la limite de la suite $\left(p_n\right)$.
	\end{enumerate}
\end{enumerate}

\subsection{\hfill 5 points}
 
\textbf{Candidats n'ayant pas suivi la spécialité mathématique}

\medskip

\parbox{0.54\linewidth}{La figure ci-contre représente un cube ABCDEFGH.

Les trois points I, J, K sont définis par les conditions
suivantes :

\begin{itemize}
\item I est le milieu du segment [AD] ;
\item J est tel que $\vect{\text{AJ}} = \dfrac{3}{4} \vect{\text{AE}}$ ;
\item K est le milieu du segment [FG].
\end{itemize}}
\hfill
\parbox{0.44\linewidth}{
\psset{unit=0.75cm}
\begin{pspicture}(-0.5,-0.5)(8,7.8)
\psframe(0,0)(4.5,4.5)%ABFE
\psline(4.5,0)(6.7,2.3)(6.7,6.8)(4.5,4.5)%BCGF
\psline(6.7,6.8)(2.2,6.8)(0,4.5)%GHE
\psline[linestyle=dashed](0,0)(2.2,2.3)(6.7,2.3)
\psline[linestyle=dashed](2.2,2.3)(2.2,6.8)
\uput[dl](0,0){A} \uput[dr](4.5,0){B} \uput[r](6.7,2.3){C} 
\uput[ur](2.2,2.3){D} \uput[l](0,4.5){E} \uput[r](4.5,4.5){F} 
\uput[r](6.7,6.8){G} \uput[u](2.2,6.8){H} \uput[ul](1.1,1.15){I} 
\uput[l](0,3.375){J} \uput[dr](5.6,5.65){K}
\psdots(1.1,1.15)(0,3.375)(5.6,5.65) 
\end{pspicture}
}

\bigskip

\textbf{Partie A}

\medskip

\begin{enumerate}
\item Sur la figure donnée en annexe, construire sans justifier le point d'intersection P du plan (IJK) et
de la droite (EH). On laissera les traits de construction sur la figure.
\item  En déduire, en justifiant, l'intersection du plan (IJK) et du plan (EFG).
\end{enumerate}
 
\bigskip

\textbf{Partie B}

\medskip

On se place désormais dans le repère orthonormé $\left(\text{A}~;~\vect{\text{AB}}, \vect{\text{AD}}, \vect{\text{AE}}\right)$.

\medskip

\begin{enumerate}
\item 
	\begin{enumerate}
		\item Donner sans justification les coordonnées des points I, J et K.
		\item Déterminer les réels $a$ et $b$ tels que le vecteur $\vect{n} (4~;~a~;~b)$ soit orthogonal aux vecteurs $\vect{\text{IJ}}$ et $\vect{\text{IK}}$.
		\item  En déduire qu'une équation cartésienne du plan (IJK) est : $4x - 6y - 4z + 3 = 0$.
	\end{enumerate}
\item 
	\begin{enumerate}
		\item Donner une représentation paramétrique de la droite (CG).
		\item Calculer les coordonnées du point N, intersection du plan (IJK) et de la droite (CG).
		\item Placer le point N sur la figure et construire en couleur la section du cube par le plan (IJK).
	\end{enumerate}
\end{enumerate}

\bigskip

\textbf{Partie C}

\medskip

On note R le projeté orthogonal du point F sur le plan (IJK). Le point R est donc l'unique point du
plan (IJK) tel que la droite (FR) est orthogonale au plan (IJK).


On définit l'intérieur du cube comme l'ensemble des points $M(x~;~y~;~z)$ tels que $\left\{\begin{array}{l}
0 < x < 1\\
0 < y < 1\\
0 < z < 1
\end{array}\right.$

Le point R est-il à l'intérieur du cube ?

\subsection{\hfill 5 points}
 
\textbf{Candidats ayant suivi la spécialité mathématique}

\medskip

Le but de cet exercice est d'envisager une méthode de cryptage à clé publique d'une information
numérique, appelée système RSA, en l'honneur des mathématiciens Ronald Rivest, Adi Shamir et
Leonard Adleman, qui ont inventé cette méthode de cryptage en 1977 et l'ont publiée en 1978.

\smallskip

Les questions 1 et 2 sont des questions préparatoires, la question 3 aborde le cryptage, la question 4
le décryptage.

\bigskip

\begin{enumerate}
\item Cette question envisage de calculer le reste dans la division euclidienne par $55$ de certaines
puissances de l'entier $8$.
	\begin{enumerate}
		\item Vérifier que $8^7 \equiv 2 \mod 55$.
		
En déduire le reste dans la division euclidienne par $55$ du nombre $8^{21}$.
		\item Vérifier que $8^2 \equiv 9 \mod 55$, puis déduire de la question a. le reste dans la division
euclidienne par $55$ de $8^{23}$.
 	\end{enumerate}
\item  Dans cette question, on considère l'équation $(E)$\: $23 x - 40 y = 1$, dont les solutions sont des
couples $(x~;~y)$ d'entiers relatifs.
	\begin{enumerate}
		\item Justifier le fait que l'équation $(E)$ admet au moins un couple solution.
		\item  Donner un couple, solution particulière de l'équation $(E)$.
		\item  Déterminer tous les couples d'entiers relatifs solutions de l'équation $(E)$.
		\item  En déduire qu'il existe un unique entier $d$ vérifiant les conditions $0 \leqslant d < 40$ et
$23 d \equiv  1 \mod 40$.
 	\end{enumerate}
\item  Cryptage dans le système RSA
	
Une personne A choisit deux nombres premiers $p$ et $q$, puis calcule les produits $N = p q$ et
$n = (p - 1)(q - 1)$. Elle choisit également un entier naturel $c$ premier avec $n$.
	
La personne A publie le couple $(N~;~c)$, qui est une clé publique permettant à quiconque de lui
envoyer un nombre crypté.
	
Les messages sont numérisés et transformés en une suite d'entiers compris entre $0$ et $N -1$.
	
Pour crypter un entier $a$ de cette suite, on procède ainsi : on calcule le reste $b$ dans la division
euclidienne par $N$ du nombre $a^c$, et le nombre crypté est l'entier $b$.

\smallskip

Dans la pratique, cette méthode est sûre si la personne A choisit des nombres premiers $p$ et $q$
très grands, s'écrivant avec plusieurs dizaines de chiffres.

On va l'envisager ici avec des nombres plus simples : $p = 5$ et $q = 11$.

La personne A choisit également $c = 23$.
	\begin{enumerate}
		\item Calculer les nombres $N$ et $n$, puis justifier que la valeur de $c$ vérifie la condition voulue.
		\item  Un émetteur souhaite envoyer à la personne A le nombre $a = 8$.
		
Déterminer la valeur du nombre crypté $b$.
	\end{enumerate}
\item  Décryptage dans le système RSA

La personne A calcule dans un premier temps l'unique entier naturel $d$ vérifiant les conditions
$0 \leqslant d < n$ et $cd \equiv 1 \mod n$.

Elle garde secret ce nombre $d$ qui lui permet, et à elle seule, de
décrypter les nombres qui lui ont été envoyés cryptés avec sa clé publique.

Pour décrypter un nombre crypté $b$, la personne A calcule le reste $a$ dans la division euclidienne
par $N$ du nombre $b^d$, et le nombre en clair -- c'est-à-dire le nombre avant cryptage -- est le
nombre $a$.

On admet l'existence et l'unicité de l'entier $d$, et le fait que le décryptage fonctionne.

Les nombres choisis par A sont encore $p = 5$, $q = 11$ et $c = 23$.
	\begin{enumerate}
		\item Quelle est la valeur de $d$ ?
		\item En appliquant la règle de décryptage, retrouver le nombre en clair lorsque le nombre crypté
est $b = 17$.
	\end{enumerate}
\end{enumerate}

\newpage

\begin{center}
\textbf{\large Annexe (à rendre avec la copie)}

\vspace{3cm}

\psset{unit=1.5cm}
\begin{pspicture}(-0.5,-0.5)(8,7.8)
\psframe(0,0)(4.5,4.5)%ABFE
\psline(4.5,0)(6.7,2.3)(6.7,6.8)(4.5,4.5)%BCGF
\psline(6.7,6.8)(2.2,6.8)(0,4.5)%GHE
\psline[linestyle=dashed](0,0)(2.2,2.3)(6.7,2.3)
\psline[linestyle=dashed](2.2,2.3)(2.2,6.8)
\uput[dl](0,0){A} \uput[dr](4.5,0){B} \uput[r](6.7,2.3){C} 
\uput[ur](2.2,2.3){D} \uput[l](0,4.5){E} \uput[r](4.5,4.5){F} 
\uput[r](6.7,6.8){G} \uput[u](2.2,6.8){H} \uput[ul](1.1,1.15){I} 
\uput[l](0,3.375){J} \uput[dr](5.6,5.65){K}
\psdots(1.1,1.15)(0,3.375)(5.6,5.65) 
\end{pspicture}
\end{center}
\end{document}
