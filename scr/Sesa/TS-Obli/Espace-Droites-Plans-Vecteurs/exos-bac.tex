\documentclass{cornouaille}
\begin{document}

\fexo{}{Exercices Bac}{}




  \begin{exercice}[D'après Bac Asie -- juin 2013 et Amérique du Sud - novembre 2012]

    Vrai faux

    \emph{Pour chacune des affirmations suivantes, dire si elle est
      vraie ou fausse, et proposer une démonstration de la réponse
      indiquée. Une réponse non justifiée n'est pas prise en compte.}

    \begin{enumerate}
    \item Dans les deux questions suivantes, l'espace est muni d'un
      repère orthonormé
      $(O;\overrightarrow{i},\overrightarrow{j},\overrightarrow{k})$.
    \item Soit S le point de coordonnées $(1~;~3~;~5)$ et $\Delta_{1}$
      la droite de représentation paramétrique
      \[\left\{\begin{array}{l c l}
                 x&=&1 + t\\ 
                 y&=&5 - 4t\\ 
                 z&=&2 - 2t
               \end{array}\right.,\:t\in\mathbb{R}\] 
             \textbf{Affirmation 1 }: la droite $\Delta_{2}$ de
             représentation\\ paramétrique
             \[\left\{\begin{array}{l c l}
                        x&=&-t\\ 
                        y&=&7 + 4t\\ 
                        z&=&7 + 2t 
                      \end{array}\right.,\:t\in\mathbb{R}\]  
                    est la droite parallèle à la droite $\Delta_{1}$
                    passant par le point S.

                  \item On considère les points I(1~;~0~;~0),
                    J(0~;~1~;~0) et K(0~;~0~;~1).

                    \textbf{Affirmation 2}: la droite $\Delta$ de
                    représentation\\
                    paramétrique \[\left\{\begin{array}{l c l}
                                            x &=& 2 - t \\
                                            y &=& 6 - 2 t\\
                                            z &=&-2 + t
                                          \end{array}\right.,\:t\in\mathbb{R}\] coupe le plan  (IJK) au point E$\left(- \dfrac{1}{2}~;~1~;~\dfrac{1}{2} \right)$.

                                      \item Dans le cube ABCDEFGH, le
                                        point T est le milieu du
                                        segment [HF].


                                        \begin{center}
                                          \begin{tikzpicture}[general]

\coordinate (A) at (0,0,2);
\coordinate (B) at (2,0,2);
\coordinate (F) at (2,2,2);
\coordinate (E) at (0,2,2);
\coordinate (D) at (0,0,0);
\coordinate (C) at (2,0,0);
\coordinate (G) at (2,2,0);
\coordinate (H) at (0,2,0);

\draw [opacity=0] (B)--(G);
\draw [dashed,J2] (A)--(D)--(C);
\draw [dashed,J2] (D)--(H);
\draw (E)--(F)--(G)--(H)--(E); % face arrière
\draw (A)--(B)--(C); % face avant
% arêtes horizontales, de l’arrière vers l’avant
\draw (B) -- (A); % bas gauche
\draw (C) -- (G); % bas droit
\draw (F) -- (B); % haut droit
\draw (A) -- (E); % haut gauche
\draw (H)--(F) node [midway,right,yshift=1mm,B2] {$T$}node [midway,B2] {$+$};

\node at (A) [left] {$A$};
\node at (B) [right] {$B$};
\node at (C) [right] {$C$};
\node at (D) [above right] {$D$};
\node at (E) [left] {$E$};
\node at (F) [right] {$F$};
\node at (G) [above] {$G$};
\node at (H) [above] {$H$};

\end{tikzpicture}
                                        \end{center}
                                        \textbf{Affirmation 3} : les
                                        droites (AT) et (EC) sont
                                        orthogonales

                                      \end{enumerate}
                                    \end{exercice}

%

\begin{exercice}[D'après Bac Polynésie -- juin 2015]
  On considère le pavé droit $ABCDEFGH$ ci-dessous, pour lequel
  $AB = 6$, $AD = 4$ et $AE = 2$.\\
  $I$, $J$ et $K$ sont les points tels que :\\
  $\overrightarrow{AI} = \dfrac{1}{6} \overrightarrow{AB}$,
  $ \overrightarrow{AJ} = \dfrac{1}{4} \overrightarrow{AD}$ et
  $ \overrightarrow{AK} = \dfrac{1}{2} \overrightarrow {AE}$.

  \begin{center}
    \begin{tikzpicture}[general]

\coordinate (A) at (0,0,2);
\coordinate (B) at (2,0,2);
\coordinate (F) at (2,1,2);
\coordinate (E) at (0,1,2);
\coordinate (D) at (0,0,0);
\coordinate (C) at (2,0,0);
\coordinate (G) at (2,1,0);
\coordinate (H) at (0,1,0);

\draw [opacity=0] (B)--(G);
\draw [dashed,J2] (A)--(D)--(C);
\draw [dashed,J2] (D)--(H);
\draw (E)--(F)--(G)--(H)--(E); % face arrière
\draw (A)--(B)--(C); % face avant
% arêtes horizontales, de l’arrière vers l’avant
\draw (B) -- (A); % bas gauche
\draw (C) -- (G); % bas droit
\draw (F) -- (B); % haut droit
\draw (A) -- (E); % haut gauche

\node at (A) [left] {$A$};
\node at (B) [right] {$B$};
\node at (C) [right] {$C$};
\node at (D) [below right] {$D$};
\node at (E) [left] {$E$};
\node at (F) [right] {$F$};
\node at (G) [above] {$G$};
\node at (H) [above] {$H$};

\end{tikzpicture}
  \end{center}

On se place dans le repère orthonormé $(A;\vec{AI},\vec{AJ},\vec{AK})$.
\begin{enumerate}
\item Déterminer une représentation paramétrique du plan $(IJG)$.
\item Déterminer les coordonnées du point d'intersection L du plan
  $(IJG)$ et de la droite $(BF)$.
\item Reproduire la figure et tracer la section du pavé $ABCDEFGH$ par
  le plan $(IJG)$. On ne demande pas de justification.
\end{enumerate}
\end{exercice}

\begin{exercice}[D'après Bac Métropole -- juin 2015]
  Dans un repère orthonormé $(O,~I,~J,~K)$ d'unité 1~cm, on considère
  les points $A(0~;~-1~;~5)$, $B(2~;~-1~;~5)$, $C(11~;~0~;~1)$ et
  $D(11~;~4~;~4)$.

  Un point $M$ se déplace sur la droite $(AB)$ dans le sens de $A$ vers $B$
  à la vitesse de 1~cm par seconde.

  Un point $N$ se déplace sur la droite $(CD)$ dans le sens de $C$ vers $D$
  à la vitesse de 1~cm par seconde.

  À l'instant $t = 0$ le point $M$ est en $A$ et le point $N$ est en $C$.

  On note $M_t$ et $N_t$ les positions des points $M$ et $N$ au bout
  de $t$ secondes, $t$ désignant un nombre réel positif.

  On admet que $M_t$ et $N_t$, ont pour coordonnées :

  $M_t(t\,;\ -1\,;\ 5)$ et $N_t(11\,;\ 0,8t\,;\ 1 + 0,6 t)$.

  \emph{Les questions $1$ et $2$ sont indépendantes.}

\begin{enumerate}
\item
  \begin{enumerate}
  \item La droite $(AB)$ est parallèle à l'un des axes $(OI)$, $(OJ)$ ou
    $(OK)$. Lequel ?
  \item La droite $(CD)$ se trouve dans un plan $\mathcal{P}$ parallèle
    à l'un des plans $(OIJ)$, $(OIK)$ ou $(OJK)$.
		
    Lequel ? On donnera une représentation paramétrique de ce plan
    $\mathcal{P}$.
  \item Vérifier que la droite $(AB)$, coupe le plan $\mathcal{P}$ au
    point $E(11~;~-1~;~5)$.
  \item Les droites $(AB)$ et $(CD)$ sont-elles sécantes ?
  \end{enumerate}
\item
  \begin{enumerate}
  \item Montrer que $M_t^{}N_t ^2 = 2 t^2 - 25,2 t + 138$.
  \item À quel instant $t$ la longueur $M_tN_t$ est-elle minimale ?
  \end{enumerate}
\end{enumerate}
\end{exercice}

\begin{exercice}[D'après Bac Métropole -- juin 2014]
  Dans l'espace, on considère un tétraèdre $ABCD$ dont les faces
  $ABC$, $ACD$ et $ABD$ sont des triangles rectangles isocèles en
  $A$. On désigne par $E$, $F$ et $G$ les milieux respectifs des côtés
  $[AB]$, $[BC]$ et $[CA]$.

  On choisit $AB$ comme unité de longueur et on se place dans le
  repère orthonormé
  $(A;\overrightarrow{AB},\overrightarrow{AC},\overrightarrow{AD})$
  de l'espace.
  \begin{enumerate}
  \item Donner les coordonnées des points $D$ et $E$.
  \item Donner une représentation paramétrique de la droite $(DF)$.
  \item On désigne par $M$ un point de la droite $(DF)$ et par $t$ le
    réel tel que $\overrightarrow{DM}=t\overrightarrow{DF}$. On note
    $\alpha$ la mesure principale en radian de l'angle géométrique
    $\widehat{EMG}$.

    Le but de cette question est de déterminer la position du point
    $M$ pour que la mesure de $\alpha$ soit maximale.
    \begin{enumerate}
    \item Démontrer que
      $ME^2=\dfrac{3}{2}t^2-\dfrac{5}{2}t+\dfrac{5}{4}$.

      

    \item Démontrer que le triangle $MEG$ est isocèle en $M$.\\
      En déduire que :
      $ME \sin(\dfrac{\alpha}{2})=\dfrac{1}{2\sqrt{2}}$.
    \item Justifier que $\alpha$ est maximale si et seulement si
      $\sin(\dfrac{\alpha}{2})$ est maximal.

      En déduire que $\alpha$ est maximale si et seulement si $ME^2$
      est minimal.
    \item Conclure.
    \end{enumerate}
  \end{enumerate}
\end{exercice}

\begin{exercice}[D'après Bac Pondichéry -- 2013]

  \emph{Pour chacune des questions, plusieurs propositions de réponse
    sont données dont une seule est exacte. Pour chacune des questions
    indiquer, sans justification, la bonne réponse sur la copie. Une
    réponse exacte rapporte $1$ point.\\ Une réponse fausse ou l'absence
    de réponse ne rapporte ni n'enlève aucun point. Il en est de même
    dans le cas où plusieurs réponses sont données pour une même
    question.}

  \medskip
 
  L'espace est rapporté à un repère orthonormal. $t$ et $t'$ désignent
  des paramètres réels.

  Le plan (P) est le plan $ABC$ avec $A(0~;~1~;~-1)$, $B(-2~;~0~;~-1)$
  et $C(-3~;~1~;~0)$.

  Le plan (S) a pour représentation paramétrique
  $\left\{\begin{array}{l c l}
            x&=&- 2 + t + 2t'\\
            y&=&- t - 2t'\\
            z&=&- 1 - t + 3t'
          \end{array}\right.$  $t$ dans $\R$ $t'$ dans $\R$.

        La droite (D) a pour représentation paramétrique
        $\left\{\begin{array}{l c l}
                  x&=&- 2 + t\\
                  y&=&- t \\
                  z&=&- 1 - t
                \end{array}\right.$ $t$ dans $\R$.
 
              On donne les points de l'espace $M(-1~;~2~;~3)$ et
              $N(1~;~-2~;~9)$.

              \medskip
 
              \begin{enumerate}
              \item Une représentation paramétrique du plan (P) est (à
                chaque fois, $t$ dans $\R$, $t'$ dans $\R$) :

                \medskip {\footnotesize\begin{tabularx}{\linewidth}{XX}
                    \textbf{a.~}$\left\{\begin{array}{l c l} x&=&
                        t\\y&=& 1- 2t\\ z&=& -1 +
                        3t \end{array}\right.$&
                    \textbf{b.~}$\left\{\begin{array}{l c l} x&=& t +
                        2t'\\y&=& 1- t + t'\\z&=& - 1 -
                        t\end{array}\right.$\\
                    \textbf{c.~}$\left\{\begin{array}{l c l} x&=&t +
                        t'\\ y&=& 1 - t- 2t'\\z&=& 1 - t -
                        3t'\end{array}\right.$&
                    \textbf{d.~}$\left\{\begin{array}{l c l} x&=& 1 +
                        2t + t'\\y&=& 1 - 2t + 2t'\\z&=& - 1 -
                        t'\end{array}\right.$
                  \end{tabularx}\par} 

                \medskip

              \item
                \begin{enumerate}
		\item La droite (D) et le plan (P) sont sécants au
                  point K$(- 8~;~3~;~2)$.
		\item La droite (D) est une droite du plan (P).
		\item La droite (D) et le plan (P) sont strictement
                  parallèles.
                \end{enumerate}
              \item
                \begin{enumerate}
		\item La droite (MN) et la droite (D) sont non
                  coplanaires.
		\item La droite (MN) et la droite (D) sont parallèles.
		\item La droite (MN) et la droite (D) sont sécantes.
		\item La droite (MN) et la droite (D) sont confondues.
                \end{enumerate}
              \item
                \begin{enumerate}
		\item Les plans (P) et (S) sont parallèles.
		\item La droite $(\Delta)$ de représentation
                  paramétrique
                  $\left\{\begin{array}{l c l}x&=&t\\y&=&- 2 - t\\z
                      &=& -3-t \end{array}\right.$
                  est la droite d'intersection des plans (P) et (S).
		\item Le point M appartient à l'intersection des plans
                  (P) et (S).
		
                \end{enumerate}
              \end{enumerate}
            \end{exercice}





%%%%%%%%%%%%%%



%%%%%%%%%%%%%%




  \begin{exercice}[Vrai ou faux ?]
    Dire si les propositions suivantes sont vraies ou fausses et le
    démontrer.
    \begin{enumerate}
    \item Si deux plans sont perpendiculaires, alors toute droite de
      l'un est orthogonale à toute droite de l'autre.
    \item Si deux droites sont parallèles, alors toute droite\\
      orthogonale à l'une est orthogonale à l'autre.
    \item Si deux droites sont perpendiculaires à une même droite
      alors elles sont parallèles entre elles.
    \end{enumerate}
  \end{exercice}

  \begin{exercice}
    Soit $ABCD$ un tétraèdre régulier et $I$ le milieu de $[BC]$.

    \begin{center}
      \begin{tikzpicture}[general]

% \coordinate (A) at (0,0);
% \coordinate (B) at (1,-3);
% \coordinate (C) at (-1,-2.5);
% \coordinate (D) at (2,-2);
% \coordinate (S) at (.25,-2.5);

% \draw [dashed,B2] (A)--(S) node {$\times$} node [right] {$S$};
% \draw [dashed,B2] (C)--(D);
% \draw (A)--(C)--(B) node [B2,midway] {$\times$} node [B2,midway,below
% left] {$I$}--(D)--(A)--(B) ;

% \node at (A) [above] {$A$};
% \node at (B) [below] {$B$};
% \node at (C) [left] {$C$};
% \node at (D) [right] {$D$};

\end{tikzpicture}

\begin{tikzpicture}[general,scale=.5]
% \clip(-4.3,-2.88) rectangle (7.14,6.3);
\draw [dashed,color=B2] (0.5938169661678563,-0.4137735165887121)-- (0.58,4.74);
\draw (0.58,4.74)-- (-2.4,-0.06);
\draw (0.58,4.74)-- (2.06,-2.3);
\draw (0.58,4.74)-- (3.,2.);
\draw (-2.4,-0.06)-- (2.06,-2.3);
\draw (2.06,-2.3)-- (3.,2.);
\draw [dashed,color=B2] (3.,2.)-- (-2.4,-0.06);
\begin{scriptsize}
\draw [fill=Noir] (3.,2.) circle (0.5pt);
\draw[color=Noir] (3.14,2.2) node {$D$};
\draw [fill=Noir] (-2.4,-0.06) circle (0.5pt);
\draw[color=Noir] (-2.56,0.26) node {$C$};
\draw [fill=Noir] (2.06,-2.3) circle (0.5pt);
\draw[color=Noir] (2.08,-2.66) node {$B$};
\draw [color=B2] (-0.17,-1.18)-- ++(-2.0pt,0 pt) -- ++(4.0pt,0 pt) ++(-2.0pt,-2.0pt) -- ++(0 pt,4.0pt);
\draw[color=B2] (-0.48,-1.36) node {$I$};
\draw [color=B2] (0.5938169661678563,-0.4137735165887121)-- ++(-1.5pt,-1.5pt) -- ++(3.0pt,3.0pt) ++(-3.0pt,0) -- ++(3.0pt,-3.0pt);
\draw[color=B2] (0.74,-0.14) node {$S$};
\draw [fill=Noir] (0.58,4.74) circle (0.5pt);
\draw[color=Noir] (0.52,5.1) node {$A$};
\end{scriptsize}
\end{tikzpicture}
    \end{center}

    \begin{enumerate}
    \item Démontrer que la droite $(BC)$ est orthogonale au plan
      $(ADI)$.
    \item En déduire que $(BC)\perp(AD)$.

      \textit{Le plan $(ADI)$ est le plan orthogonal au segment $[BC]$
        et passant par son milieu. Il est appelé plan médiateur du
        segment $[BC]$. Tous les points de ce plan sont équidistants
        de $B$ et de $C$.}
    \end{enumerate}
  \end{exercice}

  \begin{exercice}
    $ABCDEFGH$ est un cube et $I$ et $J$ sont les milieux respectifs de
    $[BC]$ et $[EH]$.  Le point $M$ est un point du segment $[AG]$
    distinct de $A$ et de $G$.
    \begin{enumerate}
    \item Montrer qu'il existe un unique réel $t\in\left] 0;1\right[ $
      tel que $\overrightarrow{AM}=t\overrightarrow{AG}$.
    \item Construire la section du cube par le plan $(IAM)$. Quelle
      est sa nature ?
    \item Existe-t-il une valeur du réel $t$ tel que les points $J$,
      $M$ et $I$ soient alignés ? Justifier.
    \item Existe-t-il une valeur du réel $t$ tel que les points $H$,
      $M$ et $I$ soient alignés ? Justifier.
    \end{enumerate}
  \end{exercice}

  \begin{exercice}
    $ABCDEFGH$ est un cube de côté $a$. Le point $M$ est défini par
    $\overrightarrow{DM}=\dfrac{1}{3}\overrightarrow{DE}$.
    \begin{enumerate}
    \item Construire la section du cube par le plan $(AGM)$.
    \item Démontrer que cette section est un losange $ANGP$ avec $N$
      et $P$ les milieux respectifs de $[BF]$ et $[DH]$.
    \item En déduire l'aire de cette section en fonction de $a$.
    \end{enumerate}
  \end{exercice}

  \begin{exercice}
    $ABCDEFGH$ est un cube et $I$ ; $J$ et $K$ les points tels que :
    $I\in [AD]$ et $AI=\dfrac{1}{3}AD$ ; $J\in [AB]$ et
    $AJ=\dfrac{1}{3}AB$.

    \begin{center}
      \begin{tikzpicture}[general]

\coordinate (A) at (0,0,2);
\coordinate (B) at (2,0,2);
\coordinate (F) at (2,2,2);
\coordinate (E) at (0,2,2);
\coordinate (D) at (0,0,0);
\coordinate (C) at (2,0,0);
\coordinate (G) at (2,2,0);
\coordinate (H) at (0,2,0);

\draw [dashed,J2] (A)--(D) node (I) [near start,B2] {+} 
               node [near start,above,B2] {I}--(C);
\draw [dashed,J2] (D)--(H);
\draw (E)--(F)--(G)--(H)--(E); % face arrière
\draw (A)--(B) node (J) [near start,B2] {+} 
               node [near start,below,B2] {J} --(C); % face avant
% arêtes horizontales, de l’arrière vers l’avant
\draw (B) -- (A); % bas gauche
\draw (C) -- (G); % bas droit
\draw (F) -- (B); % haut droit
\draw (A) -- (E); % haut gauche

\draw [B2] (I)--(J);

\node at (A) [left] {$A$};
\node at (B) [right] {$B$};
\node at (C) [right] {$C$};
\node at (D) [above right] {$D$};
\node at (E) [left] {$E$};
\node at (F) [right] {$F$};
\node at (G) [above] {$G$};
\node at (H) [above] {$H$};

\end{tikzpicture}
    \end{center}

    Les propositions sont-elles vraies ou fausse ?

    Le démontrer.
    \begin{enumerate}
    \item $(IJ)$ est orthogonale à $(EC)$ ; \item $(IJ)$ est
      orthogonale à $(BG)$ ;
    \item $(IJ)$ est orthogonale à $(HB)$ ; \item $(IJ)$ est
      orthogonale à $(HC)$.
    \end{enumerate}
  \end{exercice}

  \begin{exercice}
    Dans l'espace muni d'un repère orthonormé
    $(O\,;\vec{i},\vec{j},\vec{k})$, on considère les points
    $A(1\,;0\,;0)$, $B(0\,;1\,;0)$, $C(0\,;0\,;1)$ et $I$ le milieu de
    $[AB]$.
    \begin{enumerate}
    \item Construire la figure à l'aide d'un logiciel de géométrie dans
      l'espace.
    \item Placer un point $M$ du segment $[AC]$ et $\wp$ le plan passant
      par $I$ et orthogonal à la droite $(IM)$.
    \item Construire le point $N$ intersection de $\wp$ et de la droite
      $(OB)$.
    \item Conjecturer la position du point $M$ pour laquelle la distance
      $MN$ est minimale.
    \item Démonstration
      \begin{enumerate}
      \item Soit $t$ le réel tel que
        $\overrightarrow{AM}=t\overrightarrow{AC}$. Exprimer les\\
        coordonnées de $M$ en fonction de $t$. On admet que $N(0\,;t\,;0)$.
      \item Exprimer la longueur $MN$ en fonction de $t$.
      \item Déterminer la valeur de $t$ pour laquelle cette longueur est
        minimale.
      \end{enumerate}
    \end{enumerate}
  \end{exercice}

  \begin{exercice}
    $ABCD$ est un tétraèdre. $P$, $Q$ et $R$ sont les points tels que
    $ABPC$, $ABQD$ et $ACRD$ sont des parallélogrammes.  En se plaçant
    dans le repère
    $(A\,;\overrightarrow{AB},\overrightarrow{AC},\overrightarrow{AD})$,
    démontrer que les droites $(BR)$, $(CQ)$ et $(DP)$ sont
    concourantes en un point $K$ dont on déterminera les coordonnées.
  \end{exercice}

  \begin{exercice}
    Une autre preuve du théorème du toit  :

    \begin{center}
      \begin{tikzpicture}[general]


 \begin{scope}
    \tkzDefPoint(2,.6){A}
    \tkzDefPoint(4,1){B} 
    \tkzDefPoint(3,3.8){C}
    \tkzDefPointWith[colinear= at C](B,A)
    \tkzGetPoint{D}
    \tkzDrawPolygon[color=black,fill=A3](A,B,C,D)
  \end{scope}
 \begin{scope}
    \tkzDefPoint(0,1.2){A}
    \tkzDefPoint(2,1.6){B} 
    \tkzDefPoint(3,3.8){C}
    \tkzDefPointWith[colinear= at C](B,A)
    \tkzGetPoint{D}
    \tkzDrawPolygon[color=black,fill=H3](A,B,C,D)
  \end{scope}

  \draw (-1,3)--(4,4) node [near start,above left] {$\Delta$};
  \draw (-1,1)--(4,2) node [near start,below left] {$d$};
  \draw (-1,0)--(4,1) node [midway,below left] {$d'$};

  \node at (.6,1.75) {$\wp$};
  \node at (2.2,1) {$\wp'$};
 \end{tikzpicture}
    \end{center}

    Soit $\wp$ et $\wp'$ deux plans sécants selon une droite $\Delta$,
    $d$ une droite de $\wp$ et $d'$ une droite de $\wp'$telles que
    $d//d'$.  \\ Il s'agit de démontrer que la droite $\Delta$
    intersection de $\wp$ et $\wp'$ est parallèle à $d$ et à $d'$.
    \begin{enumerate}
    \item Soit $\vec{u}$ un vecteur directeur de $d$ et $d'$ et
      $\vec{w}$ un vecteur directeur de $\Delta$.  On considère des
      vecteurs $\vec{v}$ et $\vec{v'}$ tels que : $\vec{u}$ et
      $\vec{v}$ dirigent $\wp$ et
      $\vec{u}$ et $\vec{v'}$ dirigent $\wp'$.\\
      Traduire vectoriellement le fait que $\Delta\subset\wp$ puis que
      $\Delta\subset\wp'$.
    \item En déduire une relation vectorielle entre $\vec{u}$,
      $\vec{v}$ et $\vec{v'}$.
    \item Conclure en utilisant le fait que $\wp$ et $\wp'$ sont
      sécants.
    \end{enumerate}
  \end{exercice}

  \begin{exercice}
    Dans l'espace muni d'un repère orthonormé
    $(O\,; \vec{i},\vec{j},\vec{k})$, soit $\Delta$ la droite de
    représentation paramétrique :

    \begin{center}
      $\begin{cases}x=3-t \\y=2+4t \\z=-2t \end{cases}$ ,
      $t\in\mathbb{R}$
    \end{center}

    \begin{enumerate}
    \item Les points $A(5\,;2\,;6)$ et $B(5\,;-6\,;4)$
      appartiennent-ils à la droite $\Delta$ ?
    \item Déterminer les valeurs des réels $a$ et $b$ tels que le
      point $C(4\,;a\,;b)$ appartienne à $\Delta$.
    \item Soit $M(x\,;y\,;z)\in \Delta$. Exprimer $AM^2$ en fonction
      de $t$.
    \item Déterminer les coordonnées du point $M$ tel que la distance
      $AM$ soit minimale.
    \end{enumerate}
  \end{exercice}

  \begin{exercice}
    Dans l'espace muni d'un repère orthonormé
    $(O\,;\vec{i},\vec{j},\vec{k})$, soit $\Delta$ la droite de
    représentation paramétrique :

    \begin{center}
      $\begin{cases}x=1-t \\y=2t \\z=-t+2 \end{cases}$ , $t\in\mathbb{R}$
    \end{center} 
    et $\wp$ le plan de représentation paramétrique :

    \begin{center}
      $\begin{cases}x=-t' \\y=1+t+3t' \\z=t \end{cases}$
      $t\in\mathbb{R}, t'\in\mathbb{R}$
    \end{center}
    \begin{enumerate}
    \item Le point $C(1\,; 3\,; 2)$ appartient-il au plan $\wp$ ?
    \item Démontrer que la droite $\Delta$ est incluse dans le plan
      $\wp$.
    \item Soit $\wp'$ le plan de représentation paramétrique :

      \begin{center}
        $\begin{cases}x=t+t' \\y=1+2t+t' \\z=-1+3t+t' \end{cases}$
        $t\in\mathbb{R}, t'\in\mathbb{R}$
      \end{center}

      \begin{enumerate}
      \item Montrer que $C\in \wp'$.
      \item Montrer que $\Delta$ coupe $\wp'$ en un point $I$ dont on\\
        déterminera les coordonnées.
      \item Montrer que $CI=\sqrt{3}$.
      \end{enumerate}
    \item Soit $t$ un nombre réel et $M$ le point de $\Delta$ de
      coordonnées :\\$M(-t+1\,;2t\,; -t+2)$.
      \begin{enumerate}
      \item Montrer que \\ $CM^2=6t^2-12t+9$.
      \item Montrer que $CI$ est la distance minimale de $CM$
        lorsque $t$ décrit l'ensemble des réels.
      \end{enumerate}
    \end{enumerate}
  \end{exercice}

  \begin{exercice}
    Soient $A(-2\
    ;0\,;1)$, $B(1\,;2\,;-1)$ et $C(-2\,;2\
    ;2)$ trois points de l'espace muni d'un repère $(O\
    ;\vec{i},\vec{j},\vec{k})$.
    \begin{enumerate}
    \item Vérifier que les points $A$, $B$ et $C$ définissent un plan.
    \item Soit $D(-2\
      ;-1\,;0)$ et $E(-2\,;5\,;2)$.  Démontrer que la droite
      $(DE)$
      et le plan $(ABC)$
      sont sécants en un point $I$
      dont on déterminera les coordonnées.
    \end{enumerate}
  \end{exercice}








\end{document}