\documentclass{cornouaille}
\dscornouaille
\begin{document}

%%%%%%%%%%%%%%
\fexo{TS}{Exercices de base}{}
%%%%%%%%%%%%%%




\section{Activit\'{e}s mentales}

Pour les exercices suivants ,
$ABCDEFGH$ est un pavé droit ; $I$, $J$, $K$ et $L$ sont les milieux
respectifs de $[DH], [HG], [AB]$ et $[BF]$.

\begin{center}
  \begin{tikzpicture}[general]

\coordinate (A) at (0,0,2);
\coordinate (B) at (2,0,2);
\coordinate (F) at (2,2,2);
\coordinate (E) at (0,2,2);
\coordinate (D) at (0,0,0);
\coordinate (C) at (2,0,0);
\coordinate (G) at (2,2,0);
\coordinate (H) at (0,2,0);

\draw [dashed,J2] (A)--(D)--(C);
\draw [dashed,J2] (D)--(H);
\draw (E)--(F)--(G)--(H) node [midway,B2] {+} node [midway,B2,above] {$J$} --(E) node [midway,B2] {+} node [midway,B2,above] {$I$}; % face arrière
\draw (A)--(B)--(C) node [midway,B2] {+} node [midway,B2,right] {$L$};
% arêtes horizontales, de l’arrière vers l’avant
\draw (B) -- (A) node [midway,B2] {+} node [midway,B2,below] {$K$} ; % bas gauche
\draw (C) -- (G); % bas droit
\draw (F) -- (B); % haut droit
\draw (A) -- (E); % haut gauche


\node at (A) [left] {$A$};
\node at (B) [right] {$B$};
\node at (C) [above right] {$F$};
\node at (D) [above right] {$E$};
\node at (E) [above left] {$D$};
\node at (F) [above left] {$C$};
\node at (G) [above right] {$G$};
\node at (H) [above] {$H$};

\end{tikzpicture}
\end{center}

\begin{exercice}
Donner la position relative des deux droites citées : 
\begin{enumerate}
\item $(DB)$ et $(EF)$ ;
\item $(IJ)$ et $(AF)$ ;
\item $(IC)$ et $(AB)$ ;
\item $(JF)$ et $(EH)$.
\end{enumerate}
\end{exercice}
\begin{solution}
  \begin{enumerate}
\item $(DB)$ et $(EF)$ non coplanaires ;
\item $(IJ)$ et $(AF)$ parallèles;
\item $(IC)$ et $(AB)$ non coplanaires ;
\item $(JF)$ et $(EH)$ sécantes.
\end{enumerate}
\end{solution}

\begin{exercice}
  Donner la position relative des deux plans cités :
  \begin{enumerate}
  \item $(DCG)$ et $(AEF)$ ;
  \item $(IJA)$ et $(HDC)$ ;
  \item $(IJE)$ et $(CKL)$.
  \end{enumerate}
\end{exercice}
\begin{solution}
  \begin{enumerate}
\item $(DCG)$ et $(AEF)$ parallèles ;
\item $(IJA)$ et $(HDC)$ sécants selon $(IJ)$;
\item $(IJE)$ et $(CKL)$ parallèles.
\end{enumerate}
\end{solution}

\begin{exercice}
  Donner la position relative de la droite et du plan cités :
  \begin{enumerate}
  \item $(IJ)$ et $(ABF)$ ;
  \item $(IJ)$ et $(BCG)$ ;
  \item $(KE)$ et $(ABF)$.
  \end{enumerate}
\end{exercice}
\begin{solution}
  \begin{enumerate}
  \item $(IJ)$ parallèle à $(ABF)$ ;
  \item $(IJ)$ et $(BCG)$ sécants ;
  \item $(KE)$ est incluse dans $(ABF)$.
  \end{enumerate}  
\end{solution}

\begin{exercice}
  $ABCDEFGH$ est un cube et $I$ est le milieu de $[AB]$.

  \begin{center}
    \begin{tikzpicture}[general]

\coordinate (A) at (0,0,2);
\coordinate (B) at (2,0,2);
\coordinate (F) at (2,2,2);
\coordinate (E) at (0,2,2);
\coordinate (D) at (0,0,0);
\coordinate (C) at (2,0,0);
\coordinate (G) at (2,2,0);
\coordinate (H) at (0,2,0);

\draw [dashed,J2] (A)--(D)--(C);
\draw [dashed,J2] (D)--(H);
\draw (E)--(F)--(G)--(H)--cycle;
\draw (A)--(B)--(C);
\draw (B) -- (A) node [midway,B2] {+} node [midway,B2,below] {$I$} ;
\draw [B2] (1,0,2)--(F);
\draw (C) -- (G);  
\draw (F) -- (B);  
\draw (A) -- (E); 


\node at (A) [left] {$A$};
\node at (B) [right] {$B$};
\node at (C) [above right] {$C$};
\node at (D) [above right] {$D$};
\node at (E) [above left] {$E$};
\node at (F) [above left] {$F$};
\node at (G) [above right] {$G$};
\node at (H) [above] {$H$};

\end{tikzpicture}
  \end{center}

  Quelle est la nature de la section du cube par :
  \begin{enumerate}
  \item le plan $(IFG)$ ?
  \item le plan $(IFC)$ ?
  \end{enumerate}
\end{exercice}
\begin{solution}
\begin{enumerate}
\item un rectangle
\item un triangle isocèle en $I$
% \item  $(IF)$ et $(FG)$ sont orthogonales.
% \item  $(IF)$ et $(FH)$ ne sont pas orthogonales
% \item  $(BF)$ et $(EH)$ sont orthogonales.
% \item  $(BF)$ et $(AC)$ sont orthogonales.
\end{enumerate}
\end{solution}



\begin{exercice}
  $ABCDEFGH$ est un cube et $I$ est le milieu de $[AB]$ .

  Les droites suivantes sont-elles orthogonales ?
  \begin{enumerate}
  \item $(IF)$ et $(FG)$ ?
  \item $(IF)$ et $(FH)$?
  \item $(BF)$ et $(EH)$?
  \item $(BF)$ et $(AC)$?
  \end{enumerate}
\end{exercice}
\begin{solution}
  \begin{enumerate}
\item  $(IF)$ et $(FG)$ sont orthogonales.
\item  $(IF)$ et $(FH)$ ne sont pas orthogonales
\item  $(BF)$ et $(EH)$ sont orthogonales.
\item  $(BF)$ et $(AC)$ sont orthogonales.
\end{enumerate}
\end{solution}

\begin{exercice}
  $ABCDEFGH$ est un cube et $I$ est le milieu de $[AB]$ .

  Compléter les égalités vectorielles suivantes :
  \begin{enumerate}
  \item
    $\overrightarrow{AI}+\overrightarrow{CD}-\overrightarrow{CI}=\overrightarrow{F...}$
  \item
    $\overrightarrow{AH}+\overrightarrow{CD}-\overrightarrow{FG}=\overrightarrow{B...}$
  \item
    $\overrightarrow{FD}+\overrightarrow{CB}+\overrightarrow{DG} =...$
  \end{enumerate}
\end{exercice}
\begin{solution}
  \begin{enumerate}
\item  $\overrightarrow{AI}+\overrightarrow{CD}-\overrightarrow{CI}=\overrightarrow{FG}$ ;
\item $\overrightarrow{AH}+\overrightarrow{CD}-\overrightarrow{FG}=\overrightarrow{BE}$ ;
\item $\overrightarrow{FD}+\overrightarrow{CB}+\overrightarrow{DG} =\overrightarrow{0}$.
 \end{enumerate}
\end{solution}

\begin{exercice}
  $ABCDEFGH$ est un cube et $I$ est le milieu de $[AB]$ (voir figure
  de l'exercice 4).

  \begin{enumerate}
  \item Exprimer le vecteur $\overrightarrow{FI}$ en fonction des
    vecteurs $\overrightarrow{AB}$ et $\overrightarrow{AE}$.
  \item $O$ étant le centre du cube, exprimer le vecteur
    $\overrightarrow{AO}$ en fonction des vecteurs
    $\overrightarrow{AB}$, $\overrightarrow{AD}$ et
    $\overrightarrow{AE}$.
  \end{enumerate}
\end{exercice}
\begin{solution}
  \begin{enumerate}
\item  $\overrightarrow{FI}=-\dfrac{1}{2}\overrightarrow{AB}-\overrightarrow{AE}$.
\item $\overrightarrow{AO}=\dfrac{1}{2}\overrightarrow{AB}+\dfrac{1}{2}\overrightarrow{AD}+\dfrac{1}{2}\overrightarrow{AE}$.
\end{enumerate}
\end{solution}

\begin{exercice}
  Dans un repère $(O\ ;\ \vec{i},\vec{j},\vec{k})$ de l'espace, on
  considère les points $A(-3\,;2\,;4)$ ; $B(-1\,;1\,;0)$ et $C(2\,;-3\,;5)$.
  \begin{enumerate}
  \item Donner les coordonnées des vecteurs $\overrightarrow{AB}$ ;
    $\overrightarrow{AC}$ et $\overrightarrow{BC}$.

  \item Donner les coordonnées des vecteurs :

    $\overrightarrow{u}=2\overrightarrow{AB}-\overrightarrow{AC}$ et
    $\overrightarrow{v}=\overrightarrow{AC}+3\overrightarrow{BC}$.
  \end{enumerate}
\end{exercice}
\begin{solution}
  \begin{enumerate}
\item  $\overrightarrow{AB}(2;-1;-4)$ ; $\overrightarrow{AC}(5;-5;1)$ et $\overrightarrow{BC}(3;-4;5)$.
\item 
$\overrightarrow{u}(-1;3;-9)$ et $\overrightarrow{v}(14;-17;16)$.
\end{enumerate}
\end{solution}

\begin{exercice}
  Dans un repère$(O\,;\vec{i},\vec{j},\vec{k})$ de l'espace, on
  considère les points $A(2\,;5\,;-1)$ \,; $B(0\,;3\,;4)$ et le vecteur
  $\overrightarrow{u}(2\,;-1\,;4)$.
  \begin{enumerate}
  \item Déterminer les coordonnées du point $C$ défini par
    $\overrightarrow{AC}=\overrightarrow{u}$
  \item Déterminer les coordonnées du vecteur $\overrightarrow{AB}$
    puis celles du point $D$ tel que $ABDC$ soit un parallélogramme.
  \item Déterminer les coordonnées du centre $K$ de ce
    parallélogramme.
  \end{enumerate}
\end{exercice}
\begin{solution}
  \begin{enumerate}
\item $C(4;4;3)$ ; 
\item  $\overrightarrow{AB}(-2;-2;5)$ et $D(2;2;8)$ ;
\item  $K(2;3,5;3,5)$.
 \end{enumerate}
\end{solution}
 
\begin{exercice}
  Dans un repère$(O\,;\vec{i},\vec{j},\vec{k})$ de l'espace, on considère
  les points $A(2\,;5\,;-1)$ \,; $B(2\,;-3\,;4)$ et le vecteur
  $\overrightarrow{u}(2\,;-1\,;4)$.
  \begin{enumerate}
  \item Déterminer une représentation paramétrique de la droite
    $\Delta$ passant par $A$ et de vecteur directeur
    $\overrightarrow{u}$.
  \item Le point $B$ appartient-il à $\Delta$ ?
  \end{enumerate}
\end{exercice}
\begin{solution}
  \begin{enumerate}
\item Représentation paramétrique de la droite $\Delta$  :
%\begin{center}
$\begin{cases}
  x=2+2t \\
  y=5-t \\
  z=-1+4t 
\end{cases}$, $t\in\mathbb{R}$
%\end{center}
\item Le point $B$ n'appartient pas à $\Delta$.
 \end{enumerate}
\end{solution}

\pagebreak

\begin{exercice}
  Dans un repère$(O\,;\vec{i},\vec{j},\vec{k})$ de l'espace, on considère
  la droite $\Delta$ de représentation paramétrique :

\begin{center}
  $\begin{cases}x=-3+4t \\y=2 \\z=-t \end{cases}$, $t\in\mathbb{R}$
\end{center}
Donner un vecteur directeur de $\Delta$ et un point de $\Delta$.
\end{exercice}
\begin{solution}
  $\overrightarrow{u}(4\ ;\ 0\ ;\ -1)$ dirige $\Delta$ et
  $A(-3\ ;\ 2\ ;\ 0)$ appartient à $\Delta$.
\end{solution}

\section{Étude de positions relatives}

Pour les exercices suivants,
$ABCDEFGH$ est un cube et $I, J$ et $K$ sont les milieux respectifs de
$[FG]$, $[AD]$ et $[DH]$.

\begin{center}
  \begin{tikzpicture}[general]

\coordinate (A) at (0,0,2);
\coordinate (B) at (2,0,2);
\coordinate (F) at (2,2,2);
\coordinate (E) at (0,2,2);
\coordinate (D) at (0,0,0);
\coordinate (C) at (2,0,0);
\coordinate (G) at (2,2,0);
\coordinate (H) at (0,2,0);

\draw [dashed,J2] (A)--(D)  node [midway,B2] {+} node [midway,B2,above] {$J$}--(C);
\draw [dashed,J2] (D)--(H) node [midway,B2] {+} node [midway,B2,left] {$K$};
\draw (E)--(F)--(G) node [midway,B2] {+} node [midway,B2,below] {$I$}--(H)--(E); % face arrière
\draw (A)--(B)--(C);
% arêtes horizontales, de l’arrière vers l’avant
\draw (B) -- (A) ; % bas gauche
\draw (C) -- (G); % bas droit
\draw (F) -- (B); % haut droit
\draw (A) -- (E); % haut gauche


\node at (A) [left] {$A$};
\node at (B) [right] {$B$};
\node at (C) [above right] {$C$};
\node at (D) [above right] {$D$};
\node at (E) [above left] {$E$};
\node at (F) [above left] {$F$};
\node at (G) [above right] {$G$};
\node at (H) [above] {$H$};

\end{tikzpicture}
\end{center}

\begin{exercice}
  Déterminer en justifiant les positions relatives des droites
  ci-dessous. 

  On donnera leur intersection éventuelle.
  \begin{colenumerate}{2}
  \item $(IB)$ et $(GC)$.
  \item $(HB)$ et $(GA)$.
  \item $(GC)$ et $(BA)$.
  \end{colenumerate}
\end{exercice}

\begin{exercice}
  Déterminer en justifiant les positions relatives des droites
  ci-dessous. 

  On donnera leur intersection éventuelle.
  \begin{colenumerate}{2}
  \item $(JK)$ et $(AH)$.
  \item $(FD)$ et $(GH)$.
  \item $(IB)$ et $(HJ)$.
  \end{colenumerate}
\end{exercice}

\begin{exercice}
  Déterminer en justifiant les positions relatives des droites et
  plans ci-dessous. On donnera leur intersection éventuelle.
  \begin{colenumerate}{2}
  \item $(EJ)$ et $(HDA)$.
  \item $(JK)$ et $(ABE)$.
  \item $(IJ)$ et $(AFG)$.
  \end{colenumerate}
\end{exercice}

\begin{exercice}
  Déterminer en justifiant les positions relatives des droites et
  plans ci-dessous. On donnera leur intersection éventuelle.
  \begin{colenumerate}{2}
  \item $(FH)$ et $(ACE)$.
  \item $(EJ)$ et $(BCG)$.
  \item $(IJ)$ et $(ABE)$.
  \end{colenumerate}
\end{exercice}



\begin{exercice}
  Déterminer en justifiant les positions relatives des plans
  ci-dessous. 

  On donnera leur intersection éventuelle.
  \begin{enumerate}
  \item $(ABJ)$ et $(GIC)$.
  \item $(KGI)$ et $(EAD)$.
  \item $(KGI)$ et $(ABE)$.
  \end{enumerate}
\end{exercice}

\begin{exercice}
  Déterminer en justifiant les positions relatives des plans
  ci-dessous. 

  On donnera leur intersection éventuelle.
  \begin{enumerate}
  \item $(EBG)$ et $(HDC)$.
  \item $(EBI)$ et $(HDC)$.
  \item $(IJK)$ et $(HDC)$.
  \end{enumerate}
\end{exercice}

\begin{exercice}
  $ABCD$ est un tétraèdre, $I$, $J$ et $K$ sont les milieux respectifs
  de $[BC]$, $[CD]$ et $[AC]$.

  Déterminer en justifiant les positions relatives des éléments
  ci-dessous. 

  On donnera leur intersection éventuelle.
  \begin{enumerate}
  \item $(IK)$ et $(AD)$.
  \item $(IK)$ et $(AB)$.
  \item $(IJ)$ et $(AID)$.
  \item $(ABJ)$ et $(ACD)$.
  \item $(DIK)$ et $(ABD)$.
  \item $(IJ)$ et $(KBD)$.
  \end{enumerate}
\end{exercice}

\begin{exercice}
  $ABCDE$ est une pyramide de sommet $A$ à base rectangulaire et $I$
  est un point du segment $[AE]$.
  \begin{enumerate}
  \item Justifier que la droite $(BC)$ est parallèle au plan $(EAD)$.
  \item En déduire l'intersection des plans $(IBC)$ et $(EAD)$.
  \end{enumerate}
\end{exercice}

\begin{exercice}
  $A$, $B$, $C$ et $D$ sont quatre points non coplanaires et $\Delta$
  est la droite parallèle à $(BC)$ passant par $D$. $I$ est le milieu
  de $[AC]$.

  Quelle est l'intersection de $\Delta$ avec :
  \begin{enumerate}
  \item Le plan $(IBD)$ ?
  \item Le plan $(ABC)$ ?
  \end{enumerate}
\end{exercice}

\pagebreak

\begin{exercice}
  $ABCDS$ est une pyramide dont la base $ABCD$ est un trapèze.

  \begin{center}
    \begin{tikzpicture}[general,scale=.8]

\coordinate (A) at (0,0);
\coordinate (B) at (4,0);
\coordinate (C) at (3,1);
\coordinate (D) at (1,1);
\coordinate (S) at (1.5,4);

\draw [dashed,J1] (A)--(D)--(S)--(C)--(B);
\draw [dashed,J1] (D)--(C);
\draw (S)--(A)--(B)--cycle;

\node at (A) [left] {$A$};
\node at (B) [right] {$B$};
\node at (C) [below left] {$C$};
\node at (D) [below right] {$D$};
\node at (S) [above] {$S$};

\end{tikzpicture}
  \end{center}

  Reproduire la figure et construire les intersections des plans :
  \begin{enumerate}
  \item $(SAB)$ et $(SDC)$ ;
  \item $(SAD)$ et $(SBC)$.
  \end{enumerate}
\end{exercice}
 
\begin{exercice}
  $ABCDEFGH$ est un pavé droit, $I$ le point du segment $[AE]$ tel que
  $AI=\dfrac{3}{4}AE$ et $J$ le point du segment $[CG]$ tel que
  $CJ=\dfrac{1}{4}CG$.

  Les droites suivantes sont-elles coplanaires ?
  \begin{enumerate}
  \item $(AB)$ et $(IF)$ ;
  \item $(DJ)$ et $(IF)$ ;
  \item $(BC)$ et $(AE)$ ;
  \item $(EH)$ et $(IJ)$ .
  \end{enumerate}
\end{exercice}

\section{Sections}

\begin{exercice}


  \begin{enumerate}
  \item Reproduire la figure de l'exercice précédent.
  \item Tracer l'intersection du plan $(BIJ)$ avec la face $EABF$.
  \item Tracer l'intersection du plan $(BIJ)$ avec la face $DCGH$.
  \item Terminer la construction de la section du pavé $ABCDEFGH$ par
    le plan $(BIJ)$.
  \end{enumerate}
\end{exercice}
\begin{solution}
  \begin{enumerate}
\item L'intersection du plan $(BIJ)$ avec la face $EABF$ est le segment $[BI]$.
\item L'intersection du plan $(BIJ)$ avec la face $DCGH$ est la
  parallèle à $(IB)$ passant par $J$. C'est le segment $[JH]$.
%\begin{center}
%\hspace*{-3cm}

\resizebox{.7\linewidth}{!}{%
\begin{pspicture*}(-1.5,-5.76)(6.5,2)
\pspolygon[linewidth=0.3mm,linecolor=Noir](-1.,-1.)(-1.,-5.)(5.,-5.)(5.,-1.)
\pspolygon[linewidth=0.3mm,linecolor=Noir](5.,-5.)(6.,-3.)(6.,1.)(5.,-1.)
\pspolygon[linewidth=0.3mm,linecolor=Noir](0.,1.)(0.,-3.)(6.,-3.)(6.,1.)
\psline[linewidth=0.3mm,linecolor=Noir](-1.,-1.)(-1.,-5.)
\psline[linewidth=0.3mm,linecolor=Noir](-1.,-5.)(5.,-5.)
\psline[linewidth=0.3mm,linecolor=Noir](5.,-5.)(5.,-1.)
\psline[linewidth=0.3mm,linecolor=Noir](5.,-1.)(-1.,-1.)
\psline[linewidth=0.3mm,linecolor=Noir](5.,-5.)(6.,-3.)
\psline[linewidth=0.3mm,linecolor=Noir](6.,-3.)(6.,1.)
\psline[linewidth=0.3mm,linecolor=Noir](6.,1.)(5.,-1.)
\psline[linewidth=0.3mm,linecolor=Noir](5.,-1.)(5.,-5.)
\psline[linewidth=0.3mm,linestyle=dashed,dash=5pt 5pt,linecolor=Noir](0.,1.)(0.,-3.)
\psline[linewidth=0.3mm,linestyle=dashed,dash=5pt 5pt,linecolor=Noir](0.,-3.)(6.,-3.)
\psline[linewidth=0.3mm,linecolor=Noir](6.,-3.)(6.,1.)
\psline[linewidth=0.3mm,linecolor=Noir](6.,1.)(0.,1.)
\psline[linewidth=0.3mm,linestyle=dashed,dash=5pt 5pt,linecolor=Noir](-1.,-5.)(0.,-3.)
\psline[linewidth=0.3mm,linecolor=Noir](-1.,-1.)(0.,1.)
\psline[linewidth=1.6pt,linecolor=J1](-1.,-2.)(5.,-5.)
\psline[linewidth=1.6pt,linecolor=J1](5.,-5.)(6.,-2.)
\psline[linewidth=1.6pt,linestyle=dashed,dash=5pt 5pt,linecolor=J1](0.,1.)(6.,-2.)
\psline[linewidth=1.6pt,linestyle=dashed,dash=5pt 5pt,linecolor=J1](0.,1.)(-1.,-2.)
\psdots[dotsize=3pt 0,dotstyle=+,linecolor=Noir](-1.,-1.)
\rput[bl](-1.38,-1.12){{$E$}}
\psdots[dotsize=3pt 0,dotstyle=+,linecolor=Noir](-1.,-5.)
\rput[bl](-1.5,-5.22){{$A$}}
\psdots[dotsize=3pt 0,dotstyle=+,linecolor=J1](5.,-5.)
\rput[bl](5.28,-5.18){{$B$}}
\psdots[dotsize=3pt 0,dotstyle=+,linecolor=Noir](5.,-1.)
\rput[bl](5.1,-1.3){{$F$}}
\psdots[dotsize=3pt 0,dotstyle=+,linecolor=Noir](6.,-3.)
\rput[bl](6.18,-3.04){{$C$}}
\psdots[dotsize=3pt 0,dotstyle=+,linecolor=Noir](6.,1.)
\rput[bl](6.08,1.12){{$G$}}
\psdots[dotsize=3pt 0,dotstyle=+,linecolor=Noir](0.,1.)
\rput[bl](0.08,1.12){{$H$}}
\psdots[dotsize=3pt 0,dotstyle=+,linecolor=Noir](0.,-3.)
\rput[bl](0.16,-2.94){{$D$}}
\psdots[dotsize=3pt 0,dotstyle=+,linecolor=J1](-1.,-2.)
\rput[bl](-1.32,-1.96){{$I$}}
\psdots[dotsize=3pt 0,dotstyle=+,linecolor=J1](6.,-2.)
\rput[bl](6.08,-1.88){{$J$}}
\end{pspicture*}
}
%\end{center} 
\end{enumerate}
\end{solution}



\begin{exercice}
  \begin{enumerate}
  \item Reproduire la figure de l'exercice précédent.
  \item Tracer l'intersection du plan $(DIJ)$ avec la face $EADH$.
  \item Tracer l'intersection du plan $(DIJ)$ avec la face $DCGH$.
  \item Tracer l'intersection du plan $(DIJ)$ avec la face $BCGF$.
  \item Terminer la construction de la section du pavé $ABCDEFGH$ par
    le plan $(DIJ)$.
  \end{enumerate}
\end{exercice}

\begin{exercice}
  $ABCDEFGH$ est un cube et $I$ et $J$ les points tels que $I\in [HD]$
  et $HI=\dfrac{2}{3}HD$ ; $J\in [FG]$ et $FJ=\dfrac{3}{4}FG$.

  Construire la section du cube par le plan $(EIJ)$.
\end{exercice}

\begin{exercice}
  $ABCDEFGH$ est un cube et $I$ ; $J$ et $K$ les points tels que
  $I\in [EF]$ et $EI=\dfrac{1}{3}EF$ ; $J\in [BC]$ et
  $BJ=\dfrac{1}{2}BC$ ; $K\in [HG]$ et $HK=\dfrac{3}{4}HG$.

  Construire la section du cube par le plan $(IJK)$.
\end{exercice}

\begin{exercice}
  $ABCDEFGH$ est un cube et $I$ ; $J$ et $K$ les milieux respectifs
  des segments $[BC]$, $[CD]$ et $[EH]$.

  Construire la section du cube par le plan $(IJK)$.
\end{exercice}

\begin{exercice}
  $ABCDEFGH$ est un cube et $I$ ; $J$ et $K$ les points tels que
  $I\in [AE]$ et $AI=\dfrac{1}{4}AE$ ; $J\in [DH]$ et
  $DJ=\dfrac{3}{4}DH$ ; $K\in [FG]$ et $FK=\dfrac{1}{3}FG$.

  Construire la section du cube par le plan $(IJK)$.
\end{exercice}


% ajouts ---------------------------

\begin{exercice}
  $ABCDEFGH$ est un cube ; $I$ est le milieu de $[EH]$ ; $J$ est le
  milieu de $[BC]$ et $K$ le point du segment $[GH]$ tel que :
  $HK=\dfrac{2}{3}HG$.

  Déterminer et construire la section du cube par le plan $(IJK)$.
\end{exercice}

\begin{exercice}
  $ABCDEFGH$ est un cube et $I$ ; $J$ et $K$ les points tels que :
  $I\in [AD]$ et $AI=\dfrac{1}{3}AD$ ; $J\in [FG]$ et
  $FJ=\dfrac{2}{3}FG$ ; $K\in [AB]$ et $AK=\dfrac{1}{3}AB$.

  Déterminer et construire la section du cube par le plan $(IJK)$.
\end{exercice}

\begin{exercice}
  On considère une pyramide à base carrée $SABCD$ comme ci-dessous.
\begin{enumerate}
\item Reproduire la figure et placer les points $I$ et $J$ milieux respectifs des segments $[SD]$ et $[AB]$
\item Construire en justifiant la section de la pyramide par le plan $(CIJ)$.
\end{enumerate}

\begin{center}
  \begin{tikzpicture}[general,x=1.0cm,y=1.0cm,scale=.5]
%\clip(-1.8,-5.92) rectangle (5.48,0.82);
\draw [color=Noir] (1.,-5.)-- (-0.54,-3.);
\draw [color=Noir] (1.,-5.)-- (5.,-5.);
\draw [dashed,color=J1] (5.,-5.)-- (3.5,-3.);
\draw [dashed,color=J1] (3.5,-3.)-- (-0.54,-3.);
\draw [color=Noir] (-0.54,-3.)-- (2.02,0.6);
\draw [color=Noir] (2.02,0.6)-- (3.5,-3.);
\draw [color=Noir] (2.02,0.6)-- (5.,-5.);
\draw [color=Noir] (2.02,0.6)-- (1.,-5.);
\begin{scriptsize}
\draw [color=Noir] (1.,-5.)-- ++(-1.5pt,0 pt) -- ++(3.0pt,0 pt) ++(-1.5pt,-1.5pt) -- ++(0 pt,3.0pt);
\draw[color=Noir] (0.56,-5.06) node {$A$};
\draw [color=Noir] (5.,-5.)-- ++(-1.5pt,0 pt) -- ++(3.0pt,0 pt) ++(-1.5pt,-1.5pt) -- ++(0 pt,3.0pt);
\draw[color=Noir] (5.22,-5.02) node {$B$};
\draw [color=Noir] (3.5,-3.)-- ++(-1.5pt,0 pt) -- ++(3.0pt,0 pt) ++(-1.5pt,-1.5pt) -- ++(0 pt,3.0pt);
\draw[color=Noir] (3.,-2.64) node {$C$};
\draw [color=Noir] (-0.54,-3.)-- ++(-1.5pt,0 pt) -- ++(3.0pt,0 pt) ++(-1.5pt,-1.5pt) -- ++(0 pt,3.0pt);
\draw[color=Noir] (-0.68,-2.62) node {$D$};
\draw [color=Noir] (2.02,0.6)-- ++(-1.5pt,-1.5pt) -- ++(3.0pt,3.0pt) ++(-3.0pt,0) -- ++(3.0pt,-3.0pt);
\draw[color=Noir] (2.16,0.58) node [above] {$S$};
\end{scriptsize}
\end{tikzpicture}
\end{center}
\end{exercice}



\begin{exercice}
On considère un tétraèdre régulier $ABCD$ comme ci-dessous avec $I$, $J$ et $K$ les milieux respectifs des segments $[BC]$, $[AB]$ et $[AD]$.
\begin{enumerate}
\item Reproduire la figure. 
\item Construire en justifiant la section du tétraèdre par le plan $(IJK)$.
\item Quelle est la nature de cette section ? Justifier.
\end{enumerate}

\begin{center}
\begin{tikzpicture}[general,x=1.0cm,y=1.0cm,scale=.7]
%\clip(-3.12,-1.2) rectangle (3.66,5.32);
\draw [color=Noir] (0.,4.)-- (1.,0.);
\draw [color=Noir] (1.,0.)-- (-1.42,0.98);
\draw [color=Noir] (-1.42,0.98)-- (0.,4.);
\draw [color=Noir] (0.,4.)-- (2.04,1.74);
\draw [color=Noir] (2.04,1.74)-- (1.,0.);
\draw [color=Noir] (1.,0.)-- (0.,4.);
\draw [dashed,color=J1] (-1.42,0.98)-- (2.04,1.74);
\begin{scriptsize}
\draw [fill=Noir] (0.,4.) circle (1.5pt);
\draw[color=Noir] (0.14,4.28) node {$A$};
\draw [fill=Noir] (1.,0.) circle (1.5pt);
\draw[color=Noir] (1.3,0.06) node {$B$};
\draw [fill=Noir] (-1.42,0.98) circle (1.5pt);
\draw[color=Noir] (-1.58,1.32) node {$C$};
\draw [fill=Noir] (2.04,1.74) circle (1.5pt);
\draw[color=Noir] (2.18,2.02) node {$D$};
\draw [color=J1] (-0.21,0.49)-- ++(-1.5pt,-1.5pt) -- ++(3.0pt,3.0pt) ++(-3.0pt,0) -- ++(3.0pt,-3.0pt);
\draw[color=J1] (-0.62,0.4) node {$I$};
\draw [color=J1] (0.5,2.)-- ++(-1.5pt,-1.5pt) -- ++(3.0pt,3.0pt) ++(-3.0pt,0) -- ++(3.0pt,-3.0pt);
\draw[color=J1] (0.64,2.28) node {$J$};
\draw [color=J1] (1.02,2.87)-- ++(-1.5pt,-1.5pt) -- ++(3.0pt,3.0pt) ++(-3.0pt,0) -- ++(3.0pt,-3.0pt);
\draw[color=J1] (1.16,3.16) node {$K$};
\end{scriptsize}
\end{tikzpicture}  
\end{center}
\end{exercice}

%  fin ajouts ------------------------

\section{Orthogonalité}

Pour les exercices suivants,
$ABCDEFGH$ est un cube.

\begin{center}
  \begin{tikzpicture}[general]

\coordinate (A) at (0,0,2);
\coordinate (B) at (2,0,2);
\coordinate (F) at (2,2,2);
\coordinate (E) at (0,2,2);
\coordinate (D) at (0,0,0);
\coordinate (C) at (2,0,0);
\coordinate (G) at (2,2,0);
\coordinate (H) at (0,2,0);

\draw [opacity=0] (B)--(G);
\draw [dashed,J2] (A)--(D)--(C);
\draw [dashed,J2] (D)--(H);
\draw (E)--(F)--(G)--(H)--(E); % face arrière
\draw (A)--(B)--(C); % face avant
% arêtes horizontales, de l’arrière vers l’avant
\draw (B) -- (A); % bas gauche
\draw (C) -- (G); % bas droit
\draw (F) -- (B); % haut droit
\draw (A) -- (E); % haut gauche


\node at (A) [left] {$A$};
\node at (B) [right] {$B$};
\node at (C) [right] {$C$};
\node at (D) [above right] {$D$};
\node at (E) [left] {$E$};
\node at (F) [right] {$F$};
\node at (G) [above] {$G$};
\node at (H) [above] {$H$};

\end{tikzpicture}
\end{center}

\begin{exercice}
  \begin{enumerate}
  \item Citer six droites orthogonales à la droite $(EA)$ ;
  \item Citer six droites orthogonales à la droite $(EB)$ ;
  \item Citer deux droites orthogonales au plan $(BCG)$ ;
  \item Citer deux droites orthogonales au plan $(AFG)$.
  \end{enumerate}
\end{exercice}

\begin{exercice}
  \begin{enumerate}
  \item Démontrer que la droite $(AB)$ est orthogonale au plan
    $(BCG)$.
  \item En déduire que les droites $(AB)$ et $(CF)$ sont orthogonales.
  \end{enumerate}
\end{exercice}
\begin{solution}
  \begin{enumerate}
\item  Les droites $(BC)$ et $(BF)$ sont deux droites sécantes du plan $(BCG)$ et, par propriété du cube, $(AB)\perp(BC)$ et $(AB)\perp(BF)$.\\
Donc $(AB)$ est orthogonale au plan $(BCG)$.
\item  $(AB)$ est orthogonale au plan $(BCG)$, donc $(AB)$ est orthogonale à toute droite du plan$(BCG)$, et en particulier, $(AB)$ et $(CF)$ sont orthogonales.
\end{enumerate}
\end{solution}

\begin{exercice}
  Les droites suivantes sont-elles orthogonales ? Le démontrer.
  \begin{colenumerate}{2}
  \item $(EG)$ et $(GC)$ ;
  \item $(EB)$ et $(EG)$ ;
  \item $(AF)$ et $(BC)$ ;
  \item $(AC)$ et $(HF)$ ;
  \item $(BD)$ et $(EC)$ ;
  \item $(CE)$ et $(AG)$.
  \end{colenumerate}
\end{exercice}

\begin{exercice}
  $ABCD$ est un tétraèdre régulier, $S$ est le pied de la hauteur
  issue de A relativement à la base $BCD$ et $I$ est le milieu de
  $[BC]$.

  \begin{center}
    \begin{tikzpicture}[general]

% \coordinate (A) at (0,0);
% \coordinate (B) at (1,-3);
% \coordinate (C) at (-1,-2.5);
% \coordinate (D) at (2,-2);
% \coordinate (S) at (.25,-2.5);

% \draw [dashed,B2] (A)--(S) node {$\times$} node [right] {$S$};
% \draw [dashed,B2] (C)--(D);
% \draw (A)--(C)--(B) node [B2,midway] {$\times$} node [B2,midway,below
% left] {$I$}--(D)--(A)--(B) ;

% \node at (A) [above] {$A$};
% \node at (B) [below] {$B$};
% \node at (C) [left] {$C$};
% \node at (D) [right] {$D$};

\end{tikzpicture}

\begin{tikzpicture}[general,scale=.5]
% \clip(-4.3,-2.88) rectangle (7.14,6.3);
\draw [dashed,color=B2] (0.5938169661678563,-0.4137735165887121)-- (0.58,4.74);
\draw (0.58,4.74)-- (-2.4,-0.06);
\draw (0.58,4.74)-- (2.06,-2.3);
\draw (0.58,4.74)-- (3.,2.);
\draw (-2.4,-0.06)-- (2.06,-2.3);
\draw (2.06,-2.3)-- (3.,2.);
\draw [dashed,color=B2] (3.,2.)-- (-2.4,-0.06);
\begin{scriptsize}
\draw [fill=Noir] (3.,2.) circle (0.5pt);
\draw[color=Noir] (3.14,2.2) node {$D$};
\draw [fill=Noir] (-2.4,-0.06) circle (0.5pt);
\draw[color=Noir] (-2.56,0.26) node {$C$};
\draw [fill=Noir] (2.06,-2.3) circle (0.5pt);
\draw[color=Noir] (2.08,-2.66) node {$B$};
\draw [color=B2] (-0.17,-1.18)-- ++(-2.0pt,0 pt) -- ++(4.0pt,0 pt) ++(-2.0pt,-2.0pt) -- ++(0 pt,4.0pt);
\draw[color=B2] (-0.48,-1.36) node {$I$};
\draw [color=B2] (0.5938169661678563,-0.4137735165887121)-- ++(-1.5pt,-1.5pt) -- ++(3.0pt,3.0pt) ++(-3.0pt,0) -- ++(3.0pt,-3.0pt);
\draw[color=B2] (0.74,-0.14) node {$S$};
\draw [fill=Noir] (0.58,4.74) circle (0.5pt);
\draw[color=Noir] (0.52,5.1) node {$A$};
\end{scriptsize}
\end{tikzpicture}
  \end{center}

  \begin{enumerate}
  \item Démontrer que les droites $(AS)$ et $(BC)$ sont orthogonales.
  \item En déduire que la droite $(BC)$ est orthogonale au plan
    $(AIS)$.
  \item En déduire que les points $A$, $I$, $S$ et $D$ sont
    coplanaires et que les points $I$, $S$ et $D$ sont alignés.
  \end{enumerate}
\end{exercice}

\section{Vecteurs} 

Pour les exercices suivants, $ABCDEFGH$ est un cube et $I$ ; $J$ ; $K$
et $L$ les milieux respectifs de $[BC]$, $[GH]$, $[AD]$ et $[EH]$.

\begin{center}
  \begin{tikzpicture}[general]

\coordinate (A) at (0,0,2);
\coordinate (B) at (2,0,2);
\coordinate (F) at (2,2,2);
\coordinate (E) at (0,2,2);
\coordinate (D) at (0,0,0);
\coordinate (C) at (2,0,0);
\coordinate (G) at (2,2,0);
\coordinate (H) at (0,2,0);

\draw [opacity=0] (B)--(G);
\draw [dashed,J2](A)--(D) node [midway,B2,above left]{$K$} node [midway,B2]{$+$}--(C);
\draw [dashed,J2] (D)--(H);
\draw (E)--(F)--(G)--(H) node [midway,B2,above]{$J$} node
[midway,B2]{$+$}--(E) node [midway,B2,above left]{$L$} node [midway,B2]{$+$}; % face arrière
\draw (A)--(B)--(C) node [midway,B2,right]{$I$} node [midway,B2]{$+$}; % face avant
% arêtes horizontales, de l’arrière vers l’avant
\draw (B) -- (A); % bas gauche
\draw (C) -- (G); % bas droit
\draw (F) -- (B); % haut droit
\draw (A) -- (E); % haut gauche


\node at (A) [left] {$A$};
\node at (B) [right] {$B$};
\node at (C) [right] {$C$};
\node at (D) [above right] {$D$};
\node at (E) [left] {$E$};
\node at (F) [right] {$F$};
\node at (G) [above] {$G$};
\node at (H) [above] {$H$};

\end{tikzpicture}
\end{center}

\begin{exercice}
  Compléter les égalités vectorielles suivantes :
  \begin{enumerate}
  \item $\overrightarrow{A...}=\dfrac{1}{2}\overrightarrow{BC}$
  \item
    $\overrightarrow{KJ}=\overrightarrow{AE}+\dfrac{1}{2}\overrightarrow{E...}$
  \item
    $\overrightarrow{AK}+\overrightarrow{EF}=\overrightarrow{A...}$
  \end{enumerate}
\end{exercice}

\begin{exercice}
  Compléter les égalités vectorielles suivantes :
  \begin{enumerate}
  \item $\overrightarrow{...}=\dfrac{1}{2}\overrightarrow{AC}$
  \item
    $\overrightarrow{L...}=\overrightarrow{EA}+\overrightarrow{FE}+\overrightarrow{AI}$
  \item
    $\overrightarrow{A...}=\overrightarrow{GJ}+3\overrightarrow{AK}+\overrightarrow{AB}+\overrightarrow{JL}$
  \end{enumerate}
\end{exercice}



\begin{exercice}
  Dans chacun des cas suivants, les vecteurs sont-ils coplanaires ? Le
  justifier.
  \begin{enumerate}
  \item $\overrightarrow{AG}$, $\overrightarrow{DH}$ et
    $\overrightarrow{EG}$ ;
  \item $\overrightarrow{AB}$, $\overrightarrow{BD}$ et
    $\overrightarrow{BF}$ ;
  \item $\overrightarrow{AG}$, $\overrightarrow{BG}$ et
    $\overrightarrow{HG}$ ;
  \item $\overrightarrow{HF}$, $\overrightarrow{DC}$ et
    $\overrightarrow{AD}$.
  \end{enumerate}
\end{exercice}

\begin{exercice}
  Le point $M$ est défini par
  $\overrightarrow{EM}=2\overrightarrow{EF}$
  \begin{enumerate}
  \item En fonction des vecteurs
    $\overrightarrow{AB}$, $\overrightarrow{AD}$ et
    $\overrightarrow{AE}$ exprimer les vecteurs suivants :
    \\$\overrightarrow{EM}$
    ; $\overrightarrow{HC}$
    ; $\overrightarrow{BD}$
    ; $\overrightarrow{BJ}$
    ; $\overrightarrow{KM}$ et $\overrightarrow{MJ}$.
  \item Les droites $(BK)$ et $(MJ)$ sont-elles parallèles ?

    Le démontrer en utilisant la question précédente.
  \item Que peut-on en déduire concernant les points $B$,
    $K$, $M$ et $J$ ?
  \end{enumerate}
\end{exercice}

\begin{exercice}
  On considère les points $M$ et $N$ définis par :

  $\overrightarrow{AM}=\dfrac{1}{3}\overrightarrow{AB}+\dfrac{1}{3}\overrightarrow{AD}+\dfrac{2}{3}\overrightarrow{AE}$

  et 

  $\overrightarrow{AN}=\dfrac{2}{3}\overrightarrow{AB}+\overrightarrow{BF}+\dfrac{2}{3}\overrightarrow{FG}$.
  \begin{enumerate}
  \item Construire la figure.
  \item Démontrer que les points $C$, $E$ et $M$ sont alignés.
  \item Démontrer que les points $E$,
    $F$, $H$ et $N$ sont coplanaires.
  \end{enumerate}
\end{exercice}
\begin{solution}
\begin{enumerate}
\item ~\par
\resizebox{.8\linewidth}{!}{
\begin{pspicture*}(-1.5,-5.76)(6.5,2)
\pspolygon[linewidth=.3mm,linecolor=Noir](-1.,-1.)(-1.,-5.)(3.,-5.)(3.,-1.)
\pspolygon[linewidth=.3mm,linecolor=Noir](3.,-5.)(4.,-3.)(4.,1.)(3.,-1.)
\pspolygon[linewidth=.3mm,linecolor=Noir](0.,1.)(0.,-3.)(4.,-3.)(4.,1.)
\psline[linewidth=.3mm,linecolor=Noir](-1.,-1.)(-1.,-5.)
\psline[linewidth=.3mm,linecolor=Noir](-1.,-5.)(3.,-5.)
\psline[linewidth=.3mm,linecolor=Noir](3.,-5.)(3.,-1.)
\psline[linewidth=.3mm,linecolor=Noir](3.,-1.)(-1.,-1.)
\psline[linewidth=.3mm,linecolor=Noir](3.,-5.)(4.,-3.)
\psline[linewidth=.3mm,linecolor=Noir](4.,-3.)(4.,1.)
\psline[linewidth=.3mm,linecolor=Noir](4.,1.)(3.,-1.)
\psline[linewidth=.3mm,linecolor=Noir](3.,-1.)(3.,-5.)
\psline[linewidth=.3mm,linestyle=dashed,dash=5pt 5pt,linecolor=J1](0.,1.)(0.,-3.)
\psline[linewidth=.3mm,linestyle=dashed,dash=5pt 5pt,linecolor=J1](0.,-3.)(4.,-3.)
\psline[linewidth=.3mm,linecolor=Noir](4.,-3.)(4.,1.)
\psline[linewidth=.3mm,linecolor=Noir](4.,1.)(0.,1.)
\psline[linewidth=.3mm,linestyle=dashed,dash=5pt 5pt,linecolor=J1](-1.,-5.)(0.,-3.)
\psline[linewidth=.3mm,linecolor=Noir](-1.,-1.)(0.,1.)
\psline[linestyle=dashed,dash=5pt 5pt,linecolor=J1](-1.,-1.)(4.,-3.)
\psline[linecolor=Noir](-1.,-1.)(4.,1.)
\psdots[dotsize=3pt 0,dotstyle=+,linecolor=Noir](-1.,-1.)
\rput[bl](-1.38,-1.12){{$E$}}
\psdots[dotsize=3pt 0,dotstyle=+,linecolor=Noir](-1.,-5.)
\rput[bl](-1.5,-5.22){{$A$}}
\psdots[dotsize=3pt 0,dotstyle=+,linecolor=Noir](3.,-5.)
\rput[bl](3.28,-5.18){{$B$}}
\psdots[dotsize=3pt 0,dotstyle=+,linecolor=Noir](3.,-1.)
\rput[bl](3.1,-1.3){{$F$}}
\psdots[dotsize=3pt 0,dotstyle=+,linecolor=Noir](4.,-3.)
\rput[bl](4.18,-3.04){{$C$}}
\psdots[dotsize=3pt 0,dotstyle=+,linecolor=Noir](4.,1.)
\rput[bl](4.08,1.12){{$G$}}
\psdots[dotsize=3pt 0,dotstyle=+,linecolor=Noir](0.,1.)
\rput[bl](0.08,1.12){{$H$}}
\psdots[dotsize=3pt 0,dotstyle=+,linecolor=Noir](0.,-3.)
\rput[bl](0.16,-2.94){{$D$}}
\psdots[dotsize=3pt 0,dotstyle=+,linecolor=J1](2.333333333333333,0.33333333333333326)
\rput[bl](2.42,0.46){{$N$}}
\psdots[dotsize=3pt 0,dotstyle=+,linecolor=J1](0.666666666666667,-1.6666666666666667)
\rput[bl](0.74,-1.54){{$M$}}
\end{pspicture*}
}

\item $\overrightarrow{CM}=\overrightarrow{CA}+\overrightarrow{AM}=$\\
$-\overrightarrow{AB}-\overrightarrow{AD}+
\dfrac{1}{3}\overrightarrow{AB}+\dfrac{1}{3}\overrightarrow{AD}+\dfrac{2}{3}\overrightarrow{AE}=$\\
$-\dfrac{2}{3}\overrightarrow{AB}-\dfrac{2}{3}\overrightarrow{AD}+\dfrac{2}{3}\overrightarrow{AE}$ et \\ 
$\overrightarrow{CE}=-\overrightarrow{AB}-\overrightarrow{AD}+\overrightarrow{AE}=\dfrac{3}{2}\overrightarrow{CM}$.\\
Donc $\overrightarrow{CE}$ et $\overrightarrow{CM}$ sont colinéaires et 
les points $C$, $E$ et $M$ sont alignés.

\item $\overrightarrow{EN}=\overrightarrow{EA}+\overrightarrow{AN}=\overrightarrow{FB}+\dfrac{2}{3}\overrightarrow{AB}+\overrightarrow{BF}+\dfrac{2}{3}\overrightarrow{FG}=\dfrac{2}{3}\overrightarrow{EG}$.
Donc $N\in(EFG)$ et les points $E$, $F$, $H$ et $N$ sont coplanaires.
\end{enumerate}
\end{solution}

\begin{exercice}
  Répondre par vrai ou faux en justifiant :
  \begin{enumerate}
  \item Les vecteurs $\overrightarrow{HI}$, $\overrightarrow{AB}$ et
    $\overrightarrow{DH}$ sont coplanaires.
  \item Les vecteurs $\overrightarrow{HG}$, $\overrightarrow{KB}$ et
    $\overrightarrow{LE}$ sont coplanaires.
  \item Les vecteurs $\overrightarrow{HJ}$, $\overrightarrow{AB}$ et
    $\overrightarrow{DH}$ sont coplanaires.
  \end{enumerate}
\end{exercice}

\begin{exercice}
  $ABCDEFGH$ est un cube.\\
  On considère le point $K$ défini par
  $\overrightarrow{HK}=\dfrac{5}{4}\overrightarrow{HF}$ et $M$ un
  point du segment $[BF]$.
  \begin{enumerate}
  \item Que peut-on dire des points $D$, $M$, $K$ et $H$ ?
  \item Montrer qu'il existe un unique réel $t\in\left[ 0;1\right] $
    tel que $\overrightarrow{BM}=t\overrightarrow{BF}$.
  \item Montrer que si $t=\dfrac{4}{5}$, les points $D$, $M$ et $K$
    sont alors alignés.
  \end{enumerate}
\end{exercice}

\begin{exercice}
  Dans un repère $(O\,;\vec{i},\vec{j},\vec{k})$ de l'espace, on
  considère les points $A(-3\,;2\,;4)$ \,; $B(-1\,;1\,;0)$ et
  $C(2\,;-3\,;5)$.  Déterminer les coordonnées des points $M$, $N$ et
  $P$ définis par :
  \begin{enumerate}
  \item $\overrightarrow{AM}=2\overrightarrow{BC}-\overrightarrow{BA}$
  \item
    $\overrightarrow{NB}=4\overrightarrow{CA}-3\overrightarrow{BC}$
  \item
    $2\overrightarrow{PA}-3\overrightarrow{PB}+\overrightarrow{PC}=\overrightarrow{0}$
  \end{enumerate}
\end{exercice}



\begin{exercice}
  Dans un repère $(O\,;\vec{i},\vec{j},\vec{k})$ de l'espace, on
  considère les points $A(-4\,;2\,;3)$, $B(1\,;5\,;2)$, $C(0\,;5\,;4)$ et
  $D(-6\,;-1\,;-2)$.
  \begin{enumerate}
  \item Démontrer que
    $\overrightarrow{AD}=2\overrightarrow{AB}-3\overrightarrow{AC}$.
  \item Que peut-on en déduire concernant les points $A$, $B$, $C$ et
    $D$ ?
  \end{enumerate}
\end{exercice}
\begin{solution}
  \begin{enumerate}
\item $\overrightarrow{AD}(-2\ ;\ -3\ ;\ -5)$. D'autre part : \\ $\overrightarrow{AB}(5\ ;\ 3\ ;\ -1)$ donc $2\overrightarrow{AB}(10\ ;\ 6\ ;\ -2)$ et \\
$\overrightarrow{AC}(4\ ;\ 3\ ;\ 1)$ donc $-3\overrightarrow{AC}(-12\ ;\ -9\ ;\ -3)$.\\
Ainsi $2\overrightarrow{AB}-3\overrightarrow{AC}(-2\ ;\ -3\ ;\ -5)$.
Donc $\overrightarrow{AD}=2\overrightarrow{AB}-3\overrightarrow{AC}$
\item On en déduit que les vecteurs $\overrightarrow{AB}$, $\overrightarrow{AC}$ et $\overrightarrow{AD}$ sont coplanaires, et que les points $A$, $B$, $C$ et $D$ sont coplanaires.
 \end{enumerate}
\end{solution}

\begin{exercice}
  Dans un repère $(O\,;\vec{i},\vec{j},\vec{k})$ de l'espace, on
  considère les points $A(0\,;3\,;-1)$, $B(2\,;-2\,;0)$, $C(4\,;1\,;5)$ et
  $D(2\,;21\,;12)$.
  \begin{enumerate}
  \item Montrer que les points $A$, $B$ et $C$ définissent un plan.
  \item Le point $D$ appartient-il à ce plan ?
  \end{enumerate}
\end{exercice}

\begin{exercice}
  Dans un repère $(O\,;\vec{i},\vec{j},\vec{k})$ de l'espace,on
  considère les points $A(1\,;-1\,;-1)$, $B(5\,;0\,;-3)$, $C(2\,;-2\,;-2)$ et
  $D(0\,;5\,;-2)$.
  \begin{enumerate}
  \item Montrer que les points $A$, $B$ et $C$ définissent un plan.
  \item Le point $D$ appartient-il à ce plan ?
  \end{enumerate}
\end{exercice}

\begin{exercice}
  On reprend l'énoncé de l'exercice 42 en se plaçant dans le repère
  $(A\,;\overrightarrow{AB},\overrightarrow{AD},\overrightarrow{AE})$.
  \begin{enumerate}
  \item Écrire les coordonnées des points de la figure. On écrira les
    coordonnées de $M$ en fonction de $t$.
  \item Démontrer à l'aide des coordonnées que $D$, $M$ et $I$ sont
    alignés si et seulement si $t=\dfrac{4}{5}$.
  \end{enumerate}
\end{exercice}

\section{Représentations paramétriques}

Dans toute cette partie, on munit l'espace d'un repère
$(O\,;\vec{i},\vec{j},\vec{k})$.

\begin{exercice}
  On considère les points $A(-3\ ;\ 2\ ;\ 4)$ et  $B(-1\ ;\ 1\ ;\ 0)$.
  Écrire une représentation paramétrique de la droite $(AB)$.
\end{exercice}
%\begin{solution}
%   Le vecteur $\overrightarrow{AB}(2\ ;\ -1\ ;\ -4)$ est un vecteur
%   directeur de la droite $(AB)$. Ainsi,
%   \begin{center}
%     $\begin{cases}
%       x=-3+2t \\
%       y=2-t\\
%       z=4-4t
%     \end{cases}$,
%     $t\in\mathbb{R}$
%   \end{center}
%   est une représentation paramétrique de la droite $(AB)$.
%\end{solution}

\begin{exercice}
  Soit $\Delta$ la droite de représentation paramétrique :
{\centering
  $\begin{cases}x=1-4t \\y=3+t \\z=1-t \end{cases}$ , $t\in\mathbb{R}$
\par}
\begin{enumerate}
\item Donner un vecteur directeur de la droite $\Delta$ et un point de
  $\Delta$.
\item Le point $M(-3\,;4\,;1)$ appartient-il à la droite $\Delta$ ?
\item Donner les coordonnées de trois points de la droite~$\Delta$.
\item Déterminer une autre représentation paramétrique de $\Delta$.
\end{enumerate}
\end{exercice}

\begin{exercice}
  Soit  $\Delta$ la droite de représentation paramétrique :

  \begin{center}
    $\begin{cases}x=t-2 \\y=4 \\z=2t-1 \end{cases}$ , $t\in\mathbb{R}$
  \end{center}

  \begin{enumerate}
  \item Donner un vecteur directeur de la droite $\Delta$ et un point
    de $\Delta$.
  \item Le point $M(-3\,;4\,;-3)$ appartient-il à la droite $\Delta$ ?
  \item Donner les coordonnées de trois points de $\Delta$.
  \item Déterminer une autre représentation paramétrique de la droite
    $\Delta$.
  \end{enumerate}
\end{exercice}

\begin{exercice}
  Soient $A(-4\,;1\,;2)$ et $B(-1\,;2\,;5)$. Donner une\\
  représentation paramétrique de chacun des objets géométriques
  suivants :
  \begin{enumerate}
  \item La droite $(AB)$ ;
  \item Le segment $[AB]$ ;
  \item La demi-droite $[AB)$.
  \end{enumerate}
\end{exercice}

\begin{exercice}
  Donner une représentation paramétrique de :
  \begin{enumerate}
  \item La droite $(O\ ;\ \overrightarrow{i})$ ;
  \item La droite $(O\ ;\ \overrightarrow{j})$ ;
  \item La droite $(O\ ;\ \overrightarrow{k})$.
  \end{enumerate}
\end{exercice}

\begin{exercice}
  On considère les points $A(-3\ ;\ 2\ ;\ 4)$, $B(-1\ ;\ 1\ ;\ 0)$ et
  $C(-5\ ;\ 4\ ;\ 6)$.

  Vérifier que $A$, $B$ et $C$ définissent un plan et écrire une
  représentation paramétrique du plan $(ABC)$.
\end{exercice}
%\begin{solution}
%   $\overrightarrow{AB}(2\ ;\ -1\ ;\ -4)$ et
%   $\overrightarrow{AC}(-2\ ;\ 2\ ;\ 2)$ ne sont pas colinéaires, donc
%   les points $A$, $B$ et $C$ ne sont pas alignés.

%   Ils définissent donc un plan dirigé par les vecteurs
%   $\overrightarrow{AB}$ et $\overrightarrow{AC}$.

%   Une représentation paramétrique du plan $(ABC)$ est donc :
%   \begin{center}
%     $\begin{cases}
%       x=-3+2t-2t'\\
%       y=2-t+2t'\\
%       z=4-4t+2t'
%     \end{cases}$,
%     $t\in\mathbb{R}$ et $t'\in\mathbb{R}$
%   \end{center}
%\end{solution}

\begin{exercice}
  Soit $\wp$ le plan de représentation paramétrique :

  \begin{center}
    $\begin{cases}x=3-t+5t' \\y=1+t' \\z=-5t+3t' \end{cases}$
    $t\in\mathbb{R}, t'\in\mathbb{R}$
  \end{center}

  \begin{enumerate}
  \item Donner les coordonnées d'un couple de vecteurs\\ directeurs de
    $\wp$ et un point de $\wp$.
  \item Le point $M(6\,;2\,;-6)$ appartient-il à $\wp$ ?
  \item Donner les coordonnées de trois points de $\wp$.
  \item Déterminer une autre représentation paramétrique de $\wp$.
  \end{enumerate}
\end{exercice}

%

\begin{exercice}
  Soient $A(-4\,;1\,; 2)$ ; $B(-1\,; 2\,;\ 5)$ et $C(1\,; 0 \,; 6)$.
  \begin{enumerate}
  \item Vérifier que les points $A$, $B$ et $C$ définissent un plan.
  \item Déterminer une représentation paramétrique de la droite
    $(AB)$.
  \item Déterminer une représentation paramétrique du plan $(ABC)$.
  \item Démontrer que le point $D(-3\,;-4\,;1)$ appartient au plan
    $(ABC)$.
  \item Déterminer une autre représentation paramétrique du plan
    $(ABC)$.
  \end{enumerate}
\end{exercice}

\begin{exercice}
  Donner une représentation paramétrique des plans suivants :
  \begin{enumerate}
  \item Le plan $(O\,;\overrightarrow{i},\overrightarrow{j})$ ;
  \item Le plan $(O\,;\overrightarrow{i},\overrightarrow{k})$ ;
  \item Le plan $(O\,;\overrightarrow{j},\overrightarrow{k})$.
  \end{enumerate}
\end{exercice}
 
\begin{exercice}
  Soit $\Delta$ la droite de représentation paramétrique :

  \begin{center}
    $\begin{cases}x=1-t \\y=-2+3t \\z=-1+t \end{cases}$,
    $t\in\mathbb{R}$
  \end{center}

  Dans chacun des cas suivants, étudier la position de la droite
  $\Delta$ avec la droite $d$ de représentation paramétrique :
  \begin{enumerate}
  \item \begin{center}
      $\begin{cases}x=-k \\y=3+2k \\z=4-k \end{cases}$,
      $k\in\mathbb{R}$\\[3mm]
    \end{center}

  \item \begin{center}
      $\begin{cases}x=1+k \\y=-2k \\z=3-k \end{cases}$,
      $k\in\mathbb{R}$\\[3mm]
    \end{center}

  \item \begin{center}
      $\begin{cases}x=k-2 \\y=7-3k \\z=2-k \end{cases}$,
      $k\in\mathbb{R}$
    \end{center}
  \end{enumerate}
\end{exercice}
\begin{solution}
  $\Delta$ a pour vecteur directeur $\overrightarrow{u}(-1\ ;\ 3\ ;\ 1)$

\begin{enumerate}
\item  $d$ a pour vecteur directeur $\overrightarrow{v}(-1\ ;\ 2\ ;\
  -1)$ qui n'est pas colinéaire avec $\overrightarrow{u}$. $\Delta$ et
  $d$ sont donc soit sécantes, soit non coplanaires.

\hspace{-5mm}$\begin{cases}
  x=1-t\\y=-2+3t\\z=-1+t\\x=-k\\y=3+2k\\z=4-k \end{cases}\Leftrightarrow \begin{cases}
  -k=1-t\\3+2k=-2+3t\\4-k=-1+t\\x=-k\\y=3+2k\\z=4-k \end{cases}\Leftrightarrow$\\
\hspace*{-5mm}$\begin{cases} k=t-1\\-2+3t=1+2t\\4-t+1=-1+t\\x=-k\\y=3+2k\\z=4-k\end{cases}$
$\Leftrightarrow \begin{cases} k=2\\t=3\\t=3\\x=-2\\y=7\\z=2 \end{cases}$\\
Les droites $d$ et $\Delta$ sont donc sécantes en $A(-2\ ;\ 7\ ;\ 2)$.
\item $d$ a pour vecteur directeur $\overrightarrow{v}(1\ ;\ -2\ ;\ -1)$ qui n'est pas colinéaire avec $\overrightarrow{u}$. $\Delta$ et $d$ sont donc soit sécantes, soit non coplanaires.

  \hspace*{-5mm}$\begin{cases}
    x=1-t\\y=-2+3t\\z=-1+t\\x=1+k\\y=-2k\\z=3-k \end{cases}\hspace{-4mm}\Leftrightarrow \begin{cases}
    1+k=1-t\\-2k=-2+3t\\3-k=-1+t\\x=1+k\\y=-2k\\z=3-k \end{cases}\hspace{-4mm}\Leftrightarrow
  \begin{cases} k=-t\\t=2\\3=-1
    !!!\\x=1+k\\y=-2k\\z=3-k \end{cases}$\\
  Le système n'admet pas de solution et les droites $d$ et $\Delta$
  sont donc non coplanaires.
\item $d$ a pour vecteur directeur
  $\overrightarrow{v}(1\ ;\ -3\ ;\ -1)$ qui est pas colinéaire avec
  $\overrightarrow{u}$. $\Delta$ et $d$ sont donc parallèles.
  $B(1\ ;\ -2\ ;\ -1)\in\Delta$. Vérifions si $B\in d$ pour savoir si elles sont confondues :\\
  $\begin{cases} 1=1+k\\-2=-2k\\-1=3-k \end{cases}\Leftrightarrow \begin{cases} k=0\\k=1\\k=4 \end{cases}$.\\
  Il n'existe pas de réel $k$ tel que les coordonnées de $B$ vérifient
  le système, donc $B$ n'appartient pas à $d$ et les droites $d$ et
  $\Delta$ sont strictement parallèles.
 \end{enumerate}
\end{solution}

%

\begin{exercice}
  Soit $\Delta$ la droite de représentation paramétrique :

  \begin{center}
    $\begin{cases}x=8+2t \\y=-5-4t \\z=3+2t \end{cases}$,
    $t\in\mathbb{R}$
  \end{center}

  Dans chacun des cas suivants, étudier la position de la droite
  $\Delta$ avec la droite $d$ de représentation paramétrique :
  \begin{enumerate}
  \item \begin{center}
      $\begin{cases}x=2+3t\\y=7-6t \\z=-3+3t\end{cases}$,
      $t\in\mathbb{R}$
    \end{center}
  \item \begin{center} $\begin{cases}x=1-t \\y=4-t \\z=2 \end{cases}$, $t\in\mathbb{R}$
    \end{center}
  \item \begin{center}
      $\begin{cases}x=-1+t \\y=7-2t \\z=-2+3t \end{cases}$,
      $t\in\mathbb{R}$
    \end{center}
  \end{enumerate}
\end{exercice}

\begin{exercice}
  Soit $\wp$ le plan de représentation paramétrique :

\begin{center}
  $\begin{cases}x=t-2t' \\y=1+3t+t' \\z=2-5t \end{cases}$
  $t\in\mathbb{R}, t'\in\mathbb{R}$
\end{center}

Déterminer la nature de $\wp\cap\wp'$ dans chacun des cas\\
suivants où $\wp'$ est définie par une représentation paramétrique :
\begin{enumerate}
\item \begin{center}
    $\begin{cases}x=-2-3t-t' \\y=2-2t+4t' \\z=2+5t-5t' \end{cases}$
    $t\in\mathbb{R}, t'\in\mathbb{R}$
  \end{center}
\item \begin{center}
    $\begin{cases}x=4-3t+5t' \\y=-2t+t' \\z=5+5t-5t' \end{cases}$
    $t\in\mathbb{R}, t'\in\mathbb{R}$
  \end{center}
\item \begin{center}
    $\begin{cases}x=-3+2t+t' \\y=2-t+2t' \\z=1+t \end{cases}$
    $t\in\mathbb{R}, t'\in\mathbb{R}$
  \end{center}
\end{enumerate}
\end{exercice}

\begin{exercice}
  Soit $\wp$ le plan de représentation paramétrique :
  \begin{center}
    $\begin{cases}x=4-t+3t' \\y=1-t+5t' \\z=t-t' \end{cases}$
    $t\in\mathbb{R}, t'\in\mathbb{R}$
  \end{center}Dans chacun des cas suivants, déterminer une
  représentation paramétrique de la droite d'intersection de $\wp$ et
  du plan :
  \begin{enumerate}
  \item $(O\,;\overrightarrow{i},\overrightarrow{j})$
  \item $(O\,;\overrightarrow{i},\overrightarrow{k})$
  \item $(O\,;\overrightarrow{j},\overrightarrow{k})$
  \end{enumerate}
\end{exercice}

\begin{exercice}
  Soit $\wp$ le plan de représentation paramétrique :
  \begin{center}
    $\begin{cases}x=4-t+3t' \\y=1-t+5t' \\z=t-t' \end{cases}$
    $t\in\mathbb{R}, t'\in\mathbb{R}$
  \end{center}

  Dans chacun des cas suivants, déterminer l'intersection de $\wp$
  avec la droite $d$ donnée par une représentation\\ paramétrique :
  \begin{colenumerate}{2}
  \item \begin{center}
      $\begin{cases}x=2+t\\y=3+3t \\z=5-t\end{cases}$,
      $t\in\mathbb{R}$
    \end{center}
  \item \begin{center}
      $\begin{cases}x=1+4t \\y=10t \\z=-3t \end{cases}$,
      $t\in\mathbb{R}$
    \end{center}
  \item \begin{center}
      $\begin{cases}x=1+2t \\y=-2 \\z=-3+t \end{cases}$,
      $t\in\mathbb{R}$
    \end{center}
  \end{colenumerate}
\end{exercice}

\begin{exercice}
  Soit $ABCDEFGH$ un cube ; $I$ et $J$ les milieux respectifs de
  $[EG]$ et $[GH]$. 

  On munit l'espace du repère
  $(A;\overrightarrow{AB},\overrightarrow{AD},\overrightarrow{AE})$.
  \begin{enumerate}
  \item Déterminer une représentation paramétrique de la droite $(AI)$
    puis de la droite $(DJ)$.
  \item Démontrer que les droites $(AI)$ et $(DJ)$ sont sécantes en un
    point dont on déterminera les coordonnées.
  \end{enumerate}
\end{exercice}





\end{document}