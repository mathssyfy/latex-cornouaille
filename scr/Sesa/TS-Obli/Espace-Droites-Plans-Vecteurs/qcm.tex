\documentclass{cornouaille}

\setcounter{questionqcm}{0}

\dscornouaille
\begin{document}

\fexo{TS}{QCM}{}
\emph{Pour chaque question, plusieurs réponses sont
  proposées.  Déterminer celles qui sont correctes.}

\begin{QCM}
  \begin{EnonceCommunQCM}
    On considère le cube $ABCDEFGH$ de côté $a$, avec $I$, $J$ les
    milieux respectifs des segments $[CD]$ et $[GH]$ et $L$ est le
    milieu du segment $[GH]$.

    \begin{center}
     \begin{tikzpicture}[general]

\coordinate (A) at (0,0,2);
\coordinate (B) at (2,0,2);
\coordinate (F) at (2,2,2);
\coordinate (E) at (0,2,2);
\coordinate (D) at (0,0,0);
\coordinate (C) at (2,0,0);
\coordinate (G) at (2,2,0);
\coordinate (H) at (0,2,0);

\draw [dashed,J2] (A)--(D)--(C) node [B2,midway]{+} 
                    node [B2,midway,above]{I};
\node at (1.5,2,0) [B2,above] {L};
\node at (1.5,2,0) [B2] {+};
\draw [dashed,J2] (D)--(H);
\draw (E)--(F)--(G)--(H)node [B2,midway]{+} 
                        node [B2,midway,above]{J}--(E); % face arrière

\node at (1.5,2,0) [B2] {+};
\draw (A)--(B)--(C); % face avant
% arêtes horizontales, de l’arrière vers l’avant
\draw (B) -- (A); % bas gauche
\draw (C) -- (G); % bas droit
\draw (F) -- (B); % haut droit
\draw (A) -- (E); % haut gauche


\node at (A) [left] {$A$};
\node at (B) [right] {$B$};
\node at (C) [right] {$C$};
\node at (D) [above right] {$D$};
\node at (E) [left] {$E$};
\node at (F) [right] {$F$};
\node at (G) [above] {$G$};
\node at (H) [above] {$H$};

\end{tikzpicture}
    \end{center}
  \end{EnonceCommunQCM}
  
\begin{GroupeQCM}
  \begin{enumerate}
\setcounter{enumi}{\value{questionqcm}}\stepcounter{questionqcm}
\item 
    La droite $(BI)$ est :
    \begin{ChoixQCM}{3}
    \item orthogonale à $(IJ)$
    \item orthogonale à $(IL)$
    \item orthogonale à $(DG)$
    \end{ChoixQCM}
\begin{solution}
      \reponseQCM{a}
\end{solution}
  \end{enumerate}

  \begin{enumerate}
\setcounter{enumi}{\value{questionqcm}}\stepcounter{questionqcm}
\item 
    L'intersection du plan $(BIL)$ avec le plan $(ABF)$ est :
    \begin{ChoixQCM}{3}
    \item une droite passant par\\ le milieu de $[AB]$
    \item une droite passant par\\ le point $B$
    \item une droite parallèle à $(IL)$
    \end{ChoixQCM}
  \end{enumerate}
\begin{solution}
    \reponseQCM{b} et \reponseQCM{c}
\end{solution}

  \begin{enumerate}
\setcounter{enumi}{\value{questionqcm}}\stepcounter{questionqcm}
\item 
    La section du cube $ABCDEFGH$ par le plan $(BIL)$ est :
    \begin{ChoixQCM}{3}
    \item un triangle
    \item un parallélogramme
    \item un trapèze
    \end{ChoixQCM}
  \end{enumerate}
\begin{solution}
    \reponseQCM{c}
\end{solution}

  \begin{enumerate}
\setcounter{enumi}{\value{questionqcm}}\stepcounter{questionqcm}
\item 
    Dans le repère
    $(A;\overrightarrow{AB},\overrightarrow{AD},\overrightarrow{AE})$
    on a :
    \begin{ChoixQCM}{3}
    \item $\overrightarrow{BJ}\begin{pmatrix}
        -0,5\\
        1\\
        1
      \end{pmatrix}$
    \item les points $L$, $I$, $B$ et $F$ sont\\ coplanaires
    \item
      $\overrightarrow{AJ}=2\overrightarrow{AF}+\overrightarrow{GH}-\overrightarrow{CG}$
    \end{ChoixQCM}
  \end{enumerate}
\begin{solution}
    \reponseQCM{a}
\end{solution}
\end{GroupeQCM}
\end{QCM}

\begin{QCM}
\begin{EnonceCommunQCM}
  Dans un repère
  $(O\,;\overrightarrow{i},\overrightarrow{j},\overrightarrow{k})$ de
  l'espace, on considère les points $A(1\,;0\,;2)$, $B(2\,;1\,;2)$,
  $C(3\,;0\,;0)$ et $D(5\,;-2\,;-4)$.
\end{EnonceCommunQCM}

\begin{GroupeQCM}
  \begin{enumerate}
\setcounter{enumi}{\value{questionqcm}}\stepcounter{questionqcm}
\item 
    Les points $A$, $B$ et $C$ :
    \begin{ChoixQCM}{3}
    \item sont alignés
    \item sont coplanaires
    \item définissent un plan
    \end{ChoixQCM}
  \end{enumerate}
\begin{solution}
    \reponseQCM{b} et \reponseQCM{c}
\end{solution}

  \begin{enumerate}
\setcounter{enumi}{\value{questionqcm}}\stepcounter{questionqcm}
\item 
    Les points $A$, $B$, $C$ et $D$ :
    \begin{ChoixQCM}{3}
    \item sont coplanaires
    \item vérifient l'égalité\\
      $\overrightarrow{AD}=-2\overrightarrow{AB}+3\overrightarrow{AC}$
    \item $D\in(BC)$
    \end{ChoixQCM}
  \end{enumerate}
\begin{solution}
    \reponseQCM{a b c}
\end{solution}

  \begin{enumerate}
\setcounter{enumi}{\value{questionqcm}}\stepcounter{questionqcm}
\item 
    Une représentation paramétrique de :
    \begin{ChoixQCM}{3}
    \item la droite $(AB)$ est :\\
      $\begin{cases}x=2-t\\y=1-t \\z=2+t \end{cases}$,
      $t\in\mathbb{R}$
    \item du plan $(ABC)$ est :\\
      $\begin{cases}x=5+t+4t' \\y=-2-t-2t' \\z=-4-2t-6t' \end{cases}$\\
      $t\in\mathbb{R}$ et $t'\in\mathbb{R}$
    \item du plan $(ABC)$ est :\\
      $\begin{cases}x=1+t+2t' \\y=t \\z=2-2t' \end{cases}$\\
      $t\in\mathbb{R}$ et $t'\in\mathbb{R}$
    \end{ChoixQCM}
  \end{enumerate}
\begin{solution}
    \reponseQCM{b} et \reponseQCM{c}
\end{solution}

  \begin{enumerate}
\setcounter{enumi}{\value{questionqcm}}\stepcounter{questionqcm}
\item 
    Soit $E(3\,;4\,;5)$ :
    \begin{ChoixQCM}{3}
    \item la droite parallèle à $(AB)$ et passant par $E$ a
      pour représentation paramétrique
      $\begin{cases}x=t\\y=1+t \\z=5\end{cases}$ , $t\in\mathbb{R}$
    \item Le point $E$ appartient au plan $(ABC)$
    \item les droites $(AB)$ et $(DE)$ sont non coplanaires
    \end{ChoixQCM}
  \end{enumerate}
\begin{solution}
    \reponseQCM{a} et \reponseQCM{c}
\end{solution}

\end{GroupeQCM}
\end{QCM}

\enlargethispage{2cm}


\begin{QCM}
\begin{EnonceCommunQCM}
  Dans un repère
  $(O\ ;\ \overrightarrow{i},\overrightarrow{j},\overrightarrow{k})$ de
  l'espace, on considère les droites 

  $d$ :
  $\begin{cases}x=1-t \\y=2t \\z=3 \end{cases}$ $t\in\mathbb{R}$ et
  $d'$ :
  $\begin{cases}x=3+t \\y=1-4t \\z=2t \end{cases}$  $t\in\mathbb{R}$\\
  et le plan $\wp$  de représentation paramétrique :
  $\begin{cases}x=2+t-t' \\y=-2t+3t' \\z=4-t' \end{cases}$
  $t\in\mathbb{R}, t'\in\mathbb{R}$
\end{EnonceCommunQCM}
\begin{GroupeQCM}
  \begin{enumerate}
\setcounter{enumi}{\value{questionqcm}}\stepcounter{questionqcm}
\item 
    \begin{ChoixQCM}{3}
    \item la droite $d$ est parallèle au plan
      $(O\,;\overrightarrow{i},\overrightarrow{j})$
    \item la droite $d$ est parallèle au plan
      $(O\,;\overrightarrow{i},\overrightarrow{k})$
    \item la droite $d$ est parallèle à la droite
      $(O\,;\overrightarrow{k})$
    \end{ChoixQCM}
  \end{enumerate}
\begin{solution}
    \reponseQCM{a}
\end{solution}


  \begin{enumerate}
\setcounter{enumi}{\value{questionqcm}}\stepcounter{questionqcm}
\item 
    \begin{ChoixQCM}{3}
    \item $d$ et $\wp$ sont parallèles
    \item $d$ et $\wp$ sont sécants\\ en $A(-1;4;3)$
    \item $d$ est inclus dans $\wp$
    \end{ChoixQCM}
  \end{enumerate}
\begin{solution}
    \reponseQCM{a}
\end{solution}

  \begin{enumerate}
\setcounter{enumi}{\value{questionqcm}}\stepcounter{questionqcm}
\item 
    \begin{ChoixQCM}{3}
    \item $d$ et $d'$ sont parallèles
    \item  $d$ et $d'$ sont sécantes 
    \item  $d$ est $d'$ ne sont pas coplanaires 
    \end{ChoixQCM}
  \end{enumerate}
\begin{solution}
    \reponseQCM{c}
\end{solution}
\end{GroupeQCM}
\end{QCM}

\end{document}