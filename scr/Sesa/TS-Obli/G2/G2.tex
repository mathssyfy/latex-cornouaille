\renewcommand\standalonepath[1]{G2/figures/#1}

\chapter[Espace : droites, plans et vecteurs]{Espace : droites,\\ plans et vecteurs}

\begin{prerequis}
  \begin{itemize}
  \item Utiliser une représentation d'un objet de l'espace
  \item Calculer des aires et des volumes
  \item Utiliser la colinéarité de deux vecteurs
  \item Maîtriser le calcul vectoriel dans le plan avec ou sans\\ repère
  \item Résoudre des systèmes.
  \end{itemize}
\end{prerequis}

\begin{autoeval}
  \begin{multicols}{2}
    \begin{center}
      \includestandalone{\standalonepath{figureex1et2}}
    \end{center}

    \begin{exercice}
      $ABCDEFGH$ est un cube de côté $a$.
      \begin{enumerate}
      \item Exprimer la distance $EB$ en fonction de $a$.
      \item Préciser la nature du triangle $FBC$.
      \item Étudier la nature du triangle $EBG$.
      \end{enumerate}
    \end{exercice}
    \begin{corrige}
      \begin{enumerate}
      \item $EB=\sqrt{2}a$
      \item $FBC$ est un triangle rectangle isocèle en $B$.
      \item $EB=BG=GE=\sqrt{2}a$ donc EBG est équilatéral.
      \end{enumerate} 
    \end{corrige}
    
    \begin{exercice}
      $ABCDEFGH$ est un cube de côté $a$.
      \begin{enumerate}
      \item Calculer le volume de ce cube.
      \item Calculer le volume du tétraèdre $ABDE$.
      \item En déduire le volume du polyèdre $BCGFEHD$
      \end{enumerate}
    \end{exercice}
    \begin{corrige}
      \begin{enumerate}
      \item $V=a^3$
      \item $V_{ABDE}=\dfrac{a^3}{6}$
      \item $V_{BCGFEHD}=V-V_{ABDE}=\dfrac{5a^3}{6}$
      \end{enumerate}
    \end{corrige}

    \columnbreak

    \begin{exercice}
      Dans le plan muni d'un repère $(O\,;\vec{i},\vec{j})$, soit
      $A(1\,;4), B(-2\,;-1), C(-1\,;0,7)$ et $E(1\,;1)$.
      \begin{enumerate}
      \item $A$, $B$ et $C$ sont-ils alignés ?
      \item Soit $\mathscr{D}$ la droite parallèle à $(AB)$ passant
        par $E$ et $M(x\,;y)$ un point de $\mathscr{D}$. Montrer qu'il
        existe un unique réel $t$ tel que
        $\overrightarrow{EM}=t \overrightarrow{AB}$.
      \item Déterminer les coordonnées du point $F$ tel que
        $\overrightarrow{AF}=2 \overrightarrow{BE}-3
        \overrightarrow{AB}$.
      \end{enumerate}
    \end{exercice}
    \begin{corrige}
      \begin{enumerate}
      \item $\overrightarrow{AB}(-3\ ;\ -5)$ et $\overrightarrow{AC}(-2\ ;\ -3{,}3)$.\\
        $-3\times(-3,3)\neq(-2)\times(-5)$ donc $\overrightarrow{AB}$
        et $\overrightarrow{AC}$ ne sont pas colinéaires donc $A$, $B$
        et $C$ ne sont pas alignés.
      \item $M\in\mathscr{D}$ si et seulement si
        $ \overrightarrow{EM}$ et $\overrightarrow{AB}$ sont
        colinéaires, d'où le résultat.
      \item $\overrightarrow{BE}(3\ ;\ 2)$ donc $2\overrightarrow{BE}(6\ ;\ 4)$ et $-3\overrightarrow{AB}(9\ ;\ 15)$ donc $\overrightarrow{AF}(15\ ;\ 19)$.\\
        D'où $F(16\ ;\ 23)$.
      \end{enumerate}
    \end{corrige}

    \begin{exercice}
      Résoudre les systèmes suivants :
      \begin{enumerate}
      \item $\begin{cases}3x+2y=5 \\4x-y=3 \end{cases}$
      \item $\begin{cases}y=4x-3 \\-12x+5y=9 \end{cases}$
      \item $\begin{cases}x-y=3 \\-12x+3y=7 \end{cases}$
      \end{enumerate}
    \end{exercice}
    \begin{corrige}
      \begin{enumerate}
      \item $(1\ ;\ 1)$ est le couple solution.
      \item  $(3\ ;\ 9)$ est le couple solution.
      \item  $(\dfrac{-16}{9}\ ;\ \dfrac{-43}{9})$ est le couple solution.
      \end{enumerate}
    \end{corrige}
\end{multicols}
\end{autoeval}

%%%%%%%%%%%%%%%%%%%%%%%%%%%%%%%%%%%%%%%%%%%%%%%%%%%%%%%%%%%%%

%%%%%%%%%%
\activites
%%%%%%%%%%

\begin{activite}[Voir ou revoir dans l'espace...]
  \habillage{\includestandalone{\standalonepath{activite1}}}
  Soit $SABCD$ une pyramide dont la base $ABCD$ est un carré de centre $O$.\\
  Soit $I$ et $J$ les milieux respectifs des segments $[SC]$ et $[SD]$
  et $K$ le point du segment $[SB]$ tel que $SK=\dfrac{1}{3}SB$.

  \begin{enumerate}
  \item Reproduire et compléter la figure en perspective cavalière.

  \item Dans chacun des cas suivants, que peut-on dire des positions\\
    relatives des droites citées ?
    \begin{enumerate}
    \item $(OB)$ et $(CD)$
    \item $(IJ)$ et $(AB)$
    \item $(OK)$ et $(SD)$
    \item $(OK)$ et $(AS)$
    \end{enumerate}
 
  \item Dans chacun des cas suivants, que peut-on dire des positions
    relatives des plans cités ?
    \begin{colenumerate}{3}
    \item $(OCK)$ et $(SAD)$
    \item $(OIJ)$ et $(SAB)$
    \item $(IJK)$ et $(BAC)$
    \end{colenumerate}
 
  \item Dans chacun des cas suivants, que peut-on dire des positions
    relatives de la droite et du plan cités ?
    \begin{colenumerate}{3}
    \item $(SK)$ et $(OCD)$
    \item $(IJ)$ et $(ABC)$
    \item $(OC)$ et $(ABD)$
    \end{colenumerate}
 
  \item Résumer les positions relatives possibles dans un tableau pour
    chacun des cas suivants :
    \begin{enumerate}
    \item Pour deux droites de l'espace.
    \item Pour deux plans de l'espace.
    \item Pour un plan et une droite de l'espace.
    \end{enumerate}
  \end{enumerate}
\end{activite}

\begin{activite}[Vecteurs de l'espace]

  \begin{cadre}
    On étend à l'espace la notion de vecteur étudiée dans le plan. Un
    vecteur non nul est donc défini par sa direction, son sens et sa
    norme et les propriétés des vecteurs du plan sont aussi étendues
    aux vecteurs de l'espace (relation de Chasles, colinéarité,
    propriétés algébriques).
  \end{cadre}

  \habillage{\includestandalone{\standalonepath{activite2}}}
  On considère la figure ci-contre où $ABCDEFGH$ est un cube et $O$
  est le centre du carré $ABFE$.

  \begin{enumerate}
  \item Citer trois vecteurs égaux.

  \item Exprimer le vecteur $\overrightarrow{AO}$ en fonction de
    $\overrightarrow{AB}$ et $\overrightarrow{AE}$.
  \item Reproduire la figure et placer le point $M$ défini par $\overrightarrow{DM}=\dfrac{1}{3}\overrightarrow{DC}+2\overrightarrow{DH}$.\\
    Citer un plan contenant le point $M$.

  \item Placer le point $N$ défini par $\overrightarrow{BN}=\dfrac{1}{2}\overrightarrow{BG}-\dfrac{1}{4}\overrightarrow{BC}$.\\
    Citer un plan contenant le point $N$.

  \item Conjecturer une caractérisation vectorielle de l'appartenance\\
    d'un point à un plan défini par trois points non alignés.
  \end{enumerate}
\end{activite}

\begin{activite}[Définir une droite ou un plan par un système d'équations]

  \partie[Droite de l'espace]

  Dans un repère $(O\,;
  \overrightarrow{i},\overrightarrow{j},\overrightarrow{k})$ de
  l'espace, on considère une droite $\mathscr{D}$ de vecteur directeur
  $\overrightarrow{u} \begin {pmatrix} 3\\-4\\2 \end{pmatrix}$ et
  passant par le point $A(-6\,;1\,;5)$.\\
  Soit $M(x\,; y\,; z)$ un point de l'espace.

  \begin{enumerate}
  \item
    \begin{enumerate}
    \item Écrire une relation vectorielle traduisant l'appartenance du
      point $M$ à la droite $\mathscr{D}$.

    \item Écrire un système que doivent vérifier les coordonnées
      $(x\,;y\,;z)$ du point $M$ pour que $M$ appartienne à la droite
      $\mathscr{D}$.
    \end{enumerate}

  \item Reprendre la première question avec $\mathscr{D}$ de vecteur
    directeur
    $\vec{u} \begin {pmatrix} \alpha\\\beta\\\gamma \end{pmatrix}$ où
    $\alpha$, $\beta$ et $\gamma$ sont des réels non tous nuls et
    passant par le point $A(x_A\,;y_A\,;z_A)$.
  \end{enumerate}

  \partie[Plan de l'espace]

  Dans un repère $(O\,;\vec{i}, \vec{j},\vec{k})$ de l'espace, on
  considère un plan $\mathscr{P}$ dirigé par les vecteurs
  $\vec{u} \begin {pmatrix} 3\\-4\\2 \end{pmatrix}$  et
  $\vec{v} \begin {pmatrix} 7\\-3\\8 \end{pmatrix}$ et passant par le
  point $A(-6\,; 1\,; 5)$.

  Soit $M(x\,; y\,; z)$ un point de l'espace.

  \begin{enumerate}
  \item
    \begin{enumerate}
    \item Écrire une relation vectorielle traduisant l'appartenance du
      point $M$ au plan $\mathscr{P}$.

    \item Écrire un système que doivent vérifier les coordonnées
      $(x\,; y\,; z)$ du point $M$ pour que $M$ appartienne au plan
      $ \mathscr{P}$.
    \end{enumerate}

  \item Reprendre la première question avec $\mathscr{P}$ dirigé par
    $\vec{u} \begin {pmatrix} \alpha\\\beta\\\gamma \end{pmatrix}$ et
    $\vec{u'} \begin {pmatrix} \alpha'\\\beta'\\\gamma' \end{pmatrix}$
    deux vecteurs non nuls et non colinéaires et passant par le point
    $A(x_A\,;y_A\,;z_A)$.
  \end{enumerate}
\end{activite}

%%%%%%% 
\cours
%%%%%%%

\section{Positions relatives de droites et plans}

\begin{tikzpicture}[overlay]
  \node at (13cm,-2cm) {\includestandalone[scale=.7]{\standalonepath{1definirplan}}};
\end{tikzpicture}

\begin{rappel}
  \begin{enumerate}
  \item Un plan est défini par :
    \begin{itemize}
    \item [\textbullet] trois points non alignés ou
    \item [\textbullet] deux droites sécantes ou
    \item [\textbullet] deux droites strictement parallèles.
    \end{itemize}
  \item Si un plan $\mathscr{P}$ contient deux points distincts $A$ et
    $ B$ de l'espace,\\
    alors il contient la droite $(AB)$. On note
    $(AB)\subset \mathscr{P}$.
  \item Tous les résultats de géométrie plane (théorèmes de Thalès, de
    Pythagore...)\\ s'appliquent dans chaque plan de l'espace.
  \end{enumerate}
\end{rappel}
Dans la suite du paragraphe, $ABCDEFGH$ est un cube.\\

\begin{proprietes}
[Positions relatives de deux droites]
Deux \MotDefinition{droites}{} de l'espace sont soit
\MotDefinition{coplanaires}{} (c'est-à-dire qu'il existe un plan les
contenant toutes les deux), soit non coplanaires (c'est-à-dire qu'il
n'existe aucun plan les contenant toutes les deux).

Si elles sont coplanaires, alors elles sont soit sécantes, soit
parallèles (strictement parallèles ou confondues).
\end{proprietes}

\begin{ltableau}{\linewidth}{4}
  \hline
  \multicolumn {3}{|c|}{Droites coplanaires (dans un même plan)} & 
  Droites non coplanaires \tabularnewline \hline \cline{2-4}
  \rowcolor{FondTableaux}Droites sécantes & \multicolumn {2}{c|}{Droites strictement parallèles} & Droites confondues\tabularnewline\hline
 \includestandalone{\standalonepath{1_droites_secantes}} & 
 \includestandalone{\standalonepath{1_droites_confondues}} & 
 \includestandalone{\standalonepath{1_droites_strt_paralleles}} &
 \includestandalone{\standalonepath{1_droites_non_coplanaires}}\tabularnewline\hline
\end{ltableau}

\begin{proprietes}[Positions relatives de deux plans]
Deux plans de l'espace sont soit sécants (leur intersection est une
droite), soit parallèles.
\end{proprietes}

\begin{ltableau}{\linewidth}{3}
  \hline
  \multicolumn {1}{|c|}{Plans sécants} & \multicolumn {2}{c|}{Plans parallèles} \tabularnewline\hline
  \includestandalone{\standalonepath{1_plans_secants}} & 
  \includestandalone{\standalonepath{1_plans_st_paralleles}} &  
  \includestandalone{\standalonepath{1_plans_confondus}}\tabularnewline\hline
\end{ltableau}

\pagebreak

\begin{proprietes}[Positions relatives d'une droite et d'un plan]
  Une droite et un plan de l'espace sont soit sécants, soit parallèles.
\end{proprietes}

\begin{ltableau}{\linewidth}{4}
  \hline
  \multicolumn {1}{|c|}{Droite et plan sécants} & 
  \multicolumn{2}{c|}{Droite et plan parallèles} \tabularnewline\hline
  \includestandalone{\standalonepath{1_droites_secante_plan}} & 
  \includestandalone{\standalonepath{1_droites_parallele_plan}} &  
  \centering
  \includestandalone{\standalonepath{1_droites_incluse_dans_plan}}\tabularnewline
  \hline
\end{ltableau}

\section{Parallélisme dans l'espace}

\begin{propriete}
  \begin{itemize}
  \item Si deux droites sont parallèles à une même droite alors elles
    sont parallèles entre elles.
  \item Si deux plans sont parallèles à un même plan alors ils sont
    parallèles entre eux.
  \end{itemize}
\end{propriete}
           
\begin{propriete}
  Une droite est parallèle à un plan si et seulement si elle est
  parallèle à une droite de ce plan.
\end{propriete}

\begin{exemple*1}~\par
  \begin{minipage}{.45\linewidth}
    $d$ est parallèle à $d_1$ et $d_1$ est contenue dans le plan $\wp$
    donc $d$ est parallèle à $\wp$.
  \end{minipage}
  \hfill
  \begin{minipage}{.45\linewidth}
    \includestandalone{\standalonepath{proprieteparallelisme}}
  \end{minipage}
\end{exemple*1}

\begin{propriete}
  Si un plan $\wp$ contient deux droites sécantes respectivement
  parallèles à deux droites sécantes d' un plan $\wp'$ alors les plans
  $\wp$ et $\wp'$ sont parallèles.
\end{propriete}
          
\begin{exemple*1}~\par
  \begin{minipage}{.45\linewidth}
    $d_1$ et $d_2$ sont deux droites du plan $\wp$ ; $d_1$ et $d_2$
    sont sécantes et respectivement parallèles à deux droites du plan
    $\wp'$ donc les plans $\wp$ et $\wp'$ sont\linebreak parallèles.
  \end{minipage}
  \hfill
  \begin{minipage}{.45\linewidth}
    \includestandalone{\standalonepath{proprieteparallelisme2}}
  \end{minipage}
\end{exemple*1}

\begin{propriete}
  Si deux plans sont parallèles, alors tout plan qui coupe l'un coupe
  l'autre et les droites\linebreak d'intersection sont parallèles entre elles.
\end{propriete}

\begin{exemple*1}~\par
  \begin{minipage}{.45\linewidth}
    Les plans $\wp$ et $\wp'$ sont parallèles et $\wp$ et $\wp''$ sont
    sécants avec $\wp\cap\wp''=d$, donc $\wp'$ et $\wp''$ sont sécants
    et $\wp'\cap\wp''=d'$ où $d'$ est une droite parallèle à $d$.
  \end{minipage}
  \hfill
  \begin{minipage}{.45\linewidth}
    \includestandalone[width=\linewidth]{\standalonepath{proprieteparallelisme3}}
  \end{minipage}
\end{exemple*1}

\begin{propriete} [Théorème du toit]\label{G2theoremedutoit}
  \begin{minipage}{.45\linewidth}
    Soit $\wp$ et $\wp'$ deux plans distincts, sécants selon une
    droite $\Delta$.

    Si une droite $d$ de $\wp$ est strictement parallèle à une droite
    $d'$ de $\wp'$ alors la droite $\Delta$ intersection de $\wp$ et
    $\wp'$ est parallèle à $d$ et à $d'$.
  \end{minipage}
  \hfill
  \begin{minipage}{.45\linewidth}
    \includestandalone{\standalonepath{proprieteparallelisme4}}
  \end{minipage}
\end{propriete}

\begin{preuve}
  Par hypothèse, $\wp\cap\wp'=\Delta$ et $d//d'$. Les droites $d$ et
  $d'$ sont parallèles donc elles sont coplanaires. Donc, il existe un
  plan $Q$ qui contient à la fois $d$ et $d'$. Mais alors $d$ et
  $\Delta$ sont contenues dans $\wp$ et $d'$ et $\Delta$ sont
  contenues dans $\wp'$. Donc : $\wp\cap Q=d$ et $\wp'\cap Q=d'$.

  Montrons que $d//\Delta$.
  Supposons que $d$ et $\Delta$ ne soient pas parallèles. Donc elles
  sont sécantes en un point $A$.

  $A\in d$ et $A\in\Delta$.
  \begin{itemize}
  \item $A\in d$ et $d=\wp\cap Q$ donc $A\in Q$.
  \item $A\in\Delta$ et $\Delta=\wp\cap\wp'$ donc $A\in\wp'$.  D'où
    $A\in Q\cap\wp'=d'$.
  \end{itemize}

  Par conséquent, $A\in d'$ et $A\in d$ et par conséquent, $d$ et $d'$
  sont sécantes en A.
  Ce qui est absurde, contraire à notre hypothèse.

  Les droites $d$ et $\Delta$ sont donc parallèles.  De plus, comme
  $d$ et $d'$ sont parallèles, on en déduit que les droites $d'$ et
  $\Delta$ sont aussi parallèles.  

  \textbf{Conclusion :} L'intersection de $\wp$ et $\wp'$ est une
  droite $\Delta$ parallèle à la fois à $d$ et à $d'$.
\end{preuve}

\begin{remarque}
  Une autre démonstration de ce théorème est proposée dans l'exercice
  \RefExercice{ex71G2}.
\end{remarque}

\newpage

\begin{methode*1}[Construire la \MotDefinition{section d'un solide par
    un plan}{} %
  \MethodeRefExercice*{ex23G2}\label{Methode-1G2}]
  Il s'agit de construire l'intersection de ce plan avec chacune des
  faces du solide.

  \exercice

  On considère le cube $ABCDEFGH$ ci-contre.  On note $M$ le milieu du
  segment $[EH]$ et $N$ celui de $[FC]$.

  \habillage{\includestandalone{\standalonepath{methodesection1}}}
  Tracer la section de ce cube par le plan $(MNG)$.

  \correction

  L'intersection du plan $(MNG)$ avec la face $HEFG$ est le segment\\
  $[MG]$. Il est visible, on le trace donc en trait plein.

  $G, N$ et $B$ sont alignés, donc l'intersection du plan $(MNG)$ avec\\
  la face $FGCB$ est le segment $[GB]$. Il est visible, on le trace
  donc\\ en trait plein.

  \habillage{\includestandalone{\standalonepath{methodesection2}}}
  Les faces $EHDA$ et $FGCB$ étant parallèles, l'intersection du plan
  $(MNG)$ avec la face $EHDA$ est le segment passant par $M$ et
  parallèle à $(GN)$. Il n'est pas visible, on le trace donc en
  pointillés.

  Notons $P$ le point d'intersection de $(MNG)$ et
  $(EA)$. L'intersection du plan $(MNG)$ avec la face $ABFE$ est le
  segment $[PB]$. Il est visible, on le trace donc en trait plein.

  La section du cube par le plan $(MNG)$ est le polygone $MGBP$
  colorié en rouge. Comme $(MP)//(GB)$, il s'agit d'un trapèze.
\end{methode*1}

\section{Orthogonalité dans l'espace}

  \MotDefinition[Droites orthogonales]{}{}

\begin{definition}[Orthogonalité de deux droites]
  Deux droites sont orthogonales si leurs parallèles passant par un
  même point sont perpendiculaires dans le plan qu'elles définissent.
\end{definition}

\begin{remarque}
  Deux droites perpendiculaires sont orthogonales mais la réciproque
  est fausse.
\end{remarque}

\begin{exemple*1}~\par
  \begin{minipage}{.45\linewidth}
    Dans le cube $ABCDEFGH$ ci-contre, $(EF)//(HG)$ et
    $(HG)\perp (GC)$ donc $(EF)$ et $(GC)$ sont orthogonales. On note
    $(EF)\perp (GC)$.
  \end{minipage}
  \hfill
  \begin{minipage}{.45\linewidth}
    \includestandalone{\standalonepath{exempleorthogonalite}}
  \end{minipage}
\end{exemple*1}

\pagebreak

\begin{definition}[Orthogonalité d’une droite et d’un plan]
  Une droite est orthogonale à un plan lorsqu'elle est orthogonale à
  toutes les droites\\ de ce plan.
\end{definition}
       
\begin{theoreme}
  Si une droite est orthogonale à deux droites sécantes d'un plan
  alors elle est orthogonale\\ à ce plan.
\end{theoreme}

\begin{methode*1}[Démontrer l'orthogonalité de deux droites
  \MethodeRefExercice*{ex30G2}%
  \label{Methode-2G2}]

  \exercice

  Dans le cube $ABCDEFGH$ représenté dans l'exemple précédent,
  démontrer que $(GC)\perp (BD)$.

  \correction

  La droite $(GC)$ est perpendiculaire à $(BC)$ et à $(CD)$ qui sont
  deux droites sécantes du plan $(ABC)$ donc $(GC)$ est orthogonale au
  plan $(ABC)$ donc à toutes les droites de ce plan. En particulier,
  on en déduit que $(GC)\perp (BD)$.
\end{methode*1}

\section{Vecteurs de l'espace}

\begin{cadre}
  On étend à l'espace la définition et les propriétés des
  vecteurs étudiées dans le plan.
\end{cadre}

\begin{proprietes}[Vecteurs colinéaires]
  Deux vecteurs non nuls $\vec{u}$ et $\vec{v}$ sont colinéaires si et
  seulement si il existe un réel $k$ tel que $\vec{v}=k \vec{u}$.  Par
  convention, le vecteur nul est colinéaire à tout vecteur de
  l'espace.
\end{proprietes}

\begin{propriete}[Caractéristique]
  $A$ et  $B$ étant deux points distincts de l'espace, la droite
  $(AB)$ est l'ensemble des points $M$ de l'espace tels que
  $\overrightarrow{AB}$ et $\overrightarrow{AM}$ soient colinéaires.

  On dit que $\overrightarrow{AB}$ est un vecteur directeur de la
  droite $(AB)$.
\end{propriete}

\begin{definition}[\MotDefinition{Vecteurs coplanaires}{}]
  Trois vecteurs non nuls $\vec{u}$, $\vec{v}$ et $\vec{w}$ sont
  coplanaires si et seulement leurs représentants de même origine $A$
  ont des extrémités $B,C$ et $D$ telles que $A, B, C$ et $D$
  appartiennent à un même plan.
\end{definition}

\pagebreak

\begin{propriete}[Caractéristique]
  \begin{minipage}{.45\linewidth}
    $A$, $B$ et $C$ étant trois points non alignés de l'espace, le
    plan $(ABC)$ est l'ensemble des points $M$ de l'espace tels que :

    $\overrightarrow{AM}=\alpha\overrightarrow{AB}+\beta\overrightarrow{AC}$,
    avec $\alpha$ et $\beta$ deux nombres réels.

    On dit que $\overrightarrow{AB}$ et $\overrightarrow{AC}$ dirigent
    le plan $(ABC)$.
  \end{minipage}\quad
  \begin{minipage}{.45\linewidth}
    \includestandalone{\standalonepath{veccoplanaires2}}
  \end{minipage}
\end{propriete}


\begin{preuve}
  $A$, $B$ et $C$ ne sont pas alignés. Les vecteurs
  $\overrightarrow{AB}$ et $ \overrightarrow{AC}$ n'étant pas
  colinéaires, $(A\,;\overrightarrow{AB},\overrightarrow{AC})$ est donc
  un repère du plan $(ABC)$.
  \begin{itemize}
  \item Si $M$ appartient à $(ABC)$, alors $M$, $A$, $B$ et $C$ étant
    coplanaires, il existe $\alpha$ et $\beta$ deux nombres réels tels
    que
    $\overrightarrow{AM}=\alpha\overrightarrow{AB}+\beta\overrightarrow{AC}$.
  \item Réciproquement, si $M$ est un point de l'espace tel que \\
    $\overrightarrow{AM}=\alpha\overrightarrow{AB}+\beta\overrightarrow{AC}$,
    avec $\alpha$ et $\beta$ deux nombres réels, alors il existe un
    point $N$
    de la droite $(AB)$ tel que $\overrightarrow{AN}=\alpha\overrightarrow{AB}$.\\
    $\overrightarrow{AM}=\alpha\overrightarrow{AB}+\beta\overrightarrow{AC}\Leftrightarrow
    \overrightarrow{NM}=\beta\overrightarrow{AC}$.
    $M$ est donc un point de la droite parallèle à $(AC)$ passant par
    $N$. Donc, comme $N\in(ABC)$, $M\in(ABC)$.
  \end{itemize}
\end{preuve}

\begin{propriete}
  \begin{minipage}{.45\linewidth}
    Soit trois vecteurs non nuls $\vec{u}$, $\vec{v}$ et $\vec{w}$
    tels que $\vec{u}$ et $\vec{v}$ ne sont pas colinéaires.

    $\vec{u}$, $\vec{v}$ et $\vec{w}$ sont coplanaires si et seulement
    si il existe deux réels $\alpha$ et $\beta$ tels que\\
    $\vec{w}=\alpha\vec{u}+\beta\vec{v}$.
  \end{minipage}\quad
  \begin{minipage}{.45\linewidth}
    \includestandalone{\standalonepath{veccoplanaires}}
  \end{minipage}
\end{propriete}

\begin{preuve}
  Soit $A$, $B$, $C$ et $M$ les points de l'espace tels que
  $\vec{w}=\overrightarrow{AM}$, $\vec{u}=\overrightarrow{AB}$ et
  $\vec{w}=\overrightarrow{AC}$.

  $\vec{u}$, $\vec{v}$ et $\vec{w}$ sont coplanaires si et seulement
  si $A$, $B$, $C$ et $M$ sont coplanaires, c'est-à-dire si et
  seulement si il existe deux réels $\alpha$ et $\beta$ tels que
  $\vec{AM}=\alpha\vec{AB}+\beta\vec{AC}\Leftrightarrow
  \vec{w}=\alpha\vec{u}+\beta\vec{v}$.
\end{preuve}

% Méthode 3
\begin{methode*1}[Démontrer que quatre points sont coplanaires%
  \MethodeRefExercice*{ex37G2}%
  \label{Methode-3G2}] Il s'agit de démontrer que trois vecteurs sont
  coplanaires en écrivant l'un en fonction des deux autres.

  \begin{tikzpicture}[general, overlay,scale=.6,xshift=20cm,yshift=-10cm]
\clip(-4.3,-2.88) rectangle (7.14,6.3);
\draw [dash pattern=on 5pt off 5pt,color=J2] (3.,1.)-- (-3.,0.);
\draw [color=Noir] (-3.,0.)-- (2.06,-2.3);
\draw [color=Noir] (2.06,-2.3)-- (3.,1.);
\draw  (0.,5.)-- (-3.,0.);
\draw  (0.,5.)-- (2.06,-2.3);
\draw  (0.,5.)-- (3.,1.);
\draw [->] (3.,1.) -- (-2.06,3.3);
\draw [fill=Noir] (0.,5.) circle (0.5pt);
\draw[color=Noir] (-0.06,5.36) node {$A$};
\draw [fill=Noir] (2.06,-2.3) circle (0.5pt);
\draw[color=Noir] (2.04,-2.52) node {$B$};
\draw [fill=Noir] (3.,1.) circle (0.5pt);
\draw[color=Noir] (3.14,1.2) node {$D$};
\draw [fill=Noir] (-3.,0.) circle (0.5pt);
\draw[color=Noir] (-3.16,0.32) node {$C$};
\draw [color=B2] (1.03,1.35)-- ++(-2.0pt,0 pt) -- ++(4.0pt,0 pt) ++(-2.0pt,-2.0pt) -- ++(0 pt,4.0pt);
\draw[color=B2] (0.8,1.3) node {$I$};
\draw [color=B1] (-2.008235294117647,1.6529411764705884)-- ++(-1.5pt,-1.5pt) -- ++(3.0pt,3.0pt) ++(-3.0pt,0) -- ++(3.0pt,-3.0pt);
\draw[color=B1] (-2.12,2.) node {$E$};
\draw [color=B1] (2.0136,2.3152)-- ++(-1.5pt,-1.5pt) -- ++(3.0pt,3.0pt) ++(-3.0pt,0) -- ++(3.0pt,-3.0pt);
\draw[color=B1] (2.16,2.6) node {$F$};
\draw [fill=Noir] (-2.06,3.3) circle (0.5pt);
\draw[color=Noir] (-1.92,3.75) node {$G$};
\end{tikzpicture}

  \exercice

  Soit $ABCD$ un tétraèdre, $I$ le milieu de $[AB]$ ; $E$ et $F$ les
  points définis par
  $\overrightarrow{AE}=\dfrac{2}{3}\overrightarrow{AC}$ et
  $\overrightarrow{AF}=\dfrac{2}{3}\overrightarrow{AD}$ et G le point
  tel que $BCGD$ soit un parallélogramme.

\begin{enumerate}
\item Exprimer les vecteurs $\overrightarrow{IE}$,
  $ \overrightarrow{IF}$ et $\overrightarrow{IG}$ en fonction de
  $\overrightarrow{AB}$, $\overrightarrow{AC}$ et
  $\overrightarrow{AD}$.
\item En déduire qu'il existe deux réels $\alpha$ et $\beta$ tels que
  $\overrightarrow{IG}=\alpha\overrightarrow{IE}+\beta\overrightarrow{IF}$.
\item En déduire que les points $I, E, G$ et $F$ sont coplanaires.
\end{enumerate}

\correction

\begin{enumerate}
\item
  $ \begin{aligned}[t]
 \overrightarrow{IE} &= \overrightarrow{IA}+\overrightarrow{AE}=-\dfrac{1}{2}\overrightarrow{AB}+\dfrac{2}{3}\overrightarrow{AC}.\\
%
  \overrightarrow{IF}
  &=\overrightarrow{IA}+\overrightarrow{AF}=-\dfrac{1}{2}\overrightarrow{AB}+\dfrac{2}{3}\overrightarrow{AD}.\\
%
   \overrightarrow {IG}&=\overrightarrow{IA}+\overrightarrow{AD}+\overrightarrow{DG}\\
    &=-\dfrac{1}{2}\overrightarrow{AB}+\overrightarrow{AD}+\overrightarrow{BC}\\
    &=-\dfrac{1}{2}\overrightarrow{AB}+\overrightarrow{AD}+\overrightarrow{BA}+\overrightarrow{AC}\\
    &=-\dfrac{3}{2}\overrightarrow{AB}+\overrightarrow{AD}+\overrightarrow{AC}.
  \end{aligned}
  $

\item Il existe deux réels $\alpha$ et $\beta$ tels que
  $\overrightarrow{IG}=\alpha\overrightarrow{IE}+\beta\overrightarrow{IF}$
  \\[1mm]
  soit
  $
  -\dfrac{3}{2}\overrightarrow{AB}+\overrightarrow{AD}+\overrightarrow{AC}=-\dfrac{\alpha}{2}\overrightarrow{AB}+\dfrac{2\alpha}{3}\overrightarrow{AC}-\dfrac{\beta}{2}\overrightarrow{AB}+\dfrac{2\beta}{3}\overrightarrow{AD}$

\vspace{1mm}

Pour obtenir cette égalité, il suffit de prendre $\alpha$ et $\beta$
tels que : \\[1mm]
$-\dfrac{3}{2}=-\dfrac{\alpha}{2}-\dfrac{\beta}{2}$ et $\dfrac{2}{3} \alpha=1$ et $\dfrac{2}{3} \beta=1$ , soit, $ \alpha=\dfrac{3}{2}$ et $\beta=\dfrac{3}{2}$.\\[1mm]
D'où
$\overrightarrow{IG}=\dfrac{3}{2}\overrightarrow{IE}+\dfrac{3}{2}\overrightarrow{IF}$

\item On en déduit que les vecteurs $\overrightarrow{IE}$, $
  \overrightarrow{IF}$ et $\overrightarrow{IG}$ sont coplanaires, donc
  les points $I, E, G$ et $F$ sont coplanaires.
\end{enumerate}

\vspace{-2\baselineskip}
\end{methode*1}

\enlargethispage{1cm}


\section{Repérage dans l'espace}

\begin{theoreme}
  Si $O$ est un point de l'espace et $\overrightarrow{i}$,
  $ \overrightarrow{j}$ et $\overrightarrow{k}$ trois vecteurs non
  coplanaires, alors pour tout point $M$ de l'espace, il existe un
  unique triplet de réels $(x\,;y\,;z)$ tels que :
  \[
  \overrightarrow{OM}=x\overrightarrow{i}+y\overrightarrow{j}+z\overrightarrow{k}.
  \]
\end{theoreme}


\newpage

\begin{preuve}
  \begin{itemize}
  \item Existence\\
    \begin{minipage}{.65\linewidth}
      Soit $\wp$ le plan passant par $O$ et dirigé par les vecteurs $\overrightarrow{i}$ et $ \overrightarrow{j}$ (qui ne sont pas colinéaires car $\overrightarrow{i}$, $ \overrightarrow{j}$ et $\overrightarrow{k}$ sont non\\ coplanaires).\\
      Soit $M'$ le point d'intersection de $\wp$ et de la droite
      parallèle à $(O\overrightarrow{k})$ passant par $M$.
      $\overrightarrow{i}$, $ \overrightarrow{j}$ et
      $\overrightarrow{OM'}$ sont coplanaires avec
      $\overrightarrow{i}$ et $ \overrightarrow{j}$ non colinéaires,
      donc il existe deux réels $x$ et $y$ tels que
      $ \overrightarrow{OM'}=x\overrightarrow{i}+y\overrightarrow{j}$.
      D'autre part, $\overrightarrow{MM'}$ et $\overrightarrow{k}$
      sont colinéaires, donc il existe un réel $z$ tel que
      $\overrightarrow{MM'}=z\overrightarrow{k}$.  D'où
      $
      \overrightarrow{OM}=\overrightarrow{OM}+\overrightarrow{MM'}=x\overrightarrow{i}+y\overrightarrow{j}+z\overrightarrow{k}$
    \end{minipage}
    \begin{minipage}{.35\linewidth}
      \includegraphics[width=\linewidth]{repere}\\
    \end{minipage}
  \item Unicité\\
    Supposons qu'il existe deux triplets de réels $(x\,;y\,;z)$ et
    $(x';y';z')$ tels que\\
    $ \overrightarrow{OM}=x\overrightarrow{i}+y\overrightarrow{j}+z\overrightarrow{k}=x'\overrightarrow{i}+y'\overrightarrow{j}+z'\overrightarrow{k}$.\\
    On a alors $ (z'-z)\overrightarrow{k}=(x-x')\overrightarrow{i}+(y-y')\overrightarrow{j}$.\\
    Comme $\overrightarrow{i}$, $ \overrightarrow{j}$ et
    $\overrightarrow{k}$ ne sont pas coplanaires, il n'existe pas de
    couple de réels $(\alpha;\beta)$ tels que
    $\overrightarrow{k}=\alpha\overrightarrow{i}+\beta\overrightarrow{j}$,
    on en déduit que $z-z'=0$, et par suite, que $x=x'$, $y=y'$ et
    $z=z'$.
  \end{itemize}
\end{preuve}

\begin{definition}
  $(x\,;y\,;z)$ est le triplet de \MotDefinition{coordonnées}{} du
  point $M$ dans le repère $(O\,;\vec{i},\vec{j},\vec{k})$.

  $x$ est l'abscisse de $M$, $y$ est l'ordonnée de $M$ et $z$ est la
  cote de $M$.  

  $(x;y;z)$ sont aussi les coordonnées du vecteur
  $\overrightarrow{OM}$ dans le repère
  $(O\,;\vec{i},\vec{j},\vec{k})$.
\end{definition}
 
\begin{proprietes}
  Dans un repère $(O\ ;\vec{i},\vec{j},\vec{k})$ de l'espace,
  soit $A(x_A;y_A;z_A)$  et $B(x_B;y_B;z_B)$.  Alors :

  $\overrightarrow{AB}\begin{pmatrix}x_B-x_A\\y_B-y_A\\z_B-z_A \end{pmatrix}$

  et le milieu $K$ de $[AB]$ a pour coordonnées :   $K$
  $(\dfrac{x_A+x_B}{2};\dfrac{y_A+y_B}{2} ;\dfrac{z_A+z_B}{2})$.

  \vspace{3mm}

  Si de plus $(O\ ;\ \vec{i},\vec{j},\vec{k})$ est orthonormé,
  $AB=\sqrt{(x_B-x_A)^2+(y_B-y_A)^2+(z_B-z_A)^2}$.
\end{proprietes}

\begin{proprietes}
  Dans un repère$(O\ ;\ \vec{i},\vec{j},\vec{k})$ de l'espace,
  soit $\vec{u}\begin {pmatrix} x\\y\\z \end{pmatrix}$,
  $\vec{v}\begin {pmatrix} x'\\y'\\z' \end{pmatrix}$ deux vecteurs et
  $k$ un nombre réel. Alors :

  $\vec{u}+\vec{v} \begin {pmatrix} x+x'\\y+y'\\z+z' \end{pmatrix}$ et
  $k \vec{u} \begin {pmatrix} kx\\ky\\kz \end{pmatrix}$.

  Si de plus $(O\,;\vec{i},\vec{j},\vec{k})$ est orthonormé,
  $||\vec{u}||=\sqrt{x^2+y^2+z^2}$.
\end{proprietes}

% Méthode 4 
\begin{methode*1}[La coplanarité
  de points en utilisant leurs coordonnées \MethodeRefExercice*{ex41G2}\label{Methode-4G2}]

  Il s'agit de démontrer que trois vecteurs sont coplanaires en
  écrivant l'un des vecteurs en fonction des deux autres.

  \exercice 

  Dans un repère $(O\,;\vec{i},\vec{j},\vec{k})$ de l'espace, Démontrer
  que les points $A(1\,;2\,;0)$, $B(-1\,;1\,;1)$, $C(1\,;4\,;1)$ et $D(3\,;-1\,;-3)$
  sont coplanaires.

  \correction

  $\overrightarrow{AB} 
  \begin{pmatrix} 
    -2\\
    -1\\
    \hphantom{-}1
  \end{pmatrix}$ ; $\overrightarrow{AC} \begin{pmatrix} 0\\2\\1
  \end{pmatrix}$ et $\overrightarrow{AD} \begin{pmatrix} \hphantom{-}2\\-3\\-3
  \end{pmatrix}$.

  $\overrightarrow{AB}$ et $\overrightarrow{AC}$ ne sont pas
  colinéaires, car leurs coordonnées ne sont pas proportionnelles.

  $\overrightarrow{AD}=\alpha\overrightarrow{AB}+\beta\overrightarrow{AC}\Leftrightarrow%
  \begin{cases}
    \hphantom{-}2=-2\alpha\\
    -3=-\alpha+2\beta\\
    -3=\alpha+\beta
  \end{cases}\Leftrightarrow
  \begin{cases}
    \alpha=-1\\
    \beta=-2
  \end{cases}$.

  Le système ayant un unique couple solution, les vecteurs
  $\overrightarrow{AB}$, $\overrightarrow{AC}$ et
  $\overrightarrow{AD}$ sont coplanaires, donc les points $A$, $B$,
  $C$ et $D$ sont coplanaires.
\end{methode*1}

\section{Représentation paramétrique de droites et de plans}

\begin{propriete}
  Dans un repère $(O\ ;\vec{i},\vec{j},\vec{k})$ de l'espace, on
  considère la droite $\mathscr{D}$ passant par $A(x_A\ ;y_A\ ;z_A)$
  et de vecteur directeur
  $\vec{u} \begin {pmatrix} \alpha\\\beta\\\gamma \end{pmatrix}$.

  $M(x;y;z)\in \mathscr{D}$ si et seulement si il existe un réel $t$
  tel que :
  \begin{center}
    $\begin{cases}x=x_A+t\alpha \\y=y_A+t\beta
      \\z=z_A+t\gamma \end{cases}$
  \end{center}
\end{propriete}

\begin{preuve}
  $M(x;y;z)\in \mathscr{D}$ si et seulement si $\overrightarrow{AM}$
  et $\vec{u}$ sont colinéaires, c'est-à-dire qu'il existe un réel $t$
  tel que $\overrightarrow{AM}=t\overrightarrow{u}$.  Cela se traduit
  en terme de coordonnées par :

  $\begin{cases}x-x_A=t\alpha \\y-y_A=t\beta
    \\z-z_A=t\gamma \end{cases}\Leftrightarrow\begin{cases}x=x_A+t\alpha
    \\y=y_A+t\beta \\z=z_A+t\gamma \end{cases}$
\end{preuve}

\begin{definition}
  On dit que le système d'équations : \\
  $\begin{cases}
    x=x_A+t\alpha    \\
    y=y_A+t\beta \\
    z=z_A+t\gamma 
  \end{cases}$
  où $t\in\mathbb{R}$ est une \MotDefinition{représentation
    paramétrique}{} de la droite $\mathscr{D}$ passant par
  $A(x_A;y_A;z_A)$ et de vecteur directeur $\vec{u}
  \begin {pmatrix} 
    \alpha\\
    \beta\\
    \gamma 
  \end{pmatrix}$.
\end{definition}

\begin{remarque}
  Un exemple de cette définition est proposé dans l'exercice
  \RefExercice{ex49G2}.
\end{remarque}

\begin{propriete}
  Dans un  repère $(O\,;\vec{i},\vec{j},\vec{k})$  de l'espace, le plan
  $\mathscr{P}$ passant par $A(x_A;y_A;z_A)$ et de vecteurs\\ directeurs
  $\vec{u} \begin {pmatrix} \alpha\\\beta\\\gamma \end{pmatrix}$ et
  $\vec{v} \begin {pmatrix} \alpha'\\\beta'\\\gamma' \end{pmatrix}$. 

  $M(x\,;y\,;z)\in \mathscr{P}$ si et seulement si il existe deux réels
  $t$ et $t'$ tels que :
  \begin{center}
    $\begin{cases}
      x=x_A+t\alpha+t'\alpha' \\
      y=y_A+t\beta +t'\beta'  \\
      z=z_A+t\gamma +t'\gamma' 
    \end{cases}$
  \end{center}
\end{propriete}

\begin{preuve}
  $M(x\ ;\ y\ ;\ z)\in \mathscr{P}$ si et seulement si
  $\overrightarrow{AM}$, $\vec{u}$ et $\vec{v}$ sont coplanaires,
  c'est-à-dire qu'il existe deux réels $t$ et $t'$ tels que
  $\overrightarrow{AM}=t\overrightarrow{u}+t'\overrightarrow{v}$.
  Cela se traduit en terme de coordonnées par :
\begin{center}
  $\begin{cases}x-x_A=t\alpha+t'\alpha' \\
    y-y_A=t\beta +t'\beta' \\
    z-z_A=t\gamma + t'\gamma'
  \end{cases}
  \Leftrightarrow
  \begin{cases}
    x=x_A+t\alpha+t'\alpha' \\
    y=y_A+t\beta +t'\beta' \\
    z=z_A+t\gamma +t'\gamma' 
  \end{cases}$.
\end{center}
\end{preuve}

\begin{definition}
  On dit que le système d'équations :
  \\$\begin{cases}x=x_A+t\alpha+t'\alpha'
    \\y=y_A+t\beta +t'\beta' \\z=z_A+t\gamma +t'\gamma'
  \end{cases}$ où $t\in\mathbb{R}$ et
  $t'\in\mathbb{R}$
  est une représentation paramétrique du plan $\mathscr{P}$
  passant par $A(x_A\,;y_A\,;z_A)$ et de vecteurs directeurs $\vec{u}
  \begin {pmatrix} \alpha\\\beta\\\gamma \end{pmatrix}$ et
  $\vec{v} \begin {pmatrix} \alpha'\\\beta'\\\gamma' \end{pmatrix}$.
\end{definition}

\begin{remarque}
  Un exemple de cette définition est proposé dans l’exercice
  \RefExercice{ex54G2}.
\end{remarque}

\begin{remarque}
  Il existe une infinité de représentations paramétriques, que ce soit
  pour une droite ou pour un plan.
\end{remarque}

% Méthode 5
\begin{methode*1}[Étudier des positions relatives
  \MethodeRefExercice*{ex52G2}\label{Methode-5G2}]

  \exercice

  \vspace{-\baselineskip}

  \bgroup
  \def\columnseprule{.5pt}
  \begin{multicols}{2}
    Étudier  les positions relatives des
    droites $d$ et  $d'$ puis du plan $\wp$ et de la droite $d'$. On
    donnera leur intersection éventuelle.

    Le plan $\wp$ a pour représentation paramétrique :

    $\begin{cases} x=1-2t+3t'\\y=-2+t-t'\\z=3-t \end{cases}$ avec
    $t\in\mathbb{R}$ et $t'\in\mathbb{R}$

    \columnbreak

    Les droites $d$ et $d'$ ont pour représentation \mbox{paramétrique} :

    $d$ : $\begin{cases} x=2+4t\\y=5-2t\\z=1+2t \end{cases}$ avec
    $t\in\mathbb{R}$ et 

    $d'$ : $\begin{cases} x=4-t\\y=-2+t\\z=1+3t \end{cases}$ avec
    $t\in\mathbb{R}$
  \end{multicols}
  \egroup

  \correction

Attention : la même lettre $t$ désigne deux paramètres différents. Il faut donc changer de lettre dans les résolutions de système pour les différencier.\\

$\wp$ est dirigé par les vecteurs $\vec{u} \begin {pmatrix} -2\\1\\-1 \end{pmatrix}$ et $\vec{v} \begin {pmatrix} 3\\-1\\-1 \end{pmatrix}$.\\
$d$ et $d'$ ont pour vecteur directeur respectif $\vec{w} \begin {pmatrix} 4\\-2\\2 \end{pmatrix}$ et $\vec{w'} \begin {pmatrix} -1\\1\\3 \end{pmatrix}$.\\
On remarque que $\overrightarrow{w}=-2\overrightarrow{u}$ donc $d$ est parallèle à $\wp$. Le point $A(2;5;1)$ appartient à $d$.
S'il appartient à $\wp$ alors $d\subset\wp$, sinon $d$ est strictement parallèle à $\wp$.\\
Or,  $\begin{cases} 2=1-2t+3t'\\5=-2+t-t'\\1=3-t \end{cases}\Leftrightarrow\begin{cases} -2t+3t'=1\\t-t'=7\\t=2 \end{cases}\Leftrightarrow\begin{cases} t'=\dfrac{5}{3}\\t'=-5\\t=2 \end{cases}$ \\
Le système n'ayant pas de solution, $A\not\in \wp$ donc $d$ est strictement parallèle à $\wp$.\\
Déterminons maintenant $\wp\cap d'$ :
$M\in\wp\cap d'\Leftrightarrow$ il existe trois réels $t$, $t'$ et $k$ tels que :\\
$\begin{cases} x=1-2t+3t'\\y=-2+t-t'\\z=3-t\\x=4-k\\y=-2+k\\z=1+3k \end{cases}\Leftrightarrow \begin{cases} 4-k=1-2t+3t'\\-2+k=-2+t-t'\\1+3k=3-t\\x=4-k\\y=-2+k\\z=1+3k \end{cases}\Leftrightarrow \begin{cases} -k+2t-3t'=-3\\k-t+t'=0\\3k+t=2\\x=4-k\\y=-2+k\\z=1+3k\end{cases}$\\
En finissant la résolution du système, on obtient 
 $t'=\dfrac{14}{5}$ ; $t=\dfrac{52}{20}$ et $k=\dfrac{-1}{5}=-0,2$, ce qui nous donne $x=4,2$ ; $y=-2,2$ et $z=0,4$.\\ 
Ainsi, $\wp$ et $d'$ sont sécantes au point $K(4,2;-2,2;0,4)$

\vspace{-\baselineskip}

  % \textbf{Attention :} la même lettre $t$ désigne deux paramètres
  % différents. Il faut donc changer de lettre dans les résolutions de
  % système pour les différencier.

  % $d$ et $d'$ ont pour vecteur directeur respectif $\vec{w}(4;-2;2)$
  % et $\vec{w'}(-1;1;3)$ qui ne sont pas colinéaires, donc $d$ et $d'$
  % ne sont pas parallèles.  
  % Cherchons si elles sont sécantes ou non coplanaires : 

  % Un point $M(x;y;z)$ appartient à $d$ et à $d'$ si et seulement si : 

  % $\begin{cases}
  %   2+4t=4-k\\5-2t=-2+k\\1+2t=1+3k\\x=4-k\\y=-2+k\\z=1+3k \end{cases}\Leftrightarrow \begin{cases}
  %   4t=2-k\\6=-1+4k\\2t=3k\\x=4-k\\y=-2+k\\z=1+3k \end{cases}\Leftrightarrow \begin{cases}
  %   4t=2-k\\k=\dfrac{7}{4}\\t=\dfrac{3}{2}\times\dfrac{7}{4}=\dfrac{21}{8}\\x=4-k\\y=-2+k\\z=1+3k\end{cases}\Leftrightarrow \begin{cases}
  %   t=\dfrac{1}{16}\\k=\dfrac{7}{4}\\t=\dfrac{3}{2}\times\dfrac{7}{4}=\dfrac{21}{8}\\x=4-k\\y=-2+k\\z=1+3k\end{cases}$\\Ce
  % système n'admet donc pas de solution, les droites $d$ et $d'$ sont
  % donc non coplanaires.

  % Déterminons maintenant $\wp\cap d'$ :

  % $M\in\wp\cap d'\Leftrightarrow$ il existe trois réels $t$, $t'$ et
  % $k$ tels que :

  % $\begin{cases}
  %   x=1-2t+3t'\\y=-2+t-t'\\z=3-t\\x=4-k\\y=-2+k\\z=1+3k \end{cases}\Leftrightarrow \begin{cases}
  %   4-k=1-2t+3t'\\-2+k=-2+t-t'\\1+3k=3-t\\x=4-k\\y=-2+k\\z=1+3k \end{cases}\Leftrightarrow \begin{cases}
  %   -k+2t-3t'=-3\\k-t+t'=0\\3k+t=2\\x=4-k\\y=-2+k\\z=1+3k\end{cases}$

  % $\Leftrightarrow 
  % \begin{cases} 
  %   t-2t'=-3\\
  %   3k-3t+3t'=0
  %   3k+t=2
  %   x=4-k
  %   y=-2+k
  %   z=1+3k \end{cases}\Leftrightarrow \begin{cases} 4t-8t'=-12
  %   -4t+3t'=-2
  %   2k+t=2
  %   x=4-k
  %   y=-2+k
  %   z=1+3k \end{cases}\Leftrightarrow \begin{cases}-5t'=-14
  %   4t=3t'+2
  %   2k=-t+2
  %   x=4-k
  %   y=-2+k
  %   z=1+3k \end{cases}$
  % $\Leftrightarrow \begin{cases} t'=\dfrac{14}{5}
  %   t=\dfrac{52}{20}
  %   k=\dfrac{-1}{5}=-0,2
  %   x=4,2
  %   y=-2,2
  %   z=0,4 \end{cases}$
  
  % Ainsi, $\wp$ et $d'$ sont sécantes au point $K(4,2;-2,2;0,4)$
\end{methode*1}

%%%%%%%%%%%%%%
\exercicesbase
%%%%%%%%%%%%%%

\begin{colonne*exercice}

\serie{Activit\'{e}s mentales}

Pour les exercices \RefExercice*{G2Exo1} à \RefExercice*{G2Exo3},
$ABCDEFGH$ est un pavé droit ; $I$, $J$, $K$ et $L$ sont les milieux
respectifs de $[DH], [HG], [AB]$ et $[BF]$.

\begin{center}
  \includestandalone{\standalonepath/activitementale1}
\end{center}

\begin{exercice*}\label{G2Exo1}
Donner la position relative des deux droites\\ citées : 
\begin{enumerate}
\item $(DB)$ et $(EF)$ ;
\item $(IJ)$ et $(AF)$ ;
\item $(IC)$ et $(AB)$ ;
\item $(JF)$ et $(EH)$.
\end{enumerate}
\end{exercice*}
\begin{corrige}
  \begin{enumerate}
\item $(DB)$ et $(EF)$ non coplanaires ;
\item $(IJ)$ et $(AF)$ parallèles;
\item $(IC)$ et $(AB)$ non coplanaires ;
\item $(JF)$ et $(EH)$ sécantes.
\end{enumerate}
\end{corrige}

\begin{exercice*}
  Donner la position relative des deux plans cités :
  \begin{enumerate}
  \item $(DCG)$ et $(AEF)$ ;
  \item $(IJA)$ et $(HDC)$ ;
  \item $(IJE)$ et $(CKL)$.
  \end{enumerate}
\end{exercice*}
\begin{corrige}
  \begin{enumerate}
\item $(DCG)$ et $(AEF)$ parallèles ;
\item $(IJA)$ et $(HDC)$ sécants selon $(IJ)$;
\item $(IJE)$ et $(CKL)$ parallèles.
\end{enumerate}
\end{corrige}

\begin{exercice*}\label{G2Exo3}
  Donner la position relative de la droite et du plan cités :
  \begin{enumerate}
  \item $(IJ)$ et $(ABF)$ ;
  \item $(IJ)$ et $(BCG)$ ;
  \item $(KE)$ et $(ABF)$.
  \end{enumerate}
\end{exercice*}
\begin{corrige}
  \begin{enumerate}
  \item $(IJ)$ parallèle à $(ABF)$ ;
  \item $(IJ)$ et $(BCG)$ sécants ;
  \item $(KE)$ est incluse dans $(ABF)$.
  \end{enumerate}  
\end{corrige}

\begin{exercice*}\label{G2Exo4}
  $ABCDEFGH$ est un cube et $I$ est le milieu de $[AB]$.

  \begin{center}
    \includestandalone{\standalonepath/activitementale4}
  \end{center}

  Quelle est la nature de la section du cube par :
  \begin{enumerate}
  \item le plan $(IFG)$ ?
  \item le plan $(IFC)$ ?
  \end{enumerate}
\end{exercice*}
\begin{corrige}
\begin{enumerate}
\item un rectangle
\item un triangle isocèle en $I$
% \item  $(IF)$ et $(FG)$ sont orthogonales.
% \item  $(IF)$ et $(FH)$ ne sont pas orthogonales
% \item  $(BF)$ et $(EH)$ sont orthogonales.
% \item  $(BF)$ et $(AC)$ sont orthogonales.
\end{enumerate}
\end{corrige}

\columnbreak

\begin{exercice*}
  $ABCDEFGH$ est un cube et $I$ est le milieu de $[AB]$ (voir figure
  de l'exercice \RefExercice*{G2Exo4}).

  Les droites suivantes sont-elles orthogonales ?
  \begin{enumerate}
  \item $(IF)$ et $(FG)$ ?
  \item $(IF)$ et $(FH)$?
  \item $(BF)$ et $(EH)$?
  \item $(BF)$ et $(AC)$?
  \end{enumerate}
\end{exercice*}
\begin{corrige}
  \begin{enumerate}
\item  $(IF)$ et $(FG)$ sont orthogonales.
\item  $(IF)$ et $(FH)$ ne sont pas orthogonales
\item  $(BF)$ et $(EH)$ sont orthogonales.
\item  $(BF)$ et $(AC)$ sont orthogonales.
\end{enumerate}
\end{corrige}

\begin{exercice*}
  $ABCDEFGH$ est un cube et $I$ est le milieu de $[AB]$ (voir figure
  de l'exercice \RefExercice*{G2Exo4}).

  Compléter les égalités vectorielles suivantes :
  \begin{enumerate}
  \item
    $\overrightarrow{AI}+\overrightarrow{CD}-\overrightarrow{CI}=\overrightarrow{F...}$
  \item
    $\overrightarrow{AH}+\overrightarrow{CD}-\overrightarrow{FG}=\overrightarrow{B...}$
  \item
    $\overrightarrow{FD}+\overrightarrow{CB}+\overrightarrow{DG} =...$
  \end{enumerate}
\end{exercice*}
\begin{corrige}
  \begin{enumerate}
\item  $\overrightarrow{AI}+\overrightarrow{CD}-\overrightarrow{CI}=\overrightarrow{FG}$ ;
\item $\overrightarrow{AH}+\overrightarrow{CD}-\overrightarrow{FG}=\overrightarrow{BE}$ ;
\item $\overrightarrow{FD}+\overrightarrow{CB}+\overrightarrow{DG} =\overrightarrow{0}$.
 \end{enumerate}
\end{corrige}

\begin{exercice*}
  $ABCDEFGH$ est un cube et $I$ est le milieu de $[AB]$ (voir figure
  de l'exercice 4).

  \begin{enumerate}
  \item Exprimer le vecteur $\overrightarrow{FI}$ en fonction des
    vecteurs $\overrightarrow{AB}$ et $\overrightarrow{AE}$.
  \item $O$ étant le centre du cube, exprimer le vecteur
    $\overrightarrow{AO}$ en fonction des vecteurs
    $\overrightarrow{AB}$, $\overrightarrow{AD}$ et
    $\overrightarrow{AE}$.
  \end{enumerate}
\end{exercice*}
\begin{corrige}
  \begin{enumerate}
\item  $\overrightarrow{FI}=-\dfrac{1}{2}\overrightarrow{AB}-\overrightarrow{AE}$.
\item $\overrightarrow{AO}=\dfrac{1}{2}\overrightarrow{AB}+\dfrac{1}{2}\overrightarrow{AD}+\dfrac{1}{2}\overrightarrow{AE}$.
\end{enumerate}
\end{corrige}

\begin{exercice*}
  Dans un repère $(O\ ;\ \vec{i},\vec{j},\vec{k})$ de l'espace, on
  considère les points $A(-3\,;2\,;4)$ ; $B(-1\,;1\,;0)$ et $C(2\,;-3\,;5)$.
  \begin{enumerate}
  \item Donner les coordonnées des vecteurs $\overrightarrow{AB}$ ;
    $\overrightarrow{AC}$ et $\overrightarrow{BC}$.

  \item Donner les coordonnées des vecteurs :

    $\overrightarrow{u}=2\overrightarrow{AB}-\overrightarrow{AC}$ et
    $\overrightarrow{v}=\overrightarrow{AC}+3\overrightarrow{BC}$.
  \end{enumerate}
\end{exercice*}
\begin{corrige}
  \begin{enumerate}
\item  $\overrightarrow{AB}(2;-1;-4)$ ; $\overrightarrow{AC}(5;-5;1)$ et $\overrightarrow{BC}(3;-4;5)$.
\item 
$\overrightarrow{u}(-1;3;-9)$ et $\overrightarrow{v}(14;-17;16)$.
\end{enumerate}
\end{corrige}

\begin{exercice*}
  Dans un repère$(O\,;\vec{i},\vec{j},\vec{k})$ de l'espace, on
  considère les points $A(2\,;5\,;-1)$ \,; $B(0\,;3\,;4)$ et le vecteur
  $\overrightarrow{u}(2\,;-1\,;4)$.
  \begin{enumerate}
  \item Déterminer les coordonnées du point $C$ défini par
    $\overrightarrow{AC}=\overrightarrow{u}$
  \item Déterminer les coordonnées du vecteur $\overrightarrow{AB}$
    puis celles du point $D$ tel que $ABDC$ soit un parallélogramme.
  \item Déterminer les coordonnées du centre $K$ de ce
    parallélogramme.
  \end{enumerate}
\end{exercice*}
\begin{corrige}
  \begin{enumerate}
\item $C(4;4;3)$ ; 
\item  $\overrightarrow{AB}(-2;-2;5)$ et $D(2;2;8)$ ;
\item  $K(2;3,5;3,5)$.
 \end{enumerate}
\end{corrige}
 
\begin{exercice*}
  Dans un repère$(O\,;\vec{i},\vec{j},\vec{k})$ de l'espace, on considère
  les points $A(2\,;5\,;-1)$ \,; $B(2\,;-3\,;4)$ et le vecteur
  $\overrightarrow{u}(2\,;-1\,;4)$.
  \begin{enumerate}
  \item Déterminer une représentation paramétrique de la droite
    $\Delta$ passant par $A$ et de vecteur directeur
    $\overrightarrow{u}$.
  \item Le point $B$ appartient-il à $\Delta$ ?
  \end{enumerate}
\end{exercice*}
\begin{corrige}
  \begin{enumerate}
\item Représentation paramétrique de la droite $\Delta$  :
%\begin{center}
$\begin{cases}
  x=2+2t \\
  y=5-t \\
  z=-1+4t 
\end{cases}$, $t\in\mathbb{R}$
%\end{center}
\item Le point $B$ n'appartient pas à $\Delta$.
 \end{enumerate}
\end{corrige}

\pagebreak

\begin{exercice*}
  Dans un repère$(O\,;\vec{i},\vec{j},\vec{k})$ de l'espace, on considère
  la droite $\Delta$ de représentation paramétrique :

\begin{center}
  $\begin{cases}x=-3+4t \\y=2 \\z=-t \end{cases}$, $t\in\mathbb{R}$
\end{center}
Donner un vecteur directeur de $\Delta$ et un point de $\Delta$.
\end{exercice*}
\begin{corrige}
  $\overrightarrow{u}(4\ ;\ 0\ ;\ -1)$ dirige $\Delta$ et
  $A(-3\ ;\ 2\ ;\ 0)$ appartient à $\Delta$.
 \end{corrige}

\serie{Étude de positions relatives}

Pour les exercices \RefExercice{G2Exo12} à \RefExercice{G2Exo17},
$ABCDEFGH$ est un cube et $I, J$ et $K$ sont les milieux respectifs de
$[FG]$, $[AD]$ et $[DH]$.

\begin{center}
  \includestandalone{\standalonepath/exercice12}
\end{center}

\begin{exercice}\label{G2Exo12}
  Déterminer en justifiant les positions relatives des droites
  ci-dessous. 

  On donnera leur intersection éventuelle.
  \begin{colenumerate}{2}
  \item $(IB)$ et $(GC)$.
  \item $(HB)$ et $(GA)$.
  \item $(GC)$ et $(BA)$.
  \end{colenumerate}
\end{exercice}

\begin{exercice}
  Déterminer en justifiant les positions relatives des droites
  ci-dessous. 

  On donnera leur intersection éventuelle.
  \begin{colenumerate}{2}
  \item $(JK)$ et $(AH)$.
  \item $(FD)$ et $(GH)$.
  \item $(IB)$ et $(HJ)$.
  \end{colenumerate}
\end{exercice}

\begin{exercice}
  Déterminer en justifiant les positions relatives des droites et
  plans ci-dessous. On donnera leur intersection éventuelle.
  \begin{colenumerate}{2}
  \item $(EJ)$ et $(HDA)$.
  \item $(JK)$ et $(ABE)$.
  \item $(IJ)$ et $(AFG)$.
  \end{colenumerate}
\end{exercice}

\begin{exercice}
  Déterminer en justifiant les positions relatives des droites et
  plans ci-dessous. On donnera leur intersection éventuelle.
  \begin{colenumerate}{2}
  \item $(FH)$ et $(ACE)$.
  \item $(EJ)$ et $(BCG)$.
  \item $(IJ)$ et $(ABE)$.
  \end{colenumerate}
\end{exercice}

\columnbreak

\begin{exercice}
  Déterminer en justifiant les positions relatives des plans
  ci-dessous. 

  On donnera leur intersection éventuelle.
  \begin{enumerate}
  \item $(ABJ)$ et $(GIC)$.
  \item $(KGI)$ et $(EAD)$.
  \item $(KGI)$ et $(ABE)$.
  \end{enumerate}
\end{exercice}

\begin{exercice}\label{G2Exo17}
  Déterminer en justifiant les positions relatives des plans
  ci-dessous. 

  On donnera leur intersection éventuelle.
  \begin{enumerate}
  \item $(EBG)$ et $(HDC)$.
  \item $(EBI)$ et $(HDC)$.
  \item $(IJK)$ et $(HDC)$.
  \end{enumerate}
\end{exercice}

\begin{exercice}
  $ABCD$ est un tétraèdre, $I$, $J$ et $K$ sont les milieux respectifs
  de $[BC]$, $[CD]$ et $[AC]$.

  Déterminer en justifiant les positions relatives des éléments
  ci-dessous. 

  On donnera leur intersection éventuelle.
  \begin{enumerate}
  \item $(IK)$ et $(AD)$.
  \item $(IK)$ et $(AB)$.
  \item $(IJ)$ et $(AID)$.
  \item $(ABJ)$ et $(ACD)$.
  \item $(DIK)$ et $(ABD)$.
  \item $(IJ)$ et $(KBD)$.
  \end{enumerate}
\end{exercice}

\begin{exercice}
  $ABCDE$ est une pyramide de sommet $A$ à base rectangulaire et $I$
  est un point du segment $[AE]$.
  \begin{enumerate}
  \item Justifier que la droite $(BC)$ est parallèle au plan $(EAD)$.
  \item En déduire l'intersection des plans $(IBC)$ et $(EAD)$.
  \end{enumerate}
\end{exercice}

\begin{exercice}
  $A$, $B$, $C$ et $D$ sont quatre points non coplanaires et $\Delta$
  est la droite parallèle à $(BC)$ passant par $D$. $I$ est le milieu
  de $[AC]$.

  Quelle est l'intersection de $\Delta$ avec :
  \begin{enumerate}
  \item Le plan $(IBD)$ ?
  \item Le plan $(ABC)$ ?
  \end{enumerate}
\end{exercice}

\pagebreak

\begin{exercice}
  $ABCDS$ est une pyramide dont la base $ABCD$ est un trapèze.

  \begin{center}
    \includestandalone{\standalonepath/pyramidebasetrapeze}
  \end{center}

  Reproduire la figure et construire les intersections des plans :
  \begin{enumerate}
  \item $(SAB)$ et $(SDC)$ ;
  \item $(SAD)$ et $(SBC)$.
  \end{enumerate}
\end{exercice}
 
\begin{exercice}
  $ABCDEFGH$ est un pavé droit, $I$ le point du segment $[AE]$ tel que
  $AI=\dfrac{3}{4}AE$ et $J$ le point du segment $[CG]$ tel que
  $CJ=\dfrac{1}{4}CG$.

  Les droites suivantes sont-elles coplanaires ?
  \begin{enumerate}
  \item $(AB)$ et $(IF)$ ;
  \item $(DJ)$ et $(IF)$ ;
  \item $(BC)$ et $(AE)$ ;
  \item $(EH)$ et $(IJ)$ .
  \end{enumerate}
\end{exercice}

\serie{Sections}

\begin{exercice*}[\ExerciceRefMethode{Methode-1G2}]\label{ex23G2}

\vspace{-\baselineskip}
  \begin{enumerate}
  \item Reproduire la figure de l'exercice précédent.
  \item Tracer l'intersection du plan $(BIJ)$ avec la face $EABF$.
  \item Tracer l'intersection du plan $(BIJ)$ avec la face $DCGH$.
  \item Terminer la construction de la section du pavé $ABCDEFGH$ par
    le plan $(BIJ)$.
  \end{enumerate}
\end{exercice*}
\begin{corrige}
  \begin{enumerate}
\item L'intersection du plan $(BIJ)$ avec la face $EABF$ est le segment $[BI]$.
\item L'intersection du plan $(BIJ)$ avec la face $DCGH$ est la
  parallèle à $(IB)$ passant par $J$. C'est le segment $[JH]$.
%\begin{center}
%\hspace*{-3cm}

\resizebox{.7\linewidth}{!}{%
\begin{pspicture*}(-1.5,-5.76)(6.5,2)
\pspolygon[linewidth=0.3mm,linecolor=Noir](-1.,-1.)(-1.,-5.)(5.,-5.)(5.,-1.)
\pspolygon[linewidth=0.3mm,linecolor=Noir](5.,-5.)(6.,-3.)(6.,1.)(5.,-1.)
\pspolygon[linewidth=0.3mm,linecolor=Noir](0.,1.)(0.,-3.)(6.,-3.)(6.,1.)
\psline[linewidth=0.3mm,linecolor=Noir](-1.,-1.)(-1.,-5.)
\psline[linewidth=0.3mm,linecolor=Noir](-1.,-5.)(5.,-5.)
\psline[linewidth=0.3mm,linecolor=Noir](5.,-5.)(5.,-1.)
\psline[linewidth=0.3mm,linecolor=Noir](5.,-1.)(-1.,-1.)
\psline[linewidth=0.3mm,linecolor=Noir](5.,-5.)(6.,-3.)
\psline[linewidth=0.3mm,linecolor=Noir](6.,-3.)(6.,1.)
\psline[linewidth=0.3mm,linecolor=Noir](6.,1.)(5.,-1.)
\psline[linewidth=0.3mm,linecolor=Noir](5.,-1.)(5.,-5.)
\psline[linewidth=0.3mm,linestyle=dashed,dash=5pt 5pt,linecolor=Noir](0.,1.)(0.,-3.)
\psline[linewidth=0.3mm,linestyle=dashed,dash=5pt 5pt,linecolor=Noir](0.,-3.)(6.,-3.)
\psline[linewidth=0.3mm,linecolor=Noir](6.,-3.)(6.,1.)
\psline[linewidth=0.3mm,linecolor=Noir](6.,1.)(0.,1.)
\psline[linewidth=0.3mm,linestyle=dashed,dash=5pt 5pt,linecolor=Noir](-1.,-5.)(0.,-3.)
\psline[linewidth=0.3mm,linecolor=Noir](-1.,-1.)(0.,1.)
\psline[linewidth=1.6pt,linecolor=J1](-1.,-2.)(5.,-5.)
\psline[linewidth=1.6pt,linecolor=J1](5.,-5.)(6.,-2.)
\psline[linewidth=1.6pt,linestyle=dashed,dash=5pt 5pt,linecolor=J1](0.,1.)(6.,-2.)
\psline[linewidth=1.6pt,linestyle=dashed,dash=5pt 5pt,linecolor=J1](0.,1.)(-1.,-2.)
\psdots[dotsize=3pt 0,dotstyle=+,linecolor=Noir](-1.,-1.)
\rput[bl](-1.38,-1.12){{$E$}}
\psdots[dotsize=3pt 0,dotstyle=+,linecolor=Noir](-1.,-5.)
\rput[bl](-1.5,-5.22){{$A$}}
\psdots[dotsize=3pt 0,dotstyle=+,linecolor=J1](5.,-5.)
\rput[bl](5.28,-5.18){{$B$}}
\psdots[dotsize=3pt 0,dotstyle=+,linecolor=Noir](5.,-1.)
\rput[bl](5.1,-1.3){{$F$}}
\psdots[dotsize=3pt 0,dotstyle=+,linecolor=Noir](6.,-3.)
\rput[bl](6.18,-3.04){{$C$}}
\psdots[dotsize=3pt 0,dotstyle=+,linecolor=Noir](6.,1.)
\rput[bl](6.08,1.12){{$G$}}
\psdots[dotsize=3pt 0,dotstyle=+,linecolor=Noir](0.,1.)
\rput[bl](0.08,1.12){{$H$}}
\psdots[dotsize=3pt 0,dotstyle=+,linecolor=Noir](0.,-3.)
\rput[bl](0.16,-2.94){{$D$}}
\psdots[dotsize=3pt 0,dotstyle=+,linecolor=J1](-1.,-2.)
\rput[bl](-1.32,-1.96){{$I$}}
\psdots[dotsize=3pt 0,dotstyle=+,linecolor=J1](6.,-2.)
\rput[bl](6.08,-1.88){{$J$}}
\end{pspicture*}
}
%\end{center} 
\end{enumerate}
\end{corrige}

\vspace{-\baselineskip}

\begin{exercice}
  \begin{enumerate}
  \item Reproduire la figure de l'exercice précédent.
  \item Tracer l'intersection du plan $(DIJ)$ avec la face $EADH$.
  \item Tracer l'intersection du plan $(DIJ)$ avec la face $DCGH$.
  \item Tracer l'intersection du plan $(DIJ)$ avec la face $BCGF$.
  \item Terminer la construction de la section du pavé $ABCDEFGH$ par
    le plan $(DIJ)$.
  \end{enumerate}
\end{exercice}

\begin{exercice}
  $ABCDEFGH$ est un cube et $I$ et $J$ les points tels que $I\in [HD]$
  et $HI=\dfrac{2}{3}HD$ ; $J\in [FG]$ et $FJ=\dfrac{3}{4}FG$.

  Construire la section du cube par le plan $(EIJ)$.
\end{exercice}

\begin{exercice}
  $ABCDEFGH$ est un cube et $I$ ; $J$ et $K$ les points tels que
  $I\in [EF]$ et $EI=\dfrac{1}{3}EF$ ; $J\in [BC]$ et
  $BJ=\dfrac{1}{2}BC$ ; $K\in [HG]$ et $HK=\dfrac{3}{4}HG$.

  Construire la section du cube par le plan $(IJK)$.
\end{exercice}

\begin{exercice}
  $ABCDEFGH$ est un cube et $I$ ; $J$ et $K$ les milieux respectifs
  des segments $[BC]$, $[CD]$ et $[EH]$.

  Construire la section du cube par le plan $(IJK)$.
\end{exercice}

\begin{exercice}
  $ABCDEFGH$ est un cube et $I$ ; $J$ et $K$ les points tels que
  $I\in [AE]$ et $AI=\dfrac{1}{4}AE$ ; $J\in [DH]$ et
  $DJ=\dfrac{3}{4}DH$ ; $K\in [FG]$ et $FK=\dfrac{1}{3}FG$.

  Construire la section du cube par le plan $(IJK)$.
\end{exercice}


% ajouts ---------------------------

\begin{exercice}
  $ABCDEFGH$ est un cube ; $I$ est le milieu de $[EH]$ ; $J$ est le
  milieu de $[BC]$ et $K$ le point du segment $[GH]$ tel que :
  $HK=\dfrac{2}{3}HG$.

  Déterminer et construire la section du cube par le plan $(IJK)$.
\end{exercice}

\begin{exercice}
  $ABCDEFGH$ est un cube et $I$ ; $J$ et $K$ les points tels que :
  $I\in [AD]$ et $AI=\dfrac{1}{3}AD$ ; $J\in [FG]$ et
  $FJ=\dfrac{2}{3}FG$ ; $K\in [AB]$ et $AK=\dfrac{1}{3}AB$.

  Déterminer et construire la section du cube par le plan $(IJK)$.
\end{exercice}

\begin{exercice}
  On considère une pyramide à base carrée $SABCD$ comme ci-dessous.
\begin{enumerate}
\item Reproduire la figure et placer les points $I$ et $J$ milieux respectifs des segments $[SD]$ et $[AB]$
\item Construire en justifiant la section de la pyramide par le plan $(CIJ)$.
\end{enumerate}

\begin{center}
  \begin{tikzpicture}[general,x=1.0cm,y=1.0cm,scale=.5]
%\clip(-1.8,-5.92) rectangle (5.48,0.82);
\draw [color=Noir] (1.,-5.)-- (-0.54,-3.);
\draw [color=Noir] (1.,-5.)-- (5.,-5.);
\draw [dashed,color=J1] (5.,-5.)-- (3.5,-3.);
\draw [dashed,color=J1] (3.5,-3.)-- (-0.54,-3.);
\draw [color=Noir] (-0.54,-3.)-- (2.02,0.6);
\draw [color=Noir] (2.02,0.6)-- (3.5,-3.);
\draw [color=Noir] (2.02,0.6)-- (5.,-5.);
\draw [color=Noir] (2.02,0.6)-- (1.,-5.);
\begin{scriptsize}
\draw [color=Noir] (1.,-5.)-- ++(-1.5pt,0 pt) -- ++(3.0pt,0 pt) ++(-1.5pt,-1.5pt) -- ++(0 pt,3.0pt);
\draw[color=Noir] (0.56,-5.06) node {$A$};
\draw [color=Noir] (5.,-5.)-- ++(-1.5pt,0 pt) -- ++(3.0pt,0 pt) ++(-1.5pt,-1.5pt) -- ++(0 pt,3.0pt);
\draw[color=Noir] (5.22,-5.02) node {$B$};
\draw [color=Noir] (3.5,-3.)-- ++(-1.5pt,0 pt) -- ++(3.0pt,0 pt) ++(-1.5pt,-1.5pt) -- ++(0 pt,3.0pt);
\draw[color=Noir] (3.,-2.64) node {$C$};
\draw [color=Noir] (-0.54,-3.)-- ++(-1.5pt,0 pt) -- ++(3.0pt,0 pt) ++(-1.5pt,-1.5pt) -- ++(0 pt,3.0pt);
\draw[color=Noir] (-0.68,-2.62) node {$D$};
\draw [color=Noir] (2.02,0.6)-- ++(-1.5pt,-1.5pt) -- ++(3.0pt,3.0pt) ++(-3.0pt,0) -- ++(3.0pt,-3.0pt);
\draw[color=Noir] (2.16,0.58) node [above] {$S$};
\end{scriptsize}
\end{tikzpicture}
\end{center}
\end{exercice}

\columnbreak

\begin{exercice}
On considère un tétraèdre régulier $ABCD$ comme ci-dessous avec $I$, $J$ et $K$ les milieux respectifs des segments $[BC]$, $[AB]$ et $[AD]$.
\begin{enumerate}
\item Reproduire la figure. 
\item Construire en justifiant la section du tétraèdre par le plan $(IJK)$.
\item Quelle est la nature de cette section ? Justifier.
\end{enumerate}

\begin{center}
\begin{tikzpicture}[general,x=1.0cm,y=1.0cm,scale=.7]
%\clip(-3.12,-1.2) rectangle (3.66,5.32);
\draw [color=Noir] (0.,4.)-- (1.,0.);
\draw [color=Noir] (1.,0.)-- (-1.42,0.98);
\draw [color=Noir] (-1.42,0.98)-- (0.,4.);
\draw [color=Noir] (0.,4.)-- (2.04,1.74);
\draw [color=Noir] (2.04,1.74)-- (1.,0.);
\draw [color=Noir] (1.,0.)-- (0.,4.);
\draw [dashed,color=J1] (-1.42,0.98)-- (2.04,1.74);
\begin{scriptsize}
\draw [fill=Noir] (0.,4.) circle (1.5pt);
\draw[color=Noir] (0.14,4.28) node {$A$};
\draw [fill=Noir] (1.,0.) circle (1.5pt);
\draw[color=Noir] (1.3,0.06) node {$B$};
\draw [fill=Noir] (-1.42,0.98) circle (1.5pt);
\draw[color=Noir] (-1.58,1.32) node {$C$};
\draw [fill=Noir] (2.04,1.74) circle (1.5pt);
\draw[color=Noir] (2.18,2.02) node {$D$};
\draw [color=J1] (-0.21,0.49)-- ++(-1.5pt,-1.5pt) -- ++(3.0pt,3.0pt) ++(-3.0pt,0) -- ++(3.0pt,-3.0pt);
\draw[color=J1] (-0.62,0.4) node {$I$};
\draw [color=J1] (0.5,2.)-- ++(-1.5pt,-1.5pt) -- ++(3.0pt,3.0pt) ++(-3.0pt,0) -- ++(3.0pt,-3.0pt);
\draw[color=J1] (0.64,2.28) node {$J$};
\draw [color=J1] (1.02,2.87)-- ++(-1.5pt,-1.5pt) -- ++(3.0pt,3.0pt) ++(-3.0pt,0) -- ++(3.0pt,-3.0pt);
\draw[color=J1] (1.16,3.16) node {$K$};
\end{scriptsize}
\end{tikzpicture}  
\end{center}
\end{exercice}

%  fin ajouts ------------------------

\serie{Orthogonalité}

Pour les exercices \RefExercice*{G2Exo34} à \RefExercice{G2Exo36},
$ABCDEFGH$ est un cube.

\begin{center}
  \includestandalone{\standalonepath/cube}
\end{center}

\begin{exercice}
  \begin{enumerate}
  \item Citer six droites orthogonales à la droite $(EA)$ ;
  \item Citer six droites orthogonales à la droite $(EB)$ ;
  \item Citer deux droites orthogonales au plan $(BCG)$ ;
  \item Citer deux droites orthogonales au plan $(AFG)$.
  \end{enumerate}
\end{exercice}

\begin{exercice*}[\ExerciceRefMethode{Methode-2G2}]\label{G2Exo34}
  \begin{enumerate}
  \item Démontrer que la droite $(AB)$ est orthogonale au plan
    $(BCG)$.
  \item En déduire que les droites $(AB)$ et $(CF)$ sont orthogonales.
  \end{enumerate}
\end{exercice*}
\begin{corrige}
  \begin{enumerate}
\item  Les droites $(BC)$ et $(BF)$ sont deux droites sécantes du plan $(BCG)$ et, par propriété du cube, $(AB)\perp(BC)$ et $(AB)\perp(BF)$.\\
Donc $(AB)$ est orthogonale au plan $(BCG)$.
\item  $(AB)$ est orthogonale au plan $(BCG)$, donc $(AB)$ est orthogonale à toute droite du plan$(BCG)$, et en particulier, $(AB)$ et $(CF)$ sont orthogonales.
\end{enumerate}
\end{corrige}

\begin{exercice}
  Les droites suivantes sont-elles orthogonales ? Le démontrer.
  \begin{colenumerate}{2}
  \item $(EG)$ et $(GC)$ ;
  \item $(EB)$ et $(EG)$ ;
  \item $(AF)$ et $(BC)$ ;
  \item $(AC)$ et $(HF)$ ;
  \item $(BD)$ et $(EC)$ ;
  \item $(CE)$ et $(AG)$.
  \end{colenumerate}
\end{exercice}

\begin{exercice}\label{G2Exo36}
  $ABCD$ est un tétraèdre régulier, $S$ est le pied de la hauteur
  issue de A relativement à la base $BCD$ et $I$ est le milieu de
  $[BC]$.

  \begin{center}
    \includestandalone{\standalonepath/exortho1}
  \end{center}

  \begin{enumerate}
  \item Démontrer que les droites $(AS)$ et $(BC)$ sont orthogonales.
  \item En déduire que la droite $(BC)$ est orthogonale au plan
    $(AIS)$.
  \item En déduire que les points $A$, $I$, $S$ et $D$ sont
    coplanaires et que les points $I$, $S$ et $D$ sont alignés.
  \end{enumerate}
\end{exercice}

\serie{Vecteurs} 

Pour les exercices \RefExercice{G2Exo37} à \RefExercice{G2Exo43}, $ABCDEFGH$ est un cube et $I$ ; $J$ ; $K$
et $L$ les milieux respectifs de $[BC]$, $[GH]$, $[AD]$ et $[EH]$.

\begin{center}
  \includestandalone{\standalonepath/cube2}
\end{center}

\begin{exercice}\label{G2Exo37}
  Compléter les égalités vectorielles suivantes :
  \begin{enumerate}
  \item $\overrightarrow{A...}=\dfrac{1}{2}\overrightarrow{BC}$
  \item
    $\overrightarrow{KJ}=\overrightarrow{AE}+\dfrac{1}{2}\overrightarrow{E...}$
  \item
    $\overrightarrow{AK}+\overrightarrow{EF}=\overrightarrow{A...}$
  \end{enumerate}
\end{exercice}

\begin{exercice}
  Compléter les égalités vectorielles suivantes :
  \begin{enumerate}
  \item $\overrightarrow{...}=\dfrac{1}{2}\overrightarrow{AC}$
  \item
    $\overrightarrow{L...}=\overrightarrow{EA}+\overrightarrow{FE}+\overrightarrow{AI}$
  \item
    $\overrightarrow{A...}=\overrightarrow{GJ}+3\overrightarrow{AK}+\overrightarrow{AB}+\overrightarrow{JL}$
  \end{enumerate}
\end{exercice}

\columnbreak

\begin{exercice}
  Dans chacun des cas suivants, les vecteurs sont-ils coplanaires ? Le
  justifier.
  \begin{enumerate}
  \item $\overrightarrow{AG}$, $\overrightarrow{DH}$ et
    $\overrightarrow{EG}$ ;
  \item $\overrightarrow{AB}$, $\overrightarrow{BD}$ et
    $\overrightarrow{BF}$ ;
  \item $\overrightarrow{AG}$, $\overrightarrow{BG}$ et
    $\overrightarrow{HG}$ ;
  \item $\overrightarrow{HF}$, $\overrightarrow{DC}$ et
    $\overrightarrow{AD}$.
  \end{enumerate}
\end{exercice}

\begin{exercice}
  Le point $M$ est défini par
  $\overrightarrow{EM}=2\overrightarrow{EF}$
  \begin{enumerate}
  \item En fonction des vecteurs
    $\overrightarrow{AB}$, $\overrightarrow{AD}$ et
    $\overrightarrow{AE}$ exprimer les vecteurs suivants :
    \\$\overrightarrow{EM}$
    ; $\overrightarrow{HC}$
    ; $\overrightarrow{BD}$
    ; $\overrightarrow{BJ}$
    ; $\overrightarrow{KM}$ et $\overrightarrow{MJ}$.
  \item Les droites $(BK)$ et $(MJ)$ sont-elles parallèles ?

    Le démontrer en utilisant la question précédente.
  \item Que peut-on en déduire concernant les points $B$,
    $K$, $M$ et $J$ ?
  \end{enumerate}
\end{exercice}

\begin{exercice*}[\ExerciceRefMethode{Methode-3G2}]\label{ex37G2}
  On considère les points $M$ et $N$ définis par :

  $\overrightarrow{AM}=\dfrac{1}{3}\overrightarrow{AB}+\dfrac{1}{3}\overrightarrow{AD}+\dfrac{2}{3}\overrightarrow{AE}$

  et 

  $\overrightarrow{AN}=\dfrac{2}{3}\overrightarrow{AB}+\overrightarrow{BF}+\dfrac{2}{3}\overrightarrow{FG}$.
  \begin{enumerate}
  \item Construire la figure.
  \item Démontrer que les points $C$, $E$ et $M$ sont alignés.
  \item Démontrer que les points $E$,
    $F$, $H$ et $N$ sont coplanaires.
  \end{enumerate}
\end{exercice*}
\begin{corrige}
\begin{enumerate}
\item ~\par\vspace{-1cm}
\resizebox{.8\linewidth}{!}{
\begin{pspicture*}(-1.5,-5.76)(6.5,2)
\pspolygon[linewidth=.3mm,linecolor=Noir](-1.,-1.)(-1.,-5.)(3.,-5.)(3.,-1.)
\pspolygon[linewidth=.3mm,linecolor=Noir](3.,-5.)(4.,-3.)(4.,1.)(3.,-1.)
\pspolygon[linewidth=.3mm,linecolor=Noir](0.,1.)(0.,-3.)(4.,-3.)(4.,1.)
\psline[linewidth=.3mm,linecolor=Noir](-1.,-1.)(-1.,-5.)
\psline[linewidth=.3mm,linecolor=Noir](-1.,-5.)(3.,-5.)
\psline[linewidth=.3mm,linecolor=Noir](3.,-5.)(3.,-1.)
\psline[linewidth=.3mm,linecolor=Noir](3.,-1.)(-1.,-1.)
\psline[linewidth=.3mm,linecolor=Noir](3.,-5.)(4.,-3.)
\psline[linewidth=.3mm,linecolor=Noir](4.,-3.)(4.,1.)
\psline[linewidth=.3mm,linecolor=Noir](4.,1.)(3.,-1.)
\psline[linewidth=.3mm,linecolor=Noir](3.,-1.)(3.,-5.)
\psline[linewidth=.3mm,linestyle=dashed,dash=5pt 5pt,linecolor=J1](0.,1.)(0.,-3.)
\psline[linewidth=.3mm,linestyle=dashed,dash=5pt 5pt,linecolor=J1](0.,-3.)(4.,-3.)
\psline[linewidth=.3mm,linecolor=Noir](4.,-3.)(4.,1.)
\psline[linewidth=.3mm,linecolor=Noir](4.,1.)(0.,1.)
\psline[linewidth=.3mm,linestyle=dashed,dash=5pt 5pt,linecolor=J1](-1.,-5.)(0.,-3.)
\psline[linewidth=.3mm,linecolor=Noir](-1.,-1.)(0.,1.)
\psline[linestyle=dashed,dash=5pt 5pt,linecolor=J1](-1.,-1.)(4.,-3.)
\psline[linecolor=Noir](-1.,-1.)(4.,1.)
\psdots[dotsize=3pt 0,dotstyle=+,linecolor=Noir](-1.,-1.)
\rput[bl](-1.38,-1.12){{$E$}}
\psdots[dotsize=3pt 0,dotstyle=+,linecolor=Noir](-1.,-5.)
\rput[bl](-1.5,-5.22){{$A$}}
\psdots[dotsize=3pt 0,dotstyle=+,linecolor=Noir](3.,-5.)
\rput[bl](3.28,-5.18){{$B$}}
\psdots[dotsize=3pt 0,dotstyle=+,linecolor=Noir](3.,-1.)
\rput[bl](3.1,-1.3){{$F$}}
\psdots[dotsize=3pt 0,dotstyle=+,linecolor=Noir](4.,-3.)
\rput[bl](4.18,-3.04){{$C$}}
\psdots[dotsize=3pt 0,dotstyle=+,linecolor=Noir](4.,1.)
\rput[bl](4.08,1.12){{$G$}}
\psdots[dotsize=3pt 0,dotstyle=+,linecolor=Noir](0.,1.)
\rput[bl](0.08,1.12){{$H$}}
\psdots[dotsize=3pt 0,dotstyle=+,linecolor=Noir](0.,-3.)
\rput[bl](0.16,-2.94){{$D$}}
\psdots[dotsize=3pt 0,dotstyle=+,linecolor=J1](2.333333333333333,0.33333333333333326)
\rput[bl](2.42,0.46){{$N$}}
\psdots[dotsize=3pt 0,dotstyle=+,linecolor=J1](0.666666666666667,-1.6666666666666667)
\rput[bl](0.74,-1.54){{$M$}}
\end{pspicture*}
}

\item $\overrightarrow{CM}=\overrightarrow{CA}+\overrightarrow{AM}=$\\
$-\overrightarrow{AB}-\overrightarrow{AD}+
\dfrac{1}{3}\overrightarrow{AB}+\dfrac{1}{3}\overrightarrow{AD}+\dfrac{2}{3}\overrightarrow{AE}=$\\
$-\dfrac{2}{3}\overrightarrow{AB}-\dfrac{2}{3}\overrightarrow{AD}+\dfrac{2}{3}\overrightarrow{AE}$ et \\ 
$\overrightarrow{CE}=-\overrightarrow{AB}-\overrightarrow{AD}+\overrightarrow{AE}=\dfrac{3}{2}\overrightarrow{CM}$.\\
Donc $\overrightarrow{CE}$ et $\overrightarrow{CM}$ sont colinéaires et 
les points $C$, $E$ et $M$ sont alignés.

\item $\overrightarrow{EN}=\overrightarrow{EA}+\overrightarrow{AN}=\overrightarrow{FB}+\dfrac{2}{3}\overrightarrow{AB}+\overrightarrow{BF}+\dfrac{2}{3}\overrightarrow{FG}=\dfrac{2}{3}\overrightarrow{EG}$.
Donc $N\in(EFG)$ et les points $E$, $F$, $H$ et $N$ sont coplanaires.
\end{enumerate}
\end{corrige}

\begin{exercice}
  Répondre par vrai ou faux en justifiant :
  \begin{enumerate}
  \item Les vecteurs $\overrightarrow{HI}$, $\overrightarrow{AB}$ et
    $\overrightarrow{DH}$ sont coplanaires.
  \item Les vecteurs $\overrightarrow{HG}$, $\overrightarrow{KB}$ et
    $\overrightarrow{LE}$ sont coplanaires.
  \item Les vecteurs $\overrightarrow{HJ}$, $\overrightarrow{AB}$ et
    $\overrightarrow{DH}$ sont coplanaires.
  \end{enumerate}
\end{exercice}

\begin{exercice}\label{G2Exo43}
  $ABCDEFGH$ est un cube.\\
  On considère le point $K$ défini par
  $\overrightarrow{HK}=\dfrac{5}{4}\overrightarrow{HF}$ et $M$ un
  point du segment $[BF]$.
  \begin{enumerate}
  \item Que peut-on dire des points $D$, $M$, $K$ et $H$ ?
  \item Montrer qu'il existe un unique réel $t\in\left[ 0;1\right] $
    tel que $\overrightarrow{BM}=t\overrightarrow{BF}$.
  \item Montrer que si $t=\dfrac{4}{5}$, les points $D$, $M$ et $K$
    sont alors alignés.
  \end{enumerate}
\end{exercice}

\begin{exercice}
  Dans un repère $(O\,;\vec{i},\vec{j},\vec{k})$ de l'espace, on
  considère les points $A(-3\,;2\,;4)$ \,; $B(-1\,;1\,;0)$ et
  $C(2\,;-3\,;5)$.  Déterminer les coordonnées des points $M$, $N$ et
  $P$ définis par :
  \begin{enumerate}
  \item $\overrightarrow{AM}=2\overrightarrow{BC}-\overrightarrow{BA}$
  \item
    $\overrightarrow{NB}=4\overrightarrow{CA}-3\overrightarrow{BC}$
  \item
    $2\overrightarrow{PA}-3\overrightarrow{PB}+\overrightarrow{PC}=\overrightarrow{0}$
  \end{enumerate}
\end{exercice}

\columnbreak

\begin{exercice*}[\ExerciceRefMethode{Methode-4G2}]\label{ex41G2}
  Dans un repère $(O\,;\vec{i},\vec{j},\vec{k})$ de l'espace, on
  considère les points $A(-4\,;2\,;3)$, $B(1\,;5\,;2)$, $C(0\,;5\,;4)$ et
  $D(-6\,;-1\,;-2)$.
  \begin{enumerate}
  \item Démontrer que
    $\overrightarrow{AD}=2\overrightarrow{AB}-3\overrightarrow{AC}$.
  \item Que peut-on en déduire concernant les points $A$, $B$, $C$ et
    $D$ ?
  \end{enumerate}
\end{exercice*}
\begin{corrige}
  \begin{enumerate}
\item $\overrightarrow{AD}(-2\ ;\ -3\ ;\ -5)$. D'autre part : \\ $\overrightarrow{AB}(5\ ;\ 3\ ;\ -1)$ donc $2\overrightarrow{AB}(10\ ;\ 6\ ;\ -2)$ et \\
$\overrightarrow{AC}(4\ ;\ 3\ ;\ 1)$ donc $-3\overrightarrow{AC}(-12\ ;\ -9\ ;\ -3)$.\\
Ainsi $2\overrightarrow{AB}-3\overrightarrow{AC}(-2\ ;\ -3\ ;\ -5)$.
Donc $\overrightarrow{AD}=2\overrightarrow{AB}-3\overrightarrow{AC}$
\item On en déduit que les vecteurs $\overrightarrow{AB}$, $\overrightarrow{AC}$ et $\overrightarrow{AD}$ sont coplanaires, et que les points $A$, $B$, $C$ et $D$ sont coplanaires.
 \end{enumerate}
\end{corrige}

\begin{exercice}
  Dans un repère $(O\,;\vec{i},\vec{j},\vec{k})$ de l'espace, on
  considère les points $A(0\,;3\,;-1)$, $B(2\,;-2\,;0)$, $C(4\,;1\,;5)$ et
  $D(2\,;21\,;12)$.
  \begin{enumerate}
  \item Montrer que les points $A$, $B$ et $C$ définissent un plan.
  \item Le point $D$ appartient-il à ce plan ?
  \end{enumerate}
\end{exercice}

\begin{exercice}
  Dans un repère $(O\,;\vec{i},\vec{j},\vec{k})$ de l'espace,on
  considère les points $A(1\,;-1\,;-1)$, $B(5\,;0\,;-3)$, $C(2\,;-2\,;-2)$ et
  $D(0\,;5\,;-2)$.
  \begin{enumerate}
  \item Montrer que les points $A$, $B$ et $C$ définissent un plan.
  \item Le point $D$ appartient-il à ce plan ?
  \end{enumerate}
\end{exercice}

\begin{exercice}
  On reprend l'énoncé de l'exercice 42 en se plaçant dans le repère
  $(A\,;\overrightarrow{AB},\overrightarrow{AD},\overrightarrow{AE})$.
  \begin{enumerate}
  \item Écrire les coordonnées des points de la figure. On écrira les
    coordonnées de $M$ en fonction de $t$.
  \item Démontrer à l'aide des coordonnées que $D$, $M$ et $I$ sont
    alignés si et seulement si $t=\dfrac{4}{5}$.
  \end{enumerate}
\end{exercice}

\serie{Représentations paramétriques}

Dans toute cette partie, on munit l'espace d'un repère
$(O\,;\vec{i},\vec{j},\vec{k})$.

\begin{exercice}
  On considère les points $A(-3\ ;\ 2\ ;\ 4)$ et  $B(-1\ ;\ 1\ ;\ 0)$.
  Écrire une représentation paramétrique de la droite $(AB)$.
\end{exercice}
% \begin{corrige}
%   Le vecteur $\overrightarrow{AB}(2\ ;\ -1\ ;\ -4)$ est un vecteur
%   directeur de la droite $(AB)$. Ainsi,
%   \begin{center}
%     $\begin{cases}
%       x=-3+2t \\
%       y=2-t\\
%       z=4-4t
%     \end{cases}$,
%     $t\in\mathbb{R}$
%   \end{center}
%   est une représentation paramétrique de la droite $(AB)$.
% \end{corrige}

\begin{exercice}\label{ex49G2}
  Soit $\Delta$ la droite de représentation paramétrique :
{\centering
  $\begin{cases}x=1-4t \\y=3+t \\z=1-t \end{cases}$ , $t\in\mathbb{R}$
\par\vspace{-.5\baselineskip}}
\begin{enumerate}
\item Donner un vecteur directeur de la droite $\Delta$ et un point de
  $\Delta$.
\item Le point $M(-3\,;4\,;1)$ appartient-il à la droite $\Delta$ ?
\item Donner les coordonnées de trois points de la droite~$\Delta$.
\item Déterminer une autre représentation paramétrique de $\Delta$.
\end{enumerate}
\end{exercice}

\begin{exercice}
  Soit  $\Delta$ la droite de représentation paramétrique :

  \begin{center}
    $\begin{cases}x=t-2 \\y=4 \\z=2t-1 \end{cases}$ , $t\in\mathbb{R}$
  \end{center}

  \begin{enumerate}
  \item Donner un vecteur directeur de la droite $\Delta$ et un point
    de $\Delta$.
  \item Le point $M(-3\,;4\,;-3)$ appartient-il à la droite $\Delta$ ?
  \item Donner les coordonnées de trois points de $\Delta$.
  \item Déterminer une autre représentation paramétrique de la droite
    $\Delta$.
  \end{enumerate}
\end{exercice}

\begin{exercice}
  Soient $A(-4\,;1\,;2)$ et $B(-1\,;2\,;5)$. Donner une\\
  représentation paramétrique de chacun des objets géométriques
  suivants :
  \begin{enumerate}
  \item La droite $(AB)$ ;
  \item Le segment $[AB]$ ;
  \item La demi-droite $[AB)$.
  \end{enumerate}
\end{exercice}

\begin{exercice}
  Donner une représentation paramétrique de :
  \begin{enumerate}
  \item La droite $(O\ ;\ \overrightarrow{i})$ ;
  \item La droite $(O\ ;\ \overrightarrow{j})$ ;
  \item La droite $(O\ ;\ \overrightarrow{k})$.
  \end{enumerate}
\end{exercice}

\begin{exercice}
  On considère les points $A(-3\ ;\ 2\ ;\ 4)$, $B(-1\ ;\ 1\ ;\ 0)$ et
  $C(-5\ ;\ 4\ ;\ 6)$.

  Vérifier que $A$, $B$ et $C$ définissent un plan et écrire une
  représentation paramétrique du plan $(ABC)$.
\end{exercice}
% \begin{corrige}
%   $\overrightarrow{AB}(2\ ;\ -1\ ;\ -4)$ et
%   $\overrightarrow{AC}(-2\ ;\ 2\ ;\ 2)$ ne sont pas colinéaires, donc
%   les points $A$, $B$ et $C$ ne sont pas alignés.

%   Ils définissent donc un plan dirigé par les vecteurs
%   $\overrightarrow{AB}$ et $\overrightarrow{AC}$.

%   Une représentation paramétrique du plan $(ABC)$ est donc :
%   \begin{center}
%     $\begin{cases}
%       x=-3+2t-2t'\\
%       y=2-t+2t'\\
%       z=4-4t+2t'
%     \end{cases}$,
%     $t\in\mathbb{R}$ et $t'\in\mathbb{R}$
%   \end{center}
% \end{corrige}

\begin{exercice}\label{ex54G2}
  Soit $\wp$ le plan de représentation paramétrique :

  \begin{center}
    $\begin{cases}x=3-t+5t' \\y=1+t' \\z=-5t+3t' \end{cases}$
    $t\in\mathbb{R}, t'\in\mathbb{R}$
  \end{center}

  \begin{enumerate}
  \item Donner les coordonnées d'un couple de vecteurs\\ directeurs de
    $\wp$ et un point de $\wp$.
  \item Le point $M(6\,;2\,;-6)$ appartient-il à $\wp$ ?
  \item Donner les coordonnées de trois points de $\wp$.
  \item Déterminer une autre représentation paramétrique de $\wp$.
  \end{enumerate}
\end{exercice}

%\columnbreak

\begin{exercice}
  Soient $A(-4\,;1\,; 2)$ ; $B(-1\,; 2\,;\ 5)$ et $C(1\,; 0 \,; 6)$.
  \begin{enumerate}
  \item Vérifier que les points $A$, $B$ et $C$ définissent un plan.
  \item Déterminer une représentation paramétrique de la droite
    $(AB)$.
  \item Déterminer une représentation paramétrique du plan $(ABC)$.
  \item Démontrer que le point $D(-3\,;-4\,;1)$ appartient au plan
    $(ABC)$.
  \item Déterminer une autre représentation paramétrique du plan
    $(ABC)$.
  \end{enumerate}
\end{exercice}

\begin{exercice}
  Donner une représentation paramétrique des plans suivants :
  \begin{enumerate}
  \item Le plan $(O\,;\overrightarrow{i},\overrightarrow{j})$ ;
  \item Le plan $(O\,;\overrightarrow{i},\overrightarrow{k})$ ;
  \item Le plan $(O\,;\overrightarrow{j},\overrightarrow{k})$.
  \end{enumerate}
\end{exercice}
 
\begin{exercice*}[\ExerciceRefMethode{Methode-5G2}]\label{ex52G2}
  Soit $\Delta$ la droite de représentation paramétrique :

  \begin{center}
    $\begin{cases}x=1-t \\y=-2+3t \\z=-1+t \end{cases}$,
    $t\in\mathbb{R}$
  \end{center}

  Dans chacun des cas suivants, étudier la position de la droite
  $\Delta$ avec la droite $d$ de représentation paramétrique :
  \begin{enumerate}
  \item \begin{center}
      $\begin{cases}x=-k \\y=3+2k \\z=4-k \end{cases}$,
      $k\in\mathbb{R}$\\[3mm]
    \end{center}

  \item \begin{center}
      $\begin{cases}x=1+k \\y=-2k \\z=3-k \end{cases}$,
      $k\in\mathbb{R}$\\[3mm]
    \end{center}

  \item \begin{center}
      $\begin{cases}x=k-2 \\y=7-3k \\z=2-k \end{cases}$,
      $k\in\mathbb{R}$
    \end{center}
  \end{enumerate}
\end{exercice*}
\begin{corrige}
  $\Delta$ a pour vecteur directeur $\overrightarrow{u}(-1\ ;\ 3\ ;\ 1)$

\begin{enumerate}
\item  $d$ a pour vecteur directeur $\overrightarrow{v}(-1\ ;\ 2\ ;\
  -1)$ qui n'est pas colinéaire avec $\overrightarrow{u}$. $\Delta$ et
  $d$ sont donc soit sécantes, soit non coplanaires.

\hspace{-5mm}$\begin{cases}
  x=1-t\\y=-2+3t\\z=-1+t\\x=-k\\y=3+2k\\z=4-k \end{cases}\Leftrightarrow \begin{cases}
  -k=1-t\\3+2k=-2+3t\\4-k=-1+t\\x=-k\\y=3+2k\\z=4-k \end{cases}\Leftrightarrow$\\
\hspace*{-5mm}$\begin{cases} k=t-1\\-2+3t=1+2t\\4-t+1=-1+t\\x=-k\\y=3+2k\\z=4-k\end{cases}$
$\Leftrightarrow \begin{cases} k=2\\t=3\\t=3\\x=-2\\y=7\\z=2 \end{cases}$\\
Les droites $d$ et $\Delta$ sont donc sécantes en $A(-2\ ;\ 7\ ;\ 2)$.
\item $d$ a pour vecteur directeur $\overrightarrow{v}(1\ ;\ -2\ ;\ -1)$ qui n'est pas colinéaire avec $\overrightarrow{u}$. $\Delta$ et $d$ sont donc soit sécantes, soit non coplanaires.

  \hspace*{-5mm}$\begin{cases}
    x=1-t\\y=-2+3t\\z=-1+t\\x=1+k\\y=-2k\\z=3-k \end{cases}\hspace{-4mm}\Leftrightarrow \begin{cases}
    1+k=1-t\\-2k=-2+3t\\3-k=-1+t\\x=1+k\\y=-2k\\z=3-k \end{cases}\hspace{-4mm}\Leftrightarrow
  \begin{cases} k=-t\\t=2\\3=-1
    !!!\\x=1+k\\y=-2k\\z=3-k \end{cases}$\\
  Le système n'admet pas de solution et les droites $d$ et $\Delta$
  sont donc non coplanaires.
\item $d$ a pour vecteur directeur
  $\overrightarrow{v}(1\ ;\ -3\ ;\ -1)$ qui est pas colinéaire avec
  $\overrightarrow{u}$. $\Delta$ et $d$ sont donc parallèles.
  $B(1\ ;\ -2\ ;\ -1)\in\Delta$. Vérifions si $B\in d$ pour savoir si elles sont confondues :\\
  $\begin{cases} 1=1+k\\-2=-2k\\-1=3-k \end{cases}\Leftrightarrow \begin{cases} k=0\\k=1\\k=4 \end{cases}$.\\
  Il n'existe pas de réel $k$ tel que les coordonnées de $B$ vérifient
  le système, donc $B$ n'appartient pas à $d$ et les droites $d$ et
  $\Delta$ sont strictement parallèles.
 \end{enumerate}
\end{corrige}

%\columnbreak

\begin{exercice}
  Soit $\Delta$ la droite de représentation paramétrique :

  \begin{center}
    $\begin{cases}x=8+2t \\y=-5-4t \\z=3+2t \end{cases}$,
    $t\in\mathbb{R}$
  \end{center}

  Dans chacun des cas suivants, étudier la position de la droite
  $\Delta$ avec la droite $d$ de représentation paramétrique :
  \begin{enumerate}
  \item \begin{center}
      $\begin{cases}x=2+3t\\y=7-6t \\z=-3+3t\end{cases}$,
      $t\in\mathbb{R}$
    \end{center}
  \item \begin{center} $\begin{cases}x=1-t \\y=4-t \\z=2 \end{cases}$, $t\in\mathbb{R}$
    \end{center}
  \item \begin{center}
      $\begin{cases}x=-1+t \\y=7-2t \\z=-2+3t \end{cases}$,
      $t\in\mathbb{R}$
    \end{center}
  \end{enumerate}
\end{exercice}

\begin{exercice}
  Soit $\wp$ le plan de représentation paramétrique :

\begin{center}
  $\begin{cases}x=t-2t' \\y=1+3t+t' \\z=2-5t \end{cases}$
  $t\in\mathbb{R}, t'\in\mathbb{R}$
\end{center}

Déterminer la nature de $\wp\cap\wp'$ dans chacun des cas\\
suivants où $\wp'$ est définie par une représentation paramétrique :
\begin{enumerate}
\item \begin{center}
    $\begin{cases}x=-2-3t-t' \\y=2-2t+4t' \\z=2+5t-5t' \end{cases}$
    $t\in\mathbb{R}, t'\in\mathbb{R}$
  \end{center}
\item \begin{center}
    $\begin{cases}x=4-3t+5t' \\y=-2t+t' \\z=5+5t-5t' \end{cases}$
    $t\in\mathbb{R}, t'\in\mathbb{R}$
  \end{center}
\item \begin{center}
    $\begin{cases}x=-3+2t+t' \\y=2-t+2t' \\z=1+t \end{cases}$
    $t\in\mathbb{R}, t'\in\mathbb{R}$
  \end{center}
\end{enumerate}
\end{exercice}

\begin{exercice}
  Soit $\wp$ le plan de représentation paramétrique :
  \begin{center}
    $\begin{cases}x=4-t+3t' \\y=1-t+5t' \\z=t-t' \end{cases}$
    $t\in\mathbb{R}, t'\in\mathbb{R}$
  \end{center}Dans chacun des cas suivants, déterminer une
  représentation paramétrique de la droite d'intersection de $\wp$ et
  du plan :
  \begin{enumerate}
  \item $(O\,;\overrightarrow{i},\overrightarrow{j})$
  \item $(O\,;\overrightarrow{i},\overrightarrow{k})$
  \item $(O\,;\overrightarrow{j},\overrightarrow{k})$
  \end{enumerate}
\end{exercice}

\begin{exercice}
  Soit $\wp$ le plan de représentation paramétrique :
  \begin{center}
    $\begin{cases}x=4-t+3t' \\y=1-t+5t' \\z=t-t' \end{cases}$
    $t\in\mathbb{R}, t'\in\mathbb{R}$
  \end{center}

  Dans chacun des cas suivants, déterminer l'intersection de $\wp$
  avec la droite $d$ donnée par une représentation\\ paramétrique :
  \begin{colenumerate}{2}
  \item \begin{center}
      $\begin{cases}x=2+t\\y=3+3t \\z=5-t\end{cases}$,
      $t\in\mathbb{R}$
    \end{center}
  \item \begin{center}
      $\begin{cases}x=1+4t \\y=10t \\z=-3t \end{cases}$,
      $t\in\mathbb{R}$
    \end{center}
  \item \begin{center}
      $\begin{cases}x=1+2t \\y=-2 \\z=-3+t \end{cases}$,
      $t\in\mathbb{R}$
    \end{center}
  \end{colenumerate}
\end{exercice}

\begin{exercice}
  Soit $ABCDEFGH$ un cube ; $I$ et $J$ les milieux respectifs de
  $[EG]$ et $[GH]$. 

  On munit l'espace du repère
  $(A;\overrightarrow{AB},\overrightarrow{AD},\overrightarrow{AE})$.
  \begin{enumerate}
  \item Déterminer une représentation paramétrique de la droite $(AI)$
    puis de la droite $(DJ)$.
  \item Démontrer que les droites $(AI)$ et $(DJ)$ sont sécantes en un
    point dont on déterminera les coordonnées.
  \end{enumerate}
\end{exercice}

\end{colonne*exercice}

\exercicesbac

\begin{colonne*exercice}

  \begin{exercice}[D'après Bac (Asie -- juin 2013 et Amérique du Sud - novembre 2012)]

    Vrai faux

    \emph{Pour chacune des affirmations suivantes, dire si elle est
      vraie ou fausse, et proposer une démonstration de la réponse
      indiquée. Une réponse non justifiée n'est pas prise en compte.}

    \begin{enumerate}
    \item Dans les deux questions suivantes, l'espace est muni d'un
      repère orthonormé
      $(O;\overrightarrow{i},\overrightarrow{j},\overrightarrow{k})$.
    \item Soit S le point de coordonnées $(1~;~3~;~5)$ et $\Delta_{1}$
      la droite de représentation paramétrique
      \[\left\{\begin{array}{l c l}
                 x&=&1 + t\\ 
                 y&=&5 - 4t\\ 
                 z&=&2 - 2t
               \end{array}\right.,\:t\in\mathbb{R}\] 
             \textbf{Affirmation 1 }: la droite $\Delta_{2}$ de
             représentation\\ paramétrique
             \[\left\{\begin{array}{l c l}
                        x&=&-t\\ 
                        y&=&7 + 4t\\ 
                        z&=&7 + 2t 
                      \end{array}\right.,\:t\in\mathbb{R}\]  
                    est la droite parallèle à la droite $\Delta_{1}$
                    passant par le point S.

                  \item On considère les points I(1~;~0~;~0),
                    J(0~;~1~;~0) et K(0~;~0~;~1).

                    \textbf{Affirmation 2}: la droite $\Delta$ de
                    représentation\\
                    paramétrique \[\left\{\begin{array}{l c l}
                                            x &=& 2 - t \\
                                            y &=& 6 - 2 t\\
                                            z &=&-2 + t
                                          \end{array}\right.,\:t\in\mathbb{R}\] coupe le plan  (IJK) au point E$\left(- \dfrac{1}{2}~;~1~;~\dfrac{1}{2} \right)$.

                                      \item Dans le cube ABCDEFGH, le
                                        point T est le milieu du
                                        segment [HF].


                                        \begin{center}
                                          \includestandalone{\standalonepath/exbac1}
                                        \end{center}
                                        \textbf{Affirmation 3} : les
                                        droites (AT) et (EC) sont
                                        orthogonales

                                      \end{enumerate}
                                    \end{exercice}

%

\begin{exercice}[D'après Bac (Polynésie -- juin 2015)]
  On considère le pavé droit $ABCDEFGH$ ci-dessous, pour lequel
  $AB = 6$, $AD = 4$ et $AE = 2$.\\
  $I$, $J$ et $K$ sont les points tels que :\\
  $\overrightarrow{AI} = \dfrac{1}{6} \overrightarrow{AB}$,
  $ \overrightarrow{AJ} = \dfrac{1}{4} \overrightarrow{AD}$ et
  $ \overrightarrow{AK} = \dfrac{1}{2} \overrightarrow {AE}$.

  \begin{center}
    \includestandalone{\standalonepath/exbac2}
  \end{center}

On se place dans le repère orthonormé $(A;\vec{AI},\vec{AJ},\vec{AK})$.
\begin{enumerate}
\item Déterminer une représentation paramétrique du plan $(IJG)$.
\item Déterminer les coordonnées du point d'intersection L du plan
  $(IJG)$ et de la droite $(BF)$.
\item Reproduire la figure et tracer la section du pavé $ABCDEFGH$ par
  le plan $(IJG)$. On ne demande pas de justification.
\end{enumerate}
\end{exercice}

\begin{exercice}[D'après Bac (Métropole -- juin 2015)]
  Dans un repère orthonormé $(O,~I,~J,~K)$ d'unité 1~cm, on considère
  les points $A(0~;~-1~;~5)$, $B(2~;~-1~;~5)$, $C(11~;~0~;~1)$ et
  $D(11~;~4~;~4)$.

  Un point $M$ se déplace sur la droite $(AB)$ dans le sens de $A$ vers $B$
  à la vitesse de 1~cm par seconde.

  Un point $N$ se déplace sur la droite $(CD)$ dans le sens de $C$ vers $D$
  à la vitesse de 1~cm par seconde.

  À l'instant $t = 0$ le point $M$ est en $A$ et le point $N$ est en $C$.

  On note $M_t$ et $N_t$ les positions des points $M$ et $N$ au bout
  de $t$ secondes, $t$ désignant un nombre réel positif.

  On admet que $M_t$ et $N_t$, ont pour coordonnées :

  $M_t(t\,;\ -1\,;\ 5)$ et $N_t(11\,;\ 0,8t\,;\ 1 + 0,6 t)$.

  \emph{Les questions $1$ et $2$ sont indépendantes.}

\begin{enumerate}
\item
  \begin{enumerate}
  \item La droite $(AB)$ est parallèle à l'un des axes $(OI)$, $(OJ)$ ou
    $(OK)$. Lequel ?
  \item La droite $(CD)$ se trouve dans un plan $\mathcal{P}$ parallèle
    à l'un des plans $(OIJ)$, $(OIK)$ ou $(OJK)$.
		
    Lequel ? On donnera une représentation paramétrique de ce plan
    $\mathcal{P}$.
  \item Vérifier que la droite $(AB)$, coupe le plan $\mathcal{P}$ au
    point $E(11~;~-1~;~5)$.
  \item Les droites $(AB)$ et $(CD)$ sont-elles sécantes ?
  \end{enumerate}
\item
  \begin{enumerate}
  \item Montrer que $M_t^{}N_t ^2 = 2 t^2 - 25,2 t + 138$.
  \item À quel instant $t$ la longueur $M_tN_t$ est-elle minimale ?
  \end{enumerate}
\end{enumerate}
\end{exercice}

\begin{exercice}[D'après Bac (Métropole -- juin 2014)]
  Dans l'espace, on considère un tétraèdre $ABCD$ dont les faces
  $ABC$, $ACD$ et $ABD$ sont des triangles rectangles isocèles en
  $A$. On désigne par $E$, $F$ et $G$ les milieux respectifs des côtés
  $[AB]$, $[BC]$ et $[CA]$.

  On choisit $AB$ comme unité de longueur et on se place dans le
  repère orthonormé
  $(A;\overrightarrow{AB},\overrightarrow{AC},\overrightarrow{AD})$
  de\linebreak l'espace.
  \begin{enumerate}
  \item Donner les coordonnées des points $D$ et $E$.
  \item Donner une représentation paramétrique de la droite $(DF)$.
  \item On désigne par $M$ un point de la droite $(DF)$ et par $t$ le
    réel tel que $\overrightarrow{DM}=t\overrightarrow{DF}$. On note
    $\alpha$ la mesure principale en radian de l'angle géométrique
    $\widehat{EMG}$.

    Le but de cette question est de déterminer la position du point
    $M$ pour que la mesure de $\alpha$ soit maximale.
    \begin{enumerate}
    \item Démontrer que
      $ME^2=\dfrac{3}{2}t^2-\dfrac{5}{2}t+\dfrac{5}{4}$.

      \vspace{2mm}

    \item Démontrer que le triangle $MEG$ est isocèle en $M$.\\
      En déduire que :
      $ME \sin(\dfrac{\alpha}{2})=\dfrac{1}{2\sqrt{2}}$.
    \item Justifier que $\alpha$ est maximale si et seulement si
      $\sin(\dfrac{\alpha}{2})$ est maximal.

      En déduire que $\alpha$ est maximale si et seulement si $ME^2$
      est minimal.
    \item Conclure.
    \end{enumerate}
  \end{enumerate}
\end{exercice}

\begin{exercice}[D'après Bac (Pondichéry -- 2013)]

  \emph{Pour chacune des questions, plusieurs propositions de réponse
    sont données dont une seule est exacte. Pour chacune des questions
    indiquer, sans justification, la bonne réponse sur la copie. Une
    réponse exacte rapporte $1$ point.\\ Une réponse fausse ou l'absence
    de réponse ne rapporte ni n'enlève aucun point. Il en est de même
    dans le cas où plusieurs réponses sont données pour une même
    question.}

  \medskip
 
  L'espace est rapporté à un repère orthonormal. $t$ et $t'$ désignent
  des paramètres réels.

  Le plan (P) est le plan $ABC$ avec $A(0~;~1~;~-1)$, $B(-2~;~0~;~-1)$
  et $C(-3~;~1~;~0)$.

  Le plan (S) a pour représentation paramétrique
  $\left\{\begin{array}{l c l}
            x&=&- 2 + t + 2t'\\
            y&=&- t - 2t'\\
            z&=&- 1 - t + 3t'
          \end{array}\right.$  $t$ dans $\R$ $t'$ dans $\R$.

        La droite (D) a pour représentation paramétrique
        $\left\{\begin{array}{l c l}
                  x&=&- 2 + t\\
                  y&=&- t \\
                  z&=&- 1 - t
                \end{array}\right.$ $t$ dans $\R$.
 
              On donne les points de l'espace $M(-1~;~2~;~3)$ et
              $N(1~;~-2~;~9)$.

              \medskip
 
              \begin{enumerate}
              \item Une représentation paramétrique du plan (P) est (à
                chaque fois, $t$ dans $\R$, $t'$ dans $\R$) :

                \medskip {\footnotesize\begin{tabularx}{\linewidth}{XX}
                    \textbf{a.~}$\left\{\begin{array}{l c l} x&=&
                        t\\y&=& 1- 2t\\ z&=& -1 +
                        3t \end{array}\right.$&
                    \textbf{b.~}$\left\{\begin{array}{l c l} x&=& t +
                        2t'\\y&=& 1- t + t'\\z&=& - 1 -
                        t\end{array}\right.$\\
                    \textbf{c.~}$\left\{\begin{array}{l c l} x&=&t +
                        t'\\ y&=& 1 - t- 2t'\\z&=& 1 - t -
                        3t'\end{array}\right.$&
                    \textbf{d.~}$\left\{\begin{array}{l c l} x&=& 1 +
                        2t + t'\\y&=& 1 - 2t + 2t'\\z&=& - 1 -
                        t'\end{array}\right.$
                  \end{tabularx}\par} 

                \medskip

              \item
                \begin{enumerate}
		\item La droite (D) et le plan (P) sont sécants au
                  point K$(- 8~;~3~;~2)$.
		\item La droite (D) est une droite du plan (P).
		\item La droite (D) et le plan (P) sont strictement
                  parallèles.
                \end{enumerate}
              \item
                \begin{enumerate}
		\item La droite (MN) et la droite (D) sont non
                  coplanaires.
		\item La droite (MN) et la droite (D) sont parallèles.
		\item La droite (MN) et la droite (D) sont sécantes.
		\item La droite (MN) et la droite (D) sont confondues.
                \end{enumerate}
              \item
                \begin{enumerate}
		\item Les plans (P) et (S) sont parallèles.
		\item La droite $(\Delta)$ de représentation
                  paramétrique
                  $\left\{\begin{array}{l c l}x&=&t\\y&=&- 2 - t\\z
                      &=& -3-t \end{array}\right.$
                  est la droite d'intersection des plans (P) et (S).
		\item Le point M appartient à l'intersection des plans
                  (P) et (S).
		
                \end{enumerate}
              \end{enumerate}
            \end{exercice}


\end{colonne*exercice}

%%%%%%%%%%%%%%
\exercicesappr
%%%%%%%%%%%%%%

\begin{colonne*exercice}

  \begin{exercice}[Vrai ou faux ?]
    Dire si les propositions suivantes sont vraies ou fausses et le
    démontrer.
    \begin{enumerate}
    \item Si deux plans sont perpendiculaires, alors toute droite de
      l'un est orthogonale à toute droite de l'autre.
    \item Si deux droites sont parallèles, alors toute droite\\
      orthogonale à l'une est orthogonale à l'autre.
    \item Si deux droites sont perpendiculaires à une même droite
      alors elles sont parallèles entre elles.
    \end{enumerate}
  \end{exercice}

  \begin{exercice}
    Soit $ABCD$ un tétraèdre régulier et $I$ le milieu de $[BC]$.

    \begin{center}
      \includestandalone{\standalonepath/exortho1}
    \end{center}

    \begin{enumerate}
    \item Démontrer que la droite $(BC)$ est orthogonale au plan
      $(ADI)$.
    \item En déduire que $(BC)\perp(AD)$.

      \textit{Le plan $(ADI)$ est le plan orthogonal au segment $[BC]$
        et passant par son milieu. Il est appelé plan médiateur du
        segment $[BC]$. Tous les points de ce plan sont équidistants
        de $B$ et de $C$.}
    \end{enumerate}
  \end{exercice}

  \begin{exercice}
    $ABCDEFGH$ est un cube et $I$ et $J$ sont les milieux respectifs de
    $[BC]$ et $[EH]$.  Le point $M$ est un point du segment $[AG]$
    distinct de $A$ et de $G$.
    \begin{enumerate}
    \item Montrer qu'il existe un unique réel $t\in\left] 0;1\right[ $
      tel que $\overrightarrow{AM}=t\overrightarrow{AG}$.
    \item Construire la section du cube par le plan $(IAM)$. Quelle
      est sa nature ?
    \item Existe-t-il une valeur du réel $t$ tel que les points $J$,
      $M$ et $I$ soient alignés ? Justifier.
    \item Existe-t-il une valeur du réel $t$ tel que les points $H$,
      $M$ et $I$ soient alignés ? Justifier.
    \end{enumerate}
  \end{exercice}

  \begin{exercice}
    $ABCDEFGH$ est un cube de côté $a$. Le point $M$ est défini par
    $\overrightarrow{DM}=\dfrac{1}{3}\overrightarrow{DE}$.
    \begin{enumerate}
    \item Construire la section du cube par le plan $(AGM)$.
    \item Démontrer que cette section est un losange $ANGP$ avec $N$
      et $P$ les milieux respectifs de $[BF]$ et $[DH]$.
    \item En déduire l'aire de cette section en fonction de $a$.
    \end{enumerate}
  \end{exercice}

  \begin{exercice}
    $ABCDEFGH$ est un cube et $I$ ; $J$ et $K$ les points tels que :
    $I\in [AD]$ et $AI=\dfrac{1}{3}AD$ ; $J\in [AB]$ et
    $AJ=\dfrac{1}{3}AB$.

    \begin{center}
      \includestandalone{\standalonepath/cubeortho}
    \end{center}

    Les propositions sont-elles vraies ou fausse ?

    Le démontrer.
    \begin{enumerate}
    \item $(IJ)$ est orthogonale à $(EC)$ ; \item $(IJ)$ est
      orthogonale à $(BG)$ ;
    \item $(IJ)$ est orthogonale à $(HB)$ ; \item $(IJ)$ est
      orthogonale à $(HC)$.
    \end{enumerate}
  \end{exercice}

  \begin{exercice}[][\tice]
    Dans l'espace muni d'un repère orthonormé
    $(O\,;\vec{i},\vec{j},\vec{k})$, on considère les points
    $A(1\,;0\,;0)$, $B(0\,;1\,;0)$, $C(0\,;0\,;1)$ et $I$ le milieu de
    $[AB]$.
    \begin{enumerate}
    \item Construire la figure à l'aide d'un logiciel de géométrie dans
      l'espace.
    \item Placer un point $M$ du segment $[AC]$ et $\wp$ le plan passant
      par $I$ et orthogonal à la droite $(IM)$.
    \item Construire le point $N$ intersection de $\wp$ et de la droite
      $(OB)$.
    \item Conjecturer la position du point $M$ pour laquelle la distance
      $MN$ est minimale.
    \item Démonstration
      \begin{enumerate}
      \item Soit $t$ le réel tel que
        $\overrightarrow{AM}=t\overrightarrow{AC}$. Exprimer les\\
        coordonnées de $M$ en fonction de $t$. On admet que $N(0\,;t\,;0)$.
      \item Exprimer la longueur $MN$ en fonction de $t$.
      \item Déterminer la valeur de $t$ pour laquelle cette longueur est
        minimale.
      \end{enumerate}
    \end{enumerate}
  \end{exercice}

  \begin{exercice}
    $ABCD$ est un tétraèdre. $P$, $Q$ et $R$ sont les points tels que
    $ABPC$, $ABQD$ et $ACRD$ sont des parallélogrammes.  En se plaçant
    dans le repère
    $(A\,;\overrightarrow{AB},\overrightarrow{AC},\overrightarrow{AD})$,
    démontrer que les droites $(BR)$, $(CQ)$ et $(DP)$ sont
    concourantes en un point $K$ dont on déterminera les coordonnées.
  \end{exercice}

  \begin{exercice}\label{ex71G2}
    Une autre preuve du théorème du toit (voir page~\pageref{G2theoremedutoit}) :

    \begin{center}
      \includestandalone{\standalonepath/proprieteparallelisme4}
    \end{center}

    Soit $\wp$ et $\wp'$ deux plans sécants selon une droite $\Delta$,
    $d$ une droite de $\wp$ et $d'$ une droite de $\wp'$telles que
    $d//d'$.  \\ Il s'agit de démontrer que la droite $\Delta$
    intersection de $\wp$ et $\wp'$ est parallèle à $d$ et à $d'$.
    \begin{enumerate}
    \item Soit $\vec{u}$ un vecteur directeur de $d$ et $d'$ et
      $\vec{w}$ un vecteur directeur de $\Delta$.  On considère des
      vecteurs $\vec{v}$ et $\vec{v'}$ tels que : $\vec{u}$ et
      $\vec{v}$ dirigent $\wp$ et
      $\vec{u}$ et $\vec{v'}$ dirigent $\wp'$.\\
      Traduire vectoriellement le fait que $\Delta\subset\wp$ puis que
      $\Delta\subset\wp'$.
    \item En déduire une relation vectorielle entre $\vec{u}$,
      $\vec{v}$ et $\vec{v'}$.
    \item Conclure en utilisant le fait que $\wp$ et $\wp'$ sont
      sécants.
    \end{enumerate}
  \end{exercice}

  \begin{exercice}
    Dans l'espace muni d'un repère orthonormé
    $(O\,; \vec{i},\vec{j},\vec{k})$, soit $\Delta$ la droite de
    représentation paramétrique :

    \begin{center}
      $\begin{cases}x=3-t \\y=2+4t \\z=-2t \end{cases}$ ,
      $t\in\mathbb{R}$
    \end{center}

    \begin{enumerate}
    \item Les points $A(5\,;2\,;6)$ et $B(5\,;-6\,;4)$
      appartiennent-ils à la droite $\Delta$ ?
    \item Déterminer les valeurs des réels $a$ et $b$ tels que le
      point $C(4\,;a\,;b)$ appartienne à $\Delta$.
    \item Soit $M(x\,;y\,;z)\in \Delta$. Exprimer $AM^2$ en fonction
      de $t$.
    \item Déterminer les coordonnées du point $M$ tel que la distance
      $AM$ soit minimale.
    \end{enumerate}
  \end{exercice}

  \begin{exercice}
    Dans l'espace muni d'un repère orthonormé
    $(O\,;\vec{i},\vec{j},\vec{k})$, soit $\Delta$ la droite de
    représentation paramétrique :

    \begin{center}
      $\begin{cases}x=1-t \\y=2t \\z=-t+2 \end{cases}$ , $t\in\mathbb{R}$
    \end{center} 
    et $\wp$ le plan de représentation paramétrique :

    \begin{center}
      $\begin{cases}x=-t' \\y=1+t+3t' \\z=t \end{cases}$
      $t\in\mathbb{R}, t'\in\mathbb{R}$
    \end{center}
    \begin{enumerate}
    \item Le point $C(1\,; 3\,; 2)$ appartient-il au plan $\wp$ ?
    \item Démontrer que la droite $\Delta$ est incluse dans le plan
      $\wp$.
    \item Soit $\wp'$ le plan de représentation paramétrique :

      \begin{center}
        $\begin{cases}x=t+t' \\y=1+2t+t' \\z=-1+3t+t' \end{cases}$
        $t\in\mathbb{R}, t'\in\mathbb{R}$
      \end{center}

      \begin{enumerate}
      \item Montrer que $C\in \wp'$.
      \item Montrer que $\Delta$ coupe $\wp'$ en un point $I$ dont on\\
        déterminera les coordonnées.
      \item Montrer que $CI=\sqrt{3}$.
      \end{enumerate}
    \item Soit $t$ un nombre réel et $M$ le point de $\Delta$ de
      coordonnées :\\$M(-t+1\,;2t\,; -t+2)$.
      \begin{enumerate}
      \item Montrer que \\ $CM^2=6t^2-12t+9$.
      \item Montrer que $CI$ est la distance minimale de $CM$
        lorsque $t$ décrit l'ensemble des réels.
      \end{enumerate}
    \end{enumerate}
  \end{exercice}

  \begin{exercice}
    Soient $A(-2\
    ;0\,;1)$, $B(1\,;2\,;-1)$ et $C(-2\,;2\
    ;2)$ trois points de l'espace muni d'un repère $(O\
    ;\vec{i},\vec{j},\vec{k})$.
    \begin{enumerate}
    \item Vérifier que les points $A$, $B$ et $C$ définissent un plan.
    \item Soit $D(-2\
      ;-1\,;0)$ et $E(-2\,;5\,;2)$.  Démontrer que la droite
      $(DE)$
      et le plan $(ABC)$
      sont sécants en un point $I$
      dont on déterminera les coordonnées.
    \end{enumerate}
  \end{exercice}

\end{colonne*exercice}

\connaissances

\begin{acquis}
  \begin{itemize}
  \item Étudier les positions relatives de droites et de plans dans
    l'espace
  \item Construire et justifier la construction d'une section
  \item Calculer avec des vecteurs de l'espace
  \item Démontrer la coplanarité de points ou vecteurs
  \item Utiliser les représentations paramétriques de droites et plans
    de l'espace
  \item Étudier des positions relatives dans l'espace
  \end{itemize}
\end{acquis}

\QCMautoevaluation{Pour chaque question, plusieurs réponses sont
  proposées.  Déterminer celles qui sont correctes.}

\begin{QCM}
  \begin{EnonceCommunQCM}
    On considère le cube $ABCDEFGH$ de côté $a$, avec $I$, $J$ les
    milieux respectifs des segments $[CD]$ et $[GH]$ et $L$ est le
    milieu du segment $[GH]$.

    \begin{center}
     \includestandalone{\standalonepath/cubeqcm}
    \end{center}
  \end{EnonceCommunQCM}
  
\begin{GroupeQCM}
  \begin{exercice}
    La droite $(BI)$ est :
    \begin{ChoixQCM}{3}
    \item orthogonale à $(IJ)$
    \item orthogonale à $(IL)$
    \item orthogonale à $(DG)$
    \end{ChoixQCM}
    \begin{corrige}
      \reponseQCM{a}
    \end{corrige}
  \end{exercice}

  \begin{exercice}
    L'intersection du plan $(BIL)$ avec le plan $(ABF)$ est :
    \begin{ChoixQCM}{3}
    \item une droite passant par\\ le milieu de $[AB]$
    \item une droite passant par\\ le point $B$
    \item une droite parallèle à $(IL)$
    \end{ChoixQCM}
  \end{exercice}
  \begin{corrige}
    \reponseQCM{b} et \reponseQCM{c}
  \end{corrige}

  \begin{exercice}
    La section du cube $ABCDEFGH$ par le plan $(BIL)$ est :
    \begin{ChoixQCM}{3}
    \item un triangle
    \item un parallélogramme
    \item un trapèze
    \end{ChoixQCM}
  \end{exercice}
  \begin{corrige}
    \reponseQCM{c}
  \end{corrige}

  \begin{exercice}
    Dans le repère
    $(A;\overrightarrow{AB},\overrightarrow{AD},\overrightarrow{AE})$
    on a :
    \begin{ChoixQCM}{3}
    \item $\overrightarrow{BJ}\begin{pmatrix}
        -0,5\\
        1\\
        1
      \end{pmatrix}$
    \item les points $L$, $I$, $B$ et $F$ sont\\ coplanaires
    \item
      $\overrightarrow{AJ}=2\overrightarrow{AF}+\overrightarrow{GH}-\overrightarrow{CG}$
    \end{ChoixQCM}
  \end{exercice}
  \begin{corrige}
    \reponseQCM{a}
  \end{corrige}
\end{GroupeQCM}
\end{QCM}

\begin{QCM}
\begin{EnonceCommunQCM}
  Dans un repère
  $(O\,;\overrightarrow{i},\overrightarrow{j},\overrightarrow{k})$ de
  l'espace, on considère les points $A(1\,;0\,;2)$, $B(2\,;1\,;2)$,
  $C(3\,;0\,;0)$ et $D(5\,;-2\,;-4)$.
\end{EnonceCommunQCM}

\begin{GroupeQCM}
  \begin{exercice}
    Les points $A$, $B$ et $C$ :
    \begin{ChoixQCM}{3}
    \item sont alignés
    \item sont coplanaires
    \item définissent un plan
    \end{ChoixQCM}
  \end{exercice}
  \begin{corrige}
    \reponseQCM{b} et \reponseQCM{c}
  \end{corrige}

  \begin{exercice}
    Les points $A$, $B$, $C$ et $D$ :
    \begin{ChoixQCM}{3}
    \item sont coplanaires
    \item vérifient l'égalité\\
      $\overrightarrow{AD}=-2\overrightarrow{AB}+3\overrightarrow{AC}$
    \item $D\in(BC)$
    \end{ChoixQCM}
  \end{exercice}
  \begin{corrige}
    \reponseQCM{a b c}
  \end{corrige}

  \begin{exercice}
    Une représentation paramétrique de :
    \begin{ChoixQCM}{3}
    \item la droite $(AB)$ est :\\
      $\begin{cases}x=2-t\\y=1-t \\z=2+t \end{cases}$,
      $t\in\mathbb{R}$
    \item du plan $(ABC)$ est :\\
      $\begin{cases}x=5+t+4t' \\y=-2-t-2t' \\z=-4-2t-6t' \end{cases}$\\
      $t\in\mathbb{R}$ et $t'\in\mathbb{R}$
    \item du plan $(ABC)$ est :\\
      $\begin{cases}x=1+t+2t' \\y=t \\z=2-2t' \end{cases}$\\
      $t\in\mathbb{R}$ et $t'\in\mathbb{R}$
    \end{ChoixQCM}
  \end{exercice}
  \begin{corrige}
    \reponseQCM{b} et \reponseQCM{c}
  \end{corrige}

  \begin{exercice}
    Soit $E(3\,;4\,;5)$ :
    \begin{ChoixQCM}{3}
    \item la droite parallèle à $(AB)$ et passant par $E$ a
      pour représentation paramétrique
      $\begin{cases}x=t\\y=1+t \\z=5\end{cases}$ , $t\in\mathbb{R}$
    \item Le point $E$ appartient au plan $(ABC)$
    \item les droites $(AB)$ et $(DE)$ sont non coplanaires
    \end{ChoixQCM}
  \end{exercice}
  \begin{corrige}
    \reponseQCM{a} et \reponseQCM{c}
  \end{corrige}

\end{GroupeQCM}
\end{QCM}

\enlargethispage{2cm}
\vspace{-5mm}

\begin{QCM}
\begin{EnonceCommunQCM}
  Dans un repère
  $(O\ ;\ \overrightarrow{i},\overrightarrow{j},\overrightarrow{k})$ de
  l'espace, on considère les droites 

  $d$ :
  $\begin{cases}x=1-t \\y=2t \\z=3 \end{cases}$ $t\in\mathbb{R}$ et
  $d'$ :
  $\begin{cases}x=3+t \\y=1-4t \\z=2t \end{cases}$  $t\in\mathbb{R}$\\
  et le plan $\wp$  de représentation paramétrique :
  $\begin{cases}x=2+t-t' \\y=-2t+3t' \\z=4-t' \end{cases}$
  $t\in\mathbb{R}, t'\in\mathbb{R}$
\end{EnonceCommunQCM}
\begin{GroupeQCM}
  \begin{exercice}
    \begin{ChoixQCM}{3}
    \item la droite $d$ est parallèle au plan
      $(O\,;\overrightarrow{i},\overrightarrow{j})$
    \item la droite $d$ est parallèle au plan
      $(O\,;\overrightarrow{i},\overrightarrow{k})$
    \item la droite $d$ est parallèle à la droite
      $(O\,;\overrightarrow{k})$
    \end{ChoixQCM}
  \end{exercice}
  \begin{corrige}
    \reponseQCM{a}
  \end{corrige}


  \begin{exercice}
    \begin{ChoixQCM}{3}
    \item $d$ et $\wp$ sont parallèles
    \item $d$ et $\wp$ sont sécants\\ en $A(-1;4;3)$
    \item $d$ est inclus dans $\wp$
    \end{ChoixQCM}
  \end{exercice}
  \begin{corrige}
    \reponseQCM{a}
  \end{corrige}

  \begin{exercice}
    \begin{ChoixQCM}{3}
    \item $d$ et $d'$ sont parallèles
    \item  $d$ et $d'$ sont sécantes 
    \item  $d$ est $d'$ ne sont pas coplanaires 
    \end{ChoixQCM}
  \end{exercice}
  \begin{corrige}
    \reponseQCM{c}
  \end{corrige}
\end{GroupeQCM}
\end{QCM}

%%%%%%%%%%%%%%%%%
\TravauxPratiques
%%%%%%%%%%%%%%%%%

\begin{TP} [Section d'un cube par un plan\hfill\tice]

  \begin{enumerate}
  \item Construire un cube $ABCDEFGH$ à l'aide d'un logiciel de
    géométrie dans l'espace.
  \item Construire l'intersection de chacune des faces du cube
    $ABCDEFGH$ par le plan $(IJK)$ dans chacun des cas suivants:
    \begin{enumerate}
    \item $I$ et $J$ sont les milieux respectifs des segments $[AB]$
      et $[AD]$ et $K$ est un point du segment $[AE]$ ;
    \item $I$ et $J$ sont les milieux respectifs des segments $[AB]$
      et $[AD]$ et $K$ est un point du segment $[EH]$ ;
    \item $I$ et $J$ sont les milieux respectifs des segments $[AB]$
      et $[AD]$ et $K$ est un point du segment $[BF]$.
    \end{enumerate}
  \item Vérifier ensuite chacune des constructions en créant la
    section du cube par le plan $(IJK)$ comme ci-dessous avec, par
    exemple, le logiciel Géoplan-Géospace :

    \begin{center}
      \includegraphics[width=\linewidth]{tp}
    \end{center}
  \end{enumerate}
\end{TP}

\begin{TP}[Étudier les positions relatives à l'aide d'un logiciel de calcul formel]
Dans l'espace muni d'un repère
$(O\,;\overrightarrow{i},\overrightarrow{j},\overrightarrow{k})$, on
considère :
\begin{itemize}
\item Le plan $\wp$ de représentation paramétrique \begin{center}
    $\begin{cases}x=-3+2t \\y=7+4t-t' \\z=-1+3t+2t' \end{cases}$ ,
    $t\in\mathbb{R}$ et $t'\in\mathbb{R}$
  \end{center}
\item Le plan $\wp'$ de représentation paramétrique \begin{center}
    $\begin{cases}x=5-t+2t' \\y=2+3t-4t' \\z=1+5t' \end{cases}$ ,
    $t\in\mathbb{R}$ et $t'\in\mathbb{R}$
  \end{center}

\newpage

\item La droite $d_1$ de représentation paramétrique \begin{center}
    $\begin{cases}x=1-2t \\y=3-3t \\z=-5t \end{cases}$ ,
    $t\in\mathbb{R}$
  \end{center}
\item La droite $d_2$ de représentation paramétrique \begin{center}
    $\begin{cases}x=2-3t \\y=9+7t\\z=-4-5t \end{cases}$ ,
    $t\in\mathbb{R}$
  \end{center}
\item le point $A(-1\,;10\,;4)$.
\end{itemize}

\bigskip

\noindent \framebox{1} {\tt resoudre([1-2t=2-3k,3-3t=9+7k,-5t=-4-5k],[t,k]) } \\
\begin{equation} \label{eq:0}
[]
\end{equation}
\noindent \framebox{2} {\tt resoudre([-3+2t=2-3k,7+4t-u=9+7k,-1+3t+2u=-4-5k],[t,u,k]) } \\
\begin{equation} \label{eq:1}
\left(\begin{array}{ccc}
\frac{16}{17} & -\frac{281}{51} & \frac{53}{51}
\end{array}\right) 
\end{equation}
\noindent \framebox{3} {\tt resoudre([5-t+2u=2-3k,2+3t-4u=9+7k,1+5u=-4-5k],[t,u,k]) } \\
\begin{equation} \label{eq:2}
\left(\begin{array}{ccc}
k+1 & -k-1 & k
\end{array}\right) 
\end{equation}
\noindent \framebox{4} {\tt resoudre([-3+2t=1-2k,7+4t-u=3-3k,-1+3t+2u=-5k],[t,u,k]) } \\
\begin{equation} \label{eq:3}
[]
\end{equation}
\noindent \framebox{5} {\tt resoudre([5-t+2u=1-4k,2+3t-4u=3+3k,1+5u=5k],[t,u,k]) } \\
\begin{equation} \label{eq:4}
\left(\begin{array}{ccc}
-\frac{24}{11} & -\frac{64}{55} & -\frac{53}{55}
\end{array}\right) 
\end{equation}
\noindent \framebox{6} {\tt resoudre([-3+2t=-1,7+4t-u=10,-1+3t+2u=4],[t,u]) } \\
\begin{equation} \label{eq:5}
\left(\begin{array}{cc}
1 & 1
\end{array}\right) 
\end{equation}

% {\small
% $
% \begin{array}{ll}
% \multicolumn{1}{c}{\text{Résolution}} & \multicolumn{1}{c}{\text{Résultats}}\\\hline
% ([1 - 2t = 2 - 3k,  3 - 3t = 9 + 7k,  -5t = - 4 - 5k], [t,k]) & []\\
% ([- 3 + 2t = 2 - 3k,  7 + 4t - u = 9 + 7k,  -1 + 3t + 2u = - 4 - 5k], [t,u,k]) & (16/17      -281/51     53/51)\\
% ([5 - t + 2u = 2 - 3k,  2 + 3t - 4u = 9 + 7k,  1 + 5u = - 4 - 5k], [t,u,k]) & (K + 1      - K- 1        k)\\
% ([- 3 + 2t = 1 - 2k,  7 + 4t - u = 3 - 3k,  -1 + 3t + 2u = -5k], [t,u,k]) & []\\
% ([5 - t + 2u = 1 - 4k,  2 + 3t - 4u = 3 + 3k,  1 + 5u = 5k], [t,u,k]) & (- 24/11    - 64/55    - 53/55)\\
% ([- 3 + 2t = - 1,  7 + 4t - u = 10,  -1 + 3t + 2u = 4], [t,u]) & (1    1)\\\hline&
% \end{array}
% $\par}

% \begin{enumerate}
% \item resoudre
%   \begin{ttfamily}
%     ([1-2t=2-3k,3-3t=9+7k,-5t=-4-5k],[t,k])
%   \end{ttfamily}
%   \begin{equation} \label{eq:0} []
%   \end{equation}

% \item   resoudre
%   \begin{ttfamily}
%     ([-3+2t=2-3k,7+4t-u=9+7k,-1+3t+2u=-4-5k],[t,u,k])
%   \end{ttfamily}
%   \begin{equation} \label{eq:1} \left(
%       \begin{array}{ccc} 
%         \frac{16}{17} & -\frac{281}{51} & \frac{53}{51}
%       \end{array}\right) 
%   \end{equation}

% \item   resoudre
%   \begin{ttfamily}
%     ([5-t+2u=2-3k,2+3t-4u=9+7k,1+5u=-4-5k],[t,u,k])
%   \end{ttfamily}
%   \begin{equation} \label{eq:2} \left(
%       \begin{array}{ccc} 
%         k+1 & -k-1 & k
%       \end{array}
%     \right) 
%   \end{equation}
  
% \item   resoudre
%   \begin{ttfamily}
%     ([-3+2t=1-2k,7+4t-u=3-3k,-1+3t+2u=-5k],[t,u,k])
%   \end{ttfamily}
%   \begin{equation} \label{eq:3} []
%   \end{equation}

% \item   resoudre
%   \begin{ttfamily}
%     ([5-t+2u=1-4k,2+3t-4u=3+3k,1+5u=5k],[t,u,k])
%   \end{ttfamily}
%   \begin{equation} \label{eq:4} \left(
%       \begin{array}{ccc}
%         -\frac{24}{11} & -\frac{64}{55} & -\frac{53}{55}
%       \end{array}
%     \right) 
%   \end{equation}

% \item   resoudre
%   \begin{ttfamily}
%     ([-3+2t=-1,7+4t-u=10,-1+3t+2u=4],[t,u])
%   \end{ttfamily}
%   \begin{equation} \label{eq:5} \left(
%       \begin{array}{cc} 
%         1 & 1
%       \end{array}\right) 
%   \end{equation}
% \end{enumerate}

% À l'aide des résultats fournis par la copie d'écran ci-dessus d'un
% logiciel de calcul formel, 

Répondre aux questions suivantes, en
précisant la ligne de la copie d'écran vous permettant de conclure :
\begin{enumerate}
\item Le point $A$ appartient-il au plan $\wp$ ?
\item Déterminer la position relative des droites $d_1$ et $d_2$.
\item Déterminer la position relative de la droite $d_1$ et du plan
  $\wp$.
\item Déterminer la position relative de la droite $d_2$ et du plan
  $\wp$.
\item Déterminer la position relative de la droite $d_1$ et du plan
  $\wp'$.
\item Déterminer la position relative de la droite $d_2$ et du plan
  $\wp'$.
\end{enumerate}
\end{TP}

\newpage

\begin{TP}[En cinématique]
  La position à l'instant $t$ ($t>0$) d'un point $M$ en mouvement dans
  l'espace se définit dans un repère
  $(O\,;\overrightarrow{i},\overrightarrow{j},\overrightarrow{k})$ par
  les coordonnées $M(x(t)\,; y(t)\,; z(t))$.  Les fonctions $x(t)$,
  $y(t)$ et $z(t)$ sont appelées équations horaires du mouvement.  Par
  exemple, on considère deux points M et N dont les mouvements en
  fonction du temps, sont donnés respectivement par :
  $\begin{cases}x(t)=t \\y(t)=2-4t \\z(t)=-2+3t \end{cases}$ et
  $\begin{cases}x(t)=2-3t+2t^2 \\y(t)=1 \\z(t)=-2\end{cases}$.

  Sachant que le vecteur vitesse d'un point $P(x(t)\,;y(t)\,;z(t)$ à
  l'instant $t$ est
  $\overrightarrow{v}\begin{pmatrix} x'(t)\\y'(t)\\z'(t)
  \end{pmatrix}$
  et que le vecteur accélération du point $P$ à l'instant $t$ est
  $\overrightarrow{a}\begin{pmatrix} x''(t)\\y''(t)\\z''(t)
  \end{pmatrix}$.
  \begin{enumerate}
  \item Montrer que le point M est animé d'un mouvement rectiligne
    uniforme (sa vitesse est constante).
  \item Montrer que le point N est animé d'un mouvement rectiligne
    uniformément varié (son\\ accélération est constante).
  \end{enumerate}
\end{TP}

\recreation

\begin{enigme}
  Vous avez étudié dans ce chapitre les représentations paramétriques
  de droites, mais de nombreuses autres courbes du plan comme de
  l'espace ont des représentations paramétriques.  Dans un repère
  orthonormé du plan, saurez-vous par exemple construire à l'aide des
  valeurs remarquables du sinus l'allure de la courbe de Lissajous
  dont un
  système d'équations paramétrique est : \\
  $\begin{cases}x(t)=sin(2t)\\y(t)=sin(3t) \end{cases}$,
  $t\in]-\pi;\pi]$ ?

  \smallskip

  De même, vous avez étudié les représentations paramétriques de
  plans, mais de nombreuses autres surfaces ont des représentations
  paramétriques comme le tore ci-dessous représenté à l'aide du
  logiciel Xcas :

  \begin{center}
    \includegraphics[scale=0.2]{ToreV3}
  \end{center}
\end{enigme}


%%% Local Variables:
%%% mode: latex
%%% TeX-master: "../Manuel-TS"
%%% End:
