\begin{activite}[Boîtes dans une caisse]
\begin{multicols}{2}
Soit B une boîte en forme de pavé droit de hauteur $L$, à base carrée de côté $\ell$, où $\ell$ et $L$ sont des entiers naturels non nuls tels que $\ell < L$. 

On veut remplir la boîte B avec des cubes tous identiques dont l'arête $a$ est un entier naturel non nul (les cubes devant remplir complètement la boîte B sans laisser d'espace vide).\vspace{1.5cm}

\begin{center}
\psset{xunit=1cm, yunit=1cm}
\begin{pspicture}(5,5.5) 
\psline[arrowsize=3pt 2,arrowinset=0.25,linewidth=0.4pt]{<->}(0,0)(3.4,0)
\psline[arrowsize=3pt 2,arrowinset=0.25,linewidth=0.4pt]{<->}(3.8,0.3)(4.9,1.5)
\psline[arrowsize=3pt 2,arrowinset=0.25,linewidth=0.4pt]{<->}(5,1.5)(5,5.5) 

\pspolygon[linewidth=1.2pt](0,0.3)(3.4,0.3)(3.4,4.4)(0,4.4) 
\psline[linewidth=1.2pt](3.4,0.3)(4.5,1.5)(4.5,5.5)(3.4,4.4) 
\psline[linewidth=1.2pt](3.4,4.4)(4.5,5.5)(1.2,5.5)(0,4.4) 

\psline[linestyle=dashed](0,0.3)(1.2,1.5)(4.5,1.5) 
\psline[linestyle=dashed](1.2,1.5)(1.2,5.5)
 
\uput[d](1.7,0){$\ell$} 
\uput[r](4.35,0.8){$\ell$} 
\uput[r](5,3.5){$L$} 
\end{pspicture} 
\end{center}
\end{multicols}

\vspace{-10pt}

\begin{enumerate}
\item Dans cette question, $\ell = 882$ et $L = 945$. 
\begin{enumerate}
\item Quelle est la plus grande valeur possible pour $a$ ?

On appelle $d$ cette valeur. $d$ est appelé le PGCD de $a$ et de $b$.

%Quelle est la plus grande valeur $d$ possible pour~$a$ ?
\item Démontrer que tous les diviseurs de $d$ conviennent comme valeur de $a$.		 
\item Existe-t-il d'autres valeurs possibles pour $a$ ?
\end{enumerate} \medskip
\item Dans cette question, le volume de la boîte B est
  $v = \nombre{77760}$. On sait que, pour remplir la\linebreak boîte
  B, la plus grande valeur possible de $a$ est 12.

On pose :\enskip $\ell=a\ell'$\enskip et\enskip $L=aL'$.
\begin{enumerate}
\item Que  peut-on dire de $\pgcd(\ell';L')$ si $a=12$ ? Pourquoi ?
\item Vérifier, dans le cas\enskip $a=12$,\enskip que\enskip $\ell'^2L'=45$.
\item Montrer qu'il y a exactement deux boîtes B possibles dont on donnera les dimensions. 
\end{enumerate}  
\end{enumerate} 
\end{activite}

\vspace{-1cm}

\begin{activite}[Algorithme pour une solution particulière]
  Le but de cette activité est de déterminer un algorithme permettant
  de déterminer un couple d'entiers relatifs $(x_0\ ;\ y_0)$ solution
  de l'équation (E) : \enskip $59x+27y=1$.

On suppose que cette équation admet des solutions entières.\vspace{-10pt}

\begin{enumerate}
\item Pourquoi peut-on trouver un entier naturel $x_0>0$ tel que le
  couple $(x_0\ ;\ y_0)$ soit solution de~(E) ?\medskip
\item On s'intéresse à la quantité $59u+27v$ où $u$ et $v$ varient sur
  $\Z$. 

  On propose l'algorithme de la page suivante pour calculer les
  valeurs de $x_0$ et $y_0$. 

  E($x$) signifie la partie entière de $x$.
\begin{enumerate}
\item Que fait-on à la ligne 8 ?
\item Que calcule-t-on à la ligne 9 ?
\item Expliquer la condition ($r\neq1$) dans la boucle conditionnelle.
\item Déterminer la valeur à donner à la variable $v$ à la ligne 11,
  pour que $v$ donne la valeur $y_0$.
\item Rentrer le programme dans la calculatrice et donner une solution
  $(x_0\ ;\ y_0)$ de~(E).
\end{enumerate}

\pagebreak

%\begin{multicols}{2}

\begin{center}
\begin{minipage}{0.5\textwidth}
\begin{center}
\begin{algorithme}
\BlocVariables
\DeclareVar{$u$, $v$, $r$}{entiers}{}
\BlocEntrees
\DonnerValeur{$r$}{$0$}
\DonnerValeur{$u$}{$0$}
\BlocTraitements
\TantQue{$r\neq1$}{%
\DonnerValeur{$u$}{$u+1$}
\DonnerValeur{$r$}{\\[2pt]\hspace*{1.3cm}$59u-27\times\text{\normalfont E}\left(\dfrac{59u}{27}\right)$}
}
\DonnerValeur{$v$}{$\dots$}
\BlocAffichage
\Afficher{$u$, $v$}
\end{algorithme}
\end{center}
\end{minipage}
\end{center}
%\end{multicols}
\end{enumerate}
\end{activite}

\vspace{-1cm}

\begin{activite}[\MotDefinition{Chiffrement affine}{}]
\label{Ar2Acti3}
Le \MotDefinition{chiffrement}{} ou cryptage consiste à transformer un message en
message codé (ou chiffré). 

Le \MotDefinition{déchiffrement}{} est le procédé inverse, il consiste à décoder un message codé.\medskip

\partie[Un premier exemple]
% {\color{H1}\textbf{Partie A : Un premier exemple}}

Afin de \MotDefinition{coder}{} un message, on assimile chaque lettre de l'alphabet à un nombre entier comme l'indique le tableau ci-dessous :

\begin{center}
\renewcommand*\tabularxcolumn[1]{>{\centering\arraybackslash}m{#1}}
{\footnotesize\begin{ttableau}{0.9\linewidth}{13}\hline
\rowcolor{FondTableaux}A& B& C& D& E& F& G& H& I& J& K& L& M\\ \hline
0& 1& 2& 3& 4& 5& 6& 7& 8& 9& 10& 11& 12\\ \hline \hline
\rowcolor{FondTableaux}N& O& P& Q& R& S& T& U& V& W& X& Y& Z\\ \hline
13& 14& 15& 16& 17& 18& 19& 20& 21& 22& 23& 24& 25\\ \hline
\end{ttableau}}
\end{center}\medskip

Un chiffrement élémentaire est le chiffrement affine. On se donne une fonction de codage\linebreak affine~$f$, par exemple : \enskip $f(x)=11x+8$.

À une lettre du message :
\begin{itemize}
\item on associe un entier $x$ entre 0 et 25 suivant le tableau ci-dessus ;
\item on calcule\enskip $f(x)=11x+8$\enskip et l'on détermine le reste $y$ de la division de $f(x)$ par 26 ;
\item on traduit $y$ par une lettre d'après le tableau ci-dessus.
\end{itemize}\medskip

{\color{H1}\textbf{Exemple :}} Si l'on veut coder la lettre G par la fonction $f(x)=11x+8$, on passe par les étapes suivantes :
{\centering$\text{G}\enskip\Rightarrow\enskip x=6 \enskip\Rightarrow\enskip 11\times6+8=74\enskip\Rightarrow\enskip 74\equiv22~(26)\enskip\Rightarrow\enskip y=22 \enskip\Rightarrow\enskip \text{W}$\par}

La lettre G est donc codée par la lettre W.
\begin{enumerate}
\item  Coder la lettre W.
\item  Existence d'une fonction de décodage.
	
	\begin{cadre}
		\textbf{Théorème de Bézout} : $a$ et $b$ sont deux entiers naturels. \ofg{$a$ et $b$ sont premiers entre eux} équivaut à \ofg{il existe deux entiers relatifs $u$ et $v$ tels que $au+bv=1$}.\end{cadre}
	
	
	\begin{enumerate}
		\item Pourquoi le théorème de Bézout permet-il d'affirmer qu'il existe un entier relatif $u$ tel que : \enskip $11u +26v=1$ ?
		\item Montrer alors que l'équation $11x\equiv 1~(26)$,
                  puis que l'équation $11x\equiv j~(26)$, $j$ étant un
                  entier naturel, admettent une solution.
		\end{enumerate}\vspace{-5pt}
		
	\item  Déterminer la fonction de décodage.
		\begin{enumerate}
		\item Montrer que pour tous entiers relatifs $x$ et
                  $j$, on a
                  :\enskip$11x \equiv j~(26)\enskip
                  \Leftrightarrow\enskip x \equiv 19j~(26)$.
		\item En déduire que la fonction $f^{-1}$ de décodage
                  est $f^{-1}(y)=19y+4$.
		\item  Décoder la lettre L.
\end{enumerate}
\end{enumerate}

% {\color{H1}\textbf{Partie B : Codage et décodage}}

\partie[Codage et décodage]

La fonction de codage est définie par la fonction $f$ telle que : \enskip $f(x)=21x+11$.
\begin{enumerate}
\item \begin{multicols}{2}Coder le  mot : \enskip ENIGME.
%

On pourra éventuellement remplir le tableau ci-contre.

\begin{center}
\renewcommand*\tabularxcolumn[1]{>{\centering\arraybackslash}m{#1}}
\begin{Ctableau}{\linewidth}{7}{c}\hline
Lettre&E&N&I&G&M&E\\ 
\hline   
$x$&4&&&&& \\ 
\hline    
$f(x)$&95&&&&&\\ 
\hline
$y$&17&&&&&\\ 
\hline
Code&R&&&&&\\ 
\hline
\end{Ctableau}
\end{center}
\end{multicols}

\item On cherche la fonction de déchiffrage $f^{-1}$.
\begin{multicols}{2}
\begin{enumerate}
\item Démontrer que, pour tous relatifs $x$ et $z$, on a :\enskip $21x\equiv z~(26)\enskip \Leftrightarrow\enskip x\equiv 5z~(26)$.
\item En déduire que la fonction de décodage est définie par : \enskip $f^{-1}(y)=5y+23$.
\item Décoder le message \enskip RPERNL.
%eureka

On pourra éventuellement remplir le \mbox{tableau} ci-contre.
\end{enumerate}

\begin{center}
\renewcommand*\tabularxcolumn[1]{>{\centering\arraybackslash}m{#1}}
\begin{Ctableau}{\linewidth}{7}{c}\hline
Code&R&P&E&R&N&L\\ 
\hline   
$y$&17&&&&& \\ 
\hline    
$f^{-1}(y)$&108&&&&&\\ 
\hline
$x$&&&&&&\\ 
\hline
Lettre&&&&&&\\ 
\hline
\end{Ctableau}
\end{center}
\end{multicols}

\end{enumerate}

% {\color{H1}\textbf{Partie C : Casser une fonction de cryptage}}

\partie[Casser une fonction de cryptage]

On a reçu le message suivant : \enskip FMEYSEPGCB.\medskip

Par une étude statistique de la fréquence d'apparition des lettres sur un passage plus important, on déduit que le chiffrement est affine, que la lettre E est codée par la lettre E et que la lettre J est codée par la lettre N.\medskip

Soit la fonction affine $f$ définie par : \enskip $f(x)=ax+b$\enskip où $a$ et $b$ sont des entiers naturels compris entre 0 et 25.\vspace{-10pt}

\begin{enumerate}
\item Démontrer que $a$ et $b$ vérifient le système suivant : \enskip $\left\{\begin{aligned}
4a+b&\equiv 4~(26)\\
9a+b&\equiv 13~(26)\end{aligned}\right.$\medskip

\item \begin{enumerate}
\item Démontrer que \enskip $5a\equiv 9~(26)$,\enskip puis que\enskip $a\equiv7~(26)$.
\item En déduire que \enskip $b\equiv2~(26)$\enskip et que $f$ est
  définie par $f(x)=7x+2$.
\item Démontrer que, pour tous relatifs $x$ et $z$, on a :\enskip
$7x\equiv z~(26)\enskip \Leftrightarrow\enskip x\equiv15z~(26)$.

\item En déduire que la fonction de décodage $f^{-1}$ est définie par \enskip $f^{-1}(y)=15y+22$.
\item Décoder le message.
%tu es genial
\end{enumerate}
\end{enumerate}

\vspace{-1cm}
\end{activite}
