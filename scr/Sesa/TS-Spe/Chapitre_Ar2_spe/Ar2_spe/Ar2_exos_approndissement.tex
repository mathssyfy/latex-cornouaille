\begin{colonne*exercice}

\begin{exercice}[Carrelage d'une pièce]

Pour carreler une pièce rectangulaire mesurant 4,18 m sur 5,67 m, un carreleur propose à des propriétaires le choix entre deux modèles de dalles carrées.
\begin{enumerate}
\item Le premier modèle a 29 cm de côté et coûte 2,30 \euro{} l'unité.

Avec ce modèle, il n'utilise que des dalles entières et il complète avec du joint autour de chaque dalle.
\begin{enumerate}
\item Calculer le nombre maximal de dalles que l'on peut poser dans la largeur de la pièce.
\item Calculer le nombre maximal de dalles que l'on peut poser dans la longueur de la pièce.
\item Les joints autour des dalles auront-ils tous la même largeur ?

Si oui, quelle est cette largeur ?
\end{enumerate}
\item Le second modèle a 36 cm de côté et coûte 3,10 \euro{} l'unité.

  Avec ce modèle-là, il est préconisé d'utiliser des joints de 0,6~cm
  et le carreleur est alors dans l'obligation de couper des
  dalles. Les découpes ne sont pas réutilisées.  Calculer le nombre de
  dalles nécessaires.

\item Quel sera le choix le moins coûteux pour l'achat des dalles ?
\end{enumerate}
\end{exercice}

%%%%%%%%%%%%%%%%%%%%%%%%%%%%%%%%%%%%%%%%%%%%%%%%%%%%%%%%%%%%%%%%%%%%%%%%%%%%%%%%%%%%%%%%%%%%%

\begin{exercice}[Vrai ou faux ?]

Pour chacune des quatre propositions, indiquer si elle est vraie ou fausse, et donner une démonstration de la réponse choisie.
\begin{enumerate}
\item \textbf{Proposition 1 :}\enskip $\forall n\in\N^*,\enskip 3n$ et $2n+1$ sont premiers entre eux.
\item On appelle $S$ l'ensemble des couples $(x ; y)$ d'entiers relatifs solutions de l'équation $3x -5y = 2$.

\textbf{Proposition 2 : } \ofg{ L'ensemble $S$ est l'ensemble des couples $(5k\!-\!1~;~3k\!-\!1)$ où $k$ est un entier relatif.}
\item Soit $a$ et $b$ deux entiers naturels.

\textbf{Proposition 3 : } \ofg{S'il existe deux entiers relatifs $u$ et $v$ tels que $au +bv = 2$, alors le PGCD de $a$ et $b$
est égal à 2}.
\item On considère l'équation $(E)$ :\enskip $x^2 -52x +480= 0$,\enskip où $x$ est un entier naturel.

\textbf{Proposition 4 : } \ofg{ Il existe deux entiers naturels non nuls dont le PGCD et le PPCM sont solutions
de l'équation $(E)$.}
\end{enumerate}
\end{exercice}

%%%%%%%%%%%%%%%%%%%%%%%%%%%%%%%%%%%%%%%%%%%%%%%%%%
\columnbreak
%%%%%%%%%%%%%%%%%%%%%%%%%%%%%%%%%%

%%%%%%%%%%%%%%%%%%%%%%%%%%%%%%%%%%%%%%%%%%%%%%%%%%%%%%%%%%%%%%%%%%%%%%%%%%

\begin{exercice}[Conjonction de comètes]

Le but est de déterminer l'ensemble $\mathscr{S}$ des entiers relatifs $n$ vérifiant le système :\vspace{-5pt}
$$\left\{\begin{aligned}
				&n\equiv13\ [19]\\
				&n\equiv6\ [12]\end{aligned}\right.$$
\begin{enumerate}				
\item Recherche d'un élément de $\mathscr{S}$.

On désigne par $(u\ ;\ v)$ un couple d'entiers relatifs tel que $19u+12v=1$.
\begin{enumerate}
\item Justifier l'existence d'un tel couple $(u\ ;\ v)$.

\item On pose $n_0=6\times19u+13\times12v$.

Démontrer que $n_0$ appartient à $\mathscr{S}$.

\item Donner un exemple d'entier $n_0$ appartenant à $\mathscr{S}$.
\end{enumerate}

\item Caractérisation des éléments de $\mathscr{S}$.
\begin{enumerate}
\item Soit $n$ un entier relatif appartenant à $\mathscr{S}$.
 
Démontrer que $n-n_0\equiv0~(228)$.

\item En déduire qu'un entier relatif $n$ appartient à $\mathscr{S}$
  si, et seulement si, $n$ peut s'écrire sous la forme $n=-6+228k$ où
  $k$ est un entier relatif.
\end{enumerate}
\item Application.

  La comète A passe tous les 19 ans et apparaîtra la prochaine fois
  dans 13 ans.

  La comète B passe tous les 12 ans et apparaîtra la prochaine fois
  dans 6 ans.

  Dans combien d'années pourra-t-on observer les deux comètes la même
  année ?
\end{enumerate}
\end{exercice}

\medskip

%%%%%%%%%%%%%%%%%%%%%%%%%%%%%%%%%%%%%%%%%%%%%%%%%%%%%%%%%%%%%%%%%%%%%%%%%%%%%%%%%%%%%%

\begin{exercice}[Restes chinois]

On se propose de résoudre dans $\Z$ le système : \vspace{-5pt}
$$\left\{\begin{aligned}
N&\equiv5\ [13]\\
N&\equiv 1\ [17]
\end{aligned}\right.\vspace{-5pt}$$

\begin{enumerate}
\item Vérifier que 239 est solution de ce système.
\item Soit $N$ un entier relatif solution de ce système.
		
Démontrer que $N$ peut s'écrire sous la forme : 

$N= 1 + 17x = 5 + 13y$\enskip où $x$ et $y$ sont deux entiers relatifs vérifiant la relation $17x - 13y = 4$.
\item Résoudre l'équation $17x - 13y = 4$ où $x$ et $y$ sont des entiers relatifs.
\item En déduire qu'il existe un entier relatif  $k$ tel que $N = 18 + 221k$.
\end{enumerate}
\end{exercice}

%%%%%%%%%%%%%%%%%%%%%%%%%%%%%%%%%%%%%%%%%%%%%%%%%%%%%%%%%%%%%%%%%%%%%%%%%%%%%%%%%%%%%%%

\columnbreak

\begin{exercice}[Un problème du VIII\ieme{} siècle]
\begin{enumerate}
\item On considère l'équation $\text{(E)}~:~8x + 5y = 1$,~où $(x~;~y)$ est un couple de nombres entiers relatifs. 
\begin{enumerate}
\item Donner une solution particulière de l'équation (E). 
\item Résoudre l'équation (E). 
\end{enumerate} 

\item Soit $N$ un nombre naturel tel qu'il existe un couple $(a~;~b)$ de 
nombres entiers vérifiant :\vspace{-5pt} 
$$\left\{\begin{aligned} 
N&=8a + 1\\ 
N&=5b + 2.\\ 
\end{aligned}\right.\vspace{-5pt}$$ 

\begin{enumerate} 
\item Montrer que le couple $(a~;~-b)$ est solution de (E). 
\item Quel est le reste dans la division de $N$ par $40$ ? 
\end{enumerate} 

\item \begin{enumerate}
\item Résoudre l'équation $8x + 5 y = 100$, où $(x~;~ y)$ est un couple de  nombres entiers relatifs. 

\item Au VIII\ieme{} siècle, un groupe composé d'hommes et de femmes a  dépensé $100$ pièces de monnaie dans une auberge. Les hommes ont dépensé $8$ pièces 
chacun et les femmes $5$ pièces chacune. 

Combien pouvait-il y avoir d'hommes et de femmes dans le groupe ? 
\end{enumerate} 
\end{enumerate}
\end{exercice}

%%%%%%%%%%%%%%%%%%%%%%%%%%%%%%%%%%%%%%%%%%%%%%%%%%%%%%%%%%%%%%%%%%%%%%%%%%%%%%%%%%%%%%%%%%%%%


%\columnbreak

%%%%%%%%%%%%%%%%%%%%%%%%%%%%%%%%%%%%%%%%%%%%%%%%%%%%%%%%%%%%%%%%%%%%%%%%%%%%%%%%%%%%%%


\begin{exercice}[Algorithme et équation diophantienne \algo]

On considère l'algorithme suivant, où $A$ et $B$ sont des entiers naturels tels que $A < B$ : 

\begin{center}
\begin{algorithme}
\BlocVariables
\DeclareVar{$A$, $B$, $C$, $D$}{\normalfont entiers}{}
\BlocEntrees
\Saisir{$A$, $B$\enskip tels que $A<B$}
\BlocTraitements
\DonnerValeur{$D$}{$B-A$}
\TantQue{$D>0$}{%
\DonnerValeur{$B$}{$A$}
\DonnerValeur{$A$}{$D$}
\SiAlorsSinon{$B>A$}
{\DonnerValeur{$D$}{$B-A$}}
{\DonnerValeur{$D$}{$A-B$}}
}
\BlocAffichage
\AfficherVar{$A$}
\end{algorithme}
\end{center}

\begin{enumerate}
\item On entre $A = 12$ et $B = 14$. 

En remplissant le tableau suivant, déterminer la valeur affichée par l'algorithme.

\begin{center}
\renewcommand*\tabularxcolumn[1]{>{\centering\arraybackslash}m{#1}}
\begin{ttableau}{0.6\linewidth}{3}\hline
$A$ &  $B$&   $D$\\ \hline   
12&   14& \\ \hline    
&	&\\ \hline
&	&\\ \hline
&	&\\ \hline
&	&\\ \hline
&	&\\ \hline
&	&\\ \hline
&	&\\ \hline
&	&\\ \hline
&	&\\ \hline
&	&\\ \hline
\end{ttableau}
\end{center}
 
\item Que calcule cet algorithme ? 

\item \begin{enumerate}
\item Montrer que 221 et 331 sont premiers entre eux.

\item Justifier que l'équation \enskip $(\text{E})\ :\ \enskip 221x - 331y = 1$\enskip admet des solutions entières.  
		 
\item Déterminer une solution particulière de l'équation (E).

(On pourra éventuellement remonter l'algorithme d'Euclide.)

\item Déterminer l'ensemble des couples $(x~;~y)$ d'entiers relatifs solutions de l'équation (E). 
\end{enumerate}

\item On considère les suites d'entiers naturels $(u_n)$ et $(v_n)$ définies pour tout entier naturel $n$ par 
\[
u_n = 2 + 221n
\qquad \text{et} \qquad \left\{\begin{aligned}
    &v_0=3\\ 
    &v_{n+1}=v_n + 331
  \end{aligned}\right.
\]

\begin{enumerate}
\item Quelle est la nature de la suite $(v_n)$ ? 

En déduire l'expression de $v_n$ en fonction de l'entier naturel $n$. 
\item Déterminer tous les couples d'entiers naturels $(p~;~q)$ tels
  que :
		
$u_p = v_q,\quad 0\Inf p\Inf500$\quad  et \quad $0\Inf q\Inf500$. 
\end{enumerate}
\end{enumerate}
\end{exercice}

%%%%%%%%%%%%%%%%%%%%%%%%%%%%%%%%%%%%%%%%%%%%%%%%%%%%%%%%%%%%%%%%%%%%%%%%%%%%%%%%%%%%%%%%%%%%%


\end{colonne*exercice}
