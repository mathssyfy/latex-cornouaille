\begin{colonne*exercice}


\begin{exercice}[D'après Bac (Antilles-Guyane - juin 2014)]
  En montagne, un randonneur a effectué des réservations dans deux
  types de gîte : l'hébergement A et l'hébergement B.

Une nuit en hébergement A coûte 24 \euro{} et une nuit en  hébergement B coûte 45 \euro{}.

Il se rappelle que le coût  total de sa réservation est de 438 \euro{}.\medskip
 
On souhaite retrouver les nombres $x$ et $y$ de nuitées passées respectivement en hébergement A et en  hébergement B.
\begin{enumerate}
\item \begin{enumerate}
\item Montrer que les nombres $x$ et $y$ sont respectivement inférieurs ou égaux à 18 et 9.		
\item Recopier et compléter les pointillés de l'algorithme suivant afin qu'il affiche les couples ($x$ ; $y$) possibles.

\begin{center}
\begin{algorithme}
\BlocVariables
\DeclareVar{$x$, $y$}{\normalfont entiers}{}
\BlocTraitementsEtAffichage
\Pour{$x$}{0}{\dots}{%
	\Pour{$y$}{0}{\dots}
	{\SiAlors{\quad\dots\quad}
		{\AfficherVar{$x$, $y$}}}}
%\finAlgo
\end{algorithme}
\end{center}
\end{enumerate}

\item Justifier que le coût total de la réservation est un multiple de 3.
\item \begin{enumerate}
\item Justifier que l'équation $8x + 15y = 1$ admet pour solution au moins un couple d'entiers relatifs.
\item Déterminer une telle solution.
\item Résoudre l'équation (E) : $8x + 15y = 146$ où $x$ et $y$ sont des nombres entiers relatifs.
\end{enumerate}
\item Le randonneur se souvient avoir passé au maximum 13 nuits en hébergement A.

Montrer alors qu'il peut retrouver le nombre exact de nuits passées en hébergement A et celui des nuits passées en hébergement B.

Calculer ces nombres.
\end{enumerate}
\end{exercice}

%%%%%%%%%%%%%%%%%%%%%%%%%%%%%%%%%%%%%%%%%%%%%%%%%%%%%%%%%%%%%%%%%%%%%%%%%%%%%%%%%%%%%%%%%%

\begin{exercice}[D'après Bac (Métropole - juin 2011)]

% {\color{H1}\textbf{Partie A - Restitution organisée de connaissances}}
\partie[Restitution organisée des connaissances]

On rappelle ci-dessous le théorème de Bézout et le théorème de Gauss.

Théorème de Bézout :

\ofg{Deux entiers relatifs $a$ et $b$ sont premiers entre eux si et seulement si, il existe un couple $(u, v)$ d'entiers relatifs vérifiant $au + bv = 1$.}

Théorème de Gauss :

\ofg{Soient $a$, $b$, $c$ des entiers relatifs.
Si $a$ divise le produit $bc$ et si $a$ et $b$ sont premiers entre eux, alors $a$ divise $c$.}

\begin{enumerate}
\item En utilisant le théorème de Bézout, démontrer le théorème de Gauss.

\item Soient $p$ et $q$ deux entiers naturels tels que $p$ et $q$ sont premiers entre eux.

Déduire du théorème de Gauss que, si $a$ est un entier relatif, tel que $a\equiv0~(p)$ et $a\equiv0~(q)$, alors $a\equiv0~(pq)$.
\end{enumerate}\medskip

% {\color{H1}\textbf{Partie B}}
\partie{}

On se propose de déterminer l'ensemble $\mathscr{S}$ des entiers relatifs $n$ vérifiant le système :
$$\left\{\begin{aligned}
				&n\equiv9~(17)\\
				&n\equiv3~(5)\end{aligned}\right.$$
\begin{enumerate}			
\item Recherche d'un élément de $\mathscr{S}$.

On désigne par $(u\ ;\ v)$ un couple d'entiers relatifs tel que $17u+5v=1$.
\begin{enumerate}
\item Justifier l'existence d'un tel couple $(u\ ;\ v)$.

\item On pose $n_0=3\times17u+9\times5v$.

Démontrer que $n_0$ appartient à $\mathscr{S}$.

\item Donner un exemple d'entier $n_0$ appartenant à $\mathscr{S}$.
\end{enumerate}

\item Caractérisation des éléments de $\mathscr{S}$.
\begin{enumerate}
\item Soit $n$ un entier relatif appartenant à $\mathscr{S}$.
 
Démontrer que $n-n_0\equiv0~(85)$.

\item En déduire qu'un entier relatif $n$ appartient à $\mathscr{S}$
  si, et seulement si, $n$ peut s'écrire sous 1a forme $n=43+85k$ où
  $k$ est un entier relatif.
\end{enumerate}

\item Application

Zoé sait qu'elle a entre 300 et 400 jetons. Si elle fait des tas de 17
jetons, il lui en reste 9. 

Si elle fait des tas de 5 jetons, il lui en reste 3.

Combien a-t-elle de jetons ?
\end{enumerate}
\end{exercice}

\columnbreak

%%%%%%%%%%%%%%%%%%%%%%%%%%%%%%%%%%%%%%%%%%%%%%%%%%%%%%%%%%%%%%%%%%%%%%%%%%%%%%%%%%%%%

\begin{exercice}[D'après Bac (Nouvelle-Calédonie - 2007)]
\begin{enumerate}
\item Dans cette question $x$ et $y$ désignent des entiers relatifs.
\begin{enumerate}
\item Montrer que l'équation $(E)~:\enskip 65x - 40y = 1$\enskip n'a pas de solution.\medskip

\item Montrer que l'équation $(E')~:\enskip 17x - 40y = 1$\\ admet au moins une  solution.\medskip

\item Déterminer à l'aide de l'algorithme d'Euclide un couple d'entiers relatifs solution de l'équation $\left(E'\right)$.
	
\item Résoudre l'équation $\left(E'\right)$.
	
En déduire qu'il existe un unique naturel $x_{0}<40$ tel que : $17x_{0}\equiv 1~(40)$.\medskip
	\end{enumerate}
	
\item Pour tout naturel $a$, démontrer que si\enskip $a^{17} \equiv b~(55)$\enskip et si\enskip $a^{40} \equiv 1~(55)$,\enskip alors\enskip $b^{33} \equiv a~(55)$.
\end{enumerate}
\end{exercice}

%%%%%%%%%%%%%%%%%%%%%%%%%%%%%%%%%%%%%%%%%%%%%%%%%%%%%%%%%%%%%%%%%%%%%%%%%%%%%%%%%%%%%%%

\begin{exercice}[D'après Bac (Antilles-Guyane - 2015)]

% {\color{H1}\textbf{Partie A}}
\partie{}

Pour deux entiers naturels non nuls $a$ et $b$, on note $r(a,~b)$ le reste dans la division euclidienne de $a$ par $b$.

On considère l'algorithme suivant :

\begin{center}
\begin{algorithme}
\BlocVariables
\DeclareVar{$c$}{\normalfont entier naturel}{}
\DeclareVar{$a$, $b$}{\normalfont entiers naturels non nuls}{}
\BlocEntrees
\Saisir{$a$}
\Saisir{$b$}
\BlocTraitements
\DonnerValeur{$c$}{$r(a,~b)$}
\TantQue{$c\neq0$}{%
\DonnerValeur{$a$}{$b$}
\DonnerValeur{$b$}{$c$}
\DonnerValeur{$c$}{$r(a,~b)$}
}
\BlocAffichage
\AfficherVar{$b$}
\end{algorithme}
\end{center}

\begin{enumerate}
\item Faire fonctionner cet algorithme avec $a = 26$ et $b = 9$ en indiquant les valeurs de $a$, $b$ et $c$ à chaque étape.
\item Cet algorithme donne en sortie le PGCD des entiers naturels non nuls $a$ et $b$.

Le modifier pour qu'il indique si deux entiers naturels non nuls $a$ et $b$ sont premiers entre eux ou non.
\end{enumerate}

%{\color{H1}\textbf{Partie B}}
\partie{}

À chaque lettre de l'alphabet on associe grâce au tableau ci-dessous un nombre entier compris entre 0 et~25.

\begin{center}
\renewcommand*\tabularxcolumn[1]{>{\centering\arraybackslash}m{#1}}
{\footnotesize\begin{ttableau}{\linewidth}{13}\hline
\rowcolor{FondTableaux}A& B& C& D& E& F& G& H& I& J& K& L& M\\ \hline
0& 1& 2& 3& 4& 5& 6& 7& 8& 9& 10& 11& 12\\ \hline \hline
\rowcolor{FondTableaux}N& O& P& Q& R& S& T& U& V& W& X& Y& Z\\ \hline
13& 14& 15& 16& 17& 18& 19& 20& 21& 22& 23& 24& 25\\ \hline
\end{ttableau}}
\end{center}

On définit un procédé de codage de la façon suivante :

\textbf{Étape 1 :} on choisit deux entiers naturels $p$ et $q$ compris entre $0$ et $25$.

\textbf{Étape 2 :} à la lettre que l'on veut coder, on associe l'entier $x$ correspondant dans le tableau ci-dessus.

\textbf{Étape 3 :} on calcule l'entier $x'$ défini par les relations 
$$x' \equiv  px + q~(26)\quad \text{et}\quad  0\Inf x'\Inf 25.$$
\textbf{Étape 4 :} à l'entier $x'$, on associe la lettre correspondante dans le tableau.


\begin{enumerate}
\item Dans cette question, on choisit $p = 9$ et $q = 2$.
  \begin{enumerate}
  \item Démontrer que la lettre V est codée par la lettre J.
  \item Citer le théorème qui permet d'affirmer l'existence de deux
    entiers relatifs $u$ et $v$ tels que :
		
    $9u + 26v = 1$. Donner sans justifier un couple $(u\ ;\ v)$ qui
    convient.
  \item Démontrer que $x' \equiv 9x + 2~(26)$ équivaut à :
		
    $x \equiv 3x' + 20~(26)$.
  \item Décoder la lettre R.
  \end{enumerate}
\item Dans cette question, on choisit $q = 2$ et $p$ est inconnu. On
  sait que J est codé par D.
	
  Déterminer la valeur de $p$ (on admettra que $p$ est unique).
\item Dans cette question, on choisit $p = 13$ et $q = 2$. 

  Coder les lettres B et D. 

  Que peut-on dire de ce codage ?
\end{enumerate}
\end{exercice}

%%%%%%%%%%%%%%%%%%%%%%%%%%%%%%%%%%%%%%%%%%%%%%%%%%%%%%%%%%%%%%%%%%%%%%%%%%%%%%%%%%%%%%%%%%%%%

\end{colonne*exercice}

