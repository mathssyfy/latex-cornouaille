\begin{acquis}
\begin{itemize}
\item Calculer un PGCD grâce à l'algorithme d'Euclide.

\item Connaître l'énoncé de l'identité de Bézout.

\item Connaître le théorème et le corollaire de Bézout.

\end{itemize}
\begin{itemize}
\item Connaître le théorème et le corollaire de Gauss.
\item Trouver toutes les solutions dans $\Z^2$ de l'équation :

$ax+by=c$.
\end{itemize}
\end{acquis}

\QCMautoevaluation{Pour chaque question, plusieurs réponses sont proposées. Déterminer la réponse exacte en la justifiant.}

\begin{QCM}
\begin{GroupeQCM}
\begin{exercice}
On a \enskip $\pgcd(\nombre{25176}~,~\nombre{42722})=2$. Combien de divisions, à l'aide de l'algorithme d'Euclide, sont-elles nécessaires jusqu'à obtenir un reste nul ?
\begin{ChoixQCM}{4}
\item 12
\item 11
\item 10
\item 9
\end{ChoixQCM}
\end{exercice}
\begin{corrige}
\reponseQCM{a}
\end{corrige}

\begin{exercice}
$a$ et $b$ sont deux entiers naturels tels que : \enskip $\pgcd(a,b)=7$.

Dans l'algorithme d'Euclide, les quotients successifs sont 3, 1, 1, 2 (comprenant la dernière division de reste nul). Alors : 
\begin{ChoixQCM}{4}
\item $(a\ ;\ b)=(35,63)$
\item $(a\ ;\ b)=(35,126)$
\item $(a\ ;\ b)=(25,126)$
\item $(a\ ;\ b)=(14,35)$
\end{ChoixQCM}
\end{exercice}
\begin{corrige}
\reponseQCM{b}
\end{corrige}

\begin{exercice}
Le nombre de couples d'entiers naturels vérifiant\enskip $\pgcd(a;b)=42$\enskip et\enskip $a+2b=336$\enskip est :
\begin{ChoixQCM}{4}
\item 4
\item 3
\item 2
\item 1
\end{ChoixQCM}
\end{exercice}
\begin{corrige}
\reponseQCM{c}
\end{corrige}
\end{GroupeQCM}
\end{QCM}

\begin{QCM}
\begin{GroupeQCM}
\begin{exercice}
  Pour tout entier $n$, on pose \enskip $a=3n-5$\enskip et\enskip
  $b=2n-7$. Alors :
\begin{ChoixQCM}{2}
\item $a$ et $b$ sont premiers entre eux.
\item $\pgcd(a,b)=11$.
\item Tout diviseur commun à $a$ et $b$ divise 11.
\item $a$ et $b$ ne sont pas premiers entre eux.
\end{ChoixQCM}
\end{exercice}
\begin{corrige}
\reponseQCM{c}
\end{corrige}

\begin{exercice}
Soit $n$ un entier naturel. 

Quelle fraction est irréductible pour tout $n$ ?
\begin{ChoixQCM}{4}
\item $\dfrac{3n}{2n+1}$
\item $\dfrac{n+8}{2n+5}$
\item $\dfrac{3n^2}{2n^2+n}$
\item $\dfrac{n}{(2n+1)(3n+1)}$
\end{ChoixQCM}
\end{exercice}
\begin{corrige}
\reponseQCM{d}
\end{corrige}

\begin{exercice}
L'équation diophantienne \enskip $5x-8y=1$ admet comme solutions des couples d'entiers relatifs qui sont :
\end{exercice}
\begin{ChoixQCM}{2}
\item toujours premiers entre eux.
\item parfois premiers entre eux.
\item jamais premiers entre eux.
\item On ne peut pas déterminer si les entiers sont\\ premiers entre eux ou non.
\end{ChoixQCM}
\begin{corrige}
\reponseQCM{a}
\end{corrige}
\end{GroupeQCM}
\end{QCM}

\begin{QCM}
\begin{GroupeQCM}
\begin{exercice}
Un nombre est divisible par 15 et par 24, alors ce nombre est
divisible par :
\begin{ChoixQCM}{4}
\item 360
\item 120
\item 90
\item 72
\end{ChoixQCM}
\end{exercice}
\begin{corrige}
\reponseQCM{b}
\end{corrige}
\begin{exercice}
Soit $k$ un entier relatif. L'équation $5(x-2)=7k$\enskip d'inconnue $x$ a pour solution :
\begin{ChoixQCM}{4}
\item $x\equiv2~(5)$
\item $x\equiv5~(7)$
\item $x\equiv2~(7)$
\item $x\equiv0~(7)$
\end{ChoixQCM}
\end{exercice}
\begin{corrige}
\reponseQCM{c}
\end{corrige}
\begin{exercice}
Soit $a$, $b$, $c$ trois entiers relatifs et $n$ un entier naturel ($n\Sup2$).

La proposition : \ofg{Si \enskip $ac\equiv bc~(n)$\enskip alors
  \enskip $a\equiv b~(n)$} :
\begin{ChoixQCM}{1}
\item est toujours vraie.
\item est vraie si $c$ et $n$ sont premiers entre eux.
\item est vraie si $a$ et $b$ sont premiers entre eux.
\item n'est jamais vraie.
\end{ChoixQCM}
\end{exercice}
\begin{corrige}
\reponseQCM{b}
\end{corrige}
\end{GroupeQCM}
\end{QCM}

\begin{QCM}
\begin{GroupeQCM}

\begin{exercice}
Soit l'équation diophantienne (E) :\enskip $27x+25y=1$. 
\begin{ChoixQCM}{2}
\item $(63\ ;\ -68)$ est solution de (E).
\item (E) admet un couple d'entiers naturels comme\\ solution.
\item $(25\ ;\ 37)$ est solution de (E).
\item (E) n'admet pas de solution.
\end{ChoixQCM}
\end{exercice}
\begin{corrige}
\reponseQCM{a}
\end{corrige}
\begin{exercice}
L'équation diophantienne :\enskip $17x-13y=2$\enskip admet :
\begin{ChoixQCM}{2}
\item aucune solution.
\item comme solutions : \enskip $\left\{\begin{aligned}
&x=-6+13k\\
&y=-8+17k\end{aligned}\right.\enskip,\enskip k\in\Z$.
\item comme solutions : \enskip $\left\{\begin{aligned}
&x=7+26k\\
&y=9+34k\end{aligned}\right.\enskip,\enskip k\in\Z$.\vspace{5pt}
\item comme solutions : \enskip $\left\{\begin{aligned}
&x=7+13k\\
&y=9-17k\end{aligned}\right.\enskip,\enskip k\in\Z$.
\end{ChoixQCM}
\end{exercice}
\begin{corrige}
\reponseQCM{b}
\end{corrige}
\end{GroupeQCM}
\end{QCM}\vspace{0.5cm}

\textbf{Dans les exercices suivants, plusieurs réponses sont proposées. Déterminer celles qui sont exactes. Justifier.}

\begin{QCM}
\begin{GroupeQCM}
\begin{exercice}
Relever les affirmations vraies.
\begin{ChoixQCM}{1}
\item Si le $\pgcd(a,b)=1$,\enskip alors $\pgcd(a+b\ ;\ b)=1$.
\item Il existe deux naturels $a$ et $b$ tels que la somme vaut 150
  et $\pgcd(a,b)=8$.
\item Deux nombres impairs consécutifs sont premiers entre eux.
\item L'équation $51x+39y=2016$ admet des solutions entières.
\end{ChoixQCM}
\end{exercice}
\begin{corrige}
\reponseQCM{a}, \reponseQCM{c} et \reponseQCM{d}
\end{corrige}

\begin{exercice}
  Dans un chiffrement affine, la fonction de codage est définie par la
  fonction $f(x)=17x+22$ (voir tableau de codage de l'activité
  \ref{Ar2Acti3} p.~\pageref{Ar2Acti3}).
\begin{ChoixQCM}{1}
\item Le codage de \ofg{HUIT} est \ofg{LYCA}.
\item Le message \ofg{PZWC} veut dire \ofg{VRAI}.
\item La seule solution dans $\Z^2$ de l'équation\enskip
  $17x-26y=1$\enskip est\enskip $(23\ ;\ 15)$.
\item La fonction de décodage est :\enskip $f^{-1}(y)=23y+14$.
\end{ChoixQCM}
\end{exercice}
\begin{corrige}
\reponseQCM{b} et \reponseQCM{d}
\end{corrige}
\end{GroupeQCM}
\end{QCM}

%\begin{QCM}
%\begin{GroupeQCM}
%\begin{exercice}
%\begin{ChoixQCM}{}
%\end{ChoixQCM}
%\end{exercice}
%\begin{corrige}
%\reponseQCM{}
%\end{corrige}
%\end{GroupeQCM}
%\end{QCM}


