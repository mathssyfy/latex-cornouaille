\begin{colonne*exercice}

%%%%%%%%%%%%%%%%%%%%%%%%%%%%%%%%%%%%%%%%%%%%%%%%%%%%%%%%%%%%%%%%%%%%%%%%%%%%%%%%%%%%%%%
\serie{Activités mentales}
%%%%%%%%%%%%%%%%%%%%%%%%%%%%%%%%%%%%%%%%%%%%%%%%%%%%%%%%%%%%%%%%%%%%%%%%%%%%%%%%%%%%%%%

\begin{exercice*}
Déterminer de tête et à l’aide des règles de divisibilité,
les PGCD des entiers suivants :

\vspace{-2mm}
\begin{multicols}{2}
\begin{enumerate}
\item 12 et 42.
\item 45 et 105.
\item 92 et 69.
\item 72 et 108.
\end{enumerate}
\end{multicols}
\end{exercice*}
\begin{corrige}
\begin{colenumerate}{2}
\item $\pgcd(12\ ;\ 42)=6$
\item $\pgcd(45\ ;\ 105)=15$
\item $\pgcd(92\ ;\ 69)=23$
\item $\pgcd(72\ ;\ 108)=36$
\end{colenumerate}
\end{corrige}

\begin{exercice*}
Sur un vélodrome, deux cyclistes partent en même temps d’un point M et roulent à vitesse
constante.

Le coureur A boucle le tour en 35 secondes ; le coureur~B en 42
secondes.

Au bout de combien de temps le coureur A aura-t-il un tour d’avance sur le coureur B ?
\end{exercice*}
\begin{corrige}
  $\pgcd(35,42)=7$\enskip et \enskip $35=7\times5$,\enskip
  $42=7\times6$.

  Lorsque le coureur A aura fait 6 tours, le coureur B aura fait 5
  tours, soit un temps de \enskip $35\times 6=210$ s.
\end{corrige}

\begin{exercice*}
\begin{enumerate}
\item On veut découper un rectangle de 24 cm sur 40 cm en carrés dont le côté est le plus grand possible, sans perte.

Combien doit mesurer le côté du carré ?

\item On dispose d’un grand nombre de rectangles du type précédent que
  l’on veut assembler bord à bord pour former le carré le plus petit possible.

Combien doit mesurer le côté du carré ?
\end{enumerate}
\end{exercice*}
\begin{corrige}
  \begin{enumerate}
  \item$\pgcd(24\ ;\ 40)=8$. Le côté du carré doit diviser 24 et 40, donc
    le plus grand côté possible est de 8 cm.

  \item $40=8\times5$\enskip et \enskip $24=8\times3$

  Pour former le plus petit carré, il faut mettre 3 fois la longueur
  et 5 fois la largeur du rectangle, soit 120 cm.
\end{enumerate}
\end{corrige}

\begin{exercice*}
  Utiliser l’algorithme d’Euclide pour trouver le PGCD des nombres
  suivants :

\vspace{-2mm}
\begin{multicols}{2}
\begin{enumerate}
\item 78 et 108.
\item 144 et 840.
\item 202 et 138.
% \item 441 et 777.
\end{enumerate}
\end{multicols}
\end{exercice*}
\begin{corrige}
  \begin{enumerate}
  \item $\pgcd(78;108)=6$ car

    $\begin{aligned}
      108&=78\times1+30\\
      78&=30\times2+18\\
      30&=18\times1+12\\
      18&=12\times1+6\\
      12&=6\times2\end{aligned}$\medskip
    
  \item $\pgcd(144;840)=24$ car
    
    $\begin{aligned}
      840&=144\times5+120\\
      144&=120\times1+24\\
      120&=24\times5\end{aligned}$\medskip
    
    
    
  \item $\pgcd(202;138)=2$ car

    $\begin{aligned}
      202&=138\times1+64\\
      138&=64\times2+10\\
      64&=10\times6+4\\
      10&=4\times2+2\\
      4&=2\times2\end{aligned}$
    
  % \item Il faut l'enlever de l'exercice car présent aussi à l'exercice
  %   13.

  %   $\pgcd(441;777)=21$ car
    
  %   $\begin{aligned}
  %     777&=441\times1+336\\
  %     441&=336\times1+105\\
  %     336&=105\times3+21\\
  %     105&=21\times5\end{aligned}$
  \end{enumerate}
\end{corrige}

\begin{exercice*}
  Montrer que deux entiers naturels consécutifs non nuls sont premiers
  entre eux.
\end{exercice*}
\begin{corrige}
  $(-1)n+1(n+1)=1$. Donc d'après le théorème de Bézout, $n$ et $(n+1)$
  sont premiers entre eux.
\end{corrige}

\begin{exercice*}
En utilisant le théorème de Gauss, déterminer les couples d’entiers
relatifs $(x\ ;\ y)$ qui vérifient les équations suivantes :

\vspace{-2mm}
\begin{multicols}{2}
\begin{enumerate}
\item $5(x+3)=4y$
\item $41x+9y=0$
\end{enumerate}
\end{multicols}
\end{exercice*}
\begin{corrige}
  \begin{enumerate}
  \item 4 divise $5(x+3)$. Or $\pgcd(4,5)=1$, donc d'après le théorème
    de Gauss, 4 divise $(x+3)$.

On  a donc $x+3=4k$. 

En remplaçant dans l'équation, on obtient $y=5k$.

Les couples solutions sont :
$\left\{\begin{aligned}
&x\hskip-.5mm=-\hskip-.5mm3+4k\\
&y\hskip-.5mm=\hskip-.5mm5k\end{aligned}\right.\hskip-.5mm, k\in\Z$

\item $41x=9(-y)$\enskip (1)

  9 divise $41x$. Or $\pgcd(9,41)=1$, donc d'après le théorème de
  Gauss, 9 divise $x$.

On  a donc $x=9k$. 

En remplaçant dans l'équation, on obtient $y=-41k$.

Les couples solutions sont: 
$\left\{\begin{aligned}
&x=9k\\
&y=-41k\end{aligned}\right.,\enskip k\in\Z$
\end{enumerate}
\end{corrige}

\begin{exercice*}
  Trouver un couple d’entiers relatifs $(x ; y)$ qui vérifie
  l’équation : $7x+5y=1$.
\end{exercice*}
\begin{corrige}
  $(-2\ ;\ 3)$ est solution.
\end{corrige}

\begin{exercice*}
  Existe-il des couples d’entiers $(x\ ;\ y)$ solutions de chacune des
  équations suivantes ?
\begin{enumerate}
\item $37x + 25y = 1$
\item $51x + 39y = 1$
\item $51x + 39y = 2016$
\end{enumerate}
\end{exercice*}
\begin{corrige}
  \begin{enumerate}
  \item Oui car $\pgcd(37\ ;\ 25)=1$, donc d'après le théorème de
    Bézout, il existe au moins un couple solution.
  \item Non car $\pgcd(51\ ;\ 39)=3$ et comme 1 n'est pas multiple de
    3, d'après le corollaire de Bézout, il n'y a pas de solution.
  \item Oui car $\pgcd(51\ ;\ 39)=3$ et comme 2016 est divisible par
    3, d'après le corollaire du Bézout, il existe des solutions
    entières.
\end{enumerate}
\end{corrige}
%%%%%%%%%%%%%%%%%%%%%%%%%%%%%%%%%%%%%%%%%%%%%%%%%%%%%%%%%%%%%%%%%%%%%%%%%%%%%%%%%%%%%%%
\serie{PGCD}
%%%%%%%%%%%%%%%%%%%%%%%%%%%%%%%%%%%%%%%%%%%%%%%%%%%%%%%%%%%%%%%%%%%%%%%%%%%%%%%%%%%%%%%

\begin{exercice}
Dresser la liste des diviseurs positifs de 72 et de 60. En déduire leur PGCD.
\end{exercice}

\begin{exercice}
Si, en un point donné du ciel, un astre A apparaît tous les 28 jours et un astre B tous les 77 jours, avec quelle périodicité les verra-t-on simultanément en ce point ?
\end{exercice}

\begin{exercice}
Déterminer tous les entiers naturels $n$ inférieurs à 200 tels que : $\pgcd(n;324)=12$.
\end{exercice}

\begin{exercice}[\algo]
  $a$ et $b$ sont deux entiers naturels non nuls tels que
  $a>b$.\vspace{-10pt}

\begin{enumerate}
\item Démontrer que : $\pgcd(a;b)=\pgcd(a-b ; b)$.
\item \label{Exo12Question2}Calculer les PGCD des entiers suivants par cette méthode, répétée autant de fois que nécessaire :
\begin{colenumerate}{2}
\item 308 et 165.
\item 1 008 et 308.
\item 735 et 210.
\end{colenumerate}
\item\begin{enumerate} 
\item Recopier et compléter l’algorithme correspondant à cette méthode.

{\centering
\begin{algorithme}
\BlocVariables
\DeclareVar{$a$, $b$, $c$}{\emph{entiers naturels}}{}
\BlocEntrees
\Saisir{$a$, $b$}
\BlocTraitements
\TantQue{$a\neq b$}{%
\DonnerValeur{$\dots$}{$|a-b|$}
\DonnerValeur{$\dots$}{$b$}
\DonnerValeur{$\dots$}{$c$}
}
\BlocAffichage
\AfficherVar{$\dots$}
\end{algorithme}
\par}
\item Expliquer la condition de la ligne 6.
\item Rentrer cet algorithme dans la calculatrice puis
la tester à l’aide des valeurs de la question
\RefItem{Exo12Question2}.
\end{enumerate}
\end{enumerate}
\end{exercice}

%%%%%%%%%%%%%%%%%%%%%%%%%%%%%%%%%%%%%%%%%%%%%%%%%%%%%%%%%%%%%%%%%%%%%%%%%%
\serie{Algorithme d'Euclide}
%%%%%%%%%%%%%%%%%%%%%%%%%%%%%%%%%%%%%%%%%%%%%%%%%%%%%%%%%%%%%%%%%%%%%%%%%

\begin{exercice*}[\ExerciceRefMethode{methode-alg_euclide}]
  \label{exo-alg_euclide}
  Utiliser l’algorithme d’Euclide pour trouver le PGCD des nombres
  suivants :
\begin{colenumerate}{2}
\item 441 et 777. 
\item 2004 et 9185.
\end{colenumerate}
\end{exercice*}
\begin{corrige}
  \begin{enumerate}
\item $\pgcd(441\ ;\ 777)=21$ car

$\begin{aligned}
777&=441\times1+336\\
441&=336\times1+105\\
336&=105\times3+21\\
105&=21\times5\end{aligned}$\medskip

\item $\pgcd(9185\ ;\ 2004)=167$ car

$\begin{aligned}
9185&=2004\times4+1169\\
2004&=1169\times1+835\\
1169&=835\times1+334\\
835&=334\times2+167\\
334&=167\times2\end{aligned}$
\end{enumerate}
\end{corrige}

\begin{exercice}
  Utiliser l’algorithme d’Euclide pour trouver le PGCD des nombres
  suivants :
\begin{colenumerate}{2}
\item \nombre{2012} et \nombre{7545}. 
\item \nombre{1386} et 546.
\end{colenumerate}
\end{exercice}

\begin{exercice}
  Utiliser l’algorithme d’Euclide pour trouver le PGCD des nombres
  suivants :
\begin{colenumerate}{2}
\item \nombre{4935} et 517. 
\item \nombre{1064} et 700.
\end{colenumerate}
\end{exercice}

\begin{exercice}
Les entiers suivants sont-ils premiers entre eux ?
\begin{colenumerate}{2}
\item \nombre{4847} et \nombre{5633}. 
\item \nombre{5617} et 813.
\end{colenumerate}
\end{exercice}

\begin{exercice}
  Si on divise \nombre{4294} et \nombre{3521} par un même entier
  positif, on obtient respectivement 10 et 11 comme reste.

  Quel est cet entier ?
\end{exercice}

\begin{exercice}
  En divisant \nombre{1809} et \nombre{2527} par un même entier
  naturel, les restes sont respectivement 9 et 7. 

  Quel est le plus grand nombre que l’on peut obtenir comme diviseur ?
\end{exercice}

\begin{exercice}
On note $n$ un naturel non nul, $a=3n+1$ et $b=5n-1$.
\begin{enumerate}
\item Montrer que le $\pgcd(a,b)$ est un diviseur de 8.
\item Pour quelles valeurs de $n$, $\pgcd(a,b)$ est-il égal à 8 ?
\end{enumerate}
\end{exercice}

\begin{exercice}
$n$ est un entier relatif quelconque. On pose :\vspace{-10pt}
$$A=n-1\enskip \text{et}\enskip B=n^2-3n+6.\vspace{-5pt}$$
\begin{enumerate}
\item\begin{enumerate}
\item Démontrer que le PGCD de $A$ et de $B$ est égal au PGCD de $A$ et de 4.
\item Déterminer, selon les valeurs de l’entier $n$, le PGCD de $A$ et de $B$.
\end{enumerate}
\item Pour quelles valeurs de l’entier relatif $n$, $n\neq1$,\vspace{5pt}

$\dfrac{n^2-3n+6}{n-1}$ est-il un entier relatif ?
\end{enumerate}
\end{exercice}


%%%%%%%%%%%%%%%%%%%%%%%%%%%%%%%%%%%%%%%%%%%%%%%%%%%%%%%%%%%%%%%%%%%%%
\serie{Théorème de Bézout}
%%%%%%%%%%%%%%%%%%%%%%%%%%%%%%%%%%%%%%%%%%%%%%%%%%%%%%%%%%%%%%%%%%%%%

\begin{exercice}
Soit l’égalité de Bézout : \ofg{Soit $a$ et $b$ deux entiers non nuls et $D$ leur PGCD. Il existe un couple d’entiers relatifs telle que $au+bv=D$}.
\begin{enumerate}
\item Démontrer le théorème de Bézout \ofg{$a$ et $b$ sont premiers entre eux si, et seulement si, il existe un couple d’entiers relatifs $(u;v)$ tel que $au+bv=1$}.
\item En déduire que si $\pgcd(a;b) = D$, alors $a = Da'$ et $b = Db'$ avec $\pgcd(a'; b') = 1$.
\end{enumerate}
\end{exercice}

\columnbreak

\begin{exercice*}[\ExerciceRefMethode{methode-bezout}]
\label{exo-bezout}
Démontrer que, pour tout relatif $k$,

$(7k+3)$ et $(2k+1)$ sont premiers entre eux.
\end{exercice*}
\begin{corrige}
  $(-2)(7k+3)+7(2k+1)=-14k-6+14k+7=1$.

  D'après le théorème de Bézout, $(7k+3)$ et $(2k+1)$ sont premiers
  entre eux pour tout entier relatif $k$.
\end{corrige}

\begin{exercice}
$n$ est un entier naturel , $a=7n+4$ et $b=5n+3$.

Montrer, pour tout $n$, que $a$ et $b$ sont premiers entre eux.
\end{exercice}

\begin{exercice}
Démontrer que pour tout relatif $n$, les entiers $(14n+3)$ et $(5n+1)$ sont premiers entre eux. En déduire $\pgcd(87;31)$.
\end{exercice}

\begin{exercice}
Prouver que la fraction $\dfrac{n}{2n+1}$ est irréductible pour tout entier naturel $n$.
\end{exercice}

\begin{exercice}
Prouver que la fraction $\dfrac{2n+1}{n(n+1)}$ est irréductible pour tout entier naturel $n$.
\end{exercice}

\begin{exercice}
La fraction $\dfrac{n^3+n}{2n+1}$ est-elle irréductible pour tout entier naturel $n$ ?
\end{exercice}

\begin{exercice*}[\ExerciceRefMethode{methode-sol_particuliere}]
\label{exo-sol_particuliere}
Montrer que 17 et 40 sont premiers entre eux puis déterminer un couple d’entiers relatifs $(x;y)$ tel que : $17x-40y= 1$.
\end{exercice*}
\begin{corrige}
  On utilise l'algorithme d'Euclide :

$\begin{aligned}
40&=17\times2+6\qquad (1)\\
17&=6\times2+5\qquad (2)\\
6&=5\times1+1\qquad (3)\end{aligned}$\medskip

40 et 17 sont donc premiers entre eux.\medskip

On remonte l'algorithme d'Euclide :

de (3), on obtient\enskip $5=6-1$\medskip

On remplace dans (2) :

$17=6\times3-1$\enskip donc\enskip $6\times3=17+1$\medskip

On multiplie (1) par 3

$\begin{aligned}
40\times3&=17\times6+6\times3\\
				 &=17\times6+17+1\\
				 &=17\times7+1\end{aligned}$\medskip
				
On a alors \enskip $17\times(-7)-40\times(-3)=1$.
\end{corrige}

\begin{exercice}
Montrer que 23 et 26 sont premiers entre eux puis déterminer un couple d'entiers relatifs $(x;y)$ tel que : $23x + 26y = 1$.
\end{exercice}

\begin{exercice}
L’équation $6x + 3y = 1$ admet-elle des solutions entières ? Et l'équation $7x + 5y = 1$ ?
\end{exercice}

\begin{exercice}
  Montrer que 221 et 331 sont premiers entre eux puis déterminer un
  couple d'entiers relatifs $(x;y)$ tel que : $221x - 331y = 1$.
\end{exercice}

\begin{exercice}[Vrai ou faux ?]
  S’il existe deux entiers relatifs $u$ et $v$ tel que $au + bv = 3$,
  alors le PGCD de $a$ et de $b$ est égal à 3. Justifier.
\end{exercice}

\begin{exercice}
  Résoudre dans $\N^2$ les systèmes suivants. 

  On donnera la réponse sous forme d'un tableau.
\begin{colenumerate}{2}
\item $\left\{\begin{aligned}
					&xy=1512\\
					&\pgcd(x,y)=6\end{aligned}\right.$
\item $\left\{\begin{aligned}
					&xy=300\\
					&\pgcd(x,y)=5\end{aligned}\right.$
\end{colenumerate}
\end{exercice}

\columnbreak

%%%%%%%%%%%%%%%%%%%%%%%%%%%%%%%%%%%%%%%%%%%%%%%%%%%%%%%%%%%%%%%%%%%%%%%%%%%%%%%
\serie{Théorème de Gauss}
%%%%%%%%%%%%%%%%%%%%%%%%%%%%%%%%%%%%%%%%%%%%%%%%%%%%%%%%%%%%%%%%%%%%%%%%%%%%%%%

\begin{exercice}
En utilisant le théorème de Gauss, déterminer les couples d’entiers relatifs $(a;b)$ qui vérifient :
$$33a - 45b = 0.\vspace{-20pt}$$
\end{exercice}

\begin{exercice}
\begin{enumerate}
\item En utilisant le théorème de Gauss, déterminer les couples d’entiers relatifs $(x;y)$ qui vérifient :
$$7(x-3)=5(y-2).$$
\item De la question précédente, déterminer les entiers\linebreak naturels $x$ tels que : $7x\equiv1~(5)$.
\end{enumerate}
\end{exercice}

\begin{exercice}
En utilisant le théorème de de Gauss, démontrer le corollaire du théorème de Gauss :\ofg{Si $b$ et $c$ divisent $a$ et si $b$ et $c$ sont premiers entre eux, alors $bc$ divise $a$}.
\end{exercice}

\begin{exercice}
  Montrer que si $n\equiv0~(8)$ et $n\equiv0~(9)$, alors
  $n\equiv0~(72)$.
\end{exercice}

%%%%%%%%%%%%%%%%%%%%%%%%%%%%%%%%%%%%%%%%%%%%%%%%%%%%%%%%%%%%%%%%%%%%%%%%%%%%%%%
\serie{PPCM}
%%%%%%%%%%%%%%%%%%%%%%%%%%%%%%%%%%%%%%%%%%%%%%%%%%%%%%%%%%%%%%%%%%%%%%%%%%%%%%%
\begin{exercice}
  Soit deux entiers relatifs $a$ et $b$. 
  
  On appelle $\ppcm(a; b)$ le plus petit multiple strictement positif
  de $a$ et de $b$.
\begin{enumerate}
\item Calculer $\ppcm(18; 12)$ et $\ppcm(24\ ;\ 40)$.\vspace{5pt}
\item Calculer $\dfrac{7}{6}+\dfrac{11}{15}$. Que représente
  $\ppcm(6\ ;\ 15)$ ?
\end{enumerate}
\end{exercice}

\begin{exercice}\label{ExoPPCM}
On appelle $D = \pgcd(a\ ;\ b)$ et\\ $M = \ppcm(a\ ;\ b)$.
\begin{enumerate}
\item Montrer que si $a = Da'$ et $b = Db'$, alors $M = Da'b'$.
\item En déduire que : $D\times M = ab$.
\end{enumerate}
\end{exercice}

\begin{exercice}
Soit $a$ et $b$ deux naturels tels que $a < b$.

En utilisant les propriétés de l’exercice \RefExercice{ExoPPCM},
déterminer $a$ et $b$ tels que : $\pgcd(a\ ;\ b)=6$ et
$\ppcm(a\ ;\ b)=102$.
\end{exercice}

%%%%%%%%%%%%%%%%%%%%%%%%%%%%%%%%%%%%%%%%%%%%%%%%%%%%%%
\serie{Équation du type \textit{ax} + \textit{by} = \textit{c}}
%%%%%%%%%%%%%%%%%%%%%%%%%%%%%%%%%%%%%%%%%%%%%%%%%%%%%%

\begin{exercice}
  Soit l'identité de Bézout : \ofg{Soit $a$ et $b$ deux entiers non
    nuls et $D$ leur PGCD. Il existe un couple d'entiers relatifs
    tel que $au + bv = D$}.

  Démontrer le corollaire du théorème de Bézout : \ofg{L'équation
    $ax + by = c$ admet des solutions entières si, et seulement si,
    $c$ est un multiple du $\pgcd(a\ ;\ b)$}.
\end{exercice}

\columnbreak

\begin{exercice*}[\ExerciceRefMethode{methode-eq_diophantienne}]
\label{exo-eq_diophantienne}
Soit l'équation (E) : $4x-3y = 2$.
\begin{enumerate}
\item Déterminer une solution particulière entière à (E).
\item Déterminer l'ensemble des solutions entières.
\end{enumerate}
\end{exercice*}
\begin{corrige}
  \begin{itemize}
  \item $(2\ ;\ 2)$ est une solution particulière.
  \item Soit $(x\ ;\ y)$ la solution générale, on écrit :
    $$\left\{\begin{aligned}
        4x-3y&=2\\
        4(2)-3(2)&=2\end{aligned}\right.$$
    
En soustrayant termes à termes, on obtient : 

$4(x-2)=3(y-2)$\enskip (1)

3 divise $4(x-2)$. Or $\pgcd(4;3)=1$, donc d'après le théorème de
Gauss, 3 divise $(x-2)$. On a alors :

$x-2=3k$.

En remplaçant dans (1), on obtient :\enskip $y-2=4k$.

L'ensemble des couples solutions est de la forme :
$$\left\{\begin{aligned}
x&=2+3k\\
y&=2+4k\end{aligned}\right.\enskip k\in\Z$$
\end{itemize}
\end{corrige}

\begin{exercice}
Soit l'équation (F) : $3x - 4y = 6$.
\begin{enumerate}
\item Déterminer une solution particulière entière à (F).
\item Déterminer l'ensemble des solutions entières.
\end{enumerate}
\end{exercice}

\begin{exercice}
Soit l’équation (G) : $5x + 8y = 2$.
\begin{enumerate}
\item Déterminer une solution particulière entière à (G).
\item Déterminer l’ensemble des solutions entières.
\end{enumerate}
\end{exercice}

\begin{exercice}
Soit l'équation $13x - 23y = 1$.
\begin{enumerate}
\item Déterminer une solution particulière entière, à l'aide de
  l'algorithme d’Euclide, à cette équation.
\item Déterminer l'ensemble des solutions entières.
\end{enumerate}
\end{exercice}

\begin{exercice}
\begin{enumerate}
\item Déterminer l’ensemble des couples $(x; y)$ des nombres entiers
  relatifs, solutions de l'équation :

{\centering $\text{(E)}~:~8x - 5y = 3$.\par}

\item Soit $m$ un nombre entier relatif tel qu'il existe un couple
  $(p; q)$ de nombres entiers vérifiant :

$m=8p + 1$ et $m = 5q + 4$.

Montrer que le couple $(p, q)$ est solution de l’équation (E).

\item Déterminer le plus petit de ces nombres entiers $m$\linebreak supérieur à
  \nombre{2000}.
\end{enumerate}
\end{exercice}

\begin{exercice}
\begin{enumerate}
\item On considère l'équation (E) à résoudre dans $\Z$ :

{\centering $7x - 5y = 1.$\par}

\vspace{-\baselineskip}
\begin{enumerate}
\item Vérifier que le couple (3 ; 4) est solution de (E).
\item Montrer que le couple d'entiers $(x ; y)$ est solution de (E) si, et seulement si, $7(x - 3) = 5(y - 4)$.
\item Montrer que les solutions entières de l’équation (E) sont
  exactement les couples $(x ; y)$ d’entiers\linebreak relatifs tels que :
$$\left\{\begin{aligned}
&x = 5k + 3\\
&y = 7k + 4\end{aligned}\right.\enskip \text{où}\enskip k\in\Z.$$
\end{enumerate}
\item Une boîte contient 25 jetons, des rouges, des verts et des
  blancs. Sur les 25 jetons, il y a $x$ jetons rouges et\linebreak $y$
  jetons verts. 

  Sachant que $7x-5y = 1$, quels peuvent être les nombres de jetons
  rouges, verts et blancs ?
\end{enumerate}
\end{exercice}

%%%%%%%%%%%%%%%%%%%%%%%%%%%%%%%%%%%%%%%%%%%%%%%%%%%%%%%%%%%%%%%%%%%%%%%%%%%%%%%%

\end{colonne*exercice}