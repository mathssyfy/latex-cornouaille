\begin{prerequis}[Connaissances nécessaires à ce chapitre]
\begin{itemize}
\item Déterminer le PGCD de deux entiers
\item Savoir reconnaître deux nombres premiers entre eux
\item Connaître les congruences
\item Trouver un couple d'entiers vérifiant l'équation :\\ $ax+by=c$
\end{itemize}
\end{prerequis}
 
\vspace{0.5em}
% 
\begin{autoeval}
\begin{multicols}{2}
	\begin{exercice}
          \og{}PGCD\fg{} signifie :
          \og{}plus~grand~commun\\diviseur\fg{}.
	\begin{enumerate}
	\item Calculer le \MotDefinition[PGCD]{}[PGCD]{}PGCD de 26 et 65, puis simplifier  $\dfrac{26}{65}$.\vspace{5pt}
	\item Calculer le PGCD de 72 et 54, puis simplifier  $\dfrac{72}{54}$.\vspace{5pt}
	\item Calculer le PGCD de 255 et 35, puis simplifier $\dfrac{255}{35}$.
	%\item Calculer le PGCD de 450 et 756, puis simplifier $\dfrac{26}{65}$
	\end{enumerate}
	\end{exercice}
        \begin{corrige}
          \begin{enumerate}
          \item $\pgcd(26,65)=13$\enskip d'où \enskip
            $\dfrac{26}{65}=\dfrac{2}{5}$.\medskip
          \item $\pgcd(72,54)=18$\enskip d'où \enskip
            $\dfrac{72}{54}=\dfrac{4}{3}$.\medskip
          \item $\pgcd(255,35)=5$\enskip d'où \enskip
            $\dfrac{255}{35}=\dfrac{51}{7}$.
          \end{enumerate}
        \end{corrige}
        \smallskip{}
	
	\begin{exercice}
          Soit $n$ un entier naturel. Le PGCD de $n$ et de 72 vaut
          8. Parmi les valeurs suivantes quelles sont celles que peut
          prendre $n$ ?

          \vspace{-10pt} 
          \begin{center}
            9, 16, 18, 24, 32, 40, 48
          \end{center}
          \vspace{-10pt}
	\end{exercice}
	\begin{corrige}
	16, 32 et 40.
	\end{corrige}
	
        \smallskip{}

	\begin{exercice}
          Deux nombres sont premiers entre eux si leur PGCD est égal à
          1.
 \begin{enumerate}
 \item 9 et 16 sont-ils premiers entre eux ? 
 \item 35 et 91 sont-ils premiers entre eux ?
 \item 31 et 67 sont-ils premiers entre eux ?
 \end{enumerate}
	\end{exercice}
	\begin{corrige}
	\begin{enumerate}
	\item	9 et 16 sont premiers entre eux.
	\item 35 et 91 ne sont pas premiers entre eux. 

          $35=5\times7$ et $91=7\times13$.

	\item 31 et 67 sont premiers entre eux. En outre, ils sont tous les deux
          premiers.
        \end{enumerate}
      \end{corrige}
	
        \smallskip{}

	\begin{exercice}
          Les phrases suivantes sont-elles vraies ou fausses ?
          Justifier.
	\begin{enumerate}
	\item Un nombre entier $a$ est divisible par 6 et 9, donc $a$
          est divisible par 54.
	\item Un nombre entier $a$ est divisible par 8 et 9, donc $a$
          est divisible par 72.
	\item Un nombre entier $a$ est divisible par 4 et 18, donc
	
	$a\equiv36~(36)$.
      \item Un nombre entier $a$ est divisible par 10 et 15, donc
	
	$a\equiv0~(150)$.
	\end{enumerate}
	\end{exercice}
	\begin{corrige}
          \begin{enumerate}
          \item Fausse. 36 est divisible par 6 et 9 mais pas par 54.
          \item Vraie car 8 et 9 sont premiers entre eux.
          \item Vraie car 36 est le plus petit multiple commun à 4 et
            18.
          \item Fausse car 30 est un multiple de 10 et 15 et
            \linebreak$30\not\equiv0~(150)$
          \end{enumerate}
	\end{corrige}
	
        \smallskip{}

	\begin{exercice}
          Soit $x$ et $y$ des nombres entiers. Les phrases suivantes sont-elles
          vraies ou fausses ? Justifier.
	\begin{enumerate}
	\item Si \enskip $x\equiv0~(81)$,\enskip alors \enskip
          $x\equiv0~(9)$.
	\item L'équation \enskip $x^2+2y^2\equiv3~(4)$\enskip admet
          des solutions.
	\end{enumerate}
	\end{exercice}
	\begin{corrige}
          \begin{enumerate}
          \item Vraie car si $x$ est multiple de 81, il est aussi
            multiple de 9.
          \item Vraie car $(1\ ;\ 1)$ est un couple solution de
            l'équation.
          \end{enumerate}
	\end{corrige}
	
        \smallskip{}

	\begin{exercice}
          Trouver un couple d'entiers $(x\ ;\ y)$ vérifiant les
          équations suivantes :
	\begin{enumerate}\vspace{-10pt}
	\begin{multicols}{2}
	\item $7x-10y=1$
	\item $4x+5y=1$
	\item $3x+4y=3$
	\item $7x-12y=3$
	\end{multicols}
	\end{enumerate}
	\end{exercice}
	\begin{corrige}
          \begin{colenumerate}{2}
          \item $(3\ ;\ 2)$
          \item $(-1\ ;\ 1)$
          \item $(1\ ;\ 0)$
          \item $(-3\ ;\ -2)$
          \end{colenumerate}
	\end{corrige}
	
	\begin{exercice}
	%\begin{enumerate}
	%\item 
          Pierre a des jetons d'une valeur de 3 points et Jean a des
          jetons d'une valeur de 7 points. 

          Pierre doit donner 34 points à Jean.
	
          Comment Pierre et Jean peuvent-il procéder ? Donner une
          solution.
	%\item Céline possède des jetons de 3 et 7 points. Elle a en tout 34 points. Combien de jetons de chaque valeur possède-t-elle ? Trouver toutes les solutions
	%\end{enumerate}
	\end{exercice}
	\begin{corrige}
          Pierre peut donner 16 jetons de 3 points et Jean lui rend 2
          jetons de 7 points.
	\end{corrige}
	
	%\begin{exercice}
	%\end{exercice}
	%\begin{corrige}
	%\end{corrige}
\end{multicols}
\end{autoeval}
