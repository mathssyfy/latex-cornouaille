\documentclass[10pt]{article}
\usepackage[T1]{fontenc}
\usepackage[utf8]{inputenc}
\usepackage{fourier}
\usepackage[scaled=0.875]{helvet} 
\renewcommand{\ttdefault}{lmtt}
\usepackage{amsmath,amssymb,makeidx}
\usepackage[normalem]{ulem}
\usepackage{fancybox}
\usepackage{tabularx}
\usepackage{colortbl}
\usepackage{ulem}
\usepackage{lscape}
\usepackage{dcolumn}
\usepackage{textcomp}
\newcommand{\euro}{\eurologo{}}
\usepackage{pstricks,pst-plot,pst-tree,pstricks-add}
\usepackage[left=3.5cm, right=3.5cm, top=3cm, bottom=3cm]{geometry}
\newcommand{\vect}[1]{\overrightarrow{\,\mathstrut#1\,}}
% Tapuscrit Denis Vergès
\newcommand{\R}{\mathbb{R}}
\newcommand{\N}{\mathbb{N}}
\newcommand{\D}{\mathbb{D}}
\newcommand{\Z}{\mathbb{Z}}
\newcommand{\Q}{\mathbb{Q}}
\newcommand{\C}{\mathbb{C}}
\renewcommand{\theenumi}{\textbf{\arabic{enumi}}}
\renewcommand{\labelenumi}{\textbf{\theenumi.}}
\renewcommand{\theenumii}{\textbf{\alph{enumii}}}
\renewcommand{\labelenumii}{\textbf{\theenumii.}}
\def\Oij{$\left(\text{O}~;~\vect{\imath},~\vect{\jmath}\right)$}
\def\Oijk{$\left(\text{O}~;~\vect{\imath},~\vect{\jmath},~\vect{k}\right)$}
\def\Ouv{$\left(\text{O}~;~\vect{u},~\vect{v}\right)$}
\usepackage{fancyhdr}
\usepackage{hyperref}
\hypersetup{%
pdfauthor = {APMEP},
pdfsubject = {TS Amérique du Nord},
pdftitle = {29 mai 2018},
allbordercolors = white,
pdfstartview=FitH}
\usepackage[np]{numprint}
\usepackage[frenchb]{babel}
\begin{document}
\setlength\parindent{0mm}
\rhead{\textbf{A. P{}. M. E. P{}.}}
\lhead{\small Baccalauréat S}
\lfoot{\small{Amérique du Nord}}
\rfoot{\small 29 mai 2018}
\pagestyle{fancy}
\thispagestyle{empty}
\begin{center}\textbf{Durée : 4 heures}

\vspace{0,5cm}

{\Large \textbf{\decofourleft~Baccalauréat S Amérique du Nord 29 mai 2018~\decofourright}}
\end{center}

\vspace{0,5cm}

\textbf{Exercice 1 \hfill  6 points}

\textbf{Commun à  tous les candidats}

\medskip

On étudie certaines caractéristiques d'un supermarché d'une petite ville.

\bigskip

\textbf{Partie A - Démonstration préliminaire}

\medskip

Soit $X$ une variable aléatoire qui suit la loi exponentielle de paramètre $0,2$.

On rappelle que l'espérance de la variable aléatoire $X$, notée $E(X)$, est égale à:

\[\displaystyle\lim_{x \to + \infty}\displaystyle\int_{0}^{x}  0,2t\text{e}^{-0,2t}\:\text{d}t.\]

Le but de cette partie est de démontrer que $E(X) = 5$.

\medskip

\begin{enumerate}
\item On note $g$ la fonction définie sur l'intervalle $[0~;~+\infty[$ par $g(t) = 0,2t\text{e}^{-0,2t}$.

On définit la fonction $G$ sur l'intervalle $[0~;~+\infty[$ par $G(t) = (- t - 5)\text{e}^{-0,2t}$.

Vérifier que $G$ est une primitive de $g$ sur l'intervalle $[0~;~+\infty[$.
\item  En déduire que la valeur exacte de $E(X)$ est 5.

\emph{Indication : on pourra utiliser, sans le démontrer, le résultat suivant }:

\[\displaystyle\lim_{x \to + \infty} x \text{e}^{- 0,2x} = 0.\]
\end{enumerate}

\bigskip

\textbf{Partie B - Étude de la durée de présence d'un client dans le supermarché}

\medskip

Une étude commandée par le gérant du supermarché permet de modéliser la durée, exprimée en
minutes, passée dans le supermarché par un client choisi au hasard par une variable aléatoire $T$.

Cette variable $T$ suit une loi normale d'espérance $40$ minutes et d'écart type un réel positif noté $\sigma$.

Grâce à cette étude, on estime que $P(T < 10) = 0,067$.

\medskip

\begin{enumerate}
\item Déterminer une valeur arrondie du réel $\sigma$ à la seconde près.
\item Dans cette question, on prend $\sigma = 20$~minutes. Quelle est alors la proportion de clients qui
passent plus d'une heure dans le supermarché ?
\end{enumerate}

\bigskip

\textbf{Partie C - Durée d'attente pour le paiement}

\medskip

Ce supermarché laisse le choix au client d'utiliser seul des bornes automatiques de paiement ou
bien de passer par une caisse gérée par un opérateur.

\medskip

\begin{enumerate}
\item La durée d'attente à une borne automatique, exprimée en minutes, est modélisée par une
variable aléatoire qui suit la loi exponentielle de paramètre $0,2$~min$^{-1}$.
	\begin{enumerate}
		\item Donner la durée moyenne d'attente d'un client à une borne automatique de paiement.
		\item Calculer la probabilité, arrondie à $10^{-3}$, que la durée d'attente d'un client à une borne automatique de paiement soit supérieure à $10$ minutes.
	\end{enumerate}
\item L'étude commandée par le gérant conduit à la modélisation suivante:
	
\setlength\parindent{9mm}
\begin{itemize}
\item[$\bullet~~$] parmi les clients ayant choisi de passer à une borne automatique, 86\,\% attendent moins de $10$ minutes ;
\item[$\bullet~~$] parmi les clients passant en caisse, 63\,\% attendent moins de $10$ minutes.
\end{itemize}
\setlength\parindent{0mm}

\medskip

On choisit un client du magasin au hasard et on définit les évènements suivants :

$B$ : \og le client paye à une borne automatique \fg{} ;

$\overline{B}$ : \og le client paye à une caisse avec opérateur \fg{} ;

$S$ : \og la durée d'attente du client lors du paiement est inférieure à $10$ minutes \fg.

Une attente supérieure à dix minutes à une caisse avec opérateur ou à une borne automatique
engendre chez le client une perception négative du magasin. Le gérant souhaite que
plus de 75\,\% des clients attendent moins de $10$ minutes.

Quelle est la proportion minimale de clients qui doivent choisir une borne automatique de
paiement pour que cet objectif soit atteint ?
 \end{enumerate}
 
\bigskip

\textbf{Partie D - Bons d'achat}

\medskip

Lors du paiement, des cartes à gratter, gagnantes ou perdantes, sont distribuées aux clients. Le
nombre de cartes distribuées dépend du montant des achats. Chaque client a droit à une carte à
gratter par tranche de $10$~\euro{} d'achats.

Par exemple, si le montant des achats est 58,64~\euro, alors le client obtient $5$ cartes ; si le montant est
124,31~\euro, le client obtient $12$~cartes.

Les cartes gagnantes représentent $0,5$\,\% de l'ensemble du stock de cartes. De plus, ce stock est
suffisamment grand pour assimiler la distribution d'une carte à un tirage avec remise.

\medskip

\begin{enumerate}
\item Un client effectue des achats pour un montant de 158,02~\euro.

Quelle est la probabilité, arrondie à $10^{-2}$, qu'il obtienne au moins une carte gagnante ?
\item  À partir de quel montant d'achats, arrondi à 10~\euro, la probabilité d'obtenir au moins une carte
gagnante est-elle supérieure à 50\,\% ?
\end{enumerate}

\vspace{0,5cm}

\textbf{Exercice 2 \hfill  4 points}

\textbf{Commun à  tous les candidats}

\medskip

\parbox{0.6\linewidth}{Lors d'une expérience en laboratoire, on lance un projectile dans un milieu fluide. L'objectif est de déterminer pour quel angle de tir
$\theta$ par rapport à l'horizontale la hauteur du projectile ne dépasse
pas $1,6$ mètre.

Comme le projectile ne se déplace pas dans l'air mais dans un
fluide, le modèle parabolique usuel n'est pas adopté.

On modélise ici le projectile par un point qui se déplace, dans un
plan vertical, sur la courbe représentative de la fonction $f$ définie
sur l'intervalle [0~;~1[ par:

\[f(x) = bx + 2\ln (1- x)\]

où $b$ est un paramètre réel supérieur ou égal à $2$, $x$ est l'abscisse
du projectile, $f(x)$ son ordonnée, toutes les deux exprimées en mètres.}
\hfill
\parbox{0.38\linewidth}{
\psset{unit=4cm,comma=true}
\begin{pspicture*}(-0.15,-0.15)(1.1,1.7)
\psgrid[gridlabels=0pt,subgriddiv=10,gridwidth=0.3pt,subgridwidth=0.15pt](0,0)(1.1,1.7)
\psaxes[linewidth=1pt,Dx=0.5,Dy=0.5,labelFontSize=\scriptstyle](0,0)(0,0)(1.1,1.7)
\psaxes[linewidth=1.5pt]{->}(0,0)(1,1)
\psplot[plotpoints=3000,linewidth=1.25pt,linecolor=blue]{0}{0.932}{5.69 x mul 1 x sub ln 2 mul add}
\psline[linestyle=dotted,linewidth=1pt](0.4,1.5)
\psarc(0,0){0.15}{0}{72}
\end{pspicture*}}

\medskip

\begin{enumerate}
\item La fonction $f$ est dérivable sur l'intervalle [0~;~1[. On note $f'$ sa fonction dérivée.

On admet que la fonction $f$ possède un maximum sur l'intervalle [0~;~1[ et que, pour tout réel
$x$ de l'intervalle [0~;~1[ :

\[f'(x) = \dfrac{- bx + b - 2}{1 - x}.\]

Montrer que le maximum de la fonction $f$ est égal à $b - 2 + 2\ln \left(\dfrac{2}{b}\right)$.
\item  Déterminer pour quelles valeurs du paramètre $b$ la hauteur maximale du projectile ne dépasse
pas $1,6$~mètre.
\item  Dans cette question, on choisit $b = 5,69$.

L'angle de tir $\theta$ correspond à l'angle entre l'axe des abscisses et la tangente à la courbe de la
fonction $f$ au point d'abscisse $0$ comme indiqué sur le schéma donné ci-dessus.

Déterminer une valeur approchée au dixième de degré près de l'angle $\theta$.
\end{enumerate}
 
\vspace{0,5cm}

\textbf{Exercice 3 \hfill  5 points}

\textbf{Commun à  tous les candidats}

\medskip

On se place dans l'espace muni d'un repère orthonormé dont l'origine est le point A.

On considère les points B$(10~;~-8~;~2)$, C$(-1~;~-8~;~5)$ et D(14~;~4~;~8).

\medskip

\begin{enumerate}
\item 
	\begin{enumerate}
		\item Déterminer un système d'équations paramétriques de chacune des droites (AB) et (CD).
		\item Vérifier que les droites (AB) et (CD) ne sont pas coplanaires.
	\end{enumerate}
\item On considère le point I de la droite (AB) d'abscisse 5 et le point J de la droite (CD) d'abscisse
4.
	\begin{enumerate}
		\item Déterminer les coordonnées des points I et J et en déduire la distance IJ.
		\item Démontrer que la droite (IJ) est perpendiculaire aux droites (AB) et (CD).
		
La droite (IJ) est appelée perpendiculaire commune aux droites (AB) et (CD).
 	\end{enumerate}
\item Cette question a pour but de vérifier que la distance IJ est la distance minimale entre les
droites (AB) et (CD).
	
Sur le schéma ci -dessous on a représenté les droites (AB) et (CD), les points I et J, et la droite
$\Delta$ parallèle à la droite (CD) passant par I.
	
On considère un point $M$ de la droite (AB) distinct du point I.
	
On considère un point $M'$ de la droite (CD) distinct du point J.
	
\begin{center}
\psset{unit=1cm}
\begin{pspicture}(12.3,7.8)
\pspolygon[fillstyle=solid,fillcolor=lightgray](0.7,1.2)(5.8,0.5)(11.5,3.8)(6.4,4.5)
%\psgrid
\psline(0,2)(12.3,2)%(AB)
\psline(1.5,0)(12.3,6)%\Delta
\psline(0,2.8)(10.8,8.8)%(CD)
\psline(5.1,2)(4.3,5.22)%IJ
\psline(7.3,3.22)(6.5,6.45)%PM'
\psline(7.3,3.22)(6.5,2)(6.5,6.45)%PMM'	
\uput[u](4.3,5.22){J}\uput[dr](5.1,2){I}
\uput[d](6.5,2){$M$}\uput[dr](7.3,3.22){$P$}
\uput[u](6.5,6.45){$M'$}
\uput[d](10.5,2){(AB)}\uput[ul](5.5,5.7){(CD)}
\uput[u](10,4.7){$\Delta$}
\psline(4.38,5)(4.55,5.07)(4.48,5.28)
\psline(5.08,2.258)(4.82,2.1)(4.88,1.9)
\end{pspicture}	
\end{center}

\medskip

	\begin{enumerate}
		\item Justifier que la parallèle à la droite (IJ) passant par le point $M'$ coupe la droite $\Delta$ en un point que l'on notera $P$.
		\item Démontrer que le triangle $MPM'$ est rectangle en $P$.
		\item Justifier que $MM' > IJ$ et conclure.
	\end{enumerate}
\end{enumerate}

\vspace{0,5cm}

\textbf{Exercice 4 \hfill  5 points}

\textbf{Candidats n'ayant pas suivi l'enseignement de spécialité}

\medskip

\textbf{Les deux graphiques donnés en annexe seront à compléter et à rendre avec la copie}

\medskip

Un scooter radio commandé se déplace en ligne droite à la vitesse constante de 1 m.s$^{-1}$. Il est poursuivi
par un chien qui se déplace à la même vitesse. 

On représente la situation vue de dessus dans un repère orthonormé du plan d'unité 1 mètre. L'origine de ce repère est la position initiale du chien. Le scooter est représenté par un point appartenant à la droite d'équation $x = 5$. Il se déplace sur cette droite dans le sens des ordonnées croissantes.

\smallskip

Dans la suite de l'exercice, on étudie deux modélisations différentes de la trajectoire du chien.

\bigskip

\textbf{Partie A - Modélisation à l'aide d'une suite}

\medskip

La situation est représentée par le graphique \no 1 donné en annexe.

À l'instant initial, le scooter est représenté par le point $S_0$. Le chien qui le poursuit est représenté
par le point $M_0$. On considère qu'à chaque seconde, le chien s'oriente instantanément en direction
du scooter et se déplace en ligne droite sur une distance de 1 mètre.

Ainsi, à l'instant initial, le chien s'oriente en direction du point $S_0$, et une seconde plus tard il se
trouve un mètre plus loin au point $M_1$. À cet instant, le scooter est au point $S_1$. Le chien s'oriente
en direction de $S_1$ et se déplace en ligne droite en parcourant 1 mètre, et ainsi de suite.

On modélise alors les trajectoires du chien et du scooter par deux suites de points notées $\left(M_n\right)$ et $\left(S_n\right)$.

Au bout de $n$ secondes, les coordonnées du point $S_n$ sont $(5~;~n)$. On note $\left(x_n~;~y_n\right)$ les coordonnées du point $M_n$.

\medskip

\begin{enumerate}
\item Construire sur le graphique \no 1 donné en annexe les points $M_2$ et $M_3$.
\item  On note $d_n$ la distance entre le chien et le scooter $n$ secondes après le début de la poursuite.

On a donc $d_n = M_nS_n$.

Calculer $d_0$ et $d_1$.
\item  Justifier que le point $M_2$ a pour coordonnées $\left(1 + \dfrac{4}{\sqrt{17}}~;~\dfrac{1}{\sqrt{17}}\right)$.
\item  On admet que, pour tout entier naturel $n$ :

\[\left\{\begin{array}{l c l}
x_{n+1}& = &x_n + \dfrac{5 - x_n}{d_n}\\
y_{n+1}&=&y_n + \dfrac{n - y_n}{d_n}
\end{array}\right.\]

	\begin{enumerate}
		\item Le tableau ci-dessous, obtenu à l'aide d'un tableur, donne les coordonnées des points $M_n$
et $S_n$ ainsi que la distance $d_n$ en fonction de $n$. Quelles formules doit-on écrire dans les
cellules C5 et F5 et recopier vers le bas pour remplir les colonnes C et F ?
		
\begin{center}
		\begin{tabularx}{\linewidth}{|c|*{6}{>{\centering \arraybackslash}X|}}\cline{2-7}
\multicolumn{1}{c|}{~}&A &B &C &D &E &F\\ \hline
1 &$n$& \multicolumn{2}{|c|}{$M_n$} & \multicolumn{2}{|c|}{$S_n$} &$d_n$\\ \hline
2 &&$x_n$& $y_n$& 5 &n&\\ \hline
3 &0& 0& 0& 5 &0& 5\\ \hline
4 &1 &1 &0 &5 &1 &\np{4,12310563}\\ \hline
5 &2 &\np{1,9701425} &\np{0,24253563} &5 &2 &\np{3,50267291}\\ \hline
6 &3 &\np{2,83515547} &\np{0,74428512} &5 &3 &\np{3,12646789}\\ \hline
7 &4 &\np{3,52758047} &\np{1,46577498} &5 &4 &\np{2,93092404}\\ \hline
\ldots&\ldots&\ldots&\ldots&\ldots&\ldots&\ldots\\ \hline
28 &24 &\np{4,99979751} &\np{21,2268342} &5 &24 &\np{2,7731658}\\ \hline
29 &25 &\np{4,99987053} &\np{22,2268342} &5 &25 &\np{2,7731658}\\ \hline
\end{tabularx}	
\end{center}

\medskip

		\item On admet que la suite $\left(d_n\right)$ est strictement décroissante.

Justifier que cette suite est convergente et conjecturer sa limite à l'aide du tableau.
 	\end{enumerate}
\end{enumerate}
 
\bigskip
 
\textbf{Partie B - Modélisation à l'aide d'une fonction}
 
 \medskip
 
On modélise maintenant la trajectoire du chien à l'aide de la courbe $\mathcal{F}$ de la fonction $f$ définie
pour tout réel $x$ de l'intervalle [0~;~5[ par:
 
\[f(x) = -2,5\ln (1 - 0, 2x) - 0,5x + 0,05x^2.\]
 
 \medskip
 
Cela signifie que le chien se déplace sur la courbe $\mathcal{F}$ de la fonction $f$.
 
 \medskip
 
\begin{enumerate}
\item Lorsque le chien se trouve au point $M$ de coordonnées $(x~;~f(x))$ de la courbe $\mathcal{F}$, où $x$  appartient à l'intervalle [0~;~5[, le scooter se trouve au point $S$, d'ordonnée notée $y_S$. Ainsi le point $S$
a pour coordonnées $\left(5~;~y_S\right)$. La tangente à la courbe $\mathcal{F}$ au point $M$ passe par le point $S$. Cela traduit le fait que le chien s'oriente toujours en direction du scooter. On note $d(x)$ la distance $MS$ entre le chien et le scooter lorsque $M$ a pour abscisse $x$.
	\begin{enumerate}
		\item Sur le graphique \no 2 donné en annexe, construire, sans calcul, le point $S$ donnant la position du scooter lorsque le chien se trouve au point d'abscisse 3 de la courbe $\mathcal{F}$ et lire les
coordonnées du point $S$.
		\item On note $f'$ la fonction dérivée de la fonction $f$ sur l'intervalle [0~;~5[ et on admet que, pour tout réel $x$ de l'intervalle [0~;~5[ :

\[f'(x) = \dfrac{x(1  - 0,1x)}{5 - x}.\]

Déterminer par le calcul une valeur approchée au centième de l'ordonnée du point $S$ lorsque
le chien se trouve au point d'abscisse 3 de la courbe $\mathcal{F}$.
	\end{enumerate}
\item  On admet que $d(x) = 0,1x^2 - x + 5$ pour tout réel $x$ de l'intervalle [0~;~5[.

Justifier qu'au cours du temps la distance $MS$ se rapproche d'une valeur limite que l'on déterminera.
\end{enumerate}

\vspace{0,5cm}

\textbf{Exercice 4 \hfill  5 points}

\textbf{Candidats ayant suivi l'enseignement de spécialité}

\medskip

Dans une région, on s'intéresse à la cohabitation de deux espèces animales : les campagnols et les
renards, les renards étant les prédateurs des campagnols. 

Au 1\up{er} juillet 2012, on estime qu'il y a dans cette région approximativement deux millions de campagnols et cent-vingt renards.

On note $u_n$ le nombre de campagnols et $v_n$ le nombre de renards au 1\up{er} juillet de l'année $2012+ n$.

\bigskip

\textbf{Partie A - Un modèle simple}

\medskip

On modélise l'évolution des populations par les relations suivantes :

\[\left\{\begin{array}{l c r}
u_{n+1}& =& 1,1u_n - \np{2000}v_n\\
v_{n+1} &=& 2 \times 10^{-5}u_n + 0,6v_n
\end{array}\right. \quad \text{pour tout entier }\:n \geqslant 0,\: \text{avec } \:u_0 = \np{2000000}\:  \text{ et} \: v_0 = 120.\]

\medskip

\begin{enumerate}
\item 
	\begin{enumerate}
		\item On considère la matrice colonne $U_n = \begin{pmatrix}u_n\\v_n\end{pmatrix}$ pour tout entier $n \geqslant 0$.
		
Déterminer la matrice $A$ telle que $U_{n+1} = A \times U_n$ pour tout entier $n$ et donner la matrice $U_0$.
		\item Calculer le nombre de campagnols et de renards estimés grâce à ce modèle au 1\up{er} juillet
2018.
	\end{enumerate}
\item Soit les matrices $P = \begin{pmatrix}\np{20000}&\np{5000}\\1&1\end{pmatrix}$, \:$D = \begin{pmatrix}1&0\\0&0,7\end{pmatrix}$ et $P^{-1} = \dfrac{1}{\np{15000}}\begin{pmatrix}1& \np{-5000}\\- 1&\np{20000}\end{pmatrix}$.
	
On admet que $P^{- 1}$ est la matrice inverse de la matrice $P$ et que $A = P \times D \times P^{- 1}$.
	\begin{enumerate}
		\item Montrer que pour tout entier naturel $n$,\: $U_n = P \times D^n \times P^{- 1} \times U_0$.
		\item Donner sans justification l'expression de la matrice $D^n$ en fonction de $n$.
		\item On admet que, pour tout entier naturel $n$ :
	
\renewcommand\arraystretch{1.8}	
\[\left\{\begin{array}{l c r}
u_n &=& \dfrac{2,8 \times 10^7 + 2 \times 10^6 \times 0,7^n}{15}\\

v_n &=&\dfrac{\np{1400} + 400 \times 0,7^n}{15}
		\end{array}\right.\]
\renewcommand\arraystretch{1}	
Décrire l'évolution des deux populations.
	\end{enumerate}
\end{enumerate}

\bigskip

\textbf{Partie B - Un modèle plus conforme à la réalité}

\medskip

Dans la réalité, on observe que si le nombre de renards a suffisamment baissé, alors le nombre de
campagnols augmente à nouveau, ce qui n'est pas le cas avec le modèle précédent. 

On construit donc un autre modèle, plus précis, qui tient compte de ce type d'observations à l'aide des relations suivantes :

\[\left\{\begin{array}{l c r}
u_{n+1} &=& 1,1u_n - 0,001u_n \times v_n\\
v_{n+1} &=& 2 \times 10^{-7} u_n \times v_n + 0,6v_n
\end{array}\right.\quad \text{pour tout entier }\:n \geqslant 0,\: \text{avec }\:u_0 = \np{2000000}\: \text{et }\: v_0 = 120.\]

\medskip

Le tableau ci-dessous présente ce nouveau modèle sur les $25$ premières années en donnant les
effectifs des populations arrondis à l'unité :
\begin{center}
\begin{tabularx}{0.7\linewidth}{|>{\columncolor[gray]{0.7}}c|*{3}{>{\centering \arraybackslash}X|}}\hline
\rowcolor[gray]{0.7}&A &B &C\\ \hline
1& \multicolumn{3}{c|}{Modèle de la \textbf{partie B}}\\ \hline
2& $n$ 	&$u_n$ 			&$v_n$\\ \hline
3&0		& \np{2000000} 	&120\\ \hline
4&1		& \np{1960000} 	&120\\ \hline
5&2		& \np{1920800} 	&119\\ \hline
6&3		& \np{1884228} 	&117\\ \hline
7&4		& \np{1851905} 	&114\\ \hline
8&5		& \np{1825160} 	&111\\ \hline
9&6		& \np{1804988} 	&107\\ \hline
10&7	& \np{1792049} 	&103\\ \hline
11&8	& \np{1786692} 	&99\\ \hline
12&9	& \np{1789005} 	&94\\ \hline
13&10	& \np{1798854} 	&91\\ \hline
14&11	& \np{1815930} 	&87\\ \hline
15&12	& \np{1839780} 	&84\\ \hline
16&13	& \np{1869827} 	&81\\ \hline
17&14	& \np{1905378} 	&79\\ \hline
18&15	& \np{1945622} 	&77\\ \hline
19&16	& \np{1989620} 	&77\\ \hline
20&17	& \np{2036288} 	&76\\ \hline
21&18	& \np{2084374} 	&77\\ \hline
22&19	& \np{2132440} 	&78\\ \hline
23&20	& \np{2178846} 	&80\\ \hline
24&21	& \np{2221746} 	&83\\ \hline
25&22	& \np{2259109} 	&87\\ \hline
26&23	& \np{2288766} 	&91\\ \hline
27&24	& \np{2308508} 	&97\\ \hline
\end{tabularx}
\end{center}

\medskip

\begin{enumerate}
\item Quelles formules faut-il écrire dans les cellules B4 et C4 et recopier vers le bas pour remplir
les colonnes B et C ?
\item  Avec le deuxième modèle, à partir de quelle année observe-t-on le phénomène décrit (baisse
des renards et hausse des campagnols) ?
\end{enumerate}


\bigskip

\textbf{Partie C}

\medskip

Dans cette partie on utilise le modèle de la partie B.

Est - il possible de donner à $u_0$ et $v_0$ des valeurs afin que les deux populations restent stables d'une
année sur l'autre, c'est-à-dire telles que pour tout entier naturel $n$ on ait $u_{n+1} = u_n$ et $v_{n+1} = v_n$ ? (On parle alors d'état stable.)


\newpage

\begin{center}
\textbf{\large Annexe}

\vspace{1cm}

\textbf{À rendre avec la copie}
\vspace{1cm}
\textbf{EXERCICE 4}

\textbf{Candidats n'ayant pas suivi l'enseignement de spécialité}

\bigskip

\textbf{Partie A}, question 1

Graphique \no 1

\bigskip

\psset{unit=1.4cm}
\begin{pspicture}(-0.2,-0.2)(5.5,4.5)
\psgrid[gridlabels=0pt,subgriddiv=1](0,0)(5,4)
\psaxes[linewidth=1pt,labelFontSize=\scriptstyle](0,0)(0,0)(5.5,4.5)
\uput[dr](0,0){$M_0$}\uput[dr](1,0){$M_1$}\uput[ur](5,0){$S_0$}
\uput[ur](5,1){$S_1$}\uput[ur](5,2){$S_2$}\uput[ur](5,3){$S_3$}
\end{pspicture}

\vspace{1cm}

\textbf{Partie B}, question 1

Graphique \no 2

\bigskip

\psset{unit=1.4cm}
\begin{pspicture*}(-0.5,-0.5)(5.5,5.5)
\psgrid[gridlabels=0pt,subgriddiv=1,gridwidth=0.3pt](0,0)(5,6)
\psaxes[linewidth=1pt,labelFontSize=\scriptstyle](0,0)(0,0)(5.5,5.5)
\psaxes[linewidth=1.5pt,labelFontSize=\scriptstyle]{->}(0,0)(1,1)
\psplot[plotpoints=3000,linewidth=1.25pt,linecolor=blue]{0}{4.8}{0.05 x dup mul mul 0.5 x mul sub 1 0.2 x mul sub ln 2.5 mul sub}
\psdots(3,1.23)\uput[ul](3,1.23){$M$}\uput[l](4.5,5){\blue $\mathcal{F}$}
\end{pspicture*}
\end{center}
\end{document}