%!TEX encoding = UTF-8 Unicode
\documentclass{cornouaille}

\begin{document}
\fexo{TS}{Baccalauréat S  Métropole--La Réunion  22 juin 2018}{}

\tableofcontents

\section{Analyse - Exponentielle}
\begin{exercice}[Commun - Analyse expo][6]


\emph{Dans cet exercice, on munit le plan d'un repère orthonormé.}

On a représenté ci-dessous la courbe d'équation:
\[y = \dfrac{1}{2}\left(\text{e}^x + \text{e}^{-x} - 2\right).\]


Cette courbe est appelée une \og chaînette \fg. 

On s'intéresse ici aux \og arcs de chaînette\fg{} délimités par deux points de cette courbe
symétriques par rapport à l'axe des ordonnées.

Un tel arc est représenté sur le graphique ci-dessous en trait plein.

On définit la \og largeur \fg{} et la \og hauteur \fg{} de l'arc de chaînette délimité par les points $M$ et $M'$ comme indiqué sur le graphique.

\begin{center}
\psset{unit=1.25cm}
\begin{pspicture}(-2,-0.8)(2,2)
\psaxes[linewidth=1.25pt,Dx=2,Dy=2]{->}(0,0)(-2,0)(2,2)
\psplot[plotpoints=3000,linewidth=1.25pt]{-1.8}{1.8}{2.71828 x exp 2.71828 x neg exp add 2 sub 0.5 mul}
\psline[linestyle=dashed](1.4,0)(1.4,1.1509)(-1.4,1.1509)(-1.4,0)
\uput[d](1.4,0){$x$} \uput[d](-1.4,0){$- x$}
\psline{<->}(-1.4,-0.5)(1.4,-0.5)
\uput[d](0,-0.5){largeur}
\psline{<->}(-1.6,0)(-1.6,1.1509)
\uput[l](-1.6,0.56){hauteur}
\uput[ur](1.4,1.){$M\left(x~;~\frac{1}{2}\left(\text{e}^x + \text{e}^{- x} - 2\right)\right)$}
\uput[ur](-1.4,1.1){$M'$}
\end{pspicture}
\end{center}

\medskip

Le but de l'exercice est d'étudier les positions possibles sur la courbe du point $M$ d'abscisse $x$ strictement positive afin que la largeur de l'arc de chaînette soit égale à sa hauteur.

\medskip

\begin{enumerate}
\item Justifier que le problème étudié se ramène à la recherche des solutions strictement
positives de l'équation 

\[(E) : \text{e}^x + \text{e}^{- x} - 4x - 2 = 0.\]

\item  On note $f$ la fonction définie sur l'intervalle $[0~;~+\infty[$ par :

\[f(x) = \text{e}^x + \text{e}^{- x} - 4x - 2.\]

	\begin{enumerate}
		\item Vérifier que pour tout $x > 0,\: f(x) = x \left(\dfrac{\text{e}^x}{x}- 4\right) + \text{e}^{- x} - 2$.
		\item Déterminer $\displaystyle\lim_{x \to + \infty} f(x)$.
	\end{enumerate}
\item  
	\begin{enumerate}
		\item On note $f'$ la fonction dérivée de la fonction $f$. Calculer $f'(x)$, où $x$ appartient à l'intervalle $[0~;~ +\infty[$.
		\item Montrer que l'équation $f'(x) = 0$ équivaut à l'équation : $\left(\text{e}^x\right)^2 - 4\text{e}^x - 1 = 0$.
		\item En posant $X = \text{e}^x$, montrer que l'équation $f'(x) = 0$ admet pour unique solution réelle le nombre $\ln \left(2 + \sqrt{5}\right)$.
	\end{enumerate}
\item  On donne ci-dessous le tableau de signes de la fonction dérivée $f'$ de $f$ :
	
\begin{center}
\psset{unit=1cm}
\begin{pspicture}(7,1.5)
\psframe(7,1.5)\psline(0,0.75)(7,0.75)\psline(1,0)(1,1.5)
\uput[u](0.5,0.75){$x$}\uput[u](1.1,0.75){$0$}
\uput[u](4,0.75){$\ln \left(2 + \sqrt{5} \right)$}\uput[u](6.5,0.75){$+ \infty$}
\rput(0.5,0.375){$f'(x)$}\rput(2,0.375){$-$}
\rput(4,0.375){$0$}\rput(5,0.375){$+$}
\end{pspicture}
\end{center}

	\begin{enumerate}
		\item Dresser le tableau de variations de la fonction $f$.
		\item Démontrer que l'équation $f(x) = 0$ admet une unique solution strictement positive que l'on notera $\alpha$.
	\end{enumerate}
\item On considère l'algorithme suivant où les variables $a$, $b$ et $m$ sont des nombres réels :

\begin{center}
\begin{tabularx}{0.5\linewidth}{|X|}\hline
Tant que $b - a > 0,1$ faire:\\
\hspace{1cm}$m \gets \dfrac{a+b}{2}$\\
\hspace{1cm}Si $\text{e}^m + \text{e}^{-m} - 4m - 2 > 0$, alors:\\
\hspace{2cm}$b \gets m$\\
\hspace{1cm}Sinon :\\
\hspace{2cm}$a\gets m$\\
\hspace{1cm}Fin Si\\
Fin Tant que\\ \hline
\end{tabularx}
\end{center}

\parbox{0.53\linewidth}{\begin{enumerate}
\item Avant l'exécution de cet algorithme, les variables $a$ et $b$
contiennent respectivement les valeurs $2$ et $3$.

Que contiennent-elles à la fin de l'exécution de l'algorithme ?

On justifiera la réponse en reproduisant et en complétant le tableau ci-contre avec les différentes valeurs prises par les variables, à chaque étape de l'algorithme.
\item Comment peut-on utiliser les valeurs obtenues en fin d'algorithme à la question
précédente ?
\end{enumerate}}\hfill \parbox{0.43\linewidth}{
 \begin{tabularx}{\linewidth}{|*{4}{>{\centering \arraybackslash}X|}}\hline
$m$ 	&$a$ &$b$ &$b - a$\\ \hline
\cellcolor{lightgray}	&2& 3 &1\\ \hline
2,5		&&&\\ \hline
\ldots	&\ldots&\ldots&\\ \hline
 ~		&&&\\ \hline
\end{tabularx}}
 
\parbox{0.6\linewidth}{\item La \emph{Gateway Arch}, édifiée dans la ville de Saint-Louis aux États-Unis, a l'allure ci-contre.
 
Son profil peut être approché par un arc de chaînette renversé dont la largeur est égale à  la hauteur.}\hfill
\parbox{0.38\linewidth}{\psset{unit=1cm,arrowsize=2pt 3}
\begin{pspicture}(-2,-1)(2,2)
%\psaxes[linewidth=1.25pt,Dx=2,Dy=2]{->}(0,0)(-2,0)(2,2)
\psplot[plotpoints=3000,linewidth=1.25pt]{-1.8}{1.8}{2.71828 x exp 2.71828 x neg exp add 2 sub 0.5 mul neg 1.5 add}
\psline{<->}(-1.8,-0.6)(1.8,-0.6)
\uput[d](0,-0.6){largeur}
\psline{<->}(0,-0.6)(0,1.5)
\uput[r](0,0.45){hauteur}
\end{pspicture}
} 
 
La largeur de cet arc, exprimée en mètre, est égale au double de la solution strictement
positive de l'équation : 
 
\[\left(E'\right) : \text{e}^{\tfrac{t}{39}} + \text{e}^{-\tfrac{t}{39}} - 4\frac{t}{39} - 2 = 0.\]
 
Donner un encadrement de la hauteur de la \emph{Gateway Arch}.

\end{enumerate}
\end{exercice}

\newpage

\section{Probabilité}
\begin{exercice}[Commun - Proba][4]


\bigskip

\emph{Les parties A et B de cet exercice sont indépendantes.}

\medskip

Le virus de la grippe atteint chaque année, en période hivernale, une partie de la population d'une ville.

La vaccination contre la grippe est possible; elle doit être renouvelée chaque année.

\bigskip

\textbf{Partie A}

\medskip

L'efficacité du vaccin contre la grippe peut être diminuée en fonction des caractéristiques
individuelles des personnes vaccinées, ou en raison du vaccin, qui n'est pas toujours
totalement adapté aux souches du virus qui circulent. Il est donc possible de contracter la
grippe tout en étant vacciné.

Une étude menée dans la population de la ville à l'issue de la période hivernale a permis de constater que :

\begin{itemize}
\item[$\bullet~~$]40\,\% de la population est vaccinée ;
\item[$\bullet~~$]8\,\% des personnes vaccinées ont contracté la grippe ;
\item[$\bullet~~$]20\,\% de la population a contracté la grippe.
\end{itemize}

\smallskip

On choisit une personne au hasard dans la population de la ville et on considère les
évènements :

\begin{description}
\item[ ] $V$ : \og la personne est vaccinée contre la grippe \fg{} ;
\item[ ] $G$ : \og la personne a contracté la grippe \fg.
\end{description}

\medskip

\begin{enumerate}
\item 
	\begin{enumerate}
		\item Donner la probabilité de l'évènement $G$.
		\item Reproduire l'arbre pondéré ci-dessous et compléter les pointillés indiqués sur quatre de ses branches.
		
		\begin{center}
\pstree[treemode=R,nodesepA=0pt,nodesepB=3pt]{\TR{}}
{\pstree{\TR{$V$~}\naput{\ldots}}
	{\TR{$G$}\naput{\ldots}
	\TR{$\overline{G}$}\nbput{\ldots}
	}
\pstree{\TR{$\overline{V}$~}\nbput{\ldots}}
	{\TR{$G$}
	\TR{$\overline{G}$}
	}
}	
		
		\end{center}
	\end{enumerate}
\item Déterminer la probabilité que la personne choisie ait contracté la grippe et soit vaccinée.
\item La personne choisie n'est pas vaccinée. Montrer que la probabilité qu'elle ait contracté la grippe est égale à $0,28$.
\end{enumerate}

\bigskip

\textbf{Partie B}

\medskip

\emph{Dans cette partie, les probabilités demandées seront données à $10^{-3}$ près.}

\medskip

Un laboratoire pharmaceutique mène une étude sur la vaccination contre la grippe dans cette
ville.

\medskip

Après la période hivernale, on interroge au hasard $n$ habitants de la ville, en admettant que ce choix se ramène à $n$ tirages successifs indépendants et avec remise. On suppose que la probabilité qu'une personne choisie au hasard dans la ville soit vaccinée contre la grippe est égale à $0,4$.

On note $X$ la variable aléatoire égale au nombre de personnes vaccinées parmi les $n$
interrogées.

\medskip

\begin{enumerate}
\item Quelle est la loi de probabilité suivie par la variable aléatoire $X$ ?
\item Dans cette question, on suppose que $n = 40$.
	\begin{enumerate}
		\item Déterminer la probabilité qu'exactement $15$ des $40$ personnes interrogées soient vaccinées.
		\item Déterminer la probabilité qu'au moins la moitié des personnes interrogées soit vaccinée.
 	\end{enumerate}
\item  On interroge un échantillon de \np{3750} habitants de la ville, c'est-à-dire que l'on suppose ici que $n = \np{3750}$.
	
On note $Z$ la variable aléatoire définie par : $Z = \dfrac{X - \np{1500}}{30}$.
	
On admet que la loi de probabilité de la variable aléatoire $Z$ peut être approchée par la
loi normale centrée réduite.
	
En utilisant cette approximation, déterminer la probabilité qu'il y ait entre \np{1450} et \np{1550} individus vaccinés dans l'échantillon interrogé.
\end{enumerate}
\end{exercice}

\section{Espace}
\begin{exercice}[Commun - Espace][5]


\bigskip

Le but de cet exercice est d'examiner, dans différents cas, si les hauteurs d'un tétraèdre sont concourantes, c'est-à-dire d'étudier l'existence d'un point d'intersection de ses quatre hauteurs.

\emph{On rappelle que dans un tétraèdre} MNPQ, \emph{la hauteur issue de} M \emph{est la droite passant par} M \emph{orthogonale au plan} (NPQ).

\bigskip

\textbf{Partie A Étude de cas particuliers}

\medskip

On considère un cube ABCDEFGH.

\begin{center}
\psset{unit=1cm}
\begin{pspicture}(8,7.3)
\psline(0.5,1)(4.8,0.5)(6.8,1.3)(6.8,6.3)(4.8,5.5)(4.8,0.5)%ABCGFB
\psline(6.8,6.3)(2.5,6.8)(0.5,6)(4.8,5.5)%GHEF
\psline(0.5,1)(0.5,6)%AE
\psline[linestyle=dotted,linewidth=1pt](0.5,1)(2.5,1.8)(2.5,6.8)
\psline[linestyle=dotted,linewidth=1pt](7.2,1.3)(2.5,1.8)
\uput[dl](0.5,1){A} \uput[d](4.8,0.5){B} \uput[r](6.8,1.3){C} \uput[ur](2.5,1.8){D} 
\uput[l](0.5,6){E} \uput[u](4.8,5.5){F} \uput[ur](6.8,6.3){G} \uput[u](2.5,6.8){H} 
\end{pspicture}
\end{center}

\medskip

On admet que les droites (AG), (BH), (CE) et (DF), appelées \og grandes diagonales\fg{} du cube, sont concourantes.

\medskip

\begin{enumerate}
\item On considère le tétraèdre ABCE.
	\begin{enumerate}
		\item Préciser la hauteur issue de E et la hauteur issue de C dans ce tétraèdre.
		\item Les quatre hauteurs du tétraèdre ABCE sont-elles concourantes?
 	\end{enumerate}
\item On considère le tétraèdre ACHF et on travaille dans le repère $\left(\text{A}~;~ \vect{\text{AB}},~ \vect{\text{AD}},~ \vect{\text{AE}}\right)$.
	\begin{enumerate}
		\item Vérifier qu'une équation cartésienne du plan (ACH) est : $x - y + z = 0$.
		\item En déduire que (FD) est la hauteur issue de F du tétraèdre ACHF{}.
		\item Par analogie avec le résultat précédent, préciser les hauteurs du tétraèdre ACHF issues respectivement des sommets A, C et H.
		
Les quatre hauteurs du tétraèdre ACHF sont-elles concourantes ?
	\end{enumerate}
\end{enumerate}

\emph{Dans la suite de cet exercice, un tétraèdre dont les quatre hauteurs sont concourantes sera appelé un tétraèdre orthocentrique.}

\bigskip

\textbf{Partie B Une propriété des tétraèdres orthocentriques}

\medskip

Dans cette partie, on considère un tétraèdre MNPQ dont les hauteurs issues des sommets M et
N sont sécantes en un point K. Les droites (MK) et (NK) sont donc orthogonales aux plans
(NPQ) et (MPQ) respectivement.

\begin{center}
\psset{unit=1cm}
\begin{pspicture}(8,8.2)
\pspolygon(3.9,8)(0.5,0.5)(7.8,0.5)%MNP
\psline[linestyle=dotted,linewidth=2pt](0.5,0.5)(5,2)(3.9,8)%NQM
\psline[linestyle=dotted,linewidth=2pt](5,2)(7.8,0.5)%QP
\psline[linestyle=dotted](3.9,8)(3.9,0.8) \psframe(3.9,0.8)(4.1,1)
\psline[linestyle=dotted](0.5,0.5)(6,3.2) \rput{-60}(6,3.2){\psframe(0.2,0.2)}
\uput[u](3.9,8){M} \uput[dl](0.5,0.5){N} \uput[dr](7.8,0.5){P} 
\uput[ur](5,2){Q} \uput[ul](3.9,2.2){K} 
\end{pspicture}
\end{center}

\begin{enumerate}
\item 
	\begin{enumerate}
		\item Justifier que la droite (PQ) est orthogonale à la droite (MK) ; on admet de même que les droites (PQ) et (NK) sont orthogonales.
		\item Que peut-on déduire de la question précédente relativement à la droite (PQ) et au plan (MNK) ? Justifier la réponse.
 	\end{enumerate}
\item Montrer que les arêtes [MN] et [PQ] sont orthogonales.
	
Ainsi, on obtient la propriété suivante :
	
Si un tétraèdre est orthocentrique, alors ses arêtes opposées sont orthogonales deux à deux.
	
(On dit que deux arêtes d'un tétraèdre sont \og opposées\fg{} lorsqu'elles n'ont pas de sommet commun.)
\end{enumerate}

\bigskip

\textbf{Partie C Application}

\medskip

Dans un repère orthonormé, on considère les points :

\[\text{R}(-3~;~5~;~2) ,\text{S}(1~;~4~;~-2) , \text{T}(4~;~-1~;~5)\quad  \text{et U}(4~;~7~;~3).\]

Le tétraèdre RSTU est-il orthocentrique ? Justifier.
\end{exercice}
\newpage

\section{Complexes}
\begin{exercice}[Obli - complexes][5]

Le plan complexe est muni d'un repère orthonormé direct \Ouv.

On pose $z_0 = 8$ et, pour tout entier naturel $n$ :

\[z_{n+1} = \dfrac{3 - \text{i}\sqrt{3}}{4}z_n.\]

On note $A_n$ le point du plan d'affixe $z_n$.

\medskip

\begin{enumerate}
\item 
	\begin{enumerate}
		\item Vérifier que :
		
\[\dfrac{3 - \text{i}\sqrt{3}}{4} = \dfrac{\sqrt{3}}{2}\text{e}^{- \text{i}\frac{\pi}{6}}.\]
		
		\item En déduire l'écriture de chacun des nombres complexes $z_1$,  $z_2$ et $z_3$ sous forme exponentielle et vérifier que $z_3$ est un imaginaire pur dont on précisera la partie imaginaire.
		\item Représenter graphiquement les points $A_0$ , $A_1$ , $A_2$ et $A_3$ ; on prendra pour unité le centimètre.
 	\end{enumerate}
\item
	\begin{enumerate}
		\item Démontrer par récurrence que, pour tout entier naturel $n$,
		
\[z_n = 8 \times \left(\dfrac{\sqrt{3}}{2}\right)^n \text{e}^{- \text{i}\frac{n\pi}{6}}.\]
		
		\item Pour tout entier naturel $n$, on pose $u_n = \left|z_n\right|$.
		
Déterminer la nature et la limite de la suite $\left(u_n\right)$.
	\end{enumerate}
\item 
	\begin{enumerate}
		\item Démontrer que, pour tout entier naturel $k$,
		
		\[\dfrac{z_{k+1} - z_{k}}{z_{k+1}} = - \dfrac{1}{\sqrt{3}}\text{i}.\]

En déduire que, pour tout entier naturel $k$, on a l'égalité : $A_kA_{k+1} = \dfrac{1}{\sqrt{3}} \text{O}A_{k+1}$.
		\item Pour tout entier naturel $n$, on appelle $\ell_n$ la longueur de la ligne brisée reliant dans cet ordre les points $A_0$,\: $A_1$,\: $A_2$, \ldots , $A_n$.
		
On a ainsi : $\ell_n = A_0A_1 + A_1A_2 + \ldots + A_{n-1}A_n$.
		
Démontrer que la suite $\left(\ell_n\right)$ est convergente et calculer sa limite.
	\end{enumerate}
\end{enumerate}
\end{exercice}


\section{Spécialité}
\begin{exercice}[Spécialité][5]


\textbf{Partie A}

\medskip

On considère l'équation suivante dont les inconnues $x$ et $y$ sont des entiers naturels :

\[x^2 - 8y^2 = 1 . \quad(E)\]

\medskip

\begin{enumerate}
\item Déterminer un couple solution $(x~;~y)$ où $x$ et $y$ sont deux entiers naturels.
\item  On considère la matrice $A = \begin{pmatrix}3&8\\1&3\end{pmatrix}$.

On définit les suites d'entiers naturels $\left(x_n\right)$ et $\left(y_n\right)$ par :

\[x_0 = 1,\: y_0 = 0,\: \text{et pour tout entier naturel }\:n,\: \begin{pmatrix}x_{n+1}\\y_{n+1}\end{pmatrix} = A\begin{pmatrix}x_{n}\\y_{n}\end{pmatrix}.\]
	\begin{enumerate}
		\item Démontrer par récurrence que pour tout entier naturel $n$, le couple 
		$\left(x_n~;~y_n\right)$ est solution de l'équation $(E)$.
		\item En admettant que la suite $\left(x_n\right)$ est à valeurs strictement positives, démontrer que pour tout entier naturel $n$, on a : $x_{n+1} > x_n$.
 	\end{enumerate}
\item  En déduire que l'équation $(E)$ admet une infinité de couples solutions.
\end{enumerate}

\bigskip

\textbf{Partie B}

\medskip

Un entier naturel $n$ est appelé un nombre puissant lorsque, pour tout diviseur premier $p$ de $n$,\: $p^2$ divise $n$.

\medskip

\begin{enumerate}
\item Vérifier qu'il existe deux nombres entiers consécutifs inférieurs à $10$ qui sont puissants.
\end{enumerate}
\medskip

L'objectif de cette partie est de démontrer, à l'aide des résultats de la partie A, qu'il existe une infinité de couples de nombres entiers naturels consécutifs puissants et d'en trouver quelques exemples.

\medskip

\begin{enumerate}
\item  Soient $a$ et $b$ deux entiers naturels.

Montrer que l'entier naturel $n = a^2 b^3$ est un nombre puissant.
\item  Montrer que si $(x~;~y)$ est un couple solution de l'équation $(E)$ définie dans la partie A, alors $x^2 - 1$ et $x^2$ sont des entiers consécutifs puissants.
\item  Conclure quant à l'objectif fixé pour cette partie, en démontrant qu'il existe une infinité de couples de nombres entiers consécutifs puissants.

Déterminer deux nombres entiers consécutifs puissants supérieurs à $2018$.
\end{enumerate}
\end{exercice}
\end{document}