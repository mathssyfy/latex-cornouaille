%!TEX encoding = UTF-8 Unicode
\documentclass{cornouaille}

\begin{document}
\fexo{TS}{Baccalauréat S Pondichéry 4 mai 2018}{}
\tableofcontents

\section{Analyse - Exponentielle}
\begin{exercice}[Commun - Exponentielle][6]


\emph{Les parties } A \emph{et}   B \emph{peuvent être traitées de façon indépendante.}

\bigskip


Dans une usine, un four cuit des céramiques à la température de \np{1000}~\degres C. À la fin de la
cuisson, il est éteint et il refroidit.

\smallskip

On s'intéresse à la phase de refroidissement du four, qui débute dès l'instant où il est éteint.

\smallskip
La température du four est exprimée en degré Celsius (\degres~C).

\smallskip

La porte du four peut être ouverte sans risque pour les céramiques dès que sa température est
inférieure à $70$\degres~C. Sinon les céramiques peuvent se fissurer, voire se casser.

\bigskip

\textbf{Partie A}

\medskip

Pour un nombre entier naturel $n$, on note $T_n$ la température en degré Celsius du four au bout
de $n$ heures écoulées à partir de l'instant où il a été éteint. On a donc $T_0 = \np{1000}$.

La température $T_n$ est calculée par l'algorithme suivant :

\begin{center}
\begin{tabularx}{0.35\linewidth}{|X|}\hline
$T \gets \np{1000}$\\
Pour $i$ allant de $1$ à $n$\\
\hspace{1cm}$T \gets 0,82 \times T + 3,6$\\
Fin Pour\\\hline
\end{tabularx}
\end{center}

\medskip

\begin{enumerate}
\item Déterminer la température du four, arrondie à l'unité, au bout de $4$ heures de
refroidissement.
\item  Démontrer que, pour tout nombre entier naturel $n$, on a : $T_n = 980 \times 0,82^n + 20$.
\item  Au bout de combien d'heures le four peut-il être ouvert sans risque pour les céramiques ?
 \end{enumerate}
 
\bigskip

\textbf{Partie B}

\medskip

Dans cette partie, on note $t$ le temps (en heure) écoulé depuis l'instant où le four a été éteint.

La température du four (en degré Celsius) à l'instant $t$ est donnée par la fonction $f$ définie,
pour tout nombre réel $t$ positif, par : 

\[f(t) = a\text{e}^{- \frac{t}{5}} + b,\]

où $a$ et $b$ sont deux nombres réels.

On admet que $f$ vérifie la relation suivante : $f'(t) + \dfrac{1}{5}f(t) = 4$.
\medskip

\begin{enumerate}
\item Déterminer les valeurs de $a$ et $b$ sachant qu'initialement, la température du four est de
$\np{1000}$~\degres C, c'est-à-dire que $f(0) = \np{1000}$.
\item  Pour la suite, on admet que, pour tout nombre réel positif $t$: 

\[f(t) = 980\text{e}^{- \frac{t}{5}} + 20.\]

\medskip

	\begin{enumerate}
		\item Déterminer la limite de $f$ lorsque $t$ tend vers $+ \infty$.
		\item Étudier les variations de $f$ sur $[0~;~+ \infty[$. 
		
En déduire son tableau de variations complet.
		\item Avec ce modèle, après combien de minutes le four peut-il être ouvert sans risque pour
 les céramiques ?
 	\end{enumerate}
\item  La température moyenne (en degré Celsius) du four entre deux instants $t_1$ et $t_2$ est donnée
par: $\dfrac{1}{t_2 - t_1}\displaystyle\int_{t_1}^{t_2} f(t)\:\text{d}t$.

	\begin{enumerate}
		\item À l'aide de la représentation graphique de $f$ ci-dessous, donner une estimation de la
température moyenne $\theta$ du four sur les $15$ premières heures de refroidissement.
		
Expliquer votre démarche.
		
\begin{center}
\psset{xunit=0.6cm,yunit=0.01cm}
\begin{pspicture}(-1,-50)(19,1100)
\multido{\n=0+1}{20}{\psline[linestyle=dashed,linewidth=0.5pt](\n,0)(\n,1100)}
\multido{\n=0+100}{11}{\psline[linestyle=dashed,linewidth=0.5pt](0,\n)(19,\n)}
\psaxes[linewidth=1.25pt,Dy=200]{->}(0,0)(0,0)(19,1100)
\psaxes[linewidth=1.25pt,Dy=200](0,0)(0,0)(19,1100)
\uput[d](16.4,-40){temps écoulé (en heure)}
\uput[r](0,1080){température (en \degres C)}
\psplot[plotpoints=3000,linewidth=1.25pt,linecolor=blue]{0}{19}{980 2.71828 0.2 x mul exp div 20 add}
\end{pspicture}
\end{center}
\medskip

		\item  Calculer la valeur exacte de cette température moyenne $\theta$ et en donner la valeur
arrondie au degré Celsius.
	\end{enumerate}
\item  Dans cette question, on s'intéresse à l'abaissement de température (en degré Celsius) du
four au cours d'une heure, soit entre deux instants $t$ et $(t + 1)$. Cet abaissement est donné
par la fonction $d$ définie, pour tout nombre réel $t$ positif, par : $d(t) = f(t) - f(t + 1)$.
	\begin{enumerate}
		\item Vérifier que. pour tout nombre réel $t$ positif: $d(t) = 980\left(1 - \text{e}^{- \frac{1}{5}}\right)\text{e}^{- \frac{t}{5}}$.
		\item Déterminer la limite de $d(t)$ lorsque $t$ tend vers $+ \infty$.
		
Quelle interprétation peut-on en donner ?
 	\end{enumerate}
\end{enumerate}
\end{exercice}
\newpage

\section{Complexes}
\begin{exercice}[Commun- Complexes][4]


Le plan est muni d'un repère orthonormé \Ouv.

\smallskip

Les points A, B et C ont pour affixes respectives $a = - 4,\: b = 2$ et $c = 4$.

\medskip

\begin{enumerate}
\item On considère les trois points A$'$, B$'$ et C$'$ d'affixes respectives $a'= \text{j}a$, $b'= \text{j}b$ et $c'= \text{j}c$ où j est le nombre complexe $-\dfrac{1}{2} + \text{i}\dfrac{\sqrt{3}}{2}$.

	\begin{enumerate}
		\item Donner la forme trigonométrique et la forme exponentielle de j.
		
En déduire les formes algébriques et exponentielles de $a'$, $b'$ et $c'$.
		\item Les points A, B et C ainsi que les cercles de centre O et de rayon 2, 3 et 4 sont
représentés sur le graphique fourni en \textbf{Annexe}.
		
Placer les points A$'$, B$'$ et C$'$ sur ce graphique.
	\end{enumerate}
\item  Montrer que les points A$'$, B$'$ et C$'$ sont alignés.
\item  On note M le milieu du segment [A$'$C], N le milieu du segment [C$'$C] et P le milieu du
segment [C$'$A]. 
	
Démontrer que le triangle MNP est isocèle.
\end{enumerate}
\end{exercice}

\section{Probabilités}
\begin{exercice}[Commun - Proba][5]


\medskip

Une entreprise conditionne du sucre blanc provenant de deux exploitations U et V en paquets
de 1 kg et de différentes qualités.

\smallskip

Le sucre extra fin est conditionné séparément dans des paquets portant le label \og  extra fin \fg.

\smallskip

\emph{Les parties \rm{A}, \rm{B} et \rm{C} peuvent être traitées de façon indépendante.}

\smallskip

Dans tout l'exercice, les résultats seront arrondis, si nécessaire, au millième.

\bigskip

\textbf{Partie A}

\medskip

Pour calibrer le sucre en fonction de la taille de ses cristaux, on le fait passer au travers d'une
série de trois tamis positionnés les uns au-dessus des autres et posés sur un récipient à fond
étanche.
Les ouvertures des mailles sont les suivantes :

\begin{center}
\psset{unit=1cm}
\begin{pspicture}(12,4)
%\psgrid
\uput[r](0.5,3){Tamis 1 : 0,8 mm} \psline[linewidth=1.5pt]{->}(4.6,1)(6.8,1)
\uput[r](0.5,2){Tamis 2 : 0,5 mm} \psline[linewidth=1.5pt]{->}(4.6,2)(6.8,2)
\uput[r](0.5,1){Tamis 3 : 0,2 mm} \psline[linewidth=1.5pt]{->}(4.6,3)(6.8,3)
\uput[r](0.5,0.5){Récipient à fond étanche}  \psline[linewidth=1.5pt]{->}(4.6,0.5)(7,0.5)
\psline[linewidth=2pt](7,3.75)(7,0)(12,0)(12,3.75)
\multido{\n=7.00+0.12}{42}{\psframe(\n,2.94)(\n,3.06)}
\multido{\n=7.00+0.1}{50}{\psframe(\n,1.96)(\n,2.04)}
\multido{\n=7.00+0.06}{84}{\psframe(\n,0.97)(\n,1.03)}
\end{pspicture}
\end{center}

Les cristaux de sucre dont la taille est inférieure à $0,2$ mm se trouvent dans le récipient à fond
étanche à la fin du calibrage. Ils seront conditionnés dans des paquets portant le label \og  sucre
extra fin \fg.

\medskip

\begin{enumerate}
\item On prélève au hasard un cristal de sucre de l'exploitation U. La taille de ce cristal,
exprimée en millimètre, est modélisée par la variable aléatoire $X_{\text{ U}}$ qui suit la loi normale
de moyenne $\mu_{\text{ U}} = 0,58$~mm et d'écart type $\sigma_{\text{ U}} = 0,21$~mm.

	\begin{enumerate}
	\item Calculer les probabilités des évènements suivants : $X_{\text{ U}} < 0,2$ et $0,5 \leqslant X_{\text{ U}} < 0,8$.
	\item On fait passer \np{1800} grammes de sucre provenant de l'exploitation U au travers de la
série de tamis.
	
Déduire de la question précédente une estimation de la masse de sucre récupérée dans
le récipient à fond étanche et une estimation de la masse de sucre récupérée dans le
tamis 2.
 		\end{enumerate}
\item On prélève au hasard un cristal de sucre de l'exploitation V. La taille de ce cristal,
exprimée en millimètre, est modélisée par la variable aléatoire $X_{\text{V}}$ qui suit la loi normale
de moyenne $\mu_{\text{V}} = 0,65$ mm et d'écart type $\sigma_{\text{V}}$ à déterminer.

Lors du calibrage d'une grande quantité de cristaux de sucre provenant de l'exploitation V,
on constate que 40\,\% de ces cristaux se retrouvent dans le tamis 2.

Quelle est la valeur de l'écart type $\sigma_{\text{V}}$ de la variable aléatoire $X_{\text{V}}$ ?
\end{enumerate}

\bigskip

\textbf{Partie B}

\medskip

Dans cette partie, on admet que 3\,\% du sucre provenant de l'exploitation U est extra fin et que
5\,\% du sucre provenant de l'exploitation V est extra fin.

On prélève au hasard un paquet de sucre dans la production de l'entreprise et, dans un souci
de traçabilité, on s'intéresse à la provenance de ce paquet.

On considère les évènements suivants:

\setlength\parindent{9mm}
\begin{itemize}
\item[$\bullet~~$] $U$ : \og  Le paquet contient du sucre provenant de l'exploitation U \fg{} ;
\item[$\bullet~~$] $V$ : \og Le paquet contient du sucre provenant de l'exploitation V \fg{} ;
\item[$\bullet~~$] $E$ : \og Le paquet porte le label "extra fin" \fg{}.
 \end{itemize}
\setlength\parindent{0mm} 

\medskip
 
\begin{enumerate}
\item Dans cette question, on admet que l'entreprise fabrique 30\,\% de ses paquets avec du sucre
provenant de l'exploitation U et les autres avec du sucre provenant de l'exploitation V,
sans mélanger les sucres des deux exploitations.
	\begin{enumerate}
		\item Quelle est la probabilité que le paquet prélevé porte le label \og extra fin \fg{} ?
		\item Sachant qu'un paquet porte le label \og extra fin \fg, quelle est la probabilité que le sucre
qu'il contient provienne de l'exploitation U ?
 	\end{enumerate}
\item L'entreprise souhaite modifier son approvisionnement auprès des deux exploitations afin
que parmi les paquets portant le label « extra fin », 30\,\% d'entre eux contiennent du sucre
provenant de l'exploitation U.
	
Comment doit-elle s'approvisionner auprès des exploitations U et V ?
	
\emph{Toute trace de recherche sera valorisée dans cette question}.
\end{enumerate}

\bigskip

\textbf{Partie C}

\medskip

\begin{enumerate}
\item L'entreprise annonce que 30\,\% des paquets de sucre portant le label «extra fin» qu'elle
conditionne contiennent du sucre provenant de l'exploitation U.

Avant de valider une commande, un acheteur veut vérifier cette proportion annoncée. Il
prélève $150$ paquets pris au hasard dans la production de paquets labellisés \og extra fin \fg{} de
l'entreprise. Parmi ces paquets, $30$ contiennent du sucre provenant de l'exploitation U.

A-t-il des raisons de remettre en question l'annonce de l'entreprise ?
\item  L'année suivante, l'entreprise déclare avoir modifié sa production. L'acheteur souhaite
estimer la nouvelle proportion de paquets de sucre provenant de l'exploitation U parmi les
paquets portant le label \og extra fin \fg. 

Il prélève 150 paquets pris au hasard dans la production de paquets labellisés \og extra fin \fg{} de l'entreprise. Parmi ces paquets 42\,\% contiennent du sucre provenant de l'exploitation U.

Donner un intervalle de confiance, au niveau de confiance 95\,\%, de la nouvelle proportion
de paquets labellisés \og extra fin \fg{} contenant du sucre provenant de l'exploitation U.
\end{enumerate}
\end{exercice}
\vspace{0,5cm}

\section{Espace}
\begin{exercice}[Obli - Espace][5]


Dans l'espace muni du repère orthonormé \Oijk d'unité 1~cm, on considère les points
A, B, C et D de coordonnées respectives (2~;~1~;~4), $(4~;~-1~;~0)$, $(0~;~3~;~2)$ et $(4~;~3~;~-2)$.

\medskip

\begin{enumerate}
\item Déterminer une représentation paramétrique de la droite (CD).
\item Soit $M$ un point de la droite (CD).
	\begin{enumerate}
		\item Déterminer les coordonnées du point $M$ tel que la distance B$M$ soit minimale.
		\item On note H le point de la droite (CD) ayant pour coordonnées $(3~;~3~;~- 1)$.
Vérifier que les droites (BH) et (CD) sont perpendiculaires.
		\item Montrer que l'aire du triangle BCD est égale à 12 cm$^2$.
	\end{enumerate}
\item 
	\begin{enumerate}
		\item Démontrer que le vecteur $\vect{n}\begin{pmatrix}2\\1\\2\end{pmatrix}$  est un vecteur normal au plan (BCD).
		\item Déterminer une équation cartésienne du plan (BCD).
		\item Déterminer une représentation paramétrique de la droite $\Delta$ passant par A et orthogonale
au plan (BCD).
		\item Démontrer que le point I, intersection de la droite $\Delta$ et du plan (BCD) a pour
coordonnées $\left(\dfrac{2}{3}~;~\dfrac{1}{3}~;~\dfrac{8}{3}\right)$.
	\end{enumerate}
\item  Calculer le volume du tétraèdre ABCD.
\end{enumerate}
\end{exercice}

\section{Spécialité}
\begin{exercice}[Spécialité][5]

À toute lettre de l'alphabet on associe un nombre entier $x$ compris entre 0 et 25 comme
indiqué dans le tableau ci-dessous:

\begin{center}
\begin{tabularx}{\linewidth}{|c|*{13}{>{\centering \arraybackslash}X|}}\hline
Lettre 	&A &B &C &D &E &F &G &H &I &J &K 	&L 	&M\\ \hline
$x$ 	&0 &1 &2 &3 &4 &5 &6 &7 &8 &9 &10 	&11 &12\\ \hline\hline
Lettre 	&N &O &P &Q &R &S &T &U &V &W &X 	&Y 	&Z\\ \hline
$x$ 	&13&14&15&16&17&18&19&20&21&22&23 	&24 &25\\ \hline
\end{tabularx}
\end{center}

\medskip

Le \og chiffre de RABIN \fg{} est un dispositif de cryptage asymétrique inventé en 1979 par
l'informaticien Michael Rabin.

\smallskip

Alice veut communiquer de manière sécurisée en utilisant ce cryptosystème. Elle choisit deux
nombres premiers distincts $p$ et $q$. Ce couple de nombres est sa clé privée qu'elle garde
secrète.

Elle calcule ensuite $n = p \times q$ et elle choisit un nombre entier naturel $B$ tel que $0 \leqslant B \leqslant n -1$.

Si Bob veut envoyer un message secret à Alice, il le code lettre par lettre.

Le codage d'une lettre représentée par le nombre entier $x$ est le nombre $y$ tel que :

\[y \equiv  x(x + B)\:\: [n] \:\text{ avec }\: 0 \leqslant y \leqslant n.\]

Dans tout l'exercice on prend $p = 3,\: q = 11$ donc $n = p \times q = 33$ et $B = 13$.

\bigskip

\textbf{Partie A : Cryptage}

\medskip

Bob veut envoyer le mot \og  NO \fg{} à Alice.

\medskip
\begin{enumerate}
\item Montrer que Bob code la lettre \og N \fg{} avec le nombre 8.
\item Déterminer le nombre qui code la lettre \og O \fg.
\end{enumerate}

\bigskip

\textbf{Partie B : Décryptage}

\medskip

Alice a reçu un message crypté qui commence par le nombre 3.

Pour décoder ce premier nombre, elle doit déterminer le nombre entier $x$ tel que :

\[x(x + 13) \equiv  3 \:\: [33]\:  \text{ avec }\: 0 \leqslant  x < 26.\]

\medskip

\begin{enumerate}
\item Montrer que $x(x + 13) \equiv 3\:\: [33]$ équivaut à $(x + 23)^2 \equiv 4\:\: [33]$.
\item
	\begin{enumerate}
		\item Montrer que si $(x + 23)^2 \equiv 4\:\: [33]$ alors le système d'équations $\left\{\begin{array}{l c l}
(x + 23)^2 &\equiv &4 \:\: [3]\\ 
(x + 23)^2 &\equiv &4 \:\: [11]
\end{array}\right.$ est vérifié.
		\item Réciproquement, montrer que si  $\left\{\begin{array}{l c l}
(x + 23)^2 &\equiv &4\:\: [3]\\ 
(x + 23)^2 &\equiv &4 \:\: [11]
\end{array}\right.$ alors $(x + 23)^2 \equiv 4\:\: [33]$.
		\item En déduire que $x(x + 13) \equiv 3\:\: [33] \iff  \left\{\begin{array}{l c l}
(x + 23)^2 &\equiv&1 \:\: [3]\\
(x + 23)^2 &\equiv& 4 \:\: [11]
\end{array}\right.$
	\end{enumerate}
\item
	\begin{enumerate}
		\item Déterminer les nombres entiers naturels $a$ tels que $0 \leqslant a < 3$ et $a^2 \equiv 1 \:\:  [3]$.
		\item Déterminer les nombres entiers naturels $b$ tels que $0 \leqslant b < 11$ et $b^2 \equiv 4\:\: [11]$.
 	\end{enumerate}
\item
	\begin{enumerate}
		\item En déduire que $x(x + 13) \equiv 3 \quad[33]$ équivaut aux quatre systèmes suivants :
		
\[\left\{\begin{array}{l c l}
x &\equiv&2\quad [3]\\
x&\equiv &8\quad[11]
\end{array}\right. \: \text{ ou } \left\{\begin{array}{l c l}
 x &\equiv& 0\quad[3]\\
 x &\equiv& 1 \quad[11]
 \end{array}\right.\: \text{ ou } \left\{\begin{array}{l c l}
x  &\equiv& 2\quad[3]\\
x &\equiv&1 \quad[11]
\end{array}\right.\: \text{ ou } \left\{\begin{array}{l c l}
x &\equiv& 0\quad [3]\\
x &\equiv& 8 \quad [11]
\end{array}\right.\]

		\item On admet que chacun de ces systèmes admet une unique solution entière $x$ telle que

$0 \leqslant x < 33$.

Déterminer, sans justification, chacune de ces solutions.
	\end{enumerate}
\item Compléter l'algorithme en \textbf{Annexe} pour qu'il affiche les quatre solutions trouvées dans la
question précédente.
\item Alice peut-elle connaître la première lettre du message envoyé par Bob ? 
	
Le \og chiffre de RABIN \fg{} est-il utilisable pour décoder un message lettre par lettre ?
\end{enumerate}
\end{exercice}
\newpage
\section{Annexes}
\begin{center}

\textbf{\Large ANNEXE}

\bigskip

\textbf{\Large À COMPLÉTER ET À REMETTRE AVEC LA COPIE}

\begin{flushleft}
\textbf{\large EXERCICE 2}
\end{flushleft}

\bigskip

\psset{unit=1cm}
\begin{pspicture}(-4,-4)(4,4)
\psgrid[gridlabels=0pt,subgriddiv=1,gridwidth=0.1pt]
\psaxes[linewidth=1pt,Dx=10,Dy=10](0,0)(-4,-4)(4,4)
\psaxes[linewidth=1.5pt,Dx=10,Dy=10]{->}(0,0)(1,1)
\uput[d](0.5,0){$\vect{u}$}\uput[l](0,0.5){$\vect{v}$}\uput[dl](0,0){O}
\uput[dr](2,0){B} \uput[dr](4,0){C}\uput[dl](-4,0){A}
\pscircle(0,0){2}\pscircle(0,0){3}\pscircle(0,0){4}
\psdots(-4,0)(2,0)(4,0)
\end{pspicture}

\vspace{1.5cm}


\begin{flushleft}
\textbf{\large EXERCICE 4  (spécialité)}
\end{flushleft}

\vspace{1.5cm}

\begin{tabularx}{0.7\linewidth}{|X|}\hline
Pour ...... allant de ......à .......\\
\quad Si le reste de la division de ....... par ....... est égal à ....... alors\\
\qquad Afficher .......\\
\quad Fin Si\\
Fin Pour\\ \hline
\end{tabularx}
\end{center}
\end{document}