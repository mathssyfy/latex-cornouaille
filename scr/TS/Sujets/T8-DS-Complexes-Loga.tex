\documentclass{cornouaille}
\begin{document}
\cornouaille{TS}{Devoir Surveillé}{Lundi 30 janvier 2019}

\Bareme

%D'après Pondichery mai 2018
\begin{exercice}[Les complexes sont nos amis]
Le plan est muni d'un repère orthonormé \Ouv.

\smallskip

Les points A, B et C ont pour affixes respectives $a = - 4,\: b = 2$ et $c = 4$.

\medskip

\begin{enumerate}
\item On considère les trois points A$'$, B$'$ et C$'$ d'affixes respectives $a'= \text{j}a$, $b'= \text{j}b$ et $c'= \text{j}c$ où j est le nombre complexe $-\dfrac{1}{2} + \text{i}\dfrac{\sqrt{3}}{2}$.

	\begin{enumerate}
		\item Donner la forme trigonométrique et la forme exponentielle de j.
		
En déduire les formes algébriques et exponentielles de $a'$, $b'$ et $c'$.
		\item Les points A, B et C ainsi que les cercles de centre O et de rayon 2, 3 et 4 sont
représentés sur le graphique ci-dessous.
		
Placer les points A$'$, B$'$ et C$'$ sur ce graphique.
	\end{enumerate}
\item  Montrer que les points A$'$, B$'$ et C$'$ sont alignés.
\item  On note M le milieu du segment [A$'$C], N le milieu du segment [C$'$C] et P le milieu du
segment [C$'$A]. 
	
Démontrer que le triangle MNP est isocèle.
\end{enumerate}

\bigskip

\begin{center}

Graphique à compléter

\psset{unit=1cm}
\begin{pspicture}(-4,-4)(4,4)
\psgrid[gridlabels=0pt,subgriddiv=1,gridwidth=0.1pt]
\psaxes[linewidth=1pt,Dx=10,Dy=10](0,0)(-4,-4)(4,4)
\psaxes[linewidth=1.5pt,Dx=10,Dy=10]{->}(0,0)(1,1)
\uput[d](0.5,0){$\vect{u}$}\uput[l](0,0.5){$\vect{v}$}\uput[dl](0,0){O}
\uput[dr](2,0){B} \uput[dr](4,0){C}\uput[dl](-4,0){A}
\pscircle(0,0){2}\pscircle(0,0){3}\pscircle(0,0){4}
\psdots(-4,0)(2,0)(4,0)
\end{pspicture}
\end{center}

\end{exercice}
\end{document}