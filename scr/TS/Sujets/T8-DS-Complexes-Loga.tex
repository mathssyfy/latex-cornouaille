\documentclass{cornouaille}
\begin{document}
\cornouaille{TS}{Devoir Surveillé}{Lundi 30 janvier 2019}

\Bareme

%D'après Pondichery mai 2018
\begin{exercice}[Les complexes sont nos amis]
Le plan est muni d'un repère orthonormé \Ouv.

\smallskip

Les points A, B et C ont pour affixes respectives $a = - 4,\: b = 2$ et $c = 4$.

\medskip

\begin{enumerate}
\item On considère les trois points A$'$, B$'$ et C$'$ d'affixes respectives $a'= \text{j}a$, $b'= \text{j}b$ et $c'= \text{j}c$ où j est le nombre complexe $-\dfrac{1}{2} + \text{i}\dfrac{\sqrt{3}}{2}$.

	\begin{enumerate}
		\item Donner la forme trigonométrique et la forme exponentielle de j.
		
En déduire les formes algébriques et exponentielles de $a'$, $b'$ et $c'$.
		\item Les points A, B et C ainsi que les cercles de centre O et de rayon 2, 3 et 4 sont
représentés sur le graphique ci-dessous.
		
Placer les points A$'$, B$'$ et C$'$ sur ce graphique.
	\end{enumerate}
\item  Montrer que les points A$'$, B$'$ et C$'$ sont alignés.
\item  On note M le milieu du segment [A$'$C], N le milieu du segment [C$'$C] et P le milieu du
segment [C$'$A]. 
	
Démontrer que le triangle MNP est isocèle.
\end{enumerate}

\bigskip

\begin{center}

Graphique à compléter

\psset{unit=1cm}
\begin{pspicture}(-4,-4)(4,4)
\psgrid[gridlabels=0pt,subgriddiv=1,gridwidth=0.1pt]
\psaxes[linewidth=1pt,Dx=10,Dy=10](0,0)(-4,-4)(4,4)
\psaxes[linewidth=1.5pt,Dx=10,Dy=10]{->}(0,0)(1,1)
\uput[d](0.5,0){$\vect{u}$}\uput[l](0,0.5){$\vect{v}$}\uput[dl](0,0){O}
\uput[dr](2,0){B} \uput[dr](4,0){C}\uput[dl](-4,0){A}
\pscircle(0,0){2}\pscircle(0,0){3}\pscircle(0,0){4}
\psdots(-4,0)(2,0)(4,0)
\end{pspicture}
\end{center}

\end{exercice}

\newpage

%D'après Amérique du Nord Mai 2018
\begin{exercice}[Un peu de physique, mais pas trop...]
Lors d'une expérience en laboratoire, on lance un projectile dans un milieu fluide. L'objectif est de déterminer pour quel angle de tir
$\theta$ par rapport à l'horizontale la hauteur du projectile ne dépasse
pas $1,6$ mètre.

Comme le projectile ne se déplace pas dans l'air mais dans un
fluide, le modèle parabolique usuel n'est pas adopté.

On modélise ici le projectile par un point qui se déplace, dans un
plan vertical, sur la courbe représentative de la fonction $f$ définie
sur l'intervalle [0~;~1[ par:

\[f(x) = bx + 2\ln (1- x)\]

où $b$ est un paramètre réel supérieur ou égal à $2$, $x$ est l'abscisse
du projectile, $f(x)$ son ordonnée, toutes les deux exprimées en mètres.

\begin{center}
\psset{unit=4cm,comma=true}
\begin{pspicture*}(-0.15,-0.15)(1.1,1.7)
\psgrid[gridlabels=0pt,subgriddiv=10,gridwidth=0.3pt,subgridwidth=0.15pt](0,0)(1.1,1.7)
\psaxes[linewidth=1pt,Dx=0.5,Dy=0.5,labelFontSize=\scriptstyle](0,0)(0,0)(1.1,1.7)
\psaxes[linewidth=1.5pt]{->}(0,0)(1,1)
\psplot[plotpoints=3000,linewidth=1.25pt,linecolor=blue]{0}{0.932}{5.69 x mul 1 x sub ln 2 mul add}
\psline[linestyle=dotted,linewidth=1pt](0.4,1.5)
\psarc(0,0){0.15}{0}{72}
\end{pspicture*}

\end{center}

\begin{enumerate}
\item La fonction $f$ est dérivable sur l'intervalle [0~;~1[. On note $f'$ sa fonction dérivée.
\begin{enumerate}
	\item Démontrer que, pour tout réel
$x$ de l'intervalle [0~;~1[ :

\[f'(x) = \dfrac{- bx + b - 2}{1 - x}.\]
\item Démontrer que fonction $f$ possède un maximum sur l'intervalle [0~;~1[ et que, 

le maximum de la fonction $f$ est égal à $b - 2 + 2\ln \left(\dfrac{2}{b}\right)$.
\end{enumerate}

\item  Déterminer pour quelles valeurs du paramètre $b$ la hauteur maximale du projectile ne dépasse
pas $1,6$~mètre.
\item  Dans cette question, on choisit $b = 5,69$.

L'angle de tir $\theta$ correspond à l'angle entre l'axe des abscisses et la tangente à la courbe de la
fonction $f$ au point d'abscisse $0$ comme indiqué sur le schéma donné ci-dessus.

Déterminer une valeur approchée au dixième de degré près de l'angle $\theta$.
\end{enumerate}
\end{exercice}

%Daprès Centres étrangers juin 2018
\begin{exercice}[VRAI-FAUX]
Pour chacune des quatre affirmations suivantes,
 indiquer si elle est vraie ou fausse, 
en justifiant la réponse.
 Il est attribué un point par réponse exacte correctement justifiée.
 Une réponse inexacte ou
non justifiée ne rapporte ni n'enlève aucun point.

\begin{enumerate}

\item  On considère dans $\R$ l'équation :

\[\ln (6 x - 2) + \ln (2x - 1) = \ln (x).\]



\textbf{Affirmation} l'équation admet deux solutions dans l'intervalle $\left]\dfrac{1}{2}~;~+ \infty\right[$.
\item  On considère dans $\C$ l'équation : 

\[\left(4z^2 - 20z + 37\right)(2z -7 + 2\text{i}) = 0.\]



\textbf{Affirmation} les solutions de l'équation sont les affixes de points appartenant à un même
cercle de centre le point P d'affixe $2$.
\end{enumerate}
\end{exercice}
\end{document}