\documentclass{cornouaille}
\begin{document}
\cornouaille{TS}{Devoir Surveillé 5}{Lundi 30 janvier 2019}

\Bareme
\begin{center}
\encadre{
Le sujet est composé d'un certain nombre d'exercices indépendants.Vous devez traiter tous les exercices.\\
Pour chaque exercice, vous pouvez admettre un résultat précédemment donné dans le texte pour aborder les questions suivantes, à condition de l'indiquer clairement sur la copie.\\
Vous êtes invités à faire figurer sur votre copie toute trace de recherche, même incomplète ou non fructueuse, que vous aurez développée.\\
Il est rappelé que la qualité de la rédaction, la clarté et la précision des raisonnements seront prises en compte dans l'appréciation des copies.\\
}
\end{center}
%D'après Pondichery mai 2018
\begin{exercice}[Les complexes sont nos amis][6]
Le plan est muni d'un repère orthonormé \Ouv.

\smallskip

Les points A, B et C ont pour affixes respectives $a = - 4,\: b = 2$ et $c = 4$.

\medskip

\begin{enumerate}
\item On considère les trois points A$'$, B$'$ et C$'$ d'affixes respectives $a'= \text{j}a$, $b'= \text{j}b$ et $c'= \text{j}c$ où j est le nombre complexe:
$$j=-\dfrac{1}{2} + \text{i}\dfrac{\sqrt{3}}{2}$$

	\begin{enumerate}
		\item Donner la forme trigonométrique et la forme exponentielle de j.

\begin{solution}
$\text{j}=-\dfrac{1}{2} + \text{i}\dfrac{\sqrt{3}}{2}=\cos\left(\dfrac{2\pi}{3}\right) + \text{i}\sin\left(\dfrac{2\pi}{3}\right)=e^{\frac{2\text{i}\pi}{3}}$

$a'=a\text{j}=-4\text{j}=2-2\text{i}\sqrt{3}=4\left( -e^{\frac{2\text{i}\pi}{3}}\right)=4\left( e^{\text{i}\pi}\e{\frac{2\text{i}\pi}{3}}\right)=4e^{\text{i}\left( \pi+\frac{2\pi}{3}\right)  }=4e^{\frac{5\text{i}\pi}{3}}=4e^{-\frac{\text{i}\pi}{3}}$

$b'= b\text{j}=2\text{j}=-1+\text{i}\sqrt{3}=2e^{\frac{2\text{i}\pi}{3}}$

$c'= c\text{j}=4\text{j}=-2+2\text{i}\sqrt{3}=4e^{\frac{2\text{i}\pi}{3}}$
\end{solution}
		
En déduire les formes algébriques et exponentielles de $a'$, $b'$ et $c'$.
		\item Les points A, B et C ainsi que les cercles de centre O et de rayon 2, 3 et 4 sont
représentés sur le graphique ci-dessous.
		
Placer les points A$'$, B$'$ et C$'$ sur ce graphique.

\begin{solution}
\begin{center}
\psset{unit=0.6cm}
\begin{pspicture}(-4,-4)(4,4)
\psgrid[gridlabels=0pt,subgriddiv=1,gridwidth=0.1pt]
\psaxes[linewidth=1pt,Dx=10,Dy=10](0,0)(-4,-4)(4,4)
\psaxes[linewidth=1.5pt,Dx=10,Dy=10]{->}(0,0)(1,1)
\uput[d](0.5,0){$\vect{u}$}\uput[l](0,0.5){$\vect{v}$}\uput[dl](0,0){O}
\uput[dr](2,0){B} \uput[dr](4,0){C}\uput[dl](-4,0){A}
\pscircle(0,0){2}\pscircle(0,0){3}\pscircle(0,0){4}
\psdots(-4,0)(2,0)(4,0)

\psdots[linecolor=red](2,-3.464)(-1,1.732)(-2,3.464)
\uput[dr](2,-3.464){\red A$'$}
\uput[dr](-1,1.732){\red B$'$}
\uput[dr](-2,3.464){\red C$'$}

\psdots[linecolor=blue](3,-1.732)(1,1.732)(-3,1.732)
\uput[dr](3,-1.732){\blue M}
\uput[ur](1,1.732){\blue N}
\uput[ul](-3,1.732){\blue P}
\psline[linecolor=blue](3,-1.732)(1,1.732)(-3,1.732)(3,-1.732)
\end{pspicture}
\end{center}
$|a'|=4$ donc A$'$ est sur le cercle de centre O et de rayon 4 et on a $Re\left(a' \right) =2$ et $Im\left(a' \right)<0$, on peut donc placer A$'$


$|b'|=2$ donc B$'$ est sur le cercle de centre O et de rayon 2 et on a $Re\left(b' \right) =-1$ et $Im\left(b' \right)>0$, on peut donc placer B$'$


$|c'|=4$ donc C$'$ est sur le cercle de centre O et de rayon 4 et on a $Re\left(c' \right) =-2$ et $Im\left(c' \right)>0$, on peut donc placer C$'$

\end{solution}

	\end{enumerate}
\item  Montrer que les points A$'$, B$'$ et C$'$ sont alignés.

\begin{solution}

$a'=-c'$ donc A$'$ et C$'$ sont symétriques par rapport à O alors O, A$'$ et C$'$ sont alignés

$arg\left( b'\right) =arg\left( c'\right) =\dfrac{2\pi}{3} (2\pi)$ donc $\vect{\text{OB}'}$ et $\vect{\text{OC}'}$ sont colinéaires d'où O, B$'$ et C$'$ sont alignés.

Finalement O, A$'$, B$'$ et C$'$ sont alignés.
\end{solution}

\item  On note M le milieu du segment [A$'$C], N le milieu du segment [C$'$C] et P le milieu du
segment [C$'$A]. 
	
Démontrer que le triangle MNP est isocèle.

\begin{solution}

$z_{\text{M}}=\dfrac{a'+c}{2}=3-\text{i}\sqrt{3}$

$z_{\text{N}}=\dfrac{c'+c}{2}=1+\text{i}\sqrt{3}$

$z_{\text{P}}=\dfrac{c'+a}{2}=-3+\text{i}\sqrt{3}$

MNP semble isocèle en N d'après le dessin

MN=$\left|z_{\text{N}}-z_{\text{M}} \right| = \left|2-2\text{i}\sqrt{3} \right| =4$

 et 

PN=$\left|z_{\text{N}}-z_{\text{P}} \right| =\left|4 \right| = 4$

On a MN=NP donc MNP est bien isocèle en N
\end{solution}
\end{enumerate}

\bigskip

\begin{center}

\textbf{Graphique à compléter}

\bigskip

\psset{unit=1cm}
\begin{pspicture}(-4,-4)(4,4)
\psgrid[gridlabels=0pt,subgriddiv=1,gridwidth=0.1pt]
\psaxes[linewidth=1pt,Dx=10,Dy=10](0,0)(-4,-4)(4,4)
\psaxes[linewidth=1.5pt,Dx=10,Dy=10]{->}(0,0)(1,1)
\uput[d](0.5,0){$\vect{u}$}\uput[l](0,0.5){$\vect{v}$}\uput[dl](0,0){O}
\uput[dr](2,0){B} \uput[dr](4,0){C}\uput[dl](-4,0){A}
\pscircle(0,0){2}\pscircle(0,0){3}\pscircle(0,0){4}
\psdots(-4,0)(2,0)(4,0)
\end{pspicture}
\end{center}

\end{exercice}

\newpage

%D'après Amérique du Nord Mai 2018
\begin{exercice}[Un peu de physique, mais pas trop...][6]
Lors d'une expérience en laboratoire, on lance un projectile dans un milieu fluide. L'objectif est de déterminer pour quel angle de tir
$\theta$ par rapport à l'horizontale la hauteur du projectile ne dépasse
pas $1,6$ mètre.

Comme le projectile ne se déplace pas dans l'air mais dans un
fluide, le modèle parabolique usuel n'est pas adopté.

On modélise ici le projectile par un point qui se déplace, dans un
plan vertical, sur la courbe représentative de la fonction $f$ définie
sur l'intervalle [0~;~1[ par:

\[f(x) = bx + 2\ln (1- x)\]

où $b$ est un paramètre réel supérieur ou égal à $2$, $x$ est l'abscisse
du projectile, $f(x)$ son ordonnée, toutes les deux exprimées en mètres.

\begin{center}
\psset{unit=4cm,comma=true}
\begin{pspicture*}(-0.15,-0.15)(1.1,1.7)
\psgrid[gridlabels=0pt,subgriddiv=10,gridwidth=0.3pt,subgridwidth=0.15pt](0,0)(1.1,1.7)
\psaxes[linewidth=1pt,Dx=0.5,Dy=0.5,labelFontSize=\scriptstyle](0,0)(0,0)(1.1,1.7)
\psaxes[linewidth=1.5pt]{->}(0,0)(1,1)
\psplot[plotpoints=3000,linewidth=1.25pt,linecolor=blue]{0}{0.932}{5.69 x mul 1 x sub ln 2 mul add}
\psline[linestyle=dotted,linewidth=1pt](0.4,1.5)
\psarc(0,0){0.15}{0}{72}
\end{pspicture*}

\end{center}

\begin{enumerate}
\item La fonction $f$ est dérivable sur l'intervalle [0~;~1[. On note $f'$ sa fonction dérivée.
\begin{enumerate}
	\item Démontrer que, pour tout réel
$x$ de l'intervalle [0~;~1[ :

\[f'(x) = \dfrac{- bx + b - 2}{1 - x}.\]
\item Démontrer que fonction $f$ possède un maximum sur l'intervalle [0~;~1[ et que, 

le maximum de la fonction $f$ est égal à $b - 2 + 2\ln \left(\dfrac{2}{b}\right)$.
\end{enumerate}

\item  Déterminer pour quelles valeurs du paramètre $b$ la hauteur maximale du projectile ne dépasse
pas $1,6$~mètre.
\item  Dans cette question, on choisit $b = 5,69$.

L'angle de tir $\theta$ correspond à l'angle entre l'axe des abscisses et la tangente à la courbe de la
fonction $f$ au point d'abscisse $0$ comme indiqué sur le schéma donné ci-dessus.

Déterminer une valeur approchée au dixième de degré près de l'angle $\theta$.
\end{enumerate}
\end{exercice}

%Daprès Centres étrangers juin 2018
\begin{exercice}[VRAI-FAUX][2]
Pour chacune des  affirmations suivantes,
 indiquer si elle est vraie ou fausse, 
en justifiant la réponse.
 Il est attribué un point par réponse exacte correctement justifiée.
 Une réponse inexacte ou
non justifiée ne rapporte ni n'enlève aucun point.

\begin{enumerate}

\item  On considère dans $\R$ l'équation :

\[\ln (6 x - 2) + \ln (2x - 1) = \ln (x).\]



\textbf{Affirmation} l'équation admet deux solutions dans l'intervalle $\left]\dfrac{1}{2}~;~+ \infty\right[$.
\item  On considère dans $\C$ l'équation : 

\[\left(4z^2 - 20z + 37\right)(2z -7 + 2\text{i}) = 0.\]



\textbf{Affirmation} les solutions de l'équation sont les affixes de points appartenant à un même
cercle de centre le point P d'affixe $2$.
\end{enumerate}
\end{exercice}

%Daprès Métropole juin 2018
\begin{exercice}[Complexes: suites...][6]
Le plan complexe est muni d'un repère orthonormé direct \Ouv.

On pose $z_0 = 8$ et, pour tout entier naturel $n$ :

\[z_{n+1} = \dfrac{3 - \text{i}\sqrt{3}}{4}z_n.\]

On note $A_n$ le point du plan d'affixe $z_n$.

\medskip

\begin{enumerate}
\item 
	\begin{enumerate}
		\item Vérifier que :
		
\[\dfrac{3 - \text{i}\sqrt{3}}{4} = \dfrac{\sqrt{3}}{2}\text{e}^{- \text{i}\frac{\pi}{6}}.\]
		
		\item En déduire l'écriture de chacun des nombres complexes $z_1$,  $z_2$ et $z_3$ sous forme exponentielle et vérifier que $z_3$ est un imaginaire pur dont on précisera la partie imaginaire.
		\item Représenter graphiquement les points $A_0$ , $A_1$ , $A_2$ et $A_3$ ; on prendra pour unité le centimètre.
 	\end{enumerate}
\item
	\begin{enumerate}
		\item Démontrer par récurrence que, pour tout entier naturel $n$,
		
\[z_n = 8 \times \left(\dfrac{\sqrt{3}}{2}\right)^n \text{e}^{- \text{i}\frac{n\pi}{6}}.\]
		
		\item Pour tout entier naturel $n$, on pose $u_n = \left|z_n\right|$.
		
Déterminer la nature et la limite de la suite $\left(u_n\right)$.
	\end{enumerate}
\item 
	\begin{enumerate}
		\item Démontrer que, pour tout entier naturel $k$,
		
		\[\dfrac{z_{k+1} - z_{k}}{z_{k+1}} = - \dfrac{1}{\sqrt{3}}\text{i}.\]

En déduire que, pour tout entier naturel $k$, on a l'égalité : $A_kA_{k+1} = \dfrac{1}{\sqrt{3}} \text{O}A_{k+1}$.
		\item Pour tout entier naturel $n$, on appelle $\ell_n$ la longueur de la ligne brisée reliant dans cet ordre les points $A_0$,\: $A_1$,\: $A_2$, \ldots , $A_n$.
		
On a ainsi : $\ell_n = A_0A_1 + A_1A_2 + \ldots + A_{n-1}A_n$.
		
Démontrer que la suite $\left(\ell_n\right)$ est convergente et calculer sa limite.
	\end{enumerate}
\end{enumerate}
\end{exercice}

\newpage
\section{Un peu de culture: Tacite}
54 - 117
 
en latin Publius Cornelius Tacitus
 
Historien romain (Interamna, Ombrie, ou Rome, entre 54 et 56 apr. J.-C. — ?, v. 120).
 
Tacite était issu d'une famille de l'ordre équestre de la Gaule transalpine, 
cette classe sociale dynamique et prospère qui servait de soutien à l'Empire depuis le déclin des familles patriciennes romaines.

 Maîtrisant parfaitement l'éloquence, 
il fit une brillante carrière politique: questeur, puis préteur (88); 
toutefois, pour ne pas attirer sur lui l'attention de l'empereur Domitien (81-96), 
toujours prêt à exiler ou à faire assassiner les personnages illustres de l'Empire, 
il n'accepta le consulat qu'en 97, sous l'empereur Nerva.
 Bien que provincial et appartenant à l'ordre équestre, 
il entra au Sénat. 
Malgré une vive critique de la rhétorique désuète et stérile des sénateurs,
 il reproduisit implicitement les préjugés traditionnels de la classe dirigeante 
et défendit avec fierté son appartenance à la "caste sénatoriale"
qui retrouva son lustre sous Trajan (98-117). 
C'est à cette situation qu'il dut de connaître constamment une vie calme.

Sa production littéraire, 
s'inscrivant dans le cadre de son amitié pour Trajan et Pline le Jeune,
 était appréciée par le milieu impérial. 
Tacite fut l'historien officieux du régime,
 ce qui ne l'empêcha pas d'être aussi un historien critique.
 Dans son Dialogue des orateurs, 
dont la date de publication n'est pas connue avec certitude,
 mais qui a dû être écrit vers 81,
 il se montre déjà un subtil historien et critique littéraire. 
Étudiant les causes du déclin de l'éloquence à Rome, 
il introduit la liberté comme élément d'explication, 
et tend à montrer que le régime impérial, 
en limitant la liberté politique 
et en subordonnant le talent des orateurs à la louange de l'empereur,
 déplace et pervertit la fonction de l'éloquence.
 Au temps de Cicéron, la rhétorique était un moyen d'arriver au pouvoir;
 sous Trajan, seule une rhétorique déférente permettait 
de s'assurer les bonnes grâces de l'empereur,
 donc un certain pouvoir.
 Le discours de Tacite est teinté d'un regret du temps de la liberté.
 Cette critique subtile fut tolérée par le régime de Trajan,
 qui se voulait libéral pour faire oublier le despotisme de Domitien.
Après la Vie d'Agricola (98), 
éloge funèbre de son beau-père,
 il publia la Germanie (vers 98),
 un traité sur les moeurs des Germains. 
Ces deux oeuvres sont l'occasion d'une étude sociologique sur les "Barbares".

Le récit des campagnes d'Agricola en Bretagne (la Grande-Bretagne actuelle),
 et la compilation des récits des campagnes romaines en Germanie
 permettent à Tacite de brosser 
un tableau socio-historique de ces populations frontalières qui,
 à partir du moment où l'on commence à s'y intéresser,
 n'ont de barbare que le nom :
 dépassant les préjugés traditionnels,
 Tacite tend à reconnaître l'existence de cultures 
et de sociétés non romaines.
 Cette reconnaissance indirecte de la spécificité des Bretons et des Germains 
détruit l'image d'une barbarie homogène se définissant 
par opposition à la culture méditerranéenne.

\section{Une grande oeuvre historique}
Après les Histoires (publiées en 106), 
dont il ne reste que quatre livres, 
et qui décrivaient l'Empire de 69 à 96 
(depuis la mort de Néron jusqu'à la chute de Domitien), 
Tacite publia les Annales (écrites vers 115-117), 
qui constituent, sans doute, sa grande oeuvre historique. Elles sont consacrées à la période qui suit la mort d'Auguste; seuls nous sont parvenus les livres I à IV, un fragment des livres V et VI (sur Tibère) et les livres XI à XVI (deuxième partie du règne de Claude et quasi-totalité de celui de Néron). À la fois historien et moraliste, Tacite y dépeint avec pessimisme, dans un style d'une saisissante concision, les mentalités et les moeurs des hommes de son temps.


\end{document}